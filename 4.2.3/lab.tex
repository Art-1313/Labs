\documentclass[12pt,a4paper]{article}
\usepackage[T2A]{fontenc}
\usepackage[utf8]{inputenc}
\usepackage[english, russian]{babel}
\usepackage{indentfirst}
\usepackage{misccorr}
\usepackage{graphicx}
\usepackage{amsmath}
\usepackage{amssymb}
\usepackage{circuitikz}
\usepackage[font={small}]{caption}
\usepackage[left=20mm, top=20mm, right=20mm, bottom=20mm, nohead]{geometry}
\usepackage{float}
\usepackage{tabularx}
\usepackage{array}
\usepackage{longtable}
\usepackage{pstool}
\usepackage{pgfplots}
\usepackage{hhline}
\usepackage{multirow}
\usepackage{wrapfig}
\usepackage{pdfpages}

\DeclareCaptionLabelSeparator{fill}{.\\}

\DeclareCaptionLabelFormat{fullparents}{\bothIfFirst{#1}{~}#2}

\pgfplotsset{compat=1.17}

\begin{document}

	\begin{titlepage}
        \begin{center}
            {\LARGE Отчёт по лабораторной работе 4.2.3.\\}
            \vspace*{11cm}
                \textbf{\LARGE Интерферометр Релея.}
            \vspace*{6.5cm}
        \end{center}
        \hfill\begin{minipage}{0.37\textwidth}
            Работу выполнил Громов Артём
            \\
            ЛФИ Б02-006
        \end{minipage}
        \vspace{4.8cm}
        \begin{center}
            Долгопрудный, 2022 г.
        \end{center}
	\end{titlepage}

	\section{Аннотация}

        \begin{flushleft}
            \textbf{Цель работы:} ознакомление с интерференцией на двух щелях, устройством и принципом действия интерферометра Релея и с его
            применением для измерения показателей преломления газов.
        \end{flushleft}
        \begin{flushleft}
			\textbf{В работе используется:} технический интерферометр ИТР-1, светофильтр, баллон с углекислым газом, сильфон, манометр,
            краны.
        \end{flushleft}
		\par Для устранения недостатка попадения на экран малой доли светового потока источника при изучении интерференции был разработан
        интерферометр Релея (рис. 1).
        \begin{figure}[H]
        	\begin{center}
                \includegraphics[width=0.9\textwidth]{images/pic1.pdf}
                \caption{Принципиальная схема устройства интерферометра Релея.}
            \end{center}
        \end{figure}
		\par В этой схеме в качестве источника используют узкую щель $S$ (шириной $b$), освещаемую сколлимированным светом от лампочки Л. Щель
		расположена в фокальной плоскости объектива $O_1$. После экрана $D$ с двойной щелью два параллельных пучка проходят через кюветы 1 и 2
		и попадают на объектив $O_2$, в фокальной плоскости которого рассматривается интерференционная картина. Для обеспечения когерентности
		пучков 1 и 2 ширина входной щели $S$ выбирается так, чтобы радиус когерентности на экране $D$ был больше расстояния между щелями 1 и 2.
		На экране $F$ рассматривается дифракционное изображение щели $S$, ширина изображения определяется шириной одной щели на экране $D$. Если
		интерференция рассматривается в белом свете, ширина щелей на экране $D$ выбирается в 2–3 раза меньше расстояния между ними, так что в
		центральном максимуме видно 5–7 интерференционных полос.
		\par Для более точного измерения смещения полос из-за изменения коэффициента преломления в одном из плеч в интерферометре Релея
		используют две интерференционные картины. Верхняя часть пучков 1 и 2 проходит через кюветы K с газом. Нижняя часть, проходящая под
		кюветами, образует в фокальной плоскости объектива $O_2$ неподвижную интерференционную картину. Для отдельного рассмотрения каждой
		системы полос (пучки света 1 и 2 параллельны и системы полос в фокальной плоскости совпадают) используют окуляр с цилиндрической линзой,
		которая пространственно разделяет эти системы полос. При заполнении кювет газами с одинаковыми коэффициентами преломления обе системы
		полос (верхняя и нижняя) совпадают. Различие коэффициентов преломления газов в кюветах приводит к смещению верхней системы полос
		относительно нижней из-за возникновения между лучами 1 и 2 оптической разности хода $\Delta=L(n_2-n_1)$ ($L$ --- длина кювет). По числу
		полос между центрами обеих картин можно рассчитать изменение коэффициента преломления $n$.
		\par\textbf{Зависимость показателя преломления газа от давления и температуры.} Воспользуемся известной формулой диэлектрической
		проницаемости $\varepsilon$ для газа невзаимодействующих диполей:
		\begin{equation}
			\varepsilon=n^2=1+4\pi N\alpha,
		\end{equation}
		где $N$ --- концентрация молекул, $\alpha$ --- поляризуемость молекулы (в ед. СГС). Эта формула справедлива для разреженных газов, и
		коэффициент преломления их мало отличается от единицы. Учитывая зависимость давления $P$ газа от температуры $P = Nk_{\text{Б}}T$, где
		$k_{\text{Б}}$ --- константа Больцмана, получим соотношение
		\begin{equation}
			n-1\approx\frac{2\pi\alpha}{k_{\text{Б}}T}P.
		\end{equation}
		\par Тогда для разности показателей преломления $\delta n=n_2-n_1$, измеряемой с помощью интерферометра Релея, и разности давлений
		$\delta P$, измеряемой с помощью манометра, имеем простое соотношение:
		\begin{equation}
			\delta n=\frac{2\pi\alpha}{k_{\text{Б}}T}\delta P.
		\end{equation}

	\section{Экспериментальная установка}
		\par Схема прибора представлена на рис. 2 в вертикальной и горизонтальной проекциях. Лампа накаливания Л с помощью конденсора $K$ ярко
		освещает узкую входную щель $S$, расположенную в фокусе объектива $O_1$ (фокусное расстояние $f$). Коллиматор, состоящий из щели $S$ и
		объектива $O_1$, посылает параллельный пучок на диафрагму $D$ с двумя вертикальными щелями (расстояние между щелями $d$). Свет после
		двойной щели проходит кювету $L$, состоящую из двух одинаковых стеклянных камер, в которые вводятся исследуемые газы (в нашей установке
		--- CO$_2$ или воздух). Кювета занимает только верхнюю часть пространства между объективами $O_1$ и $O_2$, длина кюветы $l$. За кюветой
		расположены две стеклянные пластинки $J$ (компенсатор Жамена, см. ниже) и пластинка П.
		\begin{figure}[H]
        	\begin{center}
                \includegraphics[width=0.9\textwidth]{images/pic2.pdf}
                \caption{Устройство интерферометра Релея: a) вид сверху; б) вид сбоку.}
            \end{center}
        \end{figure}
		\par Интерференционная картина (картина дифракции на двух щелях), наблюдаемая в фокальной плоскости $F$ объектива $O_2$, представляет
		собой две системы равноотстоящих полос, параллельных щелям: верхняя (подвижная) образована лучами, прошедшими через кювету, нижняя
		(неподвижная) --- лучами, прошедшими под кюветой. Обе системы интерференционных полос разграничены при помощи пластины П тонкой
		разделительной линией. Для наблюдения двух систем полос в окуляре применена цилиндрическая линза диаметром 2,2 мм, ось которой
		расположена вертикально. Вторая («глазная») линза окуляра --- обычная сферическая. Она служит для подстройки чёткости картины под
		глаз наблюдателя.
		\par При малых дифракционных углах $\phi=\lambda/d$ расстояние между соседними светлыми (или тёмными) полосами $\delta y$ зависит от
		длины волны $\lambda$, фокусного расстояния $f$ объектива $O_2$ и расстояния между дифракционными щелями $d$:
		\begin{equation}
			\delta y=f\frac{\lambda}{d}.
		\end{equation}
		В техническом интерферометре ИТР-1, который используется в нашей работе, $f\simeq20$ см, $d\simeq1.5$ см, и $\delta y$ оказывается
		порядка 10$^{-3}$ см. Для наблюдения таких мелких интерференционных полос требуется окуляр с большим увеличением
		($\gamma\simeq150^{\times}$). Короткофокусная цилиндрическая линза окуляра O сильно растягивает интерференционную картину по
		горизонтали, не меняя её вертикальных размеров и тем самым мало ослабляя освещённость полос. Изображение светящейся точки в фокальной
		плоскости объектива $O_2$ при рассматривании через цилиндрическую линзу имеет вид светлой вертикальной линии, длина которой определяется
		диаметром объектива. Поэтому распределение освещённости в нижней части светлой линии зависит от действия нижней части объектива, а в
		верхней части линии --- от верхней части объектива. Таким образом, наблюдатель видит две системы полос: верхняя образована лучами,
		прошедшими через кюветы, нижняя --- лучами, прошедшими под кюветами.
		\par При заполнении камер газами с одинаковым показателем преломления $n$ обе системы полос совпадают. Оптическая разность хода
		$\Delta=\delta n\cdot l$, возникающая при прохождении света через камеры с разными газами $\delta n=n_1-n_2$, ведёт к поперечному
		смещению верхней дифракционной картины относительно неподвижной нижней. Смещение на одну полосу соответствует дополнительной разности
		хода $\Delta=\lambda$. Просчитав число полос $m$ между центрами обеих картин, можно рассчитать
		\begin{equation}
			\delta n=\frac{\Delta}{l}=m\frac{\lambda}{l}.
		\end{equation}
		\par Для точного измерения разности хода используется компенсатор Жамена ($J$ на рис. 2) --- устройство, которое позволяет вернуть
		подвижную систему полос к первоначальному положению, т. е. вновь совместить обе системы полос. В установке компенсатор Жамена
		расположен за кюветой. Он состоит из двух одинаковых плоскопараллельных стеклянных пластинок, установленных на пути лучей под углом
		$45^{\circ}$ к горизонтали. Вращение одной из пластин вокруг горизонтальной оси, перпендикулярной оси системы, вызывает увеличение или
		уменьшение оптической длины пути соответствующего луча. Ось вращения снабжена рычагом, конец которого смещается при помощи
		микрометрического винта B.
		\par Интерферометр Релея можно применять для измерения небольших изменений показателей преломления жидкостей или газов, а также для
		определения примесей различных газов в воздухе (например, для измерения концентрации рудничного газа в шахте). Показатель преломления
		$n$ исследуемого газа определяется путём сравнения с воздухом при атмосферном давлении:
		\begin{equation}
			n=n_{\text{возд}}+\frac{\Delta}{l}.
		\end{equation}
		\par Для определения величины $\Delta$ компенсатор следует прокалибровать.

	\section{Результаты измерений и обработка данных}

		\subsection{Калибровка компенсатора}
		\par Уравняв давление в камерах, подождём 2–3 минуты, пока выровняются температуры. Установим начало отсчёта, совместив с помощью
		компенсатора обе системы полос. Установка нуля в белом свете --- совмещение центральных полос --- облегчается тем, что боковые полосы
		окрашены. Совместим (приблизительно) боковые полосы с симметричной окраской, а затем (как можно точнее) — центральные.
		\par Прокалибруем компенсатор в единицах $\lambda$, выделив узкий интервал длин волн с помощью светофильтра. Занесём результаты
		калибровки в таблицу 1. Построим график зависимости значений микрометра $z$ от номера полосы $m$ (рис. 3). Аппроксимируем его прямой
		линией $z=am+b$. В итоге получим следующие коэффициенты $a=34.6 \pm 0.2$, $b=280.8 \pm 0.9$.
		\begin{figure}[H]
			\begin{minipage}{0.69\textwidth}
				\begin{figure}[H]
					\begin{center}
						% This file was created with tikzplotlib v0.9.16.
\begin{tikzpicture}

\begin{axis}[
legend cell align={left},
legend style={
  fill opacity=0.8,
  draw opacity=1,
  text opacity=1,
  at={(0.03,0.97)},
  anchor=north west,
  draw=white!80!black
},
height=11.2cm,
tick align=inside,
major tick length=0.2cm,
minor tick length=0.1cm,
tick pos=left,
xmin=0, xmax=600,
xtick={0, 50, 100, 150, 200, 250, 300, 350, 400, 450, 500, 550, 600},
minor x tick num=4,
xmajorgrids,
minor x grid style={dotted,black},
xminorgrids,
xtick style={color=black},
xlabel={$D_{\text{винта}}$, мкм},
ymin=0, ymax=500,
ytick={0, 50, 100, 150, 200, 250, 300, 350, 400, 450, 500},
ytick style={color=black},
minor y tick num=4,
ymajorgrids,
minor y grid style={dotted,black},
yminorgrids,
ytick style={color=black},
ylabel={$D_{\text{расчёт}}$, мкм}
]
\path [draw=red, semithick]
(axis cs:40,32.9861111111111)
--(axis cs:60,32.9861111111111);

\path [draw=red, semithick]
(axis cs:90,98.9583333333333)
--(axis cs:110,98.9583333333333);

\path [draw=red, semithick]
(axis cs:140,131.944444444444)
--(axis cs:160,131.944444444444);

\path [draw=red, semithick]
(axis cs:190,164.930555555556)
--(axis cs:210,164.930555555556);

\path [draw=red, semithick]
(axis cs:240,197.916666666667)
--(axis cs:260,197.916666666667);

\path [draw=red, semithick]
(axis cs:290,230.902777777778)
--(axis cs:310,230.902777777778);

\path [draw=red, semithick]
(axis cs:340,263.888888888889)
--(axis cs:360,263.888888888889);

\path [draw=red, semithick]
(axis cs:390,296.875)
--(axis cs:410,296.875);

\path [draw=red, semithick]
(axis cs:440,362.847222222222)
--(axis cs:460,362.847222222222);

\path [draw=red, semithick]
(axis cs:490,395.833333333333)
--(axis cs:510,395.833333333333);

\path [draw=red, semithick]
(axis cs:50,-0.00334608800363867)
--(axis cs:50,65.9755683102259);

\path [draw=red, semithick]
(axis cs:100,65.9421196383399)
--(axis cs:100,131.974547028327);

\path [draw=red, semithick]
(axis cs:150,98.9048365902785)
--(axis cs:150,164.98405229861);

\path [draw=red, semithick]
(axis cs:200,131.860893814377)
--(axis cs:200,198.000217296734);

\path [draw=red, semithick]
(axis cs:250,164.810309447695)
--(axis cs:250,231.023023885639);

\path [draw=red, semithick]
(axis cs:300,197.75310554622)
--(axis cs:300,264.052450009336);

\path [draw=red, semithick]
(axis cs:350,230.689308019177)
--(axis cs:350,297.088469758601);

\path [draw=red, semithick]
(axis cs:400,263.618946552474)
--(axis cs:400,330.131053447526);

\path [draw=red, semithick]
(axis cs:450,329.458668896105)
--(axis cs:450,396.235775548339);

\path [draw=red, semithick]
(axis cs:500,362.368830131627)
--(axis cs:500,429.29783653504);

\addplot [semithick, red, forget plot]
table {%
0 0
500 392.40620488787
};
\path [draw=green!50.1960784313725!black, semithick]
(axis cs:90,53.0442477876106)
--(axis cs:110,53.0442477876106);

\path [draw=green!50.1960784313725!black, semithick]
(axis cs:140,95.1428571428572)
--(axis cs:160,95.1428571428572);

\path [draw=green!50.1960784313725!black, semithick]
(axis cs:190,148)
--(axis cs:210,148);

\path [draw=green!50.1960784313725!black, semithick]
(axis cs:240,188.913461538462)
--(axis cs:260,188.913461538462);

\path [draw=green!50.1960784313725!black, semithick]
(axis cs:290,233.740384615385)
--(axis cs:310,233.740384615385);

\path [draw=green!50.1960784313725!black, semithick]
(axis cs:340,270.260869565217)
--(axis cs:360,270.260869565217);

\path [draw=green!50.1960784313725!black, semithick]
(axis cs:390,326.204081632653)
--(axis cs:410,326.204081632653);

\path [draw=green!50.1960784313725!black, semithick]
(axis cs:440,360)
--(axis cs:460,360);

\path [draw=green!50.1960784313725!black, semithick]
(axis cs:490,396)
--(axis cs:510,396);

\path [draw=green!50.1960784313725!black, semithick]
(axis cs:540,468)
--(axis cs:560,468);

\path [draw=green!50.1960784313725!black, semithick]
(axis cs:100,52.4114526583863)
--(axis cs:100,53.6770429168349);

\path [draw=green!50.1960784313725!black, semithick]
(axis cs:150,91.0666302830339)
--(axis cs:150,99.2190840026804);

\path [draw=green!50.1960784313725!black, semithick]
(axis cs:200,141.683425571105)
--(axis cs:200,154.316574428895);

\path [draw=green!50.1960784313725!black, semithick]
(axis cs:250,182.271308845487)
--(axis cs:250,195.555614231436);

\path [draw=green!50.1960784313725!black, semithick]
(axis cs:300,223.88999208276)
--(axis cs:300,243.590777148009);

\path [draw=green!50.1960784313725!black, semithick]
(axis cs:350,257.145804132173)
--(axis cs:350,283.375934998262);

\path [draw=green!50.1960784313725!black, semithick]
(axis cs:400,311.969339856676)
--(axis cs:400,340.438823408631);

\path [draw=green!50.1960784313725!black, semithick]
(axis cs:450,341.133063042629)
--(axis cs:450,378.866936957371);

\path [draw=green!50.1960784313725!black, semithick]
(axis cs:500,376.956019710008)
--(axis cs:500,415.043980289992);

\path [draw=green!50.1960784313725!black, semithick]
(axis cs:550,448.557402072788)
--(axis cs:550,487.442597927212);

\addplot [semithick, green!50.1960784313725!black, forget plot]
table {%
0 0
550 439.564038903656
};
\addplot [semithick, red, mark=*, mark size=3, mark options={solid}, only marks]
table {%
50 32.9861111111111
100 98.9583333333333
150 131.944444444444
200 164.930555555556
250 197.916666666667
300 230.902777777778
350 263.888888888889
400 296.875
450 362.847222222222
500 395.833333333333
};
\addlegendentry{Линза}
\addplot [semithick, green!50.1960784313725!black, mark=*, mark size=3, mark options={solid}, only marks]
table {%
100 53.0442477876106
150 95.1428571428572
200 148
250 188.913461538462
300 233.740384615385
350 270.260869565217
400 326.204081632653
450 360
500 396
550 468
};
\addlegendentry{Спектр}
\end{axis}

\end{tikzpicture}

						\caption{График z(m).}
					\end{center}
				\end{figure}				
			\end{minipage}
			\hfill
			\begin{minipage}{0.29\textwidth}
				\begin{table}[H]
					\captionsetup{font={small}, labelformat=fullparents, labelsep=fill, labelfont=bf, justification=raggedleft,
								  singlelinecheck=false, skip=-0.2cm}
					\caption{Результаты калибровки.}
					\begin{center}
						\begin{tabular}{|l|l|}\hline$m$&$z$, дел
\\\hline0.0&275.0
\\\hline1.0&311.0
\\\hline2.0&348.0
\\\hline3.0&381.0
\\\hline4.0&419.0
\\\hline5.0&453.0
\\\hline6.0&489.0
\\\hline7.0&529.0
\\\hline-1.0&250.0
\\\hline-2.0&214.0
\\\hline-3.0&178.0
\\\hline-4.0&148.0
\\\hline-5.0&109.0
\\\hline-6.0&73.0
\\\hline-7.0&35.0
\\\hline\end{tabular}
				  \end{center}
			  \end{table}
			\end{minipage}
		\end{figure}
		\par Длина кувейты сотавляет $l=25$ см. Светофильтр настроен на длину волны $\lambda=670$ нм и имеет ширину полосы пропускания
		$\delta\lambda=100$ нм.

		\subsection{Исследование зависимости $\Delta n$ от $P$ для воздуха}
		\par Изменяя давление с помощью сильфона и совмещая нулевые полосы, снимем зависимость показаний компенсатора $z$ от перепада давлений
		$\Delta P$. Давление будем изменять от -1000 мм вод. ст. до 1000 мм вод. ст. Занесём полученные данные в таблицу 2.
		\par Используя формулу (5) и калибровочную шкалу рассчитаем $\delta n$ и построим график зависимости от $P$ (рис. 4 слева).
		\par Как видно из графика прямая, проходящая через ноль, плохо аппроксимирует наши данные. Также две первыеб две последние и центральная
		точки выбиваются из прямой. Выкенем первые и последние точки, а остальные (кроме центральной) сдвинем на 100 мм вод. ст. вправо
		(рис. 4 справа). Теперь зависимость хорошо аппроксимируется изначальной функцией. Возможно, это было вызванно сбитым нулём на манометре.
		\par В итоге коэффициент для второго графика $a=(-268 \pm 3)\cdot10^{-10}$ (мм вод. ст.)$^{-1}$. На основе него по формуле (3)
		рассчитаем поляризуемость молекул воздуха $\alpha=-k_{\text{Б}}Ta/(2\pi)=1.76\pm0.02\cdot10^{-30}$.
		\par Найдём показатель преломления при нормальных условиях с помощью формулы (2) $n=1-aP=1.000294 \pm 0.000004$. Знак минус обучловлен
		калибровкой.
		\begin{table}[H]
			\captionsetup{font={small}, labelformat=fullparents, labelsep=fill, labelfont=bf, justification=raggedleft,
						  singlelinecheck=false, skip=-0.2cm}
			\caption{Показатель преломления при разных давлениях.}
			\begin{center}
				\begin{tabular}{|l|l|l|l|l|l|l|}\hline$P$, мм вд. c.&-1000.0&-900.0&-800.0&-700.0&-600.0&-500.0
\\\hline$z $, дел&614.0&588.0&521.0&484.0&447.0&411.0
\\\hline$\delta n\cdot10^{-7}$&258.0&238.0&186.0&157.0&129.0&101.0
\\\hline$\sigma_{\delta n}\cdot10^{-7}$&4.0&4.0&4.0&4.0&4.0&4.0
\\\hline\end{tabular}
		  	\end{center}
	  	\end{table}
		\begin{table}[H]
			\captionsetup{font={small}, labelformat=fullparents, labelsep=fill, labelfont=bf, justification=raggedleft,
						  singlelinecheck=false, skip=-0.2cm}
			\centering	
			\begin{tabular}{|l|l|l|l|l|l|l|l|}\hline$P$, мм вд. c.&-400.0&-300.0&-200.0&-100.0&0.0&100.0&200.0
\\\hline$z $, дел&374.0&341.0&309.0&273.0&275.0&196.0&166.0
\\\hline$\delta n\cdot10^{-7}$&72.0&47.0&22.0&-6.0&-4.0&-66.0&-89.0
\\\hline$\sigma_{\delta n}\cdot10^{-7}$&4.0&4.0&4.0&4.0&4.0&4.0&4.0
\\\hline\end{tabular}
	  	\end{table}
		  \begin{table}[H]
			\captionsetup{font={small}, labelformat=fullparents, labelsep=fill, labelfont=bf, justification=raggedleft,
						  singlelinecheck=false, skip=-0.2cm}
			\centering	
			\begin{tabular}{|l|l|l|l|l|l|l|l|}\hline$P$, мм вд. c.&300.0&400.0&500.0&600.0&700.0&800.0&900.0
\\\hline$z $, дел&130.0&104.0&74.0&36.0&3.0&-69.0&-140.0
\\\hline$\delta n\cdot10^{-7}$&-117.0&-137.0&-160.0&-189.0&-215.0&-271.0&-326.0
\\\hline$\sigma_{\delta n}\cdot10^{-7}$&4.0&4.0&4.0&4.0&4.0&4.0&4.0
\\\hline\end{tabular}
	  	\end{table}
		\begin{figure}[H]
			\begin{minipage}{0.49\textwidth}
				\begin{figure}[H]
					\begin{center}
						% This file was created with tikzplotlib v0.9.16.
\begin{tikzpicture}

\begin{axis}[
    height=11cm,
    tick align=inside,
    major tick length=0.2cm,
    minor tick length=0.1cm,
    tick pos=left,
    xmin=-4.1, xmax=4.1,
    xtick={-5, -4, -3, -2, -1, 0, 1, 2, 3, 4, 5},
    minor x tick num=4,
    xmajorgrids,
    minor x grid style={dotted,black},
    xminorgrids,
    xtick style={color=black},
    xlabel={$m$},
    ymin=-1.2, ymax=1.2,
    ytick={-1.5, -1, -0.5, 0, 0.5, 1, 1.5},
    minor y tick num=4,
    ymajorgrids,
    minor y grid style={dotted,black},
    yminorgrids,
    ytick style={color=black},
    ylabel={$x_m$, мм}
]
\path [draw=red, semithick]
(axis cs:1,0.14)
--(axis cs:1,0.18);

\path [draw=red, semithick]
(axis cs:2,0.48)
--(axis cs:2,0.52);

\path [draw=red, semithick]
(axis cs:3,0.7)
--(axis cs:3,0.74);

\path [draw=red, semithick]
(axis cs:4,1.02)
--(axis cs:4,1.06);

\path [draw=red, semithick]
(axis cs:-1,-0.18)
--(axis cs:-1,-0.14);

\path [draw=red, semithick]
(axis cs:-2,-0.5)
--(axis cs:-2,-0.46);

\path [draw=red, semithick]
(axis cs:-3,-0.74)
--(axis cs:-3,-0.7);

\path [draw=red, semithick]
(axis cs:-4,-1.08)
--(axis cs:-4,-1.04);

\addplot [semithick, black]
table {%
1 0.249999999998364
2 0.499999999996728
3 0.749999999995093
4 0.999999999993457
-1 -0.249999999998364
-2 -0.499999999996728
-3 -0.749999999995093
-4 -0.999999999993457
};
\addplot [semithick, red, mark=square*, mark size=3, mark options={solid}, only marks]
table {%
1 0.16
2 0.5
3 0.72
4 1.04
-1 -0.16
-2 -0.48
-3 -0.72
-4 -1.06
};
\end{axis}

\end{tikzpicture}

					\end{center}
				\end{figure}				
			\end{minipage}
			\hfill
			\begin{minipage}{0.49\textwidth}
				\begin{figure}[H]
					\begin{center}
						% This file was created with tikzplotlib v0.9.16.
\begin{tikzpicture}

\definecolor{color0}{rgb}{0.12156862745098,0.466666666666667,0.705882352941177}

\begin{axis}[
    height=7.2cm,
    tick align=inside,
    major tick length=0.2cm,
    minor tick length=0.1cm,
    tick pos=left,
    x grid style={white!69.0196078431373!black},
    xmin=-10, xmax=140,
    xtick={-40, 0, 40, 80, 120, 160},
    minor x tick num=3,
    xmajorgrids,
    minor x grid style={dotted,black},
    xminorgrids,
    xtick style={color=black},
    xlabel={$l$, \text{см}},
    ymin = -0.2, ymax = 0.8,
    ytick = {-0.4, 0, 0.4, 0.8},
    ytick style={color=black},
    minor y tick num=3,
    ymajorgrids,
    minor y grid style={dotted,black},
    yminorgrids,
    ytick style={color=black},
    ylabel={$\upsilon_2$}
]
\path [draw=red, semithick]
(axis cs:-6,0.225806451612903)
--(axis cs:-2,0.225806451612903);

\path [draw=red, semithick]
(axis cs:-2,0.6)
--(axis cs:2,0.6);

\path [draw=red, semithick]
(axis cs:0,0.133333333333333)
--(axis cs:4,0.133333333333333);

\path [draw=red, semithick]
(axis cs:2,0.214285714285714)
--(axis cs:6,0.214285714285714);

\path [draw=red, semithick]
(axis cs:4,0.2)
--(axis cs:8,0.2);

\path [draw=red, semithick]
(axis cs:6,0.161290322580645)
--(axis cs:10,0.161290322580645);

\path [draw=red, semithick]
(axis cs:8,0.225806451612903)
--(axis cs:12,0.225806451612903);

\path [draw=red, semithick]
(axis cs:10,0.225806451612903)
--(axis cs:14,0.225806451612903);

\path [draw=red, semithick]
(axis cs:14,0.161290322580645)
--(axis cs:18,0.161290322580645);

\path [draw=red, semithick]
(axis cs:26,0.0967741935483871)
--(axis cs:30,0.0967741935483871);

\path [draw=red, semithick]
(axis cs:36,0.0322580645161291)
--(axis cs:40,0.0322580645161291);

\path [draw=red, semithick]
(axis cs:46,0.0666666666666667)
--(axis cs:50,0.0666666666666667);

\path [draw=red, semithick]
(axis cs:56,0.0322580645161291)
--(axis cs:60,0.0322580645161291);

\path [draw=red, semithick]
(axis cs:66,0.0322580645161291)
--(axis cs:70,0.0322580645161291);

\path [draw=red, semithick]
(axis cs:76,0.0666666666666667)
--(axis cs:80,0.0666666666666667);

\path [draw=red, semithick]
(axis cs:86,0.0322580645161291)
--(axis cs:90,0.0322580645161291);

\path [draw=red, semithick]
(axis cs:96,0.0666666666666667)
--(axis cs:100,0.0666666666666667);

\path [draw=red, semithick]
(axis cs:106,0.0967741935483871)
--(axis cs:110,0.0967741935483871);

\path [draw=red, semithick]
(axis cs:110,0.225806451612903)
--(axis cs:114,0.225806451612903);

\path [draw=red, semithick]
(axis cs:114,0.290322580645161)
--(axis cs:118,0.290322580645161);

\path [draw=red, semithick]
(axis cs:116,0.290322580645161)
--(axis cs:120,0.290322580645161);

\path [draw=red, semithick]
(axis cs:118,0.533333333333333)
--(axis cs:122,0.533333333333333);

\path [draw=red, semithick]
(axis cs:120,0.548387096774194)
--(axis cs:124,0.548387096774194);

\path [draw=red, semithick]
(axis cs:122,0.161290322580645)
--(axis cs:126,0.161290322580645);

\path [draw=red, semithick]
(axis cs:124,0.161290322580645)
--(axis cs:128,0.161290322580645);

\path [draw=red, semithick]
(axis cs:126,0.0967741935483871)
--(axis cs:130,0.0967741935483871);

\path [draw=red, semithick]
(axis cs:128,0.0967741935483871)
--(axis cs:132,0.0967741935483871);

\path [draw=red, semithick]
(axis cs:-4,0.16988541584789)
--(axis cs:-4,0.281727487377916);

\path [draw=red, semithick]
(axis cs:0,0.524575276673435)
--(axis cs:0,0.675424723326565);

\path [draw=red, semithick]
(axis cs:2,0.0799074876436831)
--(axis cs:2,0.186759179022984);

\path [draw=red, semithick]
(axis cs:4,0.152955024080759)
--(axis cs:4,0.27561640449067);

\path [draw=red, semithick]
(axis cs:6,0.143431457505076)
--(axis cs:6,0.256568542494924);

\path [draw=red, semithick]
(axis cs:8,0.108312499224317)
--(axis cs:8,0.214268145936973);

\path [draw=red, semithick]
(axis cs:10,0.16988541584789)
--(axis cs:10,0.281727487377916);

\path [draw=red, semithick]
(axis cs:12,0.16988541584789)
--(axis cs:12,0.281727487377916);

\path [draw=red, semithick]
(axis cs:16,0.108312499224317)
--(axis cs:16,0.214268145936973);

\path [draw=red, semithick]
(axis cs:28,0.0467395826007438)
--(axis cs:28,0.14680880449603);

\path [draw=red, semithick]
(axis cs:38,-0.0148333340228294)
--(axis cs:38,0.0793494630550875);

\path [draw=red, semithick]
(axis cs:48,0.01638351778229)
--(axis cs:48,0.116949815551043);

\path [draw=red, semithick]
(axis cs:58,-0.0148333340228294)
--(axis cs:58,0.0793494630550875);

\path [draw=red, semithick]
(axis cs:68,-0.0148333340228294)
--(axis cs:68,0.0793494630550875);

\path [draw=red, semithick]
(axis cs:78,0.01638351778229)
--(axis cs:78,0.116949815551043);

\path [draw=red, semithick]
(axis cs:88,-0.0148333340228294)
--(axis cs:88,0.0793494630550875);

\path [draw=red, semithick]
(axis cs:98,0.01638351778229)
--(axis cs:98,0.116949815551043);

\path [draw=red, semithick]
(axis cs:108,0.0467395826007438)
--(axis cs:108,0.14680880449603);

\path [draw=red, semithick]
(axis cs:112,0.16988541584789)
--(axis cs:112,0.281727487377916);

\path [draw=red, semithick]
(axis cs:116,0.231458332471463)
--(axis cs:116,0.349186828818859);

\path [draw=red, semithick]
(axis cs:118,0.231458332471463)
--(axis cs:118,0.349186828818859);

\path [draw=red, semithick]
(axis cs:120,0.461051306812042)
--(axis cs:120,0.605615359854625);

\path [draw=red, semithick]
(axis cs:122,0.477749998965756)
--(axis cs:122,0.619024194582631);

\path [draw=red, semithick]
(axis cs:124,0.108312499224317)
--(axis cs:124,0.214268145936973);

\path [draw=red, semithick]
(axis cs:126,0.108312499224317)
--(axis cs:126,0.214268145936973);

\path [draw=red, semithick]
(axis cs:128,0.0467395826007438)
--(axis cs:128,0.14680880449603);

\path [draw=red, semithick]
(axis cs:130,0.0467395826007438)
--(axis cs:130,0.14680880449603);

\addplot [semithick, color0]
table {%
-4 0.289700909532296
-2.64646464646465 0.299561221223985
-1.29292929292929 0.303200008671337
0.0606060606060606 0.301611090044221
1.41414141414141 0.295698522381247
2.76767676767677 0.286281224103273
4.12121212121212 0.274097490756272
5.47474747474748 0.259809403983547
6.82828282828283 0.244007133727306
8.18181818181818 0.227213133659582
9.53535353535354 0.209886229842507
10.8888888888889 0.192425602617948
12.2424242424242 0.175174661726482
13.5959595959596 0.158424814655734
14.949494949495 0.142419128218066
16.3030303030303 0.127355883357614
17.6565656565657 0.113392023186687
19.010101010101 0.100646494251508
20.3636363636364 0.0892034810273152
21.7171717171717 0.079115533642817
23.0707070707071 0.0704065888339942
24.4242424242424 0.0630748841272588
25.7777777777778 0.0570957652519653
27.1313131313131 0.0524243867822743
28.4848484848485 0.0489983060083702
29.8383838383838 0.0467399700370295
31.1919191919192 0.0455590961215448
32.5454545454545 0.0453549452209991
33.8989898989899 0.0460184887888953
35.2525252525253 0.0474344687911365
36.6060606060606 0.0494833509533611
37.959595959596 0.0520431712376296
39.3131313131313 0.0549912755484642
40.6666666666667 0.0582059526682422
42.020202020202 0.0615679604219419
43.3737373737374 0.064961945071241
44.7272727272727 0.0682777539379683
46.0808080808081 0.0714116412569081
47.4343434343434 0.0742673672579583
48.7878787878788 0.0767571904776398
50.1414141414141 0.0788027532999603
51.4949494949495 0.0803358607266301
52.8484848484849 0.0812991523766312
54.2020202020202 0.0816466677151401
55.5555555555556 0.0813443045118011
56.9090909090909 0.0803701705283548
58.2626262626263 0.0787148284356191
59.6161616161616 0.076381433959822
60.969696969697 0.0733857672582888
62.3232323232323 0.0697561575244805
63.6767676767677 0.0655333008223871
65.030303030303 0.0607699711502716
66.3838383838384 0.0555306247337693
67.7373737373737 0.0498908975483373
69.0909090909091 0.0439369960710599
70.4444444444444 0.0377649812618038
71.7979797979798 0.0314799457737291
73.1515151515152 0.0251950843931509
74.5050505050505 0.0190306577087552
75.8585858585859 0.0131128490101669
77.2121212121212 0.00757251441587132
78.5656565656566 0.00254382623048871
79.9191919191919 -0.00183719046859988
81.2727272727273 -0.00543422801527125
82.6262626262626 -0.00811237310932827
83.979797979798 -0.00973989054181345
85.3333333333333 -0.0101901136909708
86.6868686868687 -0.00934344178885385
88.040404040404 -0.00708944395858157
89.3939393939394 -0.00332907002224061
90.7474747474748 0.00202303192056456
92.1010101010101 0.00903609114351488
93.4545454545455 0.0177606831737331
94.8080808080808 0.0282260919012925
96.1616161616162 0.0404375649180784
97.5151515151515 0.0543734620860032
98.8686868686869 0.0699822973345749
100.222222222222 0.0871796736878155
101.575757575758 0.105845111520537
102.929292929293 0.125818770043966
104.282828282828 0.146898062020724
105.636363636364 0.168834161709158
106.989898989899 0.191328406037028
108.343434343434 0.214028589004543
109.69696969697 0.236525149316752
111.050505050505 0.258347251245289
112.40404040404 0.278958758719469
113.757575757576 0.297754102646736
115.111111111111 0.314054041462469
116.464646464646 0.327101314909134
117.818181818182 0.336056191044795
119.171717171717 0.339991906480971
120.525252525253 0.337889999849856
121.878787878788 0.328635538500881
123.232323232323 0.311012238426636
124.585858585859 0.283697477418142
125.939393939394 0.24525720144948
127.292929292929 0.194140724291761
128.646464646465 0.128675420356469
130 0.047061310768137
};
\addplot [semithick, red, mark=*, mark size=3, mark options={solid}, only marks]
table {%
-4 0.225806451612903
0 0.6
2 0.133333333333333
4 0.214285714285714
6 0.2
8 0.161290322580645
10 0.225806451612903
12 0.225806451612903
16 0.161290322580645
28 0.0967741935483871
38 0.0322580645161291
48 0.0666666666666667
58 0.0322580645161291
68 0.0322580645161291
78 0.0666666666666667
88 0.0322580645161291
98 0.0666666666666667
108 0.0967741935483871
112 0.225806451612903
116 0.290322580645161
118 0.290322580645161
120 0.533333333333333
122 0.548387096774194
124 0.161290322580645
126 0.161290322580645
128 0.0967741935483871
130 0.0967741935483871
};
\end{axis}

\end{tikzpicture}

					\end{center}
				\end{figure}
			\end{minipage}
			\caption{Графики $\delta n(P)$}
		\end{figure}

		\subsection{Исследования углекислого газа}
		\par Соединим первую камеру кюветы с атмосферой, открыв кран $K_1$, и отключим манометр, закрыв кран $K_2$. Заполним углекислым
		газом камеру с открытым концом.
		\par Снимем зависимость равновесного положения компенсатора от времени, раз в минуту совмещая нулевые полосы, и оценим время
		установления равновесия. Занесём полученные данные в таблицу 3.
		\begin{table}[H]
			\captionsetup{font={small}, labelformat=fullparents, labelsep=fill, labelfont=bf, justification=raggedleft,
						  singlelinecheck=false, skip=-0.2cm}
			\caption{Положение компенсатора в зависмости от времени.}
			\begin{center}
				\begin{tabular}{|l|l|l|l|l|l|l|l|}\hline$t$, мин&0&1&2&3&4&5&6
\\\hline$z  $, дел&2494&2394&2185&2110&2033&1975&1913
\\\hline\end{tabular}
		  	\end{center}
	  	\end{table}
		\begin{table}[H]
			\captionsetup{font={small}, labelformat=fullparents, labelsep=fill, labelfont=bf, justification=raggedleft,
						  singlelinecheck=false, skip=-0.2cm}
			\centering	
			\begin{tabular}{|l|l|l|l|l|l|l|l|}\hline$t$, мин&7&8&9&10&11&12&13
\\\hline$z  $, дел&1829&1782&1745&1704&1664&1633&1603
\\\hline\end{tabular}
	  	\end{table}
		\begin{table}[H]
			\captionsetup{font={small}, labelformat=fullparents, labelsep=fill, labelfont=bf, justification=raggedleft,
						  singlelinecheck=false, skip=-0.2cm}
			\centering	
			\begin{tabular}{|l|l|l|l|l|l|l|l|}\hline$t$, мин&14&15&16&17&18&19&20
\\\hline$z  $, дел&1583&1552&1533&1514&1492&1477&1459
\\\hline\end{tabular}
	  	\end{table}
		\par Построим график зависимости $\log(z)$ от $t$. Аппроксимируем его прямой. В результате аппроксимации
		получим, что время установления равновесия равно $\tau=39 \pm 2$ мин. Аппроксимация экспонентой не дала хороших результатов. Возможно,
		закон установления равновесия гораздо сложнее из-за влияния внешних факторов. Отобразим результат на рис. 5 и 6.
		\begin{figure}[H]
			\begin{minipage}{0.49\textwidth}
				\begin{figure}[H]
					\begin{center}
						% This file was created with tikzplotlib v0.9.16.
\begin{tikzpicture}

\definecolor{color0}{rgb}{0.12156862745098,0.466666666666667,0.705882352941177}

\begin{axis}[
    height=7.2cm,
    tick align=inside,
    major tick length=0.2cm,
    minor tick length=0.1cm,
    tick pos=left,
    x grid style={white!69.0196078431373!black},
    xmin=-10, xmax=140,
    xtick={-40, 0, 40, 80, 120, 160},
    minor x tick num=3,
    xmajorgrids,
    minor x grid style={dotted,black},
    xminorgrids,
    xtick style={color=black},
    xlabel={$l$, \text{см}},
    ymin = -0.2, ymax = 0.8,
    ytick = {-0.4, 0, 0.4, 0.8},
    ytick style={color=black},
    minor y tick num=3,
    ymajorgrids,
    minor y grid style={dotted,black},
    yminorgrids,
    ytick style={color=black},
    ylabel={$\upsilon_2$}
]
\path [draw=red, semithick]
(axis cs:-6,0.225806451612903)
--(axis cs:-2,0.225806451612903);

\path [draw=red, semithick]
(axis cs:-2,0.6)
--(axis cs:2,0.6);

\path [draw=red, semithick]
(axis cs:10,0.225806451612903)
--(axis cs:14,0.225806451612903);

\path [draw=red, semithick]
(axis cs:14,0.161290322580645)
--(axis cs:18,0.161290322580645);

\path [draw=red, semithick]
(axis cs:26,0.0967741935483871)
--(axis cs:30,0.0967741935483871);

\path [draw=red, semithick]
(axis cs:36,0.0322580645161291)
--(axis cs:40,0.0322580645161291);

\path [draw=red, semithick]
(axis cs:46,0.0666666666666667)
--(axis cs:50,0.0666666666666667);

\path [draw=red, semithick]
(axis cs:56,0.0322580645161291)
--(axis cs:60,0.0322580645161291);

\path [draw=red, semithick]
(axis cs:66,0.0322580645161291)
--(axis cs:70,0.0322580645161291);

\path [draw=red, semithick]
(axis cs:76,0.0666666666666667)
--(axis cs:80,0.0666666666666667);

\path [draw=red, semithick]
(axis cs:86,0.0322580645161291)
--(axis cs:90,0.0322580645161291);

\path [draw=red, semithick]
(axis cs:96,0.0666666666666667)
--(axis cs:100,0.0666666666666667);

\path [draw=red, semithick]
(axis cs:106,0.0967741935483871)
--(axis cs:110,0.0967741935483871);

\path [draw=red, semithick]
(axis cs:110,0.225806451612903)
--(axis cs:114,0.225806451612903);

\path [draw=red, semithick]
(axis cs:114,0.290322580645161)
--(axis cs:118,0.290322580645161);

\path [draw=red, semithick]
(axis cs:118,0.533333333333333)
--(axis cs:122,0.533333333333333);

\path [draw=red, semithick]
(axis cs:120,0.548387096774194)
--(axis cs:124,0.548387096774194);

\path [draw=red, semithick]
(axis cs:124,0.161290322580645)
--(axis cs:128,0.161290322580645);

\path [draw=red, semithick]
(axis cs:128,0.0967741935483871)
--(axis cs:132,0.0967741935483871);

\path [draw=red, semithick]
(axis cs:-4,0.146722164412071)
--(axis cs:-4,0.304890738813736);

\path [draw=red, semithick]
(axis cs:0,0.493333333333333)
--(axis cs:0,0.706666666666667);

\path [draw=red, semithick]
(axis cs:12,0.146722164412071)
--(axis cs:12,0.304890738813736);

\path [draw=red, semithick]
(axis cs:16,0.0863683662851197)
--(axis cs:16,0.236212278876171);

\path [draw=red, semithick]
(axis cs:28,0.0260145681581686)
--(axis cs:28,0.167533818938606);

\path [draw=red, semithick]
(axis cs:38,-0.0343392299687825)
--(axis cs:38,0.0988553590010406);

\path [draw=red, semithick]
(axis cs:48,-0.00444444444444439)
--(axis cs:48,0.137777777777778);

\path [draw=red, semithick]
(axis cs:58,-0.0343392299687825)
--(axis cs:58,0.0988553590010406);

\path [draw=red, semithick]
(axis cs:68,-0.0343392299687825)
--(axis cs:68,0.0988553590010406);

\path [draw=red, semithick]
(axis cs:78,-0.00444444444444439)
--(axis cs:78,0.137777777777778);

\path [draw=red, semithick]
(axis cs:88,-0.0343392299687825)
--(axis cs:88,0.0988553590010406);

\path [draw=red, semithick]
(axis cs:98,-0.00444444444444439)
--(axis cs:98,0.137777777777778);

\path [draw=red, semithick]
(axis cs:108,0.0260145681581686)
--(axis cs:108,0.167533818938606);

\path [draw=red, semithick]
(axis cs:112,0.146722164412071)
--(axis cs:112,0.304890738813736);

\path [draw=red, semithick]
(axis cs:116,0.207075962539022)
--(axis cs:116,0.373569198751301);

\path [draw=red, semithick]
(axis cs:120,0.431111111111111)
--(axis cs:120,0.635555555555556);

\path [draw=red, semithick]
(axis cs:122,0.448491155046826)
--(axis cs:122,0.648283038501561);

\path [draw=red, semithick]
(axis cs:126,0.0863683662851197)
--(axis cs:126,0.236212278876171);

\path [draw=red, semithick]
(axis cs:130,0.0260145681581686)
--(axis cs:130,0.167533818938606);

\addplot [semithick, color0]
table {%
-4 0.313814118861075
-2.64646464646465 0.372719374858457
-1.29292929292929 0.414429546476477
0.0606060606060606 0.441252184268248
1.41414141414141 0.455307668066297
2.76767676767677 0.458538104134772
4.12121212121212 0.452716029232126
5.47474747474748 0.439452921584311
6.82828282828283 0.420207518768442
8.18181818181818 0.396293942506964
9.53535353535354 0.368889630372301
10.8888888888889 0.339043074402003
12.2424242424242 0.307681366624373
13.5959595959596 0.275617551494593
14.949494949495 0.243557785241332
16.3030303030303 0.21210830212385
17.6565656565657 0.181782187599588
19.010101010101 0.153005958402251
20.3636363636364 0.126125949530377
21.7171717171717 0.101414508146396
23.0707070707071 0.0790759943861873
24.4242424242424 0.059252589079113
25.7777777777778 0.0420299083785511
27.1313131313131 0.0274424253029159
28.4848484848485 0.0154786981871685
29.8383838383838 0.00608640604481389
31.1919191919192 -0.000822809159606586
32.5454545454545 -0.0053686933275373
33.8989898989899 -0.00769792413293865
35.2525252525253 -0.00797984801832828
36.6060606060606 -0.0064024102803318
37.959595959596 -0.00316827824474411
39.3131313131313 0.001508842468899
40.6666666666667 0.0074076985932389
42.020202020202 0.0143027867695034
43.3737373737374 0.0219674580144038
44.7272727272727 0.0301768290975233
46.0808080808081 0.0387105008291939
47.4343434343434 0.0473550832588656
48.7878787878788 0.0559065277839617
50.1414141414141 0.0641722661692277
51.4949494949495 0.0719731564765673
52.8484848484849 0.0791452359053693
54.2020202020202 0.0855412805433256
55.5555555555556 0.0910321720277353
56.9090909090909 0.0955080711173033
58.2626262626263 0.0988793981744253
59.6161616161616 0.101077620557964
60.969696969697 0.102055846926514
62.3232323232323 0.10178922845216
63.6767676767677 0.10027516694472
65.030303030303 0.0975333298864813
66.3838383838384 0.0936054723774249
67.7373737373737 0.088555065990942
69.0909090909091 0.0824667345400373
70.4444444444444 0.0754454967540248
71.7979797979798 0.0676158158657112
73.1515151515152 0.059120456109071
74.5050505050505 0.0501191461274099
75.8585858585859 0.0407870492920191
77.2121212121212 0.031313040931319
78.5656565656566 0.0218977924704934
79.9191919191919 0.0127516624816113
81.2727272727273 0.00409239464424182
82.6262626262626 -0.00385737738344372
83.979797979798 -0.0108728181830784
85.3333333333333 -0.0167297718840175
86.6868686868687 -0.0212078365607651
88.040404040404 -0.024093631719913
89.3939393939394 -0.025184258876587
90.7474747474748 -0.0242909552204062
92.1010101010101 -0.0212429403709502
93.4545454545455 -0.0158914562227379
94.8080808080808 -0.00811399987971576
96.1616161616162 0.00218125032074329
97.5151515151515 0.015050815694331
98.8686868686869 0.030511105006786
100.222222222222 0.048533409181362
101.575757575758 0.0690387029167914
102.929292929293 0.091892253215734
104.282828282828 0.116898034823718
105.636363636364 0.143792952578569
106.989898989899 0.172240870670332
108.343434343434 0.20182644881168
109.69696969697 0.232048785318812
111.050505050505 0.262314867102847
112.40404040404 0.291932826571698
113.757575757576 0.320105005442445
115.111111111111 0.345920825464192
116.464646464646 0.368349466051415
117.818181818182 0.386232348827805
119.171717171717 0.39827542908059
120.525252525253 0.403041294125359
121.878787878788 0.398941068581365
123.232323232323 0.384226126557329
124.585858585859 0.35697961074772
125.939393939394 0.315107758439542
127.292929292929 0.25633103442959
128.646464646465 0.178175070852219
130 0.0779614139175848
};
\addplot [semithick, red, mark=*, mark size=3, mark options={solid}, only marks]
table {%
-4 0.225806451612903
0 0.6
12 0.225806451612903
16 0.161290322580645
28 0.0967741935483871
38 0.0322580645161291
48 0.0666666666666667
58 0.0322580645161291
68 0.0322580645161291
78 0.0666666666666667
88 0.0322580645161291
98 0.0666666666666667
108 0.0967741935483871
112 0.225806451612903
116 0.290322580645161
120 0.533333333333333
122 0.548387096774194
126 0.161290322580645
130 0.0967741935483871
};
\end{axis}

\end{tikzpicture}

						\caption{График $z(t)$.}
					\end{center}
				\end{figure}				
			\end{minipage}
			\hfill
			\begin{minipage}{0.49\textwidth}
				\begin{figure}[H]
					\begin{center}
						% This file was created with tikzplotlib v0.9.16.
\begin{tikzpicture}

\begin{axis}[
    height=6.5cm,
    tick align=inside,
    major tick length=0.2cm,
    minor tick length=0.1cm,
    tick pos=left,
    x grid style={white!69.0196078431373!black},
    xmin=0, xmax=20,
    xtick={0, 5, 10, 15, 20},
    minor x tick num=3,
    xmajorgrids,
    minor x grid style={dotted,black},
    xminorgrids,
    xtick style={color=black},
    xlabel={$t$, мин},
    ymin = 7.2, ymax = 8,
    ytick = {7.2, 7.4, 7.6, 7.8, 8.0},
    ytick style={color=black},
    minor y tick num=4,
    ymajorgrids,
    minor y grid style={dotted,black},
    yminorgrids,
    ytick style={color=black},
    ylabel={$\log(z)$}
]
\addplot [semithick, black]
table {%
0 7.7308375473148
0.202020202020202 7.72575882837657
0.404040404040404 7.72068010943834
0.606060606060606 7.71560139050011
0.808080808080808 7.71052267156188
1.01010101010101 7.70544395262365
1.21212121212121 7.70036523368542
1.41414141414141 7.69528651474719
1.61616161616162 7.69020779580896
1.81818181818182 7.68512907687073
2.02020202020202 7.6800503579325
2.22222222222222 7.67497163899426
2.42424242424242 7.66989292005603
2.62626262626263 7.6648142011178
2.82828282828283 7.65973548217957
3.03030303030303 7.65465676324134
3.23232323232323 7.64957804430311
3.43434343434343 7.64449932536488
3.63636363636364 7.63942060642665
3.83838383838384 7.63434188748842
4.04040404040404 7.62926316855019
4.24242424242424 7.62418444961196
4.44444444444444 7.61910573067373
4.64646464646465 7.6140270117355
4.84848484848485 7.60894829279727
5.05050505050505 7.60386957385904
5.25252525252525 7.59879085492081
5.45454545454545 7.59371213598258
5.65656565656566 7.58863341704435
5.85858585858586 7.58355469810612
6.06060606060606 7.57847597916788
6.26262626262626 7.57339726022965
6.46464646464646 7.56831854129142
6.66666666666667 7.56323982235319
6.86868686868687 7.55816110341496
7.07070707070707 7.55308238447673
7.27272727272727 7.5480036655385
7.47474747474747 7.54292494660027
7.67676767676768 7.53784622766204
7.87878787878788 7.53276750872381
8.08080808080808 7.52768878978558
8.28282828282828 7.52261007084735
8.48484848484848 7.51753135190912
8.68686868686869 7.51245263297089
8.88888888888889 7.50737391403266
9.09090909090909 7.50229519509443
9.29292929292929 7.4972164761562
9.49494949494949 7.49213775721797
9.6969696969697 7.48705903827973
9.8989898989899 7.4819803193415
10.1010101010101 7.47690160040327
10.3030303030303 7.47182288146504
10.5050505050505 7.46674416252681
10.7070707070707 7.46166544358858
10.9090909090909 7.45658672465035
11.1111111111111 7.45150800571212
11.3131313131313 7.44642928677389
11.5151515151515 7.44135056783566
11.7171717171717 7.43627184889743
11.9191919191919 7.4311931299592
12.1212121212121 7.42611441102097
12.3232323232323 7.42103569208274
12.5252525252525 7.41595697314451
12.7272727272727 7.41087825420628
12.9292929292929 7.40579953526805
13.1313131313131 7.40072081632982
13.3333333333333 7.39564209739159
13.5353535353535 7.39056337845335
13.7373737373737 7.38548465951512
13.9393939393939 7.38040594057689
14.1414141414141 7.37532722163866
14.3434343434343 7.37024850270043
14.5454545454545 7.3651697837622
14.7474747474747 7.36009106482397
14.9494949494949 7.35501234588574
15.1515151515152 7.34993362694751
15.3535353535354 7.34485490800928
15.5555555555556 7.33977618907105
15.7575757575758 7.33469747013282
15.959595959596 7.32961875119459
16.1616161616162 7.32454003225636
16.3636363636364 7.31946131331813
16.5656565656566 7.3143825943799
16.7676767676768 7.30930387544167
16.969696969697 7.30422515650344
17.1717171717172 7.2991464375652
17.3737373737374 7.29406771862697
17.5757575757576 7.28898899968874
17.7777777777778 7.28391028075051
17.979797979798 7.27883156181228
18.1818181818182 7.27375284287405
18.3838383838384 7.26867412393582
18.5858585858586 7.26359540499759
18.7878787878788 7.25851668605936
18.989898989899 7.25343796712113
19.1919191919192 7.2483592481829
19.3939393939394 7.24328052924467
19.5959595959596 7.23820181030644
19.7979797979798 7.23312309136821
20 7.22804437242998
};
\path [draw=red, semithick]
(axis cs:-0.0833333333333333,7.82164312623998)
--(axis cs:0.0833333333333333,7.82164312623998);

\path [draw=red, semithick]
(axis cs:0.916666666666667,7.78072088611792)
--(axis cs:1.08333333333333,7.78072088611792);

\path [draw=red, semithick]
(axis cs:1.91666666666667,7.68937110752969)
--(axis cs:2.08333333333333,7.68937110752969);

\path [draw=red, semithick]
(axis cs:2.91666666666667,7.65444322647011)
--(axis cs:3.08333333333333,7.65444322647011);

\path [draw=red, semithick]
(axis cs:3.91666666666667,7.61726781362835)
--(axis cs:4.08333333333333,7.61726781362835);

\path [draw=red, semithick]
(axis cs:4.91666666666667,7.58832367733522)
--(axis cs:5.08333333333333,7.58832367733522);

\path [draw=red, semithick]
(axis cs:5.91666666666667,7.55642796944025)
--(axis cs:6.08333333333333,7.55642796944025);

\path [draw=red, semithick]
(axis cs:6.91666666666667,7.51152464839087)
--(axis cs:7.08333333333333,7.51152464839087);

\path [draw=red, semithick]
(axis cs:7.91666666666667,7.48549160803075)
--(axis cs:8.08333333333333,7.48549160803075);

\path [draw=red, semithick]
(axis cs:8.91666666666667,7.46450983463653)
--(axis cs:9.08333333333333,7.46450983463653);

\path [draw=red, semithick]
(axis cs:9.91666666666667,7.44073370738926)
--(axis cs:10.0833333333333,7.44073370738926);

\path [draw=red, semithick]
(axis cs:10.9166666666667,7.41697962138115)
--(axis cs:11.0833333333333,7.41697962138115);

\path [draw=red, semithick]
(axis cs:11.9166666666667,7.39817409297047)
--(axis cs:12.0833333333333,7.39817409297047);

\path [draw=red, semithick]
(axis cs:12.9166666666667,7.37963215260955)
--(axis cs:13.0833333333333,7.37963215260955);

\path [draw=red, semithick]
(axis cs:13.9166666666667,7.36707705988101)
--(axis cs:14.0833333333333,7.36707705988101);

\path [draw=red, semithick]
(axis cs:14.9166666666667,7.34729970074316)
--(axis cs:15.0833333333333,7.34729970074316);

\path [draw=red, semithick]
(axis cs:15.9166666666667,7.33498187887181)
--(axis cs:16.0833333333333,7.33498187887181);

\path [draw=red, semithick]
(axis cs:16.9166666666667,7.32251043399739)
--(axis cs:17.0833333333333,7.32251043399739);

\path [draw=red, semithick]
(axis cs:17.9166666666667,7.30787278076371)
--(axis cs:18.0833333333333,7.30787278076371);

\path [draw=red, semithick]
(axis cs:18.9166666666667,7.29776828253138)
--(axis cs:19.0833333333333,7.29776828253138);

\path [draw=red, semithick]
(axis cs:19.9166666666667,7.28550654852279)
--(axis cs:20.0833333333333,7.28550654852279);

\path [draw=red, semithick]
(axis cs:0,7.81963831469227)
--(axis cs:0,7.8236479377877);

\path [draw=red, semithick]
(axis cs:1,7.77863233139778)
--(axis cs:1,7.78280944083805);

\path [draw=red, semithick]
(axis cs:2,7.68708277801024)
--(axis cs:2,7.69165943704914);

\path [draw=red, semithick]
(axis cs:3,7.65207355822367)
--(axis cs:3,7.65681289471656);

\path [draw=red, semithick]
(axis cs:4,7.61480839405137)
--(axis cs:4,7.61972723320533);

\path [draw=red, semithick]
(axis cs:5,7.5857920317656)
--(axis cs:5,7.59085532290484);

\path [draw=red, semithick]
(axis cs:6,7.55381427367444)
--(axis cs:6,7.55904166520607);

\path [draw=red, semithick]
(axis cs:7,7.50879091410984)
--(axis cs:7,7.51425838267189);

\path [draw=red, semithick]
(axis cs:8,7.48268577189158)
--(axis cs:8,7.48829744416992);

\path [draw=red, semithick]
(axis cs:9,7.46164450512363)
--(axis cs:9,7.46737516414942);

\path [draw=red, semithick]
(axis cs:10,7.43779943508879)
--(axis cs:10,7.44366797968973);

\path [draw=red, semithick]
(axis cs:11,7.41397481368885)
--(axis cs:11,7.41998442907346);

\path [draw=red, semithick]
(axis cs:12,7.39511224361345)
--(axis cs:12,7.40123594232748);

\path [draw=red, semithick]
(axis cs:13,7.37651300101879)
--(axis cs:13,7.38275130420032);

\path [draw=red, semithick]
(axis cs:14,7.36391850018423)
--(axis cs:14,7.37023561957779);

\path [draw=red, semithick]
(axis cs:15,7.34407805125863)
--(axis cs:15,7.3505213502277);

\path [draw=red, semithick]
(axis cs:16,7.33172030026777)
--(axis cs:16,7.33824345747586);

\path [draw=red, semithick]
(axis cs:17,7.31920792408986)
--(axis cs:17,7.32581294390492);

\path [draw=red, semithick]
(axis cs:18,7.30452157432939)
--(axis cs:18,7.31122398719802);

\path [draw=red, semithick]
(axis cs:19,7.29438304217931)
--(axis cs:19,7.30115352288345);

\path [draw=red, semithick]
(axis cs:20,7.28207954372498)
--(axis cs:20,7.28893355332059);

\addplot [semithick, red, mark=*, mark size=1, mark options={solid}, only marks]
table {%
0 7.82164312623998
1 7.78072088611792
2 7.68937110752969
3 7.65444322647011
4 7.61726781362835
5 7.58832367733522
6 7.55642796944025
7 7.51152464839087
8 7.48549160803075
9 7.46450983463653
10 7.44073370738926
11 7.41697962138115
12 7.39817409297047
13 7.37963215260955
14 7.36707705988101
15 7.34729970074316
16 7.33498187887181
17 7.32251043399739
18 7.30787278076371
19 7.29776828253138
20 7.28550654852279
};
\end{axis}

\end{tikzpicture}

						\caption{График $\log(z)$ от $t$.}
					\end{center}
				\end{figure}
			\end{minipage}
		\end{figure}
		\par Рассчитаем показатель преломления углекислого газа используя формулу (6), значение показателя преломления воздуха из предыдущего
		опыта и калибровочную кривую. Подставим в неё значение микрометра в нулевой момент времени и получим, что
		$n_{CO_2}=1.000465 \pm 0.000004$.

		\subsection{Оценка возможностей компенсатора}
		\par Оценим интервал $\delta n$, доступный для измерений, исходя из возможностей компенсатора: минимальная величина $\delta n$,
		доступная для измерений, определяется точностью компенсатора и равна $1.4\cdot10^{-8}$, максимальная --- диапазоном его работы. И равна
		$8.7\cdot10^{-6}$

	\section{Обсуждение результатов}
		
		\par Значения полученые дл показателей преломления воздуха и углекислого газа согласуются со справочными довольно неплохо:
		$n_{\text{возд}}=1.000293$ и $n_{CO_2}=1.00045$. Также полярезуемость воздуха совпала со справочными данными $\alpha=2\cdot10^{-29}$.
		Однако плохо получилось аппроксимировать процесс установления равновесия, наверно, из-за сложного процесса и несовершенства установки.
		\par Самым плохим прибором оказался манометр, так у него был сбит нулевой уровень, из-за чего пришлось преобразовывать данные.

	\section{Вывод}
		
		\par Исследовние можно считать удачным, однако стоит пререпроверить результаты с использованием качественного оборудования.

	\newpage\pagenumbering{gobble}

	\begin{figure}[H]
		\begin{center}
			\includegraphics[width=\textwidth]{data/approved.pdf}
		\end{center}
	\end{figure}


\end{document}

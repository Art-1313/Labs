\documentclass[12pt,a4paper]{article}
\usepackage[T2A]{fontenc}
\usepackage[utf8]{inputenc}
\usepackage[english, russian]{babel}
\usepackage{indentfirst}
\usepackage{misccorr}
\usepackage{graphicx}
\usepackage{amsmath}
\usepackage{amssymb}
\usepackage{circuitikz}
\usepackage[font={small}]{caption}
\usepackage[left=20mm, top=20mm, right=20mm, bottom=20mm, nohead]{geometry}
\usepackage{float}
\usepackage{tabularx}
\usepackage{array}
\usepackage{longtable}
\usepackage{pstool}
\usepackage{pgfplots}
\usepackage{hhline}
\usepackage{multirow}
\usepackage{wrapfig}
\usepackage{pdfpages}

\DeclareCaptionLabelSeparator{fill}{.\\}

\DeclareCaptionLabelFormat{fullparents}{\bothIfFirst{#1}{~}#2}

\pgfplotsset{compat=1.17}

\begin{document}

    \begin{titlepage}
        \begin{center}
            {\LARGE Отчёт по лабораторной работе 4.4.3.\\}
            \vspace*{11cm}
                \textbf{\LARGE Изучение призмы с помощью гониометра.}
            \vspace*{6.5cm}
        \end{center}
        \hfill\begin{minipage}{0.37\textwidth}
            Работу выполнил Громов Артём
            \\
            ЛФИ Б02-006
        \end{minipage}
        \vspace{4.8cm}
        \begin{center}
            Долгопрудный, 2022 г.
        \end{center}
    \end{titlepage}

    \section{Аннотация}

        \begin{flushleft}
            \textbf{Цель работы:} Знакомство с работой гониометра, исследование дисперсии стеклянной призмы и определение характеристик призмы как
            спектрального прибора.
        \end{flushleft}
        \begin{flushleft}
            \textbf{В работе используется:} гониометр, ртутная лампа, призма, стеклянная плоскопараллельная пластинка, призменный уголковый
            отражатель.
        \end{flushleft}

        \par В настоящей работе исследуется дисперсии стеклянных призм --- зависимости показателя преломления от длины волны.
        \par Показатель преломления материала призмы удобно определять по углу наименьшего отклонения. Известно, что минимальное отклонение луча,
        преломленного призмой, от направления луча, падающего на призму, получается при симметричном ходе луча (в призме луч идёт перпендикулярно
        биссектрисе преломляющего угла). Угол минимального отклонения $\delta$, преломляющий угол $\alpha$ (угол при вершине призмы на рис. 1а) и
        показатель преломления $n$ связаны между собой соотношением
        \begin{equation}
            n=\frac{\sin{\frac{\alpha+\delta}{2}}}{\sin{\frac{\alpha}{2}}}.
        \end{equation}
        \par Измерив с помощью гониометра преломляющий угол призмы и углы наименьшего отклонения для света разных длин волн, можно рассчитать
        величину $n$ и построить дисперсионную кривую — график зависимости $n(\lambda)$.
        \par По дисперсионной кривой могут быть определены такие важные характеристики оптических стёкол, как средняя дисперсия
        \begin{equation}
            D=n_F-n_C
        \end{equation}
        и коэффициент дисперсии $\nu$ (число Аббе):
        \begin{equation}
            \nu=\frac{n_D-1}{n_F-n_C}.
        \end{equation}
        Здесь $n_D$, $n_F$ и $n_C$ --- показатели преломления для $\lambda_D=589,3$ нм (среднее значение длин волн жёлтого дублета натрия),
        $\lambda_F=486,1$ нм (голубая линия водорода), $\lambda_C=656,3$ нм (красная линия водорода).
        \par По наклону дисперсионной кривой можно оценить разрешающую способность призмы
        \begin{equation}
            R=\frac{\lambda}{\delta\lambda}=b\frac{dn}{d\lambda}.
        \end{equation}
        Здесь $\delta\lambda$ --- минимальный интервал длин волн, разрешаемый по критерию Релея, $b$ --- размер основания призмы, если вся рабочая
        грань призмы освещена параллельным пучком.

    \section{Экспериментальная установка}

        \par\textbf{Гониометр.} Гониометр служит для точного измерения углов и находит широкое применение в оптических лабораториях. С помощью
        гониометра можно определять показатели преломления и преломляющие углы призм и кристаллов, исследовать параметры дифракционных решёток,
        измерять длины волн спектральных линий и т. д. В нашей работе для измерения углов используется гониометр Г5.
        \par Оптическая схема гониометра представлена на рис. 1а. Свет от источника $S$ проходит через коллиматор (устройство, дающее параллельный
        пучок, состоящее из щели 1 и объектива 5) и преобразуется призмой или решёткой в набор параллельных пучков, каждый из которых соответствует
        определённой длине волны. Параллельные пучки собираются в фокальной плоскости объектива 9 зрительной трубы и рассматриваются глазом через
        окуляр 14. При освещении щели ртутной лампой, дающей дискретный спектр, в фокальной плоскости видны отдельные линии --- цветные изображения
        входной щели.

        \begin{wrapfigure}{r}{0.46\textwidth}
            \centering
            \includegraphics[width=0.46\textwidth]{images/pic1.pdf}
            \caption{Оптическая схема и внешний вид гониометра.}
        \end{wrapfigure}
        \par Внешний вид гониометра представлен на рис. 1б и 1в. Коллиматор 3, столик 7 и алидада 17 со зрительной трубой 12 крепятся на массивном
        основании 23. На столике 7 размещаются исследуемые объекты. Коллиматор закреплён неподвижно, а столик и алидада с трубой могут вращаться вокруг
        вертикальной оси.
        \par Ширину коллиматорной щели можно менять от 0 до 2-х мм при помощи микрометрического винта 2, высоту --- от 0 до 2-х см --- при помощи
        диафрагмы с треугольным вырезом («ласточкин хвост»), надетой на щель. Винт 4 служит для перемещения объектива 5 —-- настройки коллиматора на
        параллельный пучок.
        \par Зрительная труба 12 состоит из объектива 9 и окуляра 14 с автоколлимационным устройством 13. Объективы коллиматора и зрительной трубы
        одинаковы. Фокусировка трубы производится винтом 11. Наклон коллиматора и зрительной трубы к горизонтальной оси изменяется винтами 6 и 10
        соответственно.
        \par Зрительная труба 12 состоит из объектива 9 и окуляра 14 с автоколлимационным устройством 13. Объективы коллиматора и зрительной трубы
        одинаковы. Фокусировка трубы производится винтом 11. Наклон коллиматора и зрительной трубы к горизонтальной оси изменяется винтами 6 и 10
        соответственно.
        \par Схема окуляра зрительной трубы с автоколлимационным устрой- ством приведена на рис. 5а. Свет от лампы Л проходит через защитную стеклянную
        пластинку П и попадает на автоколлимационную сетку А, содержащую две взаимно перпендикулярные щели (рис. 2б). Свет, прошедший через сетку А,
        попадает на две прямоугольные призмы $P$ и отражается от гипотенузной грани, на которую нанесен полупрозрачный слой с коэффициентом отражения
        50\%. Для юстировки гониометра на столик ставится предмет с плоской отражающей поверхностью. После отражения от неё параллельный пучок лучей
        возвращается назад в зрительную трубу и собирается в фокальной плоскости объектива. В этом случае светящийся крест можно увидеть через окуляр
        зрительной трубы. Кроме того, в окуляре имеется ещё одна сетка С, на которой изображён чёрный отсчётный крест (рис. 2в). Совмещённые изображении
        обоих крестов рассматриваются через окулярные линзы О. Резкость видимого изображения отсчётного креста регулируется вращением оправы окуляра 14.
        \par Обе сетки окуляра, А и С (рис.2а), расположены на строго одинаковых расстояниях от гипотенузных граней призмы $P$, поэтому их одновременное
        наблюдение в окуляре возможно только при совпадении фокальных плоскостей объектива и окуляра (труба настроена на бесконечность).

        \begin{wrapfigure}{r}{0.46\textwidth}
            \centering
            \includegraphics[width=0.46\textwidth]{images/pic2.pdf}
            \caption{Автоколлимационное устройство.}
        \end{wrapfigure}
        \par Важнейшим узлом гониометра является устройство, служащее для отсчёта угла поворота зрительной трубы вокруг вертикальной оси, проходящей
        через центр столика. На этой оси крепится прозрачное кольцо (лимб), расположенное в корпусе прибора. На поверхности лимба нанесена шкала с
        делениями. Лимб разделён на $3\times360=1080$ делений. Цена деления 20', оцифровка делений произведена через $1^{\circ}$. Шкалу лимба можно
        наблюдать через окуляр отсчётного устройства 16 при включённой подсветке (тумблер 22). Резкость изображения шкалы регулируется вращением
        оправы окуляра 15.
        \par Оптическая система отсчётного устройства собрана так, что через окуляр можно наблюдать изображения штрихов двух диаметрально
        противоположных участков лимба, причём одно изображение прямое, а другое обратное (рис. 3). Кроме того, оптическая система позволяет перемещать
        эти изображения друг относительно друга, оставляя в покое как лимб, так и алидаду со зрительной трубой.
        \par Это перемещение штрихов измеряется при помощи оптического микрометра. Шкала микрометра рассчитана таким образом, что при перемещении её
        на 600 делений верхнее изображение штрихов лимба смещается относительно нижнего на 10'. Следовательно, цена деления шкалы микрометра 1''.

        \begin{wrapfigure}{r}{0.46\textwidth}
            \centering
            \includegraphics[width=0.46\textwidth]{images/pic3.pdf}
            \caption{Поле зрения отсчетного микроскопа (7$^{\circ}$51'36'').}
        \end{wrapfigure}
        \par Поле зрения отсчётного микроскопа приведено на рис. 6. В левом окне наблюдаются изображения диаметрально противоположных участков лимба и
        вертикальный штрих для отсчёта градусов, в правом --- деления шкалы оптического микрометра и горизонтальная риска А для отсчёта минут и секунд.
        \par Для удобства экспериментатора в гониометре предусмотрено несколько вариантов относительного вращения столика, алидады со зрительной трубой
        и лимба.
        \par 1. Алидада вращается относительно лимба и столика либо грубо от руки при свободном винте 24 (см. рис. 4), либо точно с помощью
        микрометрического винта 25 при зажатом винте 24.
        \par 2. Такое же вращение алидады, но вместе с лимбом и столиком, производится, если рычажок 20 находится в нижнем положении. Для возвращения
        его в верхнее положение надо нажать рычажок 19.
        \par 3. Лимб вращается относительно столика и алидады винтом 21 (рычажок 20 в нижнем положении, винт 26 свободен).
        \par 4. Вращение столика вместе с лимбом относительно алидады произ- водится либо от руки при свободном винте 26, либо микрометрическим винтом
        28 при зажатом винте 26.
        \par 5. Столик вращается относительно алидады и лимба либо от руки при свободном винте 27, либо точно микрометрическим винтом 29 при зажатом
        винте 27 (рычажок 20 в верхнем положении).
        \par Гониометр требует тщательной юстировки, которая заключается в установке: а) зрительной трубы на бесконечность; б) поверхности столика и
        оптической оси трубы --- перпендикулярно оси вращения прибора; в) коллиматора --- на параллельный пучок лучей; г) оптической оси коллиматора ---
        перпендикулярно оси вращения прибора.
        
    \section{Результаты измерений и обработка данных}

        \subsection{Измерение преломляющего угла}
        \par Для измерения преломляющего угла призмы установим трубу перпендикулярно одной из её отражающих граней и запишем значение угла
        $\alpha=0^{\circ}1'0''$. Затем, не трогая призму и столик, повернём алидаду с трубой вокруг преломляющего угла призмы и проведём
        ту же операцию для другой рабочей грани $\alpha_2=60^{\circ}5'8''\pm0^{\circ}0'1''$. По углу поворота трубы рассчитаем преломляющий угол призмы
        $\alpha=\left|\alpha_1-\alpha_2\right|=60^{\circ}4'8''\pm0^{\circ}0'0''$.

        \subsection{Минимальный угол отклонения}
        \par Найдём спектр в зрительную трубу, настроемся на одну из жёлтых линий и, вращая столик сначала рукой, а затем винтом тонкой подачи
        29 при зажатом винте 27, установим его в такое положение, при котором отклонение выбранной спектральной линии от направления оси коллиматора
        оказывается наименьшим.
        \par Для каждой линии спектра найдите свой минимум отклонения. Данные занесём в таблицу 1.
        \begin{table}[H]
            \captionsetup{font={small}, labelformat=fullparents, labelsep=fill, labelfont=bf, justification=raggedleft,
                        singlelinecheck=false, skip=-0.2cm}
            \caption{Наименьший угол отклонения для различных волн.}
            \begin{center}
                \begin{tabular}{|l|l|}\hline$m$&$z$, дел
\\\hline0.0&275.0
\\\hline1.0&311.0
\\\hline2.0&348.0
\\\hline3.0&381.0
\\\hline4.0&419.0
\\\hline5.0&453.0
\\\hline6.0&489.0
\\\hline7.0&529.0
\\\hline-1.0&250.0
\\\hline-2.0&214.0
\\\hline-3.0&178.0
\\\hline-4.0&148.0
\\\hline-5.0&109.0
\\\hline-6.0&73.0
\\\hline-7.0&35.0
\\\hline\end{tabular}
            \end{center}
        \end{table}
        \begin{figure}[H]
            \begin{center}
                % This file was created with tikzplotlib v0.9.16.
\begin{tikzpicture}

\begin{axis}[
legend cell align={left},
legend style={
  fill opacity=0.8,
  draw opacity=1,
  text opacity=1,
  at={(0.03,0.97)},
  anchor=north west,
  draw=white!80!black
},
height=11.2cm,
tick align=inside,
major tick length=0.2cm,
minor tick length=0.1cm,
tick pos=left,
xmin=0, xmax=600,
xtick={0, 50, 100, 150, 200, 250, 300, 350, 400, 450, 500, 550, 600},
minor x tick num=4,
xmajorgrids,
minor x grid style={dotted,black},
xminorgrids,
xtick style={color=black},
xlabel={$D_{\text{винта}}$, мкм},
ymin=0, ymax=500,
ytick={0, 50, 100, 150, 200, 250, 300, 350, 400, 450, 500},
ytick style={color=black},
minor y tick num=4,
ymajorgrids,
minor y grid style={dotted,black},
yminorgrids,
ytick style={color=black},
ylabel={$D_{\text{расчёт}}$, мкм}
]
\path [draw=red, semithick]
(axis cs:40,32.9861111111111)
--(axis cs:60,32.9861111111111);

\path [draw=red, semithick]
(axis cs:90,98.9583333333333)
--(axis cs:110,98.9583333333333);

\path [draw=red, semithick]
(axis cs:140,131.944444444444)
--(axis cs:160,131.944444444444);

\path [draw=red, semithick]
(axis cs:190,164.930555555556)
--(axis cs:210,164.930555555556);

\path [draw=red, semithick]
(axis cs:240,197.916666666667)
--(axis cs:260,197.916666666667);

\path [draw=red, semithick]
(axis cs:290,230.902777777778)
--(axis cs:310,230.902777777778);

\path [draw=red, semithick]
(axis cs:340,263.888888888889)
--(axis cs:360,263.888888888889);

\path [draw=red, semithick]
(axis cs:390,296.875)
--(axis cs:410,296.875);

\path [draw=red, semithick]
(axis cs:440,362.847222222222)
--(axis cs:460,362.847222222222);

\path [draw=red, semithick]
(axis cs:490,395.833333333333)
--(axis cs:510,395.833333333333);

\path [draw=red, semithick]
(axis cs:50,-0.00334608800363867)
--(axis cs:50,65.9755683102259);

\path [draw=red, semithick]
(axis cs:100,65.9421196383399)
--(axis cs:100,131.974547028327);

\path [draw=red, semithick]
(axis cs:150,98.9048365902785)
--(axis cs:150,164.98405229861);

\path [draw=red, semithick]
(axis cs:200,131.860893814377)
--(axis cs:200,198.000217296734);

\path [draw=red, semithick]
(axis cs:250,164.810309447695)
--(axis cs:250,231.023023885639);

\path [draw=red, semithick]
(axis cs:300,197.75310554622)
--(axis cs:300,264.052450009336);

\path [draw=red, semithick]
(axis cs:350,230.689308019177)
--(axis cs:350,297.088469758601);

\path [draw=red, semithick]
(axis cs:400,263.618946552474)
--(axis cs:400,330.131053447526);

\path [draw=red, semithick]
(axis cs:450,329.458668896105)
--(axis cs:450,396.235775548339);

\path [draw=red, semithick]
(axis cs:500,362.368830131627)
--(axis cs:500,429.29783653504);

\addplot [semithick, red, forget plot]
table {%
0 0
500 392.40620488787
};
\path [draw=green!50.1960784313725!black, semithick]
(axis cs:90,53.0442477876106)
--(axis cs:110,53.0442477876106);

\path [draw=green!50.1960784313725!black, semithick]
(axis cs:140,95.1428571428572)
--(axis cs:160,95.1428571428572);

\path [draw=green!50.1960784313725!black, semithick]
(axis cs:190,148)
--(axis cs:210,148);

\path [draw=green!50.1960784313725!black, semithick]
(axis cs:240,188.913461538462)
--(axis cs:260,188.913461538462);

\path [draw=green!50.1960784313725!black, semithick]
(axis cs:290,233.740384615385)
--(axis cs:310,233.740384615385);

\path [draw=green!50.1960784313725!black, semithick]
(axis cs:340,270.260869565217)
--(axis cs:360,270.260869565217);

\path [draw=green!50.1960784313725!black, semithick]
(axis cs:390,326.204081632653)
--(axis cs:410,326.204081632653);

\path [draw=green!50.1960784313725!black, semithick]
(axis cs:440,360)
--(axis cs:460,360);

\path [draw=green!50.1960784313725!black, semithick]
(axis cs:490,396)
--(axis cs:510,396);

\path [draw=green!50.1960784313725!black, semithick]
(axis cs:540,468)
--(axis cs:560,468);

\path [draw=green!50.1960784313725!black, semithick]
(axis cs:100,52.4114526583863)
--(axis cs:100,53.6770429168349);

\path [draw=green!50.1960784313725!black, semithick]
(axis cs:150,91.0666302830339)
--(axis cs:150,99.2190840026804);

\path [draw=green!50.1960784313725!black, semithick]
(axis cs:200,141.683425571105)
--(axis cs:200,154.316574428895);

\path [draw=green!50.1960784313725!black, semithick]
(axis cs:250,182.271308845487)
--(axis cs:250,195.555614231436);

\path [draw=green!50.1960784313725!black, semithick]
(axis cs:300,223.88999208276)
--(axis cs:300,243.590777148009);

\path [draw=green!50.1960784313725!black, semithick]
(axis cs:350,257.145804132173)
--(axis cs:350,283.375934998262);

\path [draw=green!50.1960784313725!black, semithick]
(axis cs:400,311.969339856676)
--(axis cs:400,340.438823408631);

\path [draw=green!50.1960784313725!black, semithick]
(axis cs:450,341.133063042629)
--(axis cs:450,378.866936957371);

\path [draw=green!50.1960784313725!black, semithick]
(axis cs:500,376.956019710008)
--(axis cs:500,415.043980289992);

\path [draw=green!50.1960784313725!black, semithick]
(axis cs:550,448.557402072788)
--(axis cs:550,487.442597927212);

\addplot [semithick, green!50.1960784313725!black, forget plot]
table {%
0 0
550 439.564038903656
};
\addplot [semithick, red, mark=*, mark size=3, mark options={solid}, only marks]
table {%
50 32.9861111111111
100 98.9583333333333
150 131.944444444444
200 164.930555555556
250 197.916666666667
300 230.902777777778
350 263.888888888889
400 296.875
450 362.847222222222
500 395.833333333333
};
\addlegendentry{Линза}
\addplot [semithick, green!50.1960784313725!black, mark=*, mark size=3, mark options={solid}, only marks]
table {%
100 53.0442477876106
150 95.1428571428572
200 148
250 188.913461538462
300 233.740384615385
350 270.260869565217
400 326.204081632653
450 360
500 396
550 468
};
\addlegendentry{Спектр}
\end{axis}

\end{tikzpicture}

                \caption{Дисперсионная кривая $n(\lambda$).}
            \end{center}
        \end{figure}
        \par Рассчитаем значения показателя преломления для спектра ртутной лампы по формуле (1). Данные поместим в таблицу 1. Построим дисперсионную
        кривую и аппроксимируем её функцией
        \begin{equation}
            n = p_1T(0, y(\lambda))+p_2T(1, y(\lambda))+p_3\exp{(-p_4y(\lambda))},
        \end{equation}
        где $T(n, y)$ --- полином Чебышева $n$-ой степени, а $y(\lambda)$ --- отображение в отрезок $[-0.25, 0.25]$.

        \subsection{Число Аббе и средняя дисперсия.}
        \par Определим по графику значения $n_D=1.6572$, $n_F=1.6700$, $n_C=1.6512$. Используя формулу (2), получим $D=0.019$. Из (3) следует, что
        $nu=34.846$.
        \par Исходя из данных на рисунке 5, мы можем сделать вывод, что стекло призмы --- это или бариевый тяжёлый флинт, или тяжёлый флинт.
        \begin{figure}[H]
            \begin{center}
                \includegraphics[width=0.9\textwidth]{images/pic4.pdf}
                \caption{Диаграмма Аббе для оптических стёкол.}
            \end{center}
        \end{figure}

        \subsection{Разрешающая способность}
        \par Рассчитаем максимальную разрешающую способность призмы по формуле (4). Значение производной на возьмём в точке с длиной волны 579.1 нм.
        Основание призмы равно $b=7.3\pm1$ см. В итоге получим $R=b\frac{dn}{d\lambda}=(74\pm1)\cdot10^{2}$.
        \par Выполним обратную задачу. Используя жёлтую линию спектра с длинной волны $\lambda=579.1$ нм вычислим размер основания призмы. Измерения дают
        следующие результаты $\delta\lambda=0^{\circ}0'5''$. Значит $b=0.7\pm0.1$ мм.
        \par Оценим, при каком размере решётки, имеющей 100 штр/мм, она обладает такой же разрешающей способностью в первом порядке, как призма с
        основанием $b=5$ см. Тогда $a\cdot100$ штр/мм $=R$, значит $a=74$ мм.
        \par Рассчитаем угловую дисперсию $d\phi/d\lambda$ по измерениям жёлтого дублета и сравним её с дисперсией решётки в первом порядке, имеющей
        100 штр/мм. Из измерений имеем $d\phi/d\lambda=1360^{\circ}$/нм. Для решётки $D=\frac{1}{100\text{штр/мм}}=5.73^{\circ}/\text{нм}\cdot10^5$.

    \section{Обсуждение результатов}

        \par Благодаря гониометру погрешности очень маленькие. Однако не полусилось определить погрешности для коэффициентов функции аппроксимации.

    \section{Вывод}
        
        \par Результаты работы можно считать удовлетворительными, но не получилось точно определить сорт стекла.

    \newpage\pagenumbering{gobble}

    \begin{figure}[H]
    \begin{center}
            \includegraphics[width=\textwidth]{data/approved.pdf}
        \end{center}
    \end{figure}

\end{document}

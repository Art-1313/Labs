\documentclass[12pt,a4paper]{article}
\usepackage[utf8]{inputenc}
\usepackage[english,russian]{babel}
\usepackage{indentfirst}
\usepackage{misccorr}
\usepackage{graphicx}
\usepackage{amsmath}
\usepackage{amssymb}
\usepackage{circuitikz}
\usepackage[font={small}]{caption}
\usepackage[left=20mm, top=20mm, right=20mm, bottom=20mm, nohead]{geometry}
\usepackage{float}
\usepackage{tabularx}
\usepackage{array}
\usepackage{longtable}
\usepackage{pstool}
\usepackage{pgfplots}
\usepackage{hhline}
\usepackage{multirow}

\DeclareCaptionLabelSeparator{fill}{.\\}
\DeclareCaptionLabelFormat{fullparents}{\bothIfFirst{#1}{~}#2}
\DeclareCaptionLabelFormat{empty}{}

\begin{document}

	\begin{titlepage}
		\begin{center}
			{\LARGE Отчёт по лабораторной работе 2.5.1.\\}
			\vspace*{11cm}
				\textbf{\LARGE Измерение коэффициента поверхностного натяжения жидкости.}	
			\vspace*{6cm}
		\end{center}
		\hfill\begin{minipage}{0.35\textwidth}
				Работу выполнил Грмов Артём\\
				ЛФИ Б02-006
		\end{minipage}
		\vspace{5cm}
		\begin{center}
 			Долгопрудный, 2021 г.
		\end{center}   
	\end{titlepage}
	
	\section{Аннотация}
		\noindent\textbf{Цель работы: }
		\begin{enumerate}
			\item измерение температурной зависимости  коэффициента поверхностного натяжения дистиллированной воды с
			использованием известного коэффициента поверхностного натяжения спирта;
			\item определение полной поверхностной энергии  и теплоты, необходимой для изотермического образования единицы
			поверхности жидкости при различной температуре.
		\end{enumerate}
		\noindent\textbf{В работе используются: } прибор  Ребиндера  с термостатом и микроманометром; исследуемые жидкости;
		стаканы.
		\vspace{0.5 cm}
		\par Наличие поверхностного слоя приводит к различию давлений по разные стороны от искривленной границы раздела двух
		сред. Для сферического пузырька с воздухом  внутри жидкости избыточное давление даётся формулой Лапласа:
		\begin{equation}
			\Delta P=P_{\text{внутри}}-P_{\text{снаружи}}=\frac{2\sigma}{r}.
		\end{equation}
		\noindent где $\sigma$ --- коэффициент поверхностного натяжения, $P_{\text{внутри}}$ и $P_{\text{снаружи}}$ --- давления
		внутри пузырька и снаружи, $r$ --- радиус кривизны поверхности раздела двух фаз. Эта формула лежит в основе
		предлагаемого мктода определения коэффициента поверхностного натяжения жидкости. Измеряется давление $\Delta P$,
		необходимое для выталкивания в жидкость пузырька газа.
	\section{Экспериментальная установка}
		\par Исследуемая жидкость (дистиллированная вода) наливается в сосуд (колбу) B (рис.1). Тестовая жидкость (этиловый спирт)
		наливается в сосуд E. Приизмерениях колбы герметично закрываются пробками. Через одну из двух пробок проходит полая
		металлическая игла C. Этой пробкой закрывается сосуд, в котором проводятся измерения. Верхний конец иглы открыт в
		атмосферу, а нижний погружен в жидкость. Другой сосуд герметично закрывается второй пробкой. При создании достаточного
		разряжения воздуха в колбе с иглой пузырьки воздуха начинают пробулькивать через жидкость. Поверхностное натяжение можно
		определить по величине разряжения $\Delta P$, необходимого для прохождения пузырьков, с помощью формулы (1) (при
		известном радиусе иглы).
		\par Разряжение в системе создается с помощью аспиратора А. Кран К2 разделяет две полости аспиратора. Верхняя полость при
		закрытом кране К2 заполняется водой. Затем кран К2 открывают и заполняют водой нижнюю полость аспиратора. Разряжение
		воздуха создается в нижней полости при открывании крана К1, когда вода вытекает из неё по каплям. В колбах В и С,
		соединённых
		трубками с нижней полостью аспиратора, создается такое же пониженное давление. Разность давлений в полостях с
		разряженным воздухом и атмосферой измеряется спиртовым микроманометром.
		\par Для стабилизации температуры исследуемой жидкости через рубашку D колбы В непрерывно прогоняется вода из
		термостата.
		\par Обычно кончик иглы лишь касается поверхности жидкости, чтобы исключить влияние гидростатического давления столба
		жидкости. Однако при измерении температурной зависимости коэффициента поверхностного натяжения возникает ряд
		сложностей. Во-первых, большая теплопроводность металлической трубки приводит к тому, что температура на конце трубки
		заметно ниже, чем в глубине жидкости. Во-вторых, тепловое расширение поднимает уровень жидкости при увеличении
		температуры.
		\begin{figure}[H]
			\begin{center}
				\includegraphics[width=\textwidth]{C:/Users/gromo/Desktop/Установка251.png}
				\caption{Схема установки для измерения температурной зависимости коэффициента поверхностного натяжения}
			\end{center}
		\end{figure}
		\par Обе погрешности можно устранить, погрузив кончик трубки до самого дна. Полное давление, измеренное при этом
		микроманометром, $P=\Delta P+\rho gh$. Заметим, что $\rho gh$ от температуры практически не зависит, так как подъём уровня
		жидкости компенсируется уменьшением её плотности (произведение $\rho h$ определяется массой всей жидкости и поэтому
		постоянно). Величину $\rho gh$ следует измерить двумя способами. Во-первых, замерить величину $P_{1}\Delta P'$, когда кончик
		трубки только касается поверхности жидкости. Затем при этой же температуре опустить иглу до дна и замерить
		$P_{2}=\rho gh+\Delta P''$ ($\Delta P'$, $\Delta P''$ --- давления Лапласа). Из-за несжимаемости жидкости можно положить
		$\Delta P'=\Delta P''$ и тогда $\rho gh=P_{2}-P_{1}$. Во-вторых, при измерениях $P_{1}$ и $P_{2}$ замерить линейкой глубину
		погружения иглы $h$. Это можно сделать, замеряя расстояние между верхним концом иглы и любой неподвижной частью
		прибора при положении иглы на поверхности и в глубине колбы.
	\section{Результаты измерений и обработка данных}
		\subsection{Измерение диаметра иглы}
		\par Измерение диаметра иглы проведём двумя методами:
		\begin{enumerate}
			\item напрямую, с помощью микроскопа;
			\item косвенно, определяя максимальное давление $\Delta P_{\text{спирта}}$ при пробулькивании пузырьков воздуха через
			спирт.
		\end{enumerate}
		\par С помощью микроскопа получим значение: $d_{\text{мик}}=1.15\pm0.05$ мм.
		\par Для измерения диаметра иглы вторым способом проведём серию из трёх измерений давления  $\Delta P_{\text{спирта}}$.
		Учтём, что погрешность микроманометра составляет $\Delta_{\text{ман}}=1$ деление. Полученные данные занесём в таблицу 1. 
		\begin{table}[H]
			\captionsetup{font={small}, labelformat=fullparents, labelsep=fill, labelfont=bf, justification=raggedleft,
							singlelinecheck=false, skip=-0.2cm}
			\caption{Результаты измерения $\Delta P_{\text{спирта}}$}
			\begin{center}
	  			\begin{tabular}{|>{\centering}p{2cm}|>{\centering}p{2cm}|>{\centering}p{2cm}
	  							|>{\centering}p{2cm}|>{\centering}p{2cm}|>{\centering}p{2cm}|}
				      \hline 
					\multicolumn{3}{|>{\centering}p{6cm}|}{Серия измерений $\Delta P_{\text{спирта}}$, дел} & 
					\multicolumn{1}{p{2cm}|}{\multirow{2}{2cm}{\centering$\Delta P_{\text{ср}}$, дел}} &
					\multicolumn{1}{p{2cm}|}{\multirow{2}{2cm}{\centering$\sigma_{\text{случ}}$, дел}} &
					\multicolumn{1}{p{2cm}|}{\multirow{2}{2cm}{\centering$\sigma_{\text{полн}}$, дел}} \\
					\hhline{---~~~} 1 & 2 & 3 & & & \\
					\hline
					\multicolumn{1}{|p{2cm}|}{39} & \multicolumn{1}{p{2cm}|}{39} 
					& \multicolumn{1}{p{2cm}|}{39} & \multicolumn{1}{p{2cm}|}{39} & \multicolumn{1}{p{2cm}|}{0} 
					& \multicolumn{1}{p{2cm}|}{1} \\
					\hline
				\end{tabular}
			\end{center}
		\end{table}
		\par Случайную и полную погрешности  рассчитаем используя следующие формулы:
		\begin{equation}
			\sigma_{\text{случ}}=\sqrt{\frac{1}{n(n-1)}\sum{\left(x_{i}-x_{\text{ср}}\right)}},
		\end{equation}
		\begin{equation}
			\sigma_{\text{полн}}=\sqrt{\sigma^{2}_{\text{случ}}+\sigma^{2}_{\text{приб}}},
		\end{equation}
		\noindent где $n$ --- количество измерений данной величины, $x_{\text{ср}}$ --- её среднее значение, $x_{i}$ --- её значение в
		конкретном опыте, $\sigma^{2}_{\text{приб}}$ --- приборная погрешность.
		\par Для перевода показаний микроманометра в паскали надо воспользоваться слдующей формулой:
		\begin{equation}
			P=C \cdot h \cdot \frac{\gamma_{\text{сп залит}}}{\gamma_{\text{сп пр}}} \cdot K \cdot 9.81,
		\end{equation}
		\noindent где $P$ --- давление в паскалях; $C=1$ --- поправочный множитель; $h$ --- значение в делениях шкалы; $K=0.2$ ---
		постоянная угла наклона; $\gamma_{\text{сп залит}}$ --- плотность спирта, залитого в прибор;
		$\gamma_{\text{сп пр}}=0.8095\text{ гм/см}^{3}$ --- плотность спирта, указанная на приборе.
		\par Определяя диаметр иглы с помощью формулы (1), получаем следущие значения: от $d_{\text{спир}}=1.12\pm0.03$ мм до
		$d_{\text{спир}}=1.17\pm0.03$ мм. Точное значение зависит от залитого в прибор спирта. Оба значения совпадают с диаметром
		измереным на микроскопе в пределах погрешностей, поэтому будем в дальнейших расчётах использовать $d_{\text{мик}}$.
		\par Результаты этого опыта используем для определения плотности залитого в прибор спирта
		$\gamma_{\text{сп залит}}\approx0.8205\text{ гм/см}^{3}$. 
		\subsection{Определение глубины погружения иглы}
		\par Для проведения дальнейших опытов необходимо узнать глубину погружения иглы. Воспользуемся двумя методами
		измерения этой величины:
		\begin{enumerate}
			\item напрямую, с помощью линейки;
			\item косвенно, определяя разность давлений $P_{1}$ и $P_{2}$ при пробулткивании пузырьков воздуха через воду.
			$P_{1}$ --- давление, когда игла находится у самой поверхности; $P_{2}$ --- давление на интересующей нас глубине.
		\end{enumerate}
		\par Измерения при помощи линейки дают нам следующие значения: $h_{1}=17\pm0.5$ мм, $h_{2}=6\pm0.5$ мм,
		$\Delta h_{\text{лин}}=11\pm0.5$ мм ($h_{1}$ и $H_{2}$ растояния от конца иглы до дна сосуда в разных положениях).
		\par Для определения глубины погружения иглы вторым способом измерим $P_{1}$ и $P_{2}$ с помощью серии из трёх опытов.
		Данные занесём в таблицу 2. 
		\begin{table}[H]
			\captionsetup{font={small}, labelformat=fullparents, labelsep=fill, labelfont=bf, justification=raggedleft,
							singlelinecheck=false, skip=-0.2cm}
			\caption{Результаты измерения $P_{1}$ и $P_{2}$}
			\begin{center}
	  			\begin{tabular}{|>{\centering}p{2cm}|>{\centering}p{2cm}|>{\centering}p{2cm}|>{\centering}p{2cm}
	  							|>{\centering}p{2cm}|>{\centering}p{2cm}|>{\centering}p{2cm}|}
				      \hline
					\multicolumn{1}{|p{2cm}|}{\multirow{2}{2cm}{\centering Величина}} & 
					\multicolumn{3}{>{\centering}p{6cm}|}{Серия измерений давления, дел} & 
					\multicolumn{1}{p{2cm}|}{\multirow{2}{2cm}{\centering$d_{\text{ср}}$, дел}} &
					\multicolumn{1}{p{2cm}|}{\multirow{2}{2cm}{\centering$\sigma_{\text{случ}}$, дел}} &
					\multicolumn{1}{p{2cm}|}{\multirow{2}{2cm}{\centering$\sigma_{\text{полн}}$, дел}} \\
					\hhline{~---~~~} & 1 & 2 & 3 & & & \\
					\hline
					\multicolumn{1}{|p{2cm}|}{$P_{1}$} & \multicolumn{1}{p{2cm}|}{95.0} & \multicolumn{1}{p{2cm}|}{96.0} 
					& \multicolumn{1}{p{2cm}|}{97.0} & \multicolumn{1}{p{2cm}|}{96.0} & \multicolumn{1}{p{2cm}|}{0.6} 
					& \multicolumn{1}{p{2cm}|}{1.2} \\
					\hline
					\multicolumn{1}{|p{2cm}|}{$P_{2}$} & \multicolumn{1}{p{2cm}|}{154.0} & \multicolumn{1}{p{2cm}|}{153.0} 
					& \multicolumn{1}{p{2cm}|}{153.0} & \multicolumn{1}{p{2cm}|}{153.3} & \multicolumn{1}{p{2cm}|}{0.3} 
					& \multicolumn{1}{p{2cm}|}{1.2} \\
					\hline
				\end{tabular}
			\end{center}
		\end{table}
		\noindent С помощью формулы (2) и формулы для расчёта давления столба жидкости получим:
		\begin{equation}
			\Delta h=\frac{P_{1}-P_{2}}{\rho_{\text{воды}}g}.
		\end{equation}
		\noindent Окончательно получим, что $\Delta h_{\text{ман}}=11.6\pm0.3$ мм. Это значение сходится с полученным нами при
		помощи другого метода в пределах погрешностей. В дальнейших расчётах будем использовать $\Delta h_{\text{ман}}$, так это
		значение меньше зависит от человеческого фактора.
		\subsection{Определение коэфициента поверхностного натяжения воды в зависимости от температуры}
		\par Снимим температурную зависимость $\sigma_{\text{воды}}\left(T\right)$ дистилированной воды. Будем проводить измерения
		в диапазоне от 20℃ до 60℃ через каждые 5℃. Для уменьшения случайной погрешности будем проводить серии из трёх опытов.
		Полученные данные занесём в таблицу 3.
		\begin{table}[H]
			\captionsetup{font={small}, labelformat=fullparents, labelsep=fill, labelfont=bf, justification=raggedleft,
							singlelinecheck=false, skip=-0.2cm}
			\caption{Результаты измерения мкасимального давления пробулькивания}
			\begin{center}
	  			\begin{tabular}{|>{\centering}p{2cm}|>{\centering}p{2cm}|>{\centering}p{2cm}|>{\centering}p{2cm}
	  							|>{\centering}p{2cm}|>{\centering}p{2cm}|>{\centering}p{2cm}|}
				      \hline
					\multicolumn{1}{|p{2cm}|}{\multirow{2}{2cm}{\centering T, ℃}} & 
					\multicolumn{3}{>{\centering}p{6cm}|}{Серия измерений $P_{\text{ман}}$, дел} & 
					\multicolumn{1}{p{2cm}|}{\multirow{2}{2cm}{\centering$P_{\text{ср}}$, дел}} &
					\multicolumn{1}{p{2cm}|}{\multirow{2}{2cm}{\centering$\sigma_{\text{случ}}$, дел}} &
					\multicolumn{1}{p{2cm}|}{\multirow{2}{2cm}{\centering$\sigma_{\text{полн}}$, дел}} \\
					\hhline{~---~~~} & 1 & 2 & 3 & & & \\
					\hline
					\multicolumn{1}{|p{2cm}|}{20.0} & \multicolumn{1}{p{2cm}|}{154.0} & \multicolumn{1}{p{2cm}|}{153.0} 
					& \multicolumn{1}{p{2cm}|}{153.0} & \multicolumn{1}{p{2cm}|}{153.3} & \multicolumn{1}{p{2cm}|}{0.3} 
					& \multicolumn{1}{p{2cm}|}{1.0} \\
					\hline
					\multicolumn{1}{|p{2cm}|}{25.0} & \multicolumn{1}{p{2cm}|}{150.0} & \multicolumn{1}{p{2cm}|}{153.0} 
					& \multicolumn{1}{p{2cm}|}{153.0} & \multicolumn{1}{p{2cm}|}{152.0} & \multicolumn{1}{p{2cm}|}{1.0} 
					& \multicolumn{1}{p{2cm}|}{1.4} \\
					\hline
					\multicolumn{1}{|p{2cm}|}{30.0} & \multicolumn{1}{p{2cm}|}{150.0} & \multicolumn{1}{p{2cm}|}{151.0} 
					& \multicolumn{1}{p{2cm}|}{151.0} & \multicolumn{1}{p{2cm}|}{150.7} & \multicolumn{1}{p{2cm}|}{0.3} 
					& \multicolumn{1}{p{2cm}|}{1.0} \\
					\hline
					\multicolumn{1}{|p{2cm}|}{35.2} & \multicolumn{1}{p{2cm}|}{148.0} & \multicolumn{1}{p{2cm}|}{148.0} 
					& \multicolumn{1}{p{2cm}|}{149.0} & \multicolumn{1}{p{2cm}|}{148.3} & \multicolumn{1}{p{2cm}|}{0.3} 
					& \multicolumn{1}{p{2cm}|}{1.0} \\
					\hline
					\multicolumn{1}{|p{2cm}|}{40.2} & \multicolumn{1}{p{2cm}|}{146.0} & \multicolumn{1}{p{2cm}|}{147.0} 
					& \multicolumn{1}{p{2cm}|}{147.0} & \multicolumn{1}{p{2cm}|}{146.7} & \multicolumn{1}{p{2cm}|}{0.3} 
					& \multicolumn{1}{p{2cm}|}{1.0} \\
					\hline
					\multicolumn{1}{|p{2cm}|}{45.1} & \multicolumn{1}{p{2cm}|}{147.0} & \multicolumn{1}{p{2cm}|}{146.0} 
					& \multicolumn{1}{p{2cm}|}{146.0} & \multicolumn{1}{p{2cm}|}{146.3} & \multicolumn{1}{p{2cm}|}{0.3} 
					& \multicolumn{1}{p{2cm}|}{1.0} \\
					\hline
					\multicolumn{1}{|p{2cm}|}{50.1} & \multicolumn{1}{p{2cm}|}{146.0} & \multicolumn{1}{p{2cm}|}{145.0} 
					& \multicolumn{1}{p{2cm}|}{146.0} & \multicolumn{1}{p{2cm}|}{145.7} & \multicolumn{1}{p{2cm}|}{0.3} 
					& \multicolumn{1}{p{2cm}|}{1.0} \\
					\hline
					\multicolumn{1}{|p{2cm}|}{55.1} & \multicolumn{1}{p{2cm}|}{144.0} & \multicolumn{1}{p{2cm}|}{144.0} 
					& \multicolumn{1}{p{2cm}|}{143.0} & \multicolumn{1}{p{2cm}|}{143.7} & \multicolumn{1}{p{2cm}|}{0.3} 
					& \multicolumn{1}{p{2cm}|}{1.0} \\
					\hline
				\end{tabular}
			\end{center}
		\end{table}
		\noindent Погрешности были расчитаны по формулам (2) и (3). 
		\par Теперь расчитаем коэффициент поверхностного натяжения воды $\sigma_{\text{воды}}$ по формле (1) при различных
		температурах и оценим его погрешность используя выражение:
		\begin{equation}
			\sigma_{\sigma_{\text{воды}}}=\sqrt{\left(\frac{\sigma_{d}}{d}\right)^{2}+\left(\frac{\sigma_{\Delta P}}{\Delta P}\right)^{2}},
		\end{equation}
		\noindent где $\Delta P=P_{\text{ман}}-p_{\text{воды}}g\Delta h$.
		\par Полученные данные занесём в таблицу 4.
		\begin{table}[H]
			\captionsetup{font={small}, labelformat=fullparents, labelsep=fill, labelfont=bf, justification=raggedleft,
							singlelinecheck=false, skip=-0.2cm}
			\caption{Результаты измерения $\sigma_{\text{воды}}$ при разных температурах}
			\begin{center}
	  			\begin{tabular}{|>{\centering}p{1.9cm}|>{\centering}p{1.4cm}|>{\centering}p{1.4cm}|>{\centering}p{1.4cm}
	  							|>{\centering}p{1.4cm}|>{\centering}p{1.4cm}|>{\centering}p{1.4cm}|>{\centering}p{1.4cm}
	  							|>{\centering}p{1.4cm}|}
				      \hline
				      	\multicolumn{1}{|p{2.2cm}|}{T, ℃} & \multicolumn{1}{p{1.2cm}|}{20.0} & \multicolumn{1}{p{1.2cm}|}{25.0} 
					& \multicolumn{1}{p{1.2cm}|}{30.0} & \multicolumn{1}{p{1.2cm}|}{35.2} & \multicolumn{1}{p{1.2cm}|}{40.2} 
					& \multicolumn{1}{p{1.2cm}|}{45.1} & \multicolumn{1}{p{1.2cm}|}{50.1} & \multicolumn{1}{p{1.2cm}|}{55.1} \\
					\hline
					\multicolumn{1}{|p{2cm}|}{$\Delta P$, Па} & \multicolumn{1}{p{1.4cm}|}{191} & \multicolumn{1}{p{1.4cm}|}{188} 
					& \multicolumn{1}{p{1.4cm}|}{186} & \multicolumn{1}{p{1.4cm}|}{181} & \multicolumn{1}{p{1.4cm}|}{178} 
					& \multicolumn{1}{p{1.4cm}|}{177} & \multicolumn{1}{p{1.4cm}|}{176} & \multicolumn{1}{p{1.4cm}|}{172} \\
					\hline
					\multicolumn{1}{|p{2cm}|}{$\sigma_{\Delta P}$, Па} &\multicolumn{1}{p{1.4cm}|}{5} &\multicolumn{1}{p{1.4cm}|}{5} 
					& \multicolumn{1}{p{1.4cm}|}{5} & \multicolumn{1}{p{1.4cm}|}{5} & \multicolumn{1}{p{1.4cm}|}{5} 
					& \multicolumn{1}{p{1.4cm}|}{5} & \multicolumn{1}{p{1.4cm}|}{5} & \multicolumn{1}{p{1.4cm}|}{5} \\
					\hline
					\multicolumn{1}{|p{2cm}|}{$\sigma_{\text{воды}}$, Н/м} 
					& \multicolumn{1}{p{1.4cm}|}{0.055} & \multicolumn{1}{p{1.4cm}|}{0.054} 
					& \multicolumn{1}{p{1.4cm}|}{0.053} & \multicolumn{1}{p{1.4cm}|}{0.052} & \multicolumn{1}{p{1.4cm}|}{0.051} 
					& \multicolumn{1}{p{1.4cm}|}{0.051} & \multicolumn{1}{p{1.4cm}|}{0.051} & \multicolumn{1}{p{1.4cm}|}{0.050} \\
					\hline
					\multicolumn{1}{|p{2.2cm}|}{$\sigma_{\sigma_{\text{воды}}}$, Н/м} 
					& \multicolumn{1}{p{1.4cm}|}{0.002} & \multicolumn{1}{p{1.4cm}|}{0.002} 
					& \multicolumn{1}{p{1.4cm}|}{0.002} & \multicolumn{1}{p{1.4cm}|}{0.002} & \multicolumn{1}{p{1.4cm}|}{0.002} 
					& \multicolumn{1}{p{1.4cm}|}{0.002} & \multicolumn{1}{p{1.4cm}|}{0.002} & \multicolumn{1}{p{1.4cm}|}{0.002} \\
					\hline
				\end{tabular}
			\end{center}
		\end{table}
		\begin{figure}[H]
			\begin{center}
				\input{C:/Users/gromo/PycharmProjects/plottest/plot6.tex}
			\end{center}
			\captionsetup{font={small}, justification=justified}
			\caption[figure]{График зависимости $\sigma_{\text{воды}}$ от $T$}
		\end{figure}
		\par Теоретическая прямая на нрафике (рис.2) была по =строена с помощью метода наименьших квадратов:
		\begin{equation}
			\frac{d\sigma_{\text{воды}}}{dT}=\frac{\langle\sigma_{\text{воды}} T\rangle
			-\langle\sigma_{\text{воды}}\rangle\langle T\rangle}{\langle T^{2}\rangle-\langle T\rangle^{2}},
			\ \ \ \ \sigma_{\text{воды}}(0)=\langle\sigma_{\text{воды}}\rangle-\frac{d\sigma_{\text{воды}}}{dT}\langle T\rangle
		\end{equation}
		\noindent где $\frac{d\sigma_{\text{воды}}}{dT}$ --- угловой коэффициент, а $\sigma_{\text{воды}}(0)$ --- свободный член.
		Погрешность определения $\frac{d\sigma_{\text{воды}}}{dT}$ оценим по формуле:
		\begin{equation}
			\sigma_{\frac{d\sigma_{\text{воды}}}{dT}}=\frac{1}{\sqrt{N}}
			\sqrt{\frac{\langle\sigma_{\text{воды}}\rangle-\langle\sigma_{\text{воды}}\rangle^{2}}
						{\langle T^{2}\rangle-\langle T\rangle^{2}}-\left(\frac{d\sigma_{\text{воды}}}{dT}\right)^2}.
		\end{equation}
		\noindent В итоге получим следущее значение: 
		$\frac{d\sigma_{\text{воды}}}{dT}=(-15.5\pm0.9)\cdot10^{-5}\text{ Н/(м}\cdot\text{К)}$.
		\subsection{Удельная поверхностная энергия и теплота образования единицы площади поверхности}
		\par Построим график зависимости теплоты образования еденицы поверхности жидкости $q$ от $T$ и график зависимости
		поверхностной энергии на еденицу площади $\frac{U}{F}$ от $T$. Для этого воспользуемся формулами:
		\begin{equation}
			q=-T\cdot\frac{d\sigma_{\text{воды}}}{dT}, 
			\ \ \ \ 
			\frac{U}{F}=\left(\sigma-T\cdot\frac{d\sigma_{\text{воды}}}{dT}\right)=\sigma_{\text{воды}}(0).
		\end{equation}
		\par Для построения гарфиков запишем значения в таблицу 5:
		\begin{table}[H]
			\captionsetup{font={small}, labelformat=fullparents, labelsep=fill, labelfont=bf, justification=raggedleft,
							singlelinecheck=false, skip=-0.2cm}
			\caption{Значения $q$ и $\frac{U}{F}$ при разных температурах}
			\begin{center}
	  			\begin{tabular}{|>{\centering}p{2.4cm}|>{\centering}p{1.4cm}|>{\centering}p{1.4cm}|>{\centering}p{1.4cm}
	  							|>{\centering}p{1.4cm}|>{\centering}p{1.4cm}|>{\centering}p{1.4cm}|>{\centering}p{1.4cm}
	  							|>{\centering}p{1.4cm}|}
				      \hline
				      	\multicolumn{1}{|p{2.6cm}|}{T, ℃} & \multicolumn{1}{p{1.3cm}|}{20.0} & \multicolumn{1}{p{1.3cm}|}{25.0} 
					& \multicolumn{1}{p{1.3cm}|}{30.0} & \multicolumn{1}{p{1.3cm}|}{35.2} & \multicolumn{1}{p{1.3cm}|}{40.2} 
					& \multicolumn{1}{p{1.3cm}|}{45.1} & \multicolumn{1}{p{1.3cm}|}{50.1} & \multicolumn{1}{p{1.3cm}|}{55.1} \\
					\hline
					\multicolumn{1}{|p{2.6cm}|}{$q$, $\text{Н/м}\cdot10^{-3}$} 
					& \multicolumn{1}{p{1.1cm}|}{3.1} & \multicolumn{1}{p{1.1cm}|}{3.9} 
					& \multicolumn{1}{p{1.1cm}|}{4.6} & \multicolumn{1}{p{1.1cm}|}{5.4} & \multicolumn{1}{p{1.1cm}|}{6.2} 
					& \multicolumn{1}{p{1.1cm}|}{7.0} & \multicolumn{1}{p{1.1cm}|}{7.8} & \multicolumn{1}{p{1.1cm}|}{8.5} \\
					\hline
					\multicolumn{1}{|p{2.7cm}|}{$\sigma_{q}$, $\text{Н/м}\cdot10^{-3}$} 
					&\multicolumn{1}{p{1.1cm}|}{0.2} &\multicolumn{1}{p{1.1cm}|}{0.2} 
					& \multicolumn{1}{p{1.1cm}|}{0.3} & \multicolumn{1}{p{1.1cm}|}{0.3} & \multicolumn{1}{p{1.1cm}|}{0.4} 
					& \multicolumn{1}{p{1.1cm}|}{0.4} & \multicolumn{1}{p{1.1cm}|}{0.4} & \multicolumn{1}{p{1.1cm}|}{0.5} \\
					\hline
					\multicolumn{1}{|p{2.7cm}|}{$\frac{U}{F}$, $\text{Н/м}$} 
					& \multicolumn{1}{p{1.1cm}|}{0.058} & \multicolumn{1}{p{1.1cm}|}{0.058} 
					& \multicolumn{1}{p{1.1cm}|}{0.058} & \multicolumn{1}{p{1.1cm}|}{0.057} & \multicolumn{1}{p{1.1cm}|}{0.057} 
					& \multicolumn{1}{p{1.1cm}|}{0.058} & \multicolumn{1}{p{1.1cm}|}{0.058} & \multicolumn{1}{p{1.1cm}|}{0.058} \\
					\hline
					\multicolumn{1}{|p{2.2cm}|}{$\sigma_{\frac{U}{F}}$, Н/м} 
					& \multicolumn{1}{p{1.1cm}|}{0.002} & \multicolumn{1}{p{1.1cm}|}{0.002} 
					& \multicolumn{1}{p{1.1cm}|}{0.002} & \multicolumn{1}{p{1.1cm}|}{0.002} & \multicolumn{1}{p{1.1cm}|}{0.002} 
					& \multicolumn{1}{p{1.1cm}|}{0.002} & \multicolumn{1}{p{1.1cm}|}{0.002} & \multicolumn{1}{p{1.1cm}|}{0.002} \\
					\hline
				\end{tabular}
			\end{center}
		\end{table}
		\begin{figure}[H]
			\begin{minipage}[H]{0.49\linewidth}
				\center\input{C:/Users/gromo/PycharmProjects/plottest/plot7.tex}
				\captionsetup{font={small}, labelformat=empty}
				\center\caption{\ \ \ \ \ \ \ \ \ График зависимости $q$ от $T$}
			\end{minipage}
			\hfill
			\begin{minipage}[H]{0.49\linewidth}
				\center\input{C:/Users/gromo/PycharmProjects/plottest/plot8.tex}
				\captionsetup{font={small}, labelformat=empty}
				\center\caption{\ \ \ \ \ \ \ \ \ График зависимости $\frac{U}{F}$ от $T$}
			\end{minipage}
		\end{figure}
	\section{Обсуждение результатов}
		\par В данной работе мы определяли коэффициент поверхностного натяжения воды. Согласно справочнику ни одно из значений
		не согласуется с экспериментальными. Так погрешность измерения оказалась порядка 5\%, то результаты нельзя списать на
		плохие приборы. Одной из причин столь неудовлетворительных результатов, возможно, является попадание в иследуемую
		жидкость остатков спирта. Так же последние опыты были плохо проведены из-за нехватки времени, однако они не могли
		сильно повлиять на конечный результат. Зависимость $\sigma(T)$ хоть и получилась линейной в пределах погрешностей,
		вызывает к себе много вопросов из-за маленького диапазона и неточных измерений.
		\par Также в данной работе определялись удельная поверхностная энергия и теплота образования еденицы площади
		поверхности. О второй величине никаких справочных материалов обнаружено не было. О качестве определения удельной
		поверхностной энергии можно судить из общих представлений: так как это энергия взаимодействия молекул поверхностного слоя
		с остальными, то она не должна сильно меняться с температурой из-за слабого расширения воды. В нашей это предположение
		наблюдается.
	\section{Вывод}
		\par Данную работу можно охарактеризовать как неудачу. Последние измерения были выполнены довольно небрежно, а
		качественные измерения дают не ожидаемый изначально результат. Однако небольшие погрешности измерений указывают на
		то, что при более грамотном и аккуратном подходе к используемому оборудованию мы можем получить лучшие результаты.
		
\end{document}
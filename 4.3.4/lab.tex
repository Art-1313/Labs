\documentclass[12pt,a4paper]{article}
\usepackage[T2A]{fontenc}
\usepackage[utf8]{inputenc}
\usepackage[english, russian]{babel}
\usepackage{indentfirst}
\usepackage{misccorr}
\usepackage{graphicx}
\usepackage{amsmath}
\usepackage{amssymb}
\usepackage{circuitikz}
\usepackage[font={small}]{caption}
\usepackage[left=20mm, top=20mm, right=20mm, bottom=20mm, nohead]{geometry}
\usepackage{float}
\usepackage{tabularx}
\usepackage{array}
\usepackage{longtable}
\usepackage{pstool}
\usepackage{pgfplots}
\usepackage{hhline}
\usepackage{multirow}
\usepackage{wrapfig}
\usepackage{pdfpages}

\DeclareCaptionLabelSeparator{fill}{.\\}

\DeclareCaptionLabelFormat{fullparents}{\bothIfFirst{#1}{~}#2}

\pgfplotsset{compat=1.17}

\begin{document}

    \begin{titlepage}
        \begin{center}
            {\LARGE Отчёт по лабораторной работе 4.3.4.\\}
            \vspace*{11cm}
                \textbf{\LARGE Метод преобразования Фурье в оптике.}
            \vspace*{6.5cm}
        \end{center}
        \hfill\begin{minipage}{0.37\textwidth}
            Работу выполнил Громов Артём
            \\
            ЛФИ Б02-006
        \end{minipage}
        \vspace{4.8cm}
        \begin{center}
            Долгопрудный, 2022 г.
        \end{center}
    \end{titlepage}

    \section{Аннотация}

        \begin{flushleft}
            \textbf{Цель работы:} исследование особенностей применения пространственного преобразования Фурье для анализа дифракционных явлений.
        \end{flushleft}
        \begin{flushleft}
            \textbf{В работе используется:} гелий-неоновый лазер, кассета с набором сеток разного периода, щель с микрометрическим винтом, линзы, экран,
            линейка.
        \end{flushleft}

        \par Анализ сложного волнового поля во многих случаях целесообразно проводить, разлагая его на простеийшие составляющие, например, представляя
        его в виде разложения по плоским волнам. При этом оказывается, что если мы рассматриваем поле, полученное после прохождения плоской
        монохроматической волны через предмет или транспарант (изображение предмета на фотоплёнке или стеклянной пластинке) с функцией пропускания
        $t(x)$, то разложение по плоским волнам соответствует преобразованию Фурье от этой функции. Если за предметом поставить линзу, то каждая
        плоская волна сфокусируется в свою точку в задней фокальной плоскости линзы. Таким образом, картина, наблюдаемая в фокальной плоскости
        линзы, даёт нам представление о спектре плоских волн падающего на линзу волнового поля. Поэтому можно утверждать, что с помощью линзы в
        оптике осуществляется пространственное преобразование Фурье.
        
        \subsection{Спектр функции пропускания амплитудной синусоидальной решётки}
        \par Рассмотрим вначале простой пример: дифракцию плоской монохроматической волны на синусоидальной амплитудной решётке. Пусть решётка с
        периодом $d$ расположена в плоскости $Z=0$, а её штрихи ориентированы вдоль оси $Y$ . Функция пропускания такой решётки имеет вид
        \begin{equation}
            t(x)=\beta+\alpha\cos{(ux)}=\beta+\alpha\frac{e^{iux}+e^{-iux}}{2}
        \end{equation}
        с постоянными $\alpha$, $\beta$ и $u$ ($u=2\pi/d$ --- пространственная частота).
        \par Если на решётку падает плоская монохроматическая волна, распространяющаяся вдоль оси $Z$,
        \begin{equation}
            E(\vec{r},t)=E_0e^{-i(\omega t-kz)},
        \end{equation}
        где $\omega$ --- круговая частота, $k$ --- волновой вектор ($k=2\pi/\lambda$), $E_0$ --- амплитуда, то на выходе из решётки мы получим три
        плоских волны:
        \begin{equation}
            \begin{gathered}
                E_1=\beta E_0e^{-i(\omega t-kz)};
                \\
                E_2=\frac{\alpha}{2}E_0e^{-i(\omega t-ux-z\sqrt{k^2-u^2})};
                \\
                E_3=\frac{\alpha}{2}E_0e^{-i(\omega t+ux-z\sqrt{k^2-u^2})}.
            \end{gathered}
        \end{equation}
        \par Действительно, легко видеть, что в плоскости $Z=0$ амплитуда колебаний, создаваемая суммой этих волн, описывается функцией (1), а фаза
        колебаний постоянна. Таким образом, в силу единственности решения волнового уравнения при заданных граничных условиях мы нашли искомую
        суперпозицию плоских волн. Каждая из этих трёх плоских волн фокусируется линзой в точку в задней фокальной плоскости.
        \par Волна $E_1=\beta E_0e^{-i(\omega t-kz)}$, распространяющаяся вдоль оси линзы (оси $Z$), фокусируется в начало координат, а волны $E_2$ и
        $E_3$, распространяющиеся в направлении $\sin{\theta}=\pm(u/k)$, фокусируются в точках $x_{1,2}=\pm Fu/k=\pm F\lambda/d$ ($F$ --- фокусное
        расстояние линзы).
        \par Функция $t(x)$ с самого начала задана в виде суммы гармонических составляющих, т. е. в виде ряда Фурье. Каждой гармонической составляющей
        мы поставили в соответствие с (3) плоскую волну, собираемую линзой в точку в задней фокальной плоскости (её обычно называют фурье-плоскостью).
        Проводя аналогию с «временной» координатой, мы можем заключить, что спектр функции $t(x)$ представлен в фурье-плоскости тремя пространственными
        частотами: 0, $+u$, $-u$; с амплитудами соответственно: $\beta$, $\alpha/2$, $\alpha/2$.
        \par Теорема Фурье, доказываемая в курсе математического анализа, утверждает, что широкий класс периодических функций $t(x)$ может быть
        представлен в виде суммы бесконечного множества гармонических составляющих, имеющих кратные частоты, т. е. в виде ряда Фурье. В комплексной
        форме этот ряд имеет вид
        \begin{equation}
            t(x)=\sum^{\infty}_{n=-\infty}{C_ne^{inux}}.
        \end{equation}
        \par Рассуждая так же, как в случае амплитудной синусоидальной решётки, мы придём к выводу, что картина, наблюдаемая в фурье-плоскости,
        представляет собой эквидистантный набор точек с координатами
        $$
            x_n=\frac{Fu}{k}n=\frac{f\lambda}{d}n
        $$
        и амплитудами, пропорциональными $C_n$. Таким образом, с помощью линзы в оптике осуществляется пространственное преобразование Фурье: при
        освещении транспаранта плоской монохроматической волной картина, наблюдаемая в задней фокальной плоскости линзы, установленной за
        транспарантом, представляет собой фурье-образ функции пропускания транспаранта.
        \par Последнее утверждение нуждается в уточнении. Распределение света в задней фокальной плоскости линзы будет воспроизводить распределение
        амплитуд плоских волн, продифрагировавших на транспаранте, но фазовые соотношения при этом, вообще говоря, оказываются искажёнными и не
        соответствуют аргументам комплексных амплитуд в выражении (4). При изменении расстояния между транспарантом и линзой фазовые соотношения
        изменяются. Можно доказать, что если транспарант установлен в передней фокальной плоскости линзы, то в её задней фокальной плоскости
        восстанавливаются и амплитудные, и фазовые соотношения между плоскими волнами, и таким образом строго осуществляется комплексное
        фурье-преобразование (4).
        \par Во многих практически важных случаях функция пропускания транспаранта чисто амплитудная, как, например, в случае амплитудной
        синусоидальной решётки (1). Тогда для того, чтобы найти фурье-образ функции пропускания транспаранта, достаточно определить только
        пространственные частоты и соотношение между амплитудами плоских волн на выходе из транспаранта. Для амплитудной синусоидальной решётки мы
        получили три плоских волны с пространственными частотами 0, $+u$, $-u$ и амплитудами, пропорциональными $\beta$, $\alpha/2$, $\alpha/2$. В
        соответствии с (1) мы можем утверждать, что нашли пространственный фурье-образ функции пропускания амплитудной синусоидальной решётки.
        \par Интересно заметить, что наблюдаемая визуально картина фраунгоферовой дифракции в задней фокальной плоскости линзы не зависит от
        расстояния между транспарантом и линзой, так как глаз не реагирует на фазу волны, а регистрирует только интенсивность (усреднённый по времени
        квадрат амплитуды поля). Условия наблюдения дифракции Фра- унгофера можно выполнить и без применения линзы, если наблюдать дифракционную
        картину на достаточно удалённом экране. Таким образом, пространственное преобразование Фурье может осуществляться и в свободном пространстве
        при наблюдении дифракции Фраунгофера.
        
        \subsection{Спектр функции пропускания щелевой диафрагмы и периодической последовательности таких функций}
        \par Картина дифракции Фраунгофера на щели и на дифракционной решётке, имеющей вид периодического набора щелей, хорошо известна из курса
        оптики. Спектр дифракционной решётки представлен на рис. 1. Если размеры дифракционной решётки неограничены, то дифракционные максимумы в
        спектре бесконечно узки. Чем меньше размер решётки (полное число щелей), тем шире каждый отдельный максимум.
        \par Направление на главные максимумы $\theta_n=un/k=\lambda n/d$ ($n$ --- целое число) определяется периодом решётки $d$, а распределение
        амплитуд в спектре (огибающая) --- фурье-образом функции пропускания отдельного штриха:
        \begin{equation}
            g_2(x) = 
            \begin{cases}
                1 &\text{при $-D/2\le x\le D/2$};\\
                0 &\text{при $\-D/2> x> D/2$}.
            \end{cases}
        \end{equation}
        \par Так как функция $g_2$ непериодична, её фурье-образ представляется непрерывным множеством точек и определяется интегральным
        преобразованием Фурье:
        \begin{equation}
            \begin{gathered}
                g(x)=\frac{1}{2\pi}\int^{\infty}_{\infty}G(u)e^{iux}du,
                \\
                G(u)=\frac{1}{2\pi}\int^{\infty}_{\infty}g(x)e^{-iux}dx,
            \end{gathered}
        \end{equation}
        \par Говорят, что в таком виде $g(x)$ и $G(u)$ представляют собой пару преобразований Фурье: $G(u)$ --- спектр или фурье-образ функции $g(x)$.
        \begin{figure}[H]
            \begin{center}
                \includegraphics[width=0.9\textwidth]{images/pic1.pdf}
                \caption{а) $g_1(x)$ --- функция пропускания дифракционной решётки (последовательности прозрачных и непрозрачных полос); б) $G_1(u)$
                --- спектр функции пропускания дифракционной решётки}
            \end{center}
        \end{figure}
        \par Спектр функции $g_2(x)$ хорошо известен, он соответствует картине дифракции Фраунгофера на щели и описывается функцией вида $\sin{x}$ 
        (рис. 2).
        \begin{figure}[H]
            \begin{center}
                \includegraphics[width=0.9\textwidth]{images/pic2.pdf}
                \caption{а) g2(x) — функция пропускания щелевой диафрагмы; б) G2(u) — спектр функции пропускания щелевой диафрагмы}
            \end{center}
        \end{figure}
        \par Получим спектр $G_2(u)$ ещё раз с помощью преобразования Фурье:
        $$
            G_2(u)=\int^{\infty}_{\infty}g_2(x)e^{-iux}dx=\int^{D/2}_{-D/2}e^{-iux}=D\frac{\sin{uD/2}}{uD/2}.
        $$
        \par Отсюда видно, что направление на первый минимум $\theta_1$ в огибающей спектра пропускания дифракционной решётки определяется шириной
        функции пропускания отдельного штриха: $\theta_1=u/k=\lambda/D$. Если ввести понятия протяжённости функции пропускания транспаранта по
        координате ($\Delta x$) и ширины её спектра ($\Delta u$), то
        \begin{equation}
            \Delta u \cdot \Delta x = \text{const}.
        \end{equation}
        \par Для частного случая функции пропускания щелевой диафрагмы, определяя ширину её спектра по первому нулю функции $\frac{\sin{uD/2}}{uD/2}$,
        получаем
        $$
            \Delta u \cdot \Delta x = \frac{2\pi}{D}\cdot D=2\pi.
        $$
        Соотношение (7) в волновой физике играет чрезвычайно важную роль. Его называют соотношением неопределённости.
        \par Измерив на удалённом экране расстояния между максимумами или минимумами в спектре пропускания щели (рис. 2б) или решётки (рис. 1б),
        можно рассчитать размер щели или период решётки.
        \par Размер малого объекта можно рассчитать, если получить его изображение, увеличенное с помощью линзы.

        \subsection{Метод Аббе}
        \begin{figure}[H]
            \begin{center}
                \includegraphics[width=0.9\textwidth]{images/pic3.pdf}
                \caption{Схема, поясняющая метод Аббе построения изображения}
            \end{center}
        \end{figure}
        \par Рассмотрим кратко схему образования изображения (рис. 3). Пусть предмет расположен в плоскости $P_1$ на расстоянии от линзы большем,
        чем фокусное. Тогда существует сопряжённая предметной плоскости $P_1$ плоскость $P_2$, где образуется изображение предмета-щели.
        \par Аббе предложил рассматривать схему прохождения лучей от предмета к изображению в два этапа. Сначала рассматривается изображение-спектр
        в задней фокальной плоскости Ф линзы Л$_1$ (это изображение Аббе назвал первичным).

        \begin{wrapfigure}{r}{0.55\textwidth}
            \centering
            \includegraphics[width=0.55\textwidth]{images/pic4.pdf}
            \caption{а) $G_2(x)$ --- спектр функции пропускания щелевой диафрагмы; $x$ --- координаты в задней фокальной плоскости линзы;
            б) Ф$(x)$ --- функция пропускания решётки, установленной в фурье-плоскости линзы; в) $G_1(x)$ --- отфильтрованный спектр щелевой диафрагмы
            (ср. с рис. 1)}
        \end{wrapfigure}
        \par Затем это изображение рассматривается как источник волн, создающий изображение предмета в плоскости $P_2$ (вторичное изображение ).
        Такой подход опирается на принцип Гюйгенса–Френеля, согласно которому любой участок волнового фронта можно рассматривать как источник
        излучения.
        \par Картина, наблюдаемая в плоскости $P_2$, зависит от распределения амплитуды и фазы в плоскости Ф — в первичном изображении. Если плоскость
        $P_2$ сопряжена с предметной плоскостью $P$ , то фазовые соотношения в первичном изображении оказываются именно такими, что в плоскости $P_2$
        мы наблюдаем соответственно увеличенное или уменьшенное изображение предмета. Поэтому иногда говорят, что линза дважды осуществляет
        преобразование Фурье: сначала в задней фокальной плоскости Ф линзы получается световое поле, соответствующее фурье-образу функции пропускания
        предмета (с точностью до фазы), а затем на промежутке между фокальной плоскостью Ф и плоскостью изображений $P_2$ осуществляется обратное
        преобразование Фурье, и в плоскости $P_2$ восстанавливается таким образом изображение предмета.

        \subsection{Мультипликация изображения предмета}
        \par Рассмотрим, что произойдёт с изображением предмета, если мы установим в задней фокальной плоскости линзы решётку. Сопоставим вначале
        спектры щелевой диафрагмы (рис. 2) и периодической последовательности щелевых диафрагм (рис. 1).
        \par Легко видеть, что спектр, изображённый на рис. 1, можно получить из спектра, изображённого на рис. 2, если исключить из него часть
        пространственных частот, поместив в фурье-плоскость решётку --- последовательность прозрачных и непрозрачных линий (рис. 4).
        \par Отфильтрованный таким образом спектр не будет отличаться ни по амплитуде, ни по фазе от спектра периодической последовательности щелевых
        диафрагм, и в плоскости $P_2$ мы получим вместо изображения одиночной щели изображение периодической последовательности щелей.
        \par Эти рассуждения можно повторить и для предмета с произвольным спектром, необходимо только, чтобы период решётки был заметно меньше
        ширины спектра (точное соотношение можно получить из теоремы Котельникова). Таким образом, установив в задней фокальной плоскости линзы
        решётку, мы вместо изображения одиночного предмета получим эквидистантный набор изображений таких предметов, т. е. осуществим мультипликацию
        изображения предмета (увидим изображение несуществующей «фиктивной» решётки).
        \par Поменяв местами сетку и щель, можно проследить влияние размера щели на изображение сетки.

    \section{Экспериментальная установка}

        \par Схема установки представлена на рис. 5. Щель переменной ширины $D$, снабжённая микрометрическим винтом В, освещается параллельным пучком
        света, излучаемым He-Ne лазером (радиус кривизны фронта волны велик по сравнению с фокусными расстояниями используемых в схеме линз).
        \par Увеличенное изображение щели с помощью линзы Л$_1$ проецирется на экран Э. Величина изображения $D_1$ зависит от расстояний от линзы до
        предмета --- $a_1$ и до изображения --- $b_1$, т. е. от увеличения Г системы:
        \begin{equation}
            \text{Г}=\frac{D_1}{D}=\frac{b_1}{a_1}.
        \end{equation}
        Изображение спектра щели образуется в задней фокальной плоскости Ф линзы Л$_1$. Размещая в плоскости Ф двумерные решётки-сетки, можно влиять
        на первичное изображение и получать мультиплицированное изображение щели.
        \par Убрав линзу, можно наблюдать на экране спектр щели (рис. 6), а если заменить щель решёткой — спектр решётки. Крупные решётки дают на
        экране очень мелкую картину спектра, которую трудно промерить. В этом случае используют две линзы (рис. 7): первая (длиннофокусная) формирует
        первичное изображение --- спектр, вторая (короткофо-кусная) --- проецирует на экран увеличенное изображение спектра.

    \section{Результаты измерений и обработка данных}

        \subsection{Определение ширины щели с помощью линзы}
        \begin{figure}[H]
            \begin{center}
                \includegraphics[width=0.8\textwidth]{images/pic5.pdf}
                \caption{Схема, поясняющая метод Аббе построения изображения}
            \end{center}
        \end{figure}
        \par С помощью короткофокусной линзы Л$_1$ ($F_1=38$ мм) получим на экране Э увеличенное изображение щели. Меняя ширину щели от 50 до 500 мкм,
        снимим зависимость размера изображения $D_1$ от ширины щели $D$. Занесём данные в таблицу 1.
        \par Измерим расстояния $a_1=5\pm1$ см и $b_1=120\pm1$ см для определения увеличения Г системы. Погрешность этих измерений велика, поэтому
        дополнительно измерим $L=a_1+b_1=125\pm1$ см и, зная $F_1$, вычислим $a_1=LF_1/b_1=3.96 \pm 0.05$ см. Полученное косвенно значение лучше
        согласуется с ожидаемым ($a_1\simeq F_1$), поэтому будем использовать его. Зная увеличение линзы и размер изображения, вычислим размер щели по
        формуле (8) и занесём результаты в таблицу 1.
        \begin{table}[H]
            \captionsetup{font={small}, labelformat=fullparents, labelsep=fill, labelfont=bf, justification=raggedleft,
                        singlelinecheck=false, skip=-0.2cm}
            \caption{Ширина щели и изображения}
            \begin{center}
                \begin{tabular}{|l|l|}\hline$m$&$z$, дел
\\\hline0.0&275.0
\\\hline1.0&311.0
\\\hline2.0&348.0
\\\hline3.0&381.0
\\\hline4.0&419.0
\\\hline5.0&453.0
\\\hline6.0&489.0
\\\hline7.0&529.0
\\\hline-1.0&250.0
\\\hline-2.0&214.0
\\\hline-3.0&178.0
\\\hline-4.0&148.0
\\\hline-5.0&109.0
\\\hline-6.0&73.0
\\\hline-7.0&35.0
\\\hline\end{tabular}
            \end{center}
        \end{table}

        \subsection{Определение ширины щели по её спектру}
        \begin{figure}[H]
            \begin{center}
                \includegraphics[width=0.8\textwidth]{images/pic6.pdf}
                \caption{Схема, поясняющая метод Аббе построения изображения}
            \end{center}
        \end{figure}
        \par Получим на удалённом экране спектр щели (рис. 6). Измерим ширину спектра для самой маленькой щели. Для большей точности будем измерять
        расстояние $X$ между наиболее удалёнными минимумами. Измерения проведём для нескольких значений $D$ и занесём данные в таблицу 1.
        \par По результатм измерения спектра рассчитаем ширину щели $D_{\text{с}}$ используя соотношения
        \begin{equation}
            \Delta X=\frac{X}{m}=\frac{\lambda}{D_{\text{с}}}L.
        \end{equation}
        Длина волны He-Ne лазера $\lambda=5320$ \AA. Полученные данные занесём в таблицу 2.
        \begin{table}[H]
            \captionsetup{font={small}, labelformat=fullparents, labelsep=fill, labelfont=bf, justification=raggedleft,
                        singlelinecheck=false, skip=-0.2cm}
            \caption{Ширина щели и изображения}
            \begin{center}
                \begin{tabular}{|l|l|l|l|l|l|l|l|l|}\hline$x$, дел&8&25&36&52&-8&-24&-36&-53
\\\hline$m $&1&2&3&4&-1&-2&-3&-4
\\\hline\end{tabular}
            \end{center}
        \end{table}
        \par Построим на одном графике зависимости $D_{\text{л}}=f(D)$ и $D_{\text{и}}=f(D)$.
        \begin{figure}[H]
            \begin{center}
                % This file was created with tikzplotlib v0.9.16.
\begin{tikzpicture}

\begin{axis}[
legend cell align={left},
legend style={
  fill opacity=0.8,
  draw opacity=1,
  text opacity=1,
  at={(0.03,0.97)},
  anchor=north west,
  draw=white!80!black
},
height=11.2cm,
tick align=inside,
major tick length=0.2cm,
minor tick length=0.1cm,
tick pos=left,
xmin=0, xmax=600,
xtick={0, 50, 100, 150, 200, 250, 300, 350, 400, 450, 500, 550, 600},
minor x tick num=4,
xmajorgrids,
minor x grid style={dotted,black},
xminorgrids,
xtick style={color=black},
xlabel={$D_{\text{винта}}$, мкм},
ymin=0, ymax=500,
ytick={0, 50, 100, 150, 200, 250, 300, 350, 400, 450, 500},
ytick style={color=black},
minor y tick num=4,
ymajorgrids,
minor y grid style={dotted,black},
yminorgrids,
ytick style={color=black},
ylabel={$D_{\text{расчёт}}$, мкм}
]
\path [draw=red, semithick]
(axis cs:40,32.9861111111111)
--(axis cs:60,32.9861111111111);

\path [draw=red, semithick]
(axis cs:90,98.9583333333333)
--(axis cs:110,98.9583333333333);

\path [draw=red, semithick]
(axis cs:140,131.944444444444)
--(axis cs:160,131.944444444444);

\path [draw=red, semithick]
(axis cs:190,164.930555555556)
--(axis cs:210,164.930555555556);

\path [draw=red, semithick]
(axis cs:240,197.916666666667)
--(axis cs:260,197.916666666667);

\path [draw=red, semithick]
(axis cs:290,230.902777777778)
--(axis cs:310,230.902777777778);

\path [draw=red, semithick]
(axis cs:340,263.888888888889)
--(axis cs:360,263.888888888889);

\path [draw=red, semithick]
(axis cs:390,296.875)
--(axis cs:410,296.875);

\path [draw=red, semithick]
(axis cs:440,362.847222222222)
--(axis cs:460,362.847222222222);

\path [draw=red, semithick]
(axis cs:490,395.833333333333)
--(axis cs:510,395.833333333333);

\path [draw=red, semithick]
(axis cs:50,-0.00334608800363867)
--(axis cs:50,65.9755683102259);

\path [draw=red, semithick]
(axis cs:100,65.9421196383399)
--(axis cs:100,131.974547028327);

\path [draw=red, semithick]
(axis cs:150,98.9048365902785)
--(axis cs:150,164.98405229861);

\path [draw=red, semithick]
(axis cs:200,131.860893814377)
--(axis cs:200,198.000217296734);

\path [draw=red, semithick]
(axis cs:250,164.810309447695)
--(axis cs:250,231.023023885639);

\path [draw=red, semithick]
(axis cs:300,197.75310554622)
--(axis cs:300,264.052450009336);

\path [draw=red, semithick]
(axis cs:350,230.689308019177)
--(axis cs:350,297.088469758601);

\path [draw=red, semithick]
(axis cs:400,263.618946552474)
--(axis cs:400,330.131053447526);

\path [draw=red, semithick]
(axis cs:450,329.458668896105)
--(axis cs:450,396.235775548339);

\path [draw=red, semithick]
(axis cs:500,362.368830131627)
--(axis cs:500,429.29783653504);

\addplot [semithick, red, forget plot]
table {%
0 0
500 392.40620488787
};
\path [draw=green!50.1960784313725!black, semithick]
(axis cs:90,53.0442477876106)
--(axis cs:110,53.0442477876106);

\path [draw=green!50.1960784313725!black, semithick]
(axis cs:140,95.1428571428572)
--(axis cs:160,95.1428571428572);

\path [draw=green!50.1960784313725!black, semithick]
(axis cs:190,148)
--(axis cs:210,148);

\path [draw=green!50.1960784313725!black, semithick]
(axis cs:240,188.913461538462)
--(axis cs:260,188.913461538462);

\path [draw=green!50.1960784313725!black, semithick]
(axis cs:290,233.740384615385)
--(axis cs:310,233.740384615385);

\path [draw=green!50.1960784313725!black, semithick]
(axis cs:340,270.260869565217)
--(axis cs:360,270.260869565217);

\path [draw=green!50.1960784313725!black, semithick]
(axis cs:390,326.204081632653)
--(axis cs:410,326.204081632653);

\path [draw=green!50.1960784313725!black, semithick]
(axis cs:440,360)
--(axis cs:460,360);

\path [draw=green!50.1960784313725!black, semithick]
(axis cs:490,396)
--(axis cs:510,396);

\path [draw=green!50.1960784313725!black, semithick]
(axis cs:540,468)
--(axis cs:560,468);

\path [draw=green!50.1960784313725!black, semithick]
(axis cs:100,52.4114526583863)
--(axis cs:100,53.6770429168349);

\path [draw=green!50.1960784313725!black, semithick]
(axis cs:150,91.0666302830339)
--(axis cs:150,99.2190840026804);

\path [draw=green!50.1960784313725!black, semithick]
(axis cs:200,141.683425571105)
--(axis cs:200,154.316574428895);

\path [draw=green!50.1960784313725!black, semithick]
(axis cs:250,182.271308845487)
--(axis cs:250,195.555614231436);

\path [draw=green!50.1960784313725!black, semithick]
(axis cs:300,223.88999208276)
--(axis cs:300,243.590777148009);

\path [draw=green!50.1960784313725!black, semithick]
(axis cs:350,257.145804132173)
--(axis cs:350,283.375934998262);

\path [draw=green!50.1960784313725!black, semithick]
(axis cs:400,311.969339856676)
--(axis cs:400,340.438823408631);

\path [draw=green!50.1960784313725!black, semithick]
(axis cs:450,341.133063042629)
--(axis cs:450,378.866936957371);

\path [draw=green!50.1960784313725!black, semithick]
(axis cs:500,376.956019710008)
--(axis cs:500,415.043980289992);

\path [draw=green!50.1960784313725!black, semithick]
(axis cs:550,448.557402072788)
--(axis cs:550,487.442597927212);

\addplot [semithick, green!50.1960784313725!black, forget plot]
table {%
0 0
550 439.564038903656
};
\addplot [semithick, red, mark=*, mark size=3, mark options={solid}, only marks]
table {%
50 32.9861111111111
100 98.9583333333333
150 131.944444444444
200 164.930555555556
250 197.916666666667
300 230.902777777778
350 263.888888888889
400 296.875
450 362.847222222222
500 395.833333333333
};
\addlegendentry{Линза}
\addplot [semithick, green!50.1960784313725!black, mark=*, mark size=3, mark options={solid}, only marks]
table {%
100 53.0442477876106
150 95.1428571428572
200 148
250 188.913461538462
300 233.740384615385
350 270.260869565217
400 326.204081632653
450 360
500 396
550 468
};
\addlegendentry{Спектр}
\end{axis}

\end{tikzpicture}

                \caption{Графики зависимости рассчитаных диаметров от показаний микрометрического винта}
            \end{center}
        \end{figure}

        \subsection{Определение периода решётки по спектру на удалённом экране}
        \par Поствим кассету с двумерными решётками вплотную к выходному окну лазера. Вращением наружнего кольца будем менять сетки. Для каждой
        сетки измерим расстояние $x$ между соседними максимумами и отметим $m$ --- порядок максимума. Измерим расстояние $L=136$ см от кассеты до
        экрана.
        \par Рассчитаем расстояние $\Delta X$ между соседними максимумаи и определим период каждой решётки $d_{\text{с}}=f(\text{№})$, используя
        соотношения
        \begin{equation}
            \Delta X=\frac{X}{2m}=\frac{\lambda}{d_{\text{с}}}L.
        \end{equation}
        Данные занесём в таблицу 3.
        
        \subsection{Определение периода решёток по увеличенному изображению спектра}
        \par Линзу Л$_2$ с максимальным фокусом $F_2=110$ мм поставим на расстоянии $\simeq F_2$. В плоскости Ф линза Л$_2$ даёт фурье-образ сетки ---
        её спектр, а короткофокусная линза Л$_3$ ($F_3=25$ мм) создаёт на экране увеличеннное изображение этого спектра.
        \par Так как экран достаточно удалён ($b_3\gg a_3$), то практически $a_3=F_3$, и расстояние между линзами $\simeq F_2+F_3$.
        \par Измерим $X$ --- растояние между наиболее удалёнными минимумаи и их количество $m$ для разных сеток. Зная увеличение линзы Л$_3$ можно
        рассчитать расстояние между максимумами $\Delta x$ в плоскости Ф, а затем период сетки $d_{\text{л}}$:
        \begin{equation}
            \Delta x=\frac{\Delta X}{\Gamma_3}=\frac{\lambda}{d_{\text{л}}}F_2.
        \end{equation}
        Данные занесём в таблицу 4.
        \begin{figure}[H]
            \begin{center}
                \includegraphics[width=0.8\textwidth]{images/pic7.pdf}
                \caption{Схема, поясняющая метод Аббе построения изображения}
            \end{center}
        \end{figure}
        \begin{figure}[H]
            \begin{minipage}{0.49\textwidth}
                \begin{table}[H]
                    \captionsetup{font={small}, labelformat=fullparents, labelsep=fill, labelfont=bf, justification=raggedright,
                                singlelinecheck=false, skip=-0.2cm}
                    \caption{Ширина щели и изображения}
                    \begin{center}
                        \begin{tabular}{|l|l|l|l|l|}\hline№&$X $, мм&$m $&$d_{\text{с}}$, мкм&$\sigma_{d_\text{с}}$, мкм
\\\hline1.0&218.0&3.0&19.9&0.2
\\\hline2.0&267.0&5.0&27.1&0.2
\\\hline3.0&241.0&10.0&60.0&0.5
\\\hline4.0&177.0&15.0&123.0&1.0
\\\hline5.0&92.0&10.0&157.0&2.0
\\\hline\end{tabular}
                    \end{center}
                \end{table}
            \end{minipage}
            \hfill
            \begin{minipage}{0.49\textwidth}
                \begin{table}[H]
                    \captionsetup{font={small}, labelformat=fullparents, labelsep=fill, labelfont=bf, justification=raggedleft,
                                singlelinecheck=false, skip=-0.2cm}
                    \caption{Ширина щели и изображения}
                    \begin{center}
                        \begin{tabular}{|l|l|l|l|l|}\hline№ &$X   $, мм&$m. $&$d_{\text{л}}$, мкм&$\sigma_{d_\text{л}}$, мкм
\\\hline1.0&295.0&1.0&18.1&0.2
\\\hline2.0&196.0&1.0&27.2&0.3
\\\hline3.0&295.0&3.0&54.2&0.5
\\\hline4.0&355.0&7.0&105.1&0.9
\\\hline5.0&355.0&10.0&150.0&1.0
\\\hline\end{tabular}
                    \end{center}
                \end{table}
            \end{minipage}
        \end{figure}

        \subsection{Мультиплицированние}
        \par Снова поставим тубус со щелью к окну лазера (рис. 8) и найдём на экране резкое изображение щели с помощью линзы Л$_2$.
        \par В фокальной плоскости Ф линзы Л$_2$ поставим кассету с сетками, которые будут «рассекать» фурье-образ щели --- осуществлять
        пространственную фильтрацию.
        \par Подберём такую ширину входной щели $D$, чтобы на экране можно было наблюдать мультиплицированное изображение для всех сеток. Чем уже
        щель, тем шире её фурье-образ и тем легче рассечь его сетками.
        \par Снимем хависимость $Y$ (расстояние между симметрично удалёнными изображениями щели) и $K$ (число промежутков между изображением и
        центром) от номера сетки для фиксированной ширины входной щели. Данные занесём в таблицу 5.
        \par Измерим расстояния $a_2=14\pm1$ см и $b_2=114\pm1$ см для рассёта увеличения Г$_2$.
        \begin{figure}[H]
            \begin{center}
                \includegraphics[width=0.8\textwidth]{images/pic8.pdf}
                \caption{Схема, поясняющая метод Аббе построения изображения}
            \end{center}
        \end{figure}
        \par Рассчитаем периоды $\Delta y$ «фиктивных» решёток, которые дают такую же периодичность на экране: $\Delta y=\Delta Y/\Gamma_2$, где
        $\Delta Y=Y/(2K)$. Результаты поместим в таблицу 5.
        \begin{table}[H]
            \captionsetup{font={small}, labelformat=fullparents, labelsep=fill, labelfont=bf, justification=raggedleft,
                        singlelinecheck=false, skip=-0.2cm}
            \caption{Ширина щели и изображения}
            \begin{center}
                \begin{tabular}{|l|l|l|l|l|}\hline№  &$Y$, мм&$K$&$\Delta y$, мм&$\sigma_{\Delta y}$, мм
\\\hline1.0&222.0&4.0&3.4&0.2
\\\hline2.0&260.0&7.0&2.3&0.2
\\\hline3.0&205.0&11.0&1.14&0.08
\\\hline4.0&178.0&20.0&0.55&0.05
\\\hline5.0&162.0&25.0&0.4&0.03
\\\hline\end{tabular}
            \end{center}
        \end{table}
        \par Построим график зависимости $\Delta y$ от $1 / d_c$.
        \begin{figure}[H]
            \centering
                % This file was created with tikzplotlib v0.9.16.
\begin{tikzpicture}

\begin{axis}[
    height=11cm,
    tick align=inside,
    major tick length=0.2cm,
    minor tick length=0.1cm,
    tick pos=left,
    xmin=-4.1, xmax=4.1,
    xtick={-5, -4, -3, -2, -1, 0, 1, 2, 3, 4, 5},
    minor x tick num=4,
    xmajorgrids,
    minor x grid style={dotted,black},
    xminorgrids,
    xtick style={color=black},
    xlabel={$m$},
    ymin=-1.2, ymax=1.2,
    ytick={-1.5, -1, -0.5, 0, 0.5, 1, 1.5},
    minor y tick num=4,
    ymajorgrids,
    minor y grid style={dotted,black},
    yminorgrids,
    ytick style={color=black},
    ylabel={$x_m$, мм}
]
\path [draw=red, semithick]
(axis cs:1,0.14)
--(axis cs:1,0.18);

\path [draw=red, semithick]
(axis cs:2,0.48)
--(axis cs:2,0.52);

\path [draw=red, semithick]
(axis cs:3,0.7)
--(axis cs:3,0.74);

\path [draw=red, semithick]
(axis cs:4,1.02)
--(axis cs:4,1.06);

\path [draw=red, semithick]
(axis cs:-1,-0.18)
--(axis cs:-1,-0.14);

\path [draw=red, semithick]
(axis cs:-2,-0.5)
--(axis cs:-2,-0.46);

\path [draw=red, semithick]
(axis cs:-3,-0.74)
--(axis cs:-3,-0.7);

\path [draw=red, semithick]
(axis cs:-4,-1.08)
--(axis cs:-4,-1.04);

\addplot [semithick, black]
table {%
1 0.249999999998364
2 0.499999999996728
3 0.749999999995093
4 0.999999999993457
-1 -0.249999999998364
-2 -0.499999999996728
-3 -0.749999999995093
-4 -0.999999999993457
};
\addplot [semithick, red, mark=square*, mark size=3, mark options={solid}, only marks]
table {%
1 0.16
2 0.5
3 0.72
4 1.04
-1 -0.16
-2 -0.48
-3 -0.72
-4 -1.06
};
\end{axis}

\end{tikzpicture}

                \caption{Графики зависимости $\Delta y \frac{1}{d_c}$}
        \end{figure}

    \section{Обсуждение результатов}
    \par В данной работе мы измеряли парметры решёток и щели разными методами. Методы дали несовпадающие результаты. Вероятно, это вызвано плохой
    методикой измерений, так как для большенства опытов было трудно точно определить требуемые расстояния.
    \par Построенные графики совпадают с теоретическими зависимостями (линейнве и проходят через ноль). Для построения аппроксимирующих прямых $y=kx$
    использовался МНК:
    \begin{equation}
        k = \frac{\left\langle xy \right\rangle}{\left\langle x^2 \right\rangle}
    \end{equation}
    \par Погрешности оценивались по общей формуле погрешностей:
    \begin{equation}
        \sigma_{f(a, b, c, ...)}=f(a, b, c, ...)\sqrt{\left(\frac{\sigma_a}{a}\right)^2+\left(\frac{\sigma_b}{b}\right)^2+
        \left(\frac{\sigma_c}{c}\right)^2+...}
    \end{equation}

    \section{Вывод}
    \par Несмотря на плохое соответсвие значений между разными методами, совпадающие с теоретической зависимостью графики дают надежду, что при более
    аккуратном проведении эксперимента, данные будут согласовываться лучше.

    \newpage\pagenumbering{gobble}

    \begin{figure}[H]
        \begin{center}
            \includegraphics[width=\textwidth]{data/approved.pdf}
        \end{center}
    \end{figure}

\end{document}

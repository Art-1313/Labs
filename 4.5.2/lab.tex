\documentclass[12pt,a4paper]{article}
\usepackage[T2A]{fontenc}
\usepackage[utf8]{inputenc}
\usepackage[english, russian]{babel}
\usepackage{indentfirst}
\usepackage{misccorr}
\usepackage{graphicx}
\usepackage{amsmath}
\usepackage{amssymb}
\usepackage{circuitikz}
\usepackage[font={small}]{caption}
\usepackage[left=20mm, top=20mm, right=20mm, bottom=20mm, nohead]{geometry}
\usepackage{float}
\usepackage{tabularx}
\usepackage{array}
\usepackage{longtable}
\usepackage{pstool}
\usepackage{pgfplots}
\usepackage{hhline}
\usepackage{multirow}
\usepackage{wrapfig}
\usepackage{pdfpages}

\DeclareCaptionLabelSeparator{fill}{.\\}

\DeclareCaptionLabelFormat{fullparents}{\bothIfFirst{#1}{~}#2}

\pgfplotsset{compat=1.17}

\begin{document}

	\begin{titlepage}
		\begin{center}
			{\LARGE Отчёт по лабораторной работе 4.5.2.\\}
			\vspace*{11cm}
				\textbf{\LARGE Интерференция лазерного излучения.}
			\vspace*{6.5cm}
		\end{center}
		\hfill\begin{minipage}{0.37\textwidth}
				Работу выполнил Громов Артём
				\\
				ЛФИ Б02-006
		\end{minipage}
		\vspace{4.8cm}
		\begin{center}
			Долгопрудный, 2022 г.
		\end{center}
	\end{titlepage}

	\section{Аннотация}

		\begin{flushleft}
			\textbf{Цель работы:} исследование видности интерференционной картины излучения гелий-неонового лазера и определение длины
			когерентности излучения.
		\end{flushleft}
		\begin{flushleft}
			\textbf{В работе используется:} Не–Nе-лазер, интерферометр Майкельсона с подвижным зеркалом, фотодиод с усилителем,
			осциллограф, поляроид, линейка.
		\end{flushleft}
		\par Лазер состоит из двух зеркал, составляющих лазерный резонатор, и расположенной между ними газообразной усиливающей среды,
		состоящей из смеси гелия и неона. Типичное расстояние между зеркалами --- $0,2\div1$м. Смесь гелия и неона может находиться в
		отдельной трубке, закрытой с торцов стеклянными окошками, приклеенными под углом Брюстера с целью устранения потерь на
		отражение для одной из поляризаций. Лазер такой конструкции излучает свет с линейной поляризацией. Зеркала резонатора должны
		быть с высокой точностью настроены параллельно друг другу. В современных моделях лазеров, как правило, зеркала приклеиваются
		непосредственно на торцы трубки, одновременно играя роль окошек и образуя резонатор. Этим достигается жёсткость резонатора и
		отпадает необходимость периодической подстройки зеркал. Такой лазер излучает неполяризованный свет (точнее, свет с хаотической
		поляризацией).
		\par В лазере излучение распространяется по резонатору «туда и обратно». При этом максимальным усилением обладают волны, для
		которых набег фазы при полном обходе резонатора кратен $2\pi$. Это приводит к условию на разрешённые частоты и длины волн:
		\begin{equation}
			2\pi2L=2\pi m, \ \ \ L=m\lambda, \ \ \ \nu_{m}=\frac{mc}{2L}, \ \ \ \Delta\nu_{m}=\nu_{m+1}-\nu_{m}=\frac{c}{2L},
		\end{equation}
		\noindent где $L$ --- длина резонатора, $m$ --- целое число. Поэтому лазер генерирует отдельные типы колебаний, называемые
		модами, которые удовлетворяют условию (1). Отметим, что условие (1) в точности совпадает с условием максимума пропускания
		интерферометра Фабри—Перо, что не является удивительным, поскольку лазерный резонатор действительно представляет собой
		интерферометр Фабри—Перо.
		\par Спектральная ширина отдельной моды определяется добротностью резонатора лазера и мощностью излучения. В гелий-неоновом
		лазере из-за малого усиления активной среды используются зеркала c высоким отражением, добротность резонатора большая и
		спектральная ширина моды может быть очень узкой, вплоть до единиц Гц. Реально из-за тепловых нестабильностей длины резонатора
		типичная ширина моды составляет 105 Гц, что всё равно много меньше расстояния между модами (108 Гц). Количество генерируемых
		мод определяется шириной спектра усиления активной среды. Эта ширина складывается из естественной ширины линии излучения
		атомов неона и доплеровского уширения, вызванного тепловым движением атомов. Время когерентности и ширина спектра связаны
		соотношением неопределённости $\tau\cdot\Delta\nu \sim 1$. Для лазерного перехода атома неона с длиной волны 632,8 нм время
		затухания $\tau\approx 10^{-8}$ с, длина цуга в пространстве $L_{\text{цуга}}\approx 3$ м, ширина спектра
		$\Delta\nu\approx108$ Гц. Доплеровское уширение на порядок больше, поэтому общая ширина спектра определяется именно
		доплеровским уширением. При температуре 400 K ширина по полувысоте спектра излучения газообразного неона равна $1,5\cdot10^9$
		Гц. На такой ширине укладывается несколько мод при типичном расстоянии между модами $200\div300$ МГц, поэтому гелий-неоновый
		лазер с длиной резонатора 0,5–0,7 м обычно одновременно излучает 3–7 мод. Амплитуды и фазы этих мод флуктуируют во времени,
		но в среднем амплитуда мод вблизи максимума кривой усиления больше, чем на краях.
		\par Вследствие тепловых нестабильностей длина резонатора все время меняется, в результате чего моды «переползают» с одного
		края контура усиления на другой, там исчезают, а на другом краю рождаются новые. Поэтому количество одновременно генерируемых
		мод и их положение нестабильно. При типичном коэффициенте теплового расширения твёрдых тел $10^{-5}$ достаточно изменения
		температуры на $0,05^{\circ}$, чтобы длина резонатора изменилась на $\lambda/2$ и моды переместились на величину межмодового
		расстояния. Таким образом, температурная нестабильность резонатора приводит к медленным изменениям амплитуд колебаний в
		лазерных модах и числа самих мод. После примерно часа работы лазера характерное время перестройки резонатора составляет около
		1 мин.
		\par\textbf{Видность интерференционной картины.} При сложении двух когерентных световых волн возникает интерференционная
		картина. Если в плоскости наблюдения сходятся под малым углом $\alpha$ две плоских волны с длиной волны $\lambda_0$, то
		наблюдается интерференционная картина в виде последовательности тёмных и светлых полос с расстоянием между полосами:
		\begin{equation}
			\Delta x=\frac{\lambda_0}{\alpha}.
		\end{equation}
		Для оценки чёткости интерференционной картины в окрестности неко- торой точки используют параметр видности:
		\begin{equation}
			\upsilon = \frac{I_{max}-I_{min}}{I_{max}+I_{min}}
		\end{equation}
		где $I_{max}$ и $I_{min}$ --- максимальная и минимальная интенсивности света интерференционной картины вблизи выбранной
		точки. Параметр $\upsilon$ меняется в пределах от 0 (полное исчезновение интерференционной картины) до 1 (наиболее чёткая
		картина). Человеческий глаз может уверенно различать чередование светлых и тёмных интерференционных полос, если $\upsilon>0,1$.
		\par Видность зависит от спектрального состава света, отношения амплитуд интерферирующих волн, разности хода между ними,
		поляризации интерферирующих пучков. Рассмотрим эту зависимость. Найдём выражение для интенсивности света в интерференционной
		картине. При этом можно не учитывать так называемые межмодовые биения, приводящие к быстрым (с частотой $\Delta\nu\approx200$
		МГц) пульсациям интенсивности света в точке наблюдения. При визуальном наблюдении интерференционной картины или при
		использовании достаточно инерционного фотоприёмника эти пульсации усредняются.
		\par Найдём видность интерференционной картины для одной моды лазерного излучения с частотой $\nu_{m}$. Пусть в плоскости
		наблюдения интерферируют под небольшим углом две волны с амплитудами $A_m$ и $B_m$. Если в точке наблюдения разность фаз между
		волнами равна $k_ml$, где $k_m = 2\pi/\lambda m=2\pi\lambda m/c$ --- волновое число, $l$ --- разность хода, то интенсивность
		света в этой точке
		\begin{equation}
			I_m=A^2_m+B^2_m+2A_mB_m\cos{(k_ml)}.
		\end{equation}
		При перемещении поперёк интерференционной картины разность фаз изменяется, и мы переходим от одного максимума к другому. При
		этом интенсивность света в максимуме интерференционной картины $I_{max}=(A_m+B_m)^2$, а в минимуме $I_{min}=(A_m-B_m)^2$.
		Поэтому видность
		\begin{equation}
			\upsilon_1=\frac{2\sqrt{\delta}}{1+\delta},
		\end{equation}
		где введён параметр $\delta=B_m^2/A^2_m$. Видность $\upsilon_1=1$ только тогда, когда интерферирующие волны имеют равную
		интенсивность. В установке разделение пучка на два производится специальным делительным кубиком, при этом $\delta\approx1$.
		\par Найдём видность в случае, когда лазер генерирует одновременно несколько мод, но амплитуды разделяемых пучков одинаковы,
		т.е. $\delta=1$ для всех мод. Для этого следует сложить мгновенные амплитуды всех интерферирующих волн, возвести в квадрат
		и усреднить по времени в течение $10^{-5}$ с (время отклика используемого в данной работе фотодиода), так как более быстрые
		изменения интенсивности фотодиод не воспринимает. Пусть генерируется одновременно $n$ мод с индексом $m$ от $m_0$ до $m_0+n-1$
		и один из пучков проходит путь на величину $l$ больший, чем второй пучок, до места, где наблюдается интерференция.
		Интенсивность находится по формуле
		\begin{equation}
			I(l)=\overline{\left(\sum^{m_0+n-1}_{m=m_0}{A_m\cos{(\omega_mt+\phi_m+k_ml)}+A_m\cos{(\omega_mt+\phi_m)}}\right)^2},
		\end{equation}
		где $A_m$ --- амплитуда $m$-й моды, $\omega_m=2\pi\lambda_m=\pi cm/L$ --- круговая частота, а $\phi_m$ --- фаза этой моды,
		черта означает усреднение по времени.
		\par После возведения в квадрат этой суммы выражение будет содержать члены двух сортов: квадраты каждого члена суммы вида
		$$
			A_m\cos{(\omega_mt+\dots)}
		$$
		и перекрёстные члены вида
		$$
			2A_mA_i\cos{(\omega_mt+\dots)}\cos{(\omega_it+\dots)}.
		$$
		Среднее значение $\overline{\cos{(\omega_mt+\phi)}}$ равно 1/2, перекрёстные члены, содержащие частоты разных мод с $m\neq i$,
		осциллируют с частотой межмодовых биений порядка сотен МГц и поэтому при усреднении по времени занулятся. Останутся только
		квадраты и перекрёстные члены с одинаковой частотой. Получим
		\begin{equation}
			I=\sum^{m_0+n-1}_{m=m_0}{\left(A_m^2+\overline{A_m^2\cos{(\omega_mt+\phi_m+k_ml)}\cos{(\omega_mt+\phi_m)}}\right)}
		\end{equation}
		После преобразования произведения косинусов в сумму и усреднения по времени один из членов суммы также занулится и окончательно
		получим
		\begin{equation}
			I=\sum^{m_0+n-1}_{m=m_0}{A_m^2(1+\cos{(k_mll)})}=\sum^{m_0+n-1}_{m=m_0}{A_m^2(1+\cos{\frac{\pi m}{L}l})}
		\end{equation}
		Из этого выражения видно, что разные моды не интерферируют друг с другом, а суммарный результат интерференции равен сумме
		интерференционных картин разных мод. Кроме того, фазы мод не влияют на картину интерференции, а соотношения амплитуд разных
		мод, напротив, имеют большое значение. Дальнейший анализ затрудняется тем, что амплитуды мод флуктуируют. Легко рассмотреть
		простейший случай, когда амплитуды всех мод одинаковы, $A_m=A=\text{const}$. Тогда, заменив косинусы комплексными экспонентами,
		$\cos{x}=\frac{e^{ix}+e^{-ix}}{2}$, и проведя суммирование получающейся геометрической прогрессии
		$\sum{e^{imx}}=e^{im_0x}\frac{1-e^{inx}}{1-e^{ix}}$, находим
		\begin{equation}
			I=A^2\left(n+\frac{\sin{\frac{\pi l}{2L}n}}{\sin{\frac{\pi l}{2L}}}\cos{\left[m_0+\frac{n-1}{2}\right]}\right)
		\end{equation}
		В этой формуле последний множитель периодически меняет свой знак при небольших изменениях $l$, обеспечивая появление
		максимумов и минимумов при смещении поперёк интерференционной картины, а дробь с синусами определяет амплитуду изменения
		интенсивности. Поэтому видность получается равной
		\begin{equation}
			\upsilon_2=\left| \frac{1}{n}\frac{\sin{\frac{\pi l}{2L}n}}{\sin{\frac{\pi l}{2L}}} \right|
		\end{equation}
		\par Мы ввели обозначение $\upsilon_2$ для видности, обусловленной немонохроматичностью, в отличие от рассмотренной ранее
		$\upsilon_1$, обусловленной разницей амплитуд интерферирующих пучков. Из формулы (10) видно, что при разности хода
		$l=2L\cdot j$, где $j$ --- целое число, видность $V_2=1$, а число минимумов при изменении $l$ от 0 до $2L$ равно $n-1$,
		где $n$ --- число мод. На рис. 1 изображена зависимость видности от задержки $l$ для 3 мод (кривая а). По числу минимумов в
		принципе можно было бы определить число генерируемых лазером мод, но только если амплитуды мод равны, что в действительности
		не выполняется. В общем случае зависимость $V_2$ от $l$ будет похожей на кривую (а) на рис. 1, но промежуточные максимумы и
		минимумы могут размыться или даже совсем исчезнуть. Например, для случая трёх мод, когда квадрат амплитуды центральной моды
		в два раза больше, чем у крайних мод, получается кривая (б) на рис. 1, у которой вместо двух минимумов остался один. Поэтому
		число мод лучше оценивать по ширине главного максимума в окрестности нулевой задержки. На рис. 1 приведено ещё несколько
		расчётных кривых для разного числа мод, интенсивность (квадрат амплитуды) которых плавно спадает от центра линии усиления к
		краям.
		\begin{figure}[H]
			\begin{center}
				\includegraphics[width=0.9\textwidth]{images/pic1.pdf}
				\caption{Зависимость видности от задержки для разного количества генерируемых мод: а) и б) 3 моды, в) 5 мод, г) 6 мод.
				Справа приведены соотношения интенсивностей мод.}
			\end{center}
		\end{figure}
		\par При анализе видности мы предполагали, что световые волны поляризованы одинаково. Если же поляризация в волнах различна,
		то интерферируют компоненты только с одинаковой поляризацией, присутствие в одном из пучков излучения с поляризацией,
		перпендикулярной поляризации другого пучка, только даёт фон, уменьшая видность. В частности, если обе волны линейно
		поляризованы, их амплитуды не флуктуируют, а угол между плоскостями их поляризаций равен $\beta$, то в формуле (8) и в
		последующих выражениях для видности появится сомножитель $\upsilon_3=\cos{\beta}$. При равных амплитудах интерферирующих
		волн этот сомножитель имеет смысл видности, обусловленной разной поляризацией волн. Если световые волны поляризованы во
		взаимно перпендикулярных плоскостях ($\beta=\pi/2$), то видность $\upsilon_3$ обращается в ноль.
		\par Несколько сложнее случай, когда источник света генерирует излучение с линейной поляризацией, но направление поляризации
		хаотически меняется в пределах от 0 до $\pi$. Если такое излучение разделить на два пучка и на пути каждого поставить по
		поляроиду с углом между направлениями разрешённой поляризации этих поляроидов $\beta$, то опять получим две волны с углом
		между плоскостями их поляризаций $\beta$, но амплитуды этих волн будут флуктуировать. Можно показать, что в этом случае
		$\upsilon_3=\cos^2{\beta}$.
		\par Если имеют место все три фактора уменьшения видности: неравенство амплитуд, несовпадение поляризаций и разная оптическая
		задержка между интерферирующими пучками, то можно доказать, что результирующая видность является произведением
		\begin{equation}
			\upsilon=\upsilon_1\upsilon_2\upsilon_3
		\end{equation}
		\par Мы разобрали главные причины уменьшения видности, но есть и другие. Например, лазерные пучки, проходя через большое
		количество оптических элементов, могут портиться из-за дефектов этих элементов и дифракции на осевших пылинках, в результате
		чего интенсивность света становится неоднородной по поперечному сечению. При наложении таких пучков друг на друга происходит
		интерференция излучения разной амплитуды, даже при изначально равной интенсивности пучков. Поэтому в эксперименте практически
		невозможно достичь видности, строго равной 1.
		\par\textbf{Определение ширины спектра лазерного излучения и числа генерируемых мод}. Геометрическая задержка $l$, при которой
		кривая видности заметно спадает, фактически является длиной когерентности. Разделив её на скорость света, получим время
		когерентности $\tau_{\text{ког}}=l/c$. Из принципа неопределённости $\tau_{\text{ког}}\Delta\nu_{\text{полн}}\sim1$, где
		$\Delta\nu_{\text{полн}}$ --- полная ширина спектра излучения лазера (с учётом всех имеющихся мод). Это соотношение
		выполняется только по порядку величины, точное значение произведения $\tau_{\text{ког}}\Delta\nu$ зависит от формы спектра,
		а также от того уровня, по которому измеряется ширина. В нашем случае для определения ширины спектра рациональнее исходить из
		расчётных кривых, представленных на рис. 1. Их анализ показывает, что полная ширина спектра связана с геометрической задержкой
		$l_{1/2}$, при которой видность падает вдвое, приблизительным соотношением
		\begin{equation}
			\Delta\nu_{\text{полн}}\approx0,6\frac{c}{l_{1/2}},
		\end{equation}
		а число мод равно
		\begin{equation}
			n\approx1+1,2\frac{L}{l_{1/2}}.
		\end{equation}

	\section{Экспериментальная установка}

		\par Для получения интерференционной картины используется интерферометр Майкельсона, смонтированный на вертикально стоящей
		массивной металлической плите. Схема установки приведена на рис. 2.
		\par Источником света служит гелий-неоновый лазер (средняя длина волны $\lambda_0=632,8$ нм). Пучок лазерного излучения
		отражается от зеркала З и проходит призму полного внутреннего отражения РФ (ромб Френеля), которая превращает линейную
		поляризацию излучения в круговую. Если в установке используется лазер, излучающий неполяризованный свет, то ромб Френеля не
		нужен, но он и не мешает выполнению работы. Далее лазерное излучение делится диагональной плоскостью делительного кубика ДК на
		два пучка.
		\begin{figure}[H]
			\begin{center}
				\includegraphics[width=0.9\textwidth]{images/pic2.pdf}
				\caption{Схема установки. З, З$_1$, З$_2$, З$_3$ --- зеркала. П$_1$ и П$_2$ --- поляроиды. Б$_1$ и Б$_2$ --- блоки
				1 и 2. ДК --- делительный кубик, РФ --- ромб Френеля. ФД --- фотодиод, Э --- экран, ПК --- пьезокерамика, Л -- линза.}
			\end{center}
		\end{figure}
		\par Пучок 1 проходит поляроид П$_1$, отражается под небольшим углом от зеркала З$_1$, снова проходит поляроид П$_1$ и,
		частично отражаясь от диагональной плоскости делительного кубика, выходит из интерферометра, попадает на зеркало З$_3$ и далее
		на фотодиод ФД. Зеркало З$_1$ наклеено на пьезокерамику ПК, которая может осуществлять малые колебания зеркала вдоль
		направления распространения падающего пучка. Поляроид и зеркало с пьезокерамикой собраны в единый блок Б$_1$, который крепится
		к вертикально стоящей плите. В блоке Б$_1$ имеются юстировочные винты, которые позволяют регулировать угол наклона зеркала
		З$_1$. В установке предусмотрена возможность вращения поляроида П$_1$. Угол поворота отсчитывается по шкале, нанесённой на
		оправу поляроида.
		\par Пучок 2 проходит линзу Л, поляроид П$_2$, отражается от зеркала З$_2$, снова проходит поляроид П$_2$, линзу Л и
		делительный кубик, выходит из интерферометра, попадает на зеркало З$_3$ и далее на фотодиод ФД. Таким образом, от зеркала З$_3$
		под небольшим углом друг к другу идут на фотодиод два пучка, прошедшие разные плечи интерферометра. Между ними происходит
		интерференция и образуются интерференционные полосы. Линза Л, поляроид П$_2$ и зеркало З$_2$ собраны в единый блок Б$_2$.
		Зеркало З$_2$ установлено в фокальной плоскости линзы Л. Это сделано для того, чтобы падающий и выходящий из блока Б$_2$ пучки
		всегда были параллельны друг другу. Блок Б$_2$ может перемещаться вдоль пучка 2 по штанге, жёстко связанной с плитой
		интерферометра. Длина штанги 90 см. В установке предусмотрена возможность небольшого поперечного перемещения блока Б$_2$, что
		позволяет регулировать расстояние между падающим и выходящим из блока пучками. При измерениях блок Б$_2$ крепится к штанге при
		помощи двух винтов. Вдоль штанги нанесены деления через один сантиметр. При перемещении блока Б$_2$ вдоль штанги на величину
		$x_1$ геометрическая разность хода между пучками 1 и 2 изменяется на величину $l=2x_1$.
		\par Сферическое зеркало З$_3$ с небольшим фокусным расстоянием увеличивает картину интерференционных полос и позволяет
		наблюдать её на экране Э, расположенном в плоскости входного окна фотодиода.
		\par Свет попадает на фотодиод ФД через узкую щель в центре экрана. Щель ориентируется параллельно интерференционным полосам.
		Ширина щели меньше расстояния между полосами. Сигнал фотодиода усиливается и подаётся на вход осциллографа. Для питания
		усилителя сигнала фотодиода и управления пьезокерамикой используется блок питания БП.
		\par На пьезокерамику подаётся напряжение с частотой 50 Гц. При этом её длина изменяется с частотой 100 Гц. Величина удлинения
		зависит от приложенного напряжения и регулируется ручкой «Качание» на блоке питания. Обычно удлинение составляет несколько
		длин волн света. На эту величину перемещается вдоль пучка 1 зеркало З$_1$. Интерференционная картина смещается на ширину полосы
		(одно колебание на экране осциллографа), если зеркало З$_1$ смещается на $\lambda_0/2\sim0,3$ мкм. При измерениях через входную
		щель фотодиода последовательно проходит несколько полос интерференционной картины, а на экране осциллографа наблюдаются
		колебания с изменяющимся периодом.

		\begin{wrapfigure}{r}{0.6\textwidth}
			\centering
			\includegraphics[width=0.6\textwidth]{images/pic3.pdf}
			\caption{Осциллограмма сигналов с фотодиода.}
		\end{wrapfigure}
		\par\textbf{Измерение видности.} Типичная осциллограмма сигнала фотодиода приведена на рис. 3. По осциллограмме можно найти
		следующие величины: фоновую засветку (линия 0 --- перекрыты оба пучка 1 и 2); интенсивность света каждого из пучков (линии 1
		или 2 --- перекрыт пучок 2 или 1); максимума и минимума интенсивности интерференционной картины (открыты оба пучка). При этом
		параметр $\delta$, необходимый для расчёта $\upsilon_1$ в формуле (5), определяется отношением
		\begin{equation}
			\delta=\frac{h_1}{h_2}.
		\end{equation}
		Видность интерференционной картины рассчитывается по формуле
		\begin{equation}
			\upsilon=\frac{h_3-h_4}{h_3+h_4}.
		\end{equation}
		Измерив величины $h_1$, $h_2$, $h_3$ и $h_4$, можно рассчитать $\upsilon$ и $\upsilon_1$, а затем определить видность при
		данной разности хода $l$ для угла между плоскостями поляризации пучков $\beta=0$ ($\upsilon_3 = 1$):
		\begin{equation}
			\upsilon_2(l)=\frac{\upsilon}{\upsilon_1}
		\end{equation}
		или при $l=0$, ($\upsilon_2 = 1$) для известного угла $\beta$:
		\begin{equation}
			\upsilon_3=\frac{\upsilon}{\upsilon_1}.
		\end{equation}

	\section{Результаты измерений и обработка данных}

		\subsection{Исследование зависимости видимости от угла поляризации}
		\par При внесении поляроида в пучок, который показывал нудевую интенсивность, она перестаёт быть
		нулевой и изменяется в зависимости от поворота поляроида.
		\par Определим, поворачивая поляроид, угол, соответствующий минимальной разности между величинами $h_3$ и $h_4$. Он составил
		$165\pm1^\circ$.
		\par Исследуем зависимость видности интерфереционной картины от угла $\beta$ поворта поляроида П$_1$ при нулевой разности хода
		($\upsilon_2=0$). Для этого измерим $h_1$, $h_2$, $h_3$, $h_4$ в зависимости от угла поворота поляроида в
		(от $165^\circ$ до $5^\circ$). Погрешность измерения угла составляет $\sigma_\beta=1^\circ$, а на осцилографе ---
		$0,2$ деления. Полученные данные занесём в таблицу 1.
		\par Воспользуемся формулами (5), (14), (15) и (17) для расчёта значеия $\upsilon_3$, результаты занесём
		в таблицу 1.
		\begin{table}[H]
			\captionsetup{font={small}, labelformat=fullparents, labelsep=fill, labelfont=bf, justification=raggedleft,
					singlelinecheck=false, skip=-0.2cm}
			\caption{Результаты измерений $\upsilon_3$ при разных $\beta$}
				\begin{center}
					\begin{tabular}{|l|l|}\hline$m$&$z$, дел
\\\hline0.0&275.0
\\\hline1.0&311.0
\\\hline2.0&348.0
\\\hline3.0&381.0
\\\hline4.0&419.0
\\\hline5.0&453.0
\\\hline6.0&489.0
\\\hline7.0&529.0
\\\hline-1.0&250.0
\\\hline-2.0&214.0
\\\hline-3.0&178.0
\\\hline-4.0&148.0
\\\hline-5.0&109.0
\\\hline-6.0&73.0
\\\hline-7.0&35.0
\\\hline\end{tabular}
				\end{center}    
		\end{table}
		\par Построим графики зависимости $\upsilon_3$ от $\beta$ с разными аппроксимациями.
		\begin{figure}[H]
			\begin{minipage}{0.49\textwidth}
				\begin{center}
					% This file was created with tikzplotlib v0.9.16.
\begin{tikzpicture}

\begin{axis}[
legend cell align={left},
legend style={
  fill opacity=0.8,
  draw opacity=1,
  text opacity=1,
  at={(0.03,0.97)},
  anchor=north west,
  draw=white!80!black
},
height=11.2cm,
tick align=inside,
major tick length=0.2cm,
minor tick length=0.1cm,
tick pos=left,
xmin=0, xmax=600,
xtick={0, 50, 100, 150, 200, 250, 300, 350, 400, 450, 500, 550, 600},
minor x tick num=4,
xmajorgrids,
minor x grid style={dotted,black},
xminorgrids,
xtick style={color=black},
xlabel={$D_{\text{винта}}$, мкм},
ymin=0, ymax=500,
ytick={0, 50, 100, 150, 200, 250, 300, 350, 400, 450, 500},
ytick style={color=black},
minor y tick num=4,
ymajorgrids,
minor y grid style={dotted,black},
yminorgrids,
ytick style={color=black},
ylabel={$D_{\text{расчёт}}$, мкм}
]
\path [draw=red, semithick]
(axis cs:40,32.9861111111111)
--(axis cs:60,32.9861111111111);

\path [draw=red, semithick]
(axis cs:90,98.9583333333333)
--(axis cs:110,98.9583333333333);

\path [draw=red, semithick]
(axis cs:140,131.944444444444)
--(axis cs:160,131.944444444444);

\path [draw=red, semithick]
(axis cs:190,164.930555555556)
--(axis cs:210,164.930555555556);

\path [draw=red, semithick]
(axis cs:240,197.916666666667)
--(axis cs:260,197.916666666667);

\path [draw=red, semithick]
(axis cs:290,230.902777777778)
--(axis cs:310,230.902777777778);

\path [draw=red, semithick]
(axis cs:340,263.888888888889)
--(axis cs:360,263.888888888889);

\path [draw=red, semithick]
(axis cs:390,296.875)
--(axis cs:410,296.875);

\path [draw=red, semithick]
(axis cs:440,362.847222222222)
--(axis cs:460,362.847222222222);

\path [draw=red, semithick]
(axis cs:490,395.833333333333)
--(axis cs:510,395.833333333333);

\path [draw=red, semithick]
(axis cs:50,-0.00334608800363867)
--(axis cs:50,65.9755683102259);

\path [draw=red, semithick]
(axis cs:100,65.9421196383399)
--(axis cs:100,131.974547028327);

\path [draw=red, semithick]
(axis cs:150,98.9048365902785)
--(axis cs:150,164.98405229861);

\path [draw=red, semithick]
(axis cs:200,131.860893814377)
--(axis cs:200,198.000217296734);

\path [draw=red, semithick]
(axis cs:250,164.810309447695)
--(axis cs:250,231.023023885639);

\path [draw=red, semithick]
(axis cs:300,197.75310554622)
--(axis cs:300,264.052450009336);

\path [draw=red, semithick]
(axis cs:350,230.689308019177)
--(axis cs:350,297.088469758601);

\path [draw=red, semithick]
(axis cs:400,263.618946552474)
--(axis cs:400,330.131053447526);

\path [draw=red, semithick]
(axis cs:450,329.458668896105)
--(axis cs:450,396.235775548339);

\path [draw=red, semithick]
(axis cs:500,362.368830131627)
--(axis cs:500,429.29783653504);

\addplot [semithick, red, forget plot]
table {%
0 0
500 392.40620488787
};
\path [draw=green!50.1960784313725!black, semithick]
(axis cs:90,53.0442477876106)
--(axis cs:110,53.0442477876106);

\path [draw=green!50.1960784313725!black, semithick]
(axis cs:140,95.1428571428572)
--(axis cs:160,95.1428571428572);

\path [draw=green!50.1960784313725!black, semithick]
(axis cs:190,148)
--(axis cs:210,148);

\path [draw=green!50.1960784313725!black, semithick]
(axis cs:240,188.913461538462)
--(axis cs:260,188.913461538462);

\path [draw=green!50.1960784313725!black, semithick]
(axis cs:290,233.740384615385)
--(axis cs:310,233.740384615385);

\path [draw=green!50.1960784313725!black, semithick]
(axis cs:340,270.260869565217)
--(axis cs:360,270.260869565217);

\path [draw=green!50.1960784313725!black, semithick]
(axis cs:390,326.204081632653)
--(axis cs:410,326.204081632653);

\path [draw=green!50.1960784313725!black, semithick]
(axis cs:440,360)
--(axis cs:460,360);

\path [draw=green!50.1960784313725!black, semithick]
(axis cs:490,396)
--(axis cs:510,396);

\path [draw=green!50.1960784313725!black, semithick]
(axis cs:540,468)
--(axis cs:560,468);

\path [draw=green!50.1960784313725!black, semithick]
(axis cs:100,52.4114526583863)
--(axis cs:100,53.6770429168349);

\path [draw=green!50.1960784313725!black, semithick]
(axis cs:150,91.0666302830339)
--(axis cs:150,99.2190840026804);

\path [draw=green!50.1960784313725!black, semithick]
(axis cs:200,141.683425571105)
--(axis cs:200,154.316574428895);

\path [draw=green!50.1960784313725!black, semithick]
(axis cs:250,182.271308845487)
--(axis cs:250,195.555614231436);

\path [draw=green!50.1960784313725!black, semithick]
(axis cs:300,223.88999208276)
--(axis cs:300,243.590777148009);

\path [draw=green!50.1960784313725!black, semithick]
(axis cs:350,257.145804132173)
--(axis cs:350,283.375934998262);

\path [draw=green!50.1960784313725!black, semithick]
(axis cs:400,311.969339856676)
--(axis cs:400,340.438823408631);

\path [draw=green!50.1960784313725!black, semithick]
(axis cs:450,341.133063042629)
--(axis cs:450,378.866936957371);

\path [draw=green!50.1960784313725!black, semithick]
(axis cs:500,376.956019710008)
--(axis cs:500,415.043980289992);

\path [draw=green!50.1960784313725!black, semithick]
(axis cs:550,448.557402072788)
--(axis cs:550,487.442597927212);

\addplot [semithick, green!50.1960784313725!black, forget plot]
table {%
0 0
550 439.564038903656
};
\addplot [semithick, red, mark=*, mark size=3, mark options={solid}, only marks]
table {%
50 32.9861111111111
100 98.9583333333333
150 131.944444444444
200 164.930555555556
250 197.916666666667
300 230.902777777778
350 263.888888888889
400 296.875
450 362.847222222222
500 395.833333333333
};
\addlegendentry{Линза}
\addplot [semithick, green!50.1960784313725!black, mark=*, mark size=3, mark options={solid}, only marks]
table {%
100 53.0442477876106
150 95.1428571428572
200 148
250 188.913461538462
300 233.740384615385
350 270.260869565217
400 326.204081632653
450 360
500 396
550 468
};
\addlegendentry{Спектр}
\end{axis}

\end{tikzpicture}

					\caption{$\upsilon_3=\cos{\beta}$}
				\end{center}
			\end{minipage}
			\hfill
			\begin{minipage}{0.49\textwidth}
				\begin{center}
					% This file was created with tikzplotlib v0.9.16.
\begin{tikzpicture}

\begin{axis}[
    height=11cm,
    tick align=inside,
    major tick length=0.2cm,
    minor tick length=0.1cm,
    tick pos=left,
    xmin=-4.1, xmax=4.1,
    xtick={-5, -4, -3, -2, -1, 0, 1, 2, 3, 4, 5},
    minor x tick num=4,
    xmajorgrids,
    minor x grid style={dotted,black},
    xminorgrids,
    xtick style={color=black},
    xlabel={$m$},
    ymin=-1.2, ymax=1.2,
    ytick={-1.5, -1, -0.5, 0, 0.5, 1, 1.5},
    minor y tick num=4,
    ymajorgrids,
    minor y grid style={dotted,black},
    yminorgrids,
    ytick style={color=black},
    ylabel={$x_m$, мм}
]
\path [draw=red, semithick]
(axis cs:1,0.14)
--(axis cs:1,0.18);

\path [draw=red, semithick]
(axis cs:2,0.48)
--(axis cs:2,0.52);

\path [draw=red, semithick]
(axis cs:3,0.7)
--(axis cs:3,0.74);

\path [draw=red, semithick]
(axis cs:4,1.02)
--(axis cs:4,1.06);

\path [draw=red, semithick]
(axis cs:-1,-0.18)
--(axis cs:-1,-0.14);

\path [draw=red, semithick]
(axis cs:-2,-0.5)
--(axis cs:-2,-0.46);

\path [draw=red, semithick]
(axis cs:-3,-0.74)
--(axis cs:-3,-0.7);

\path [draw=red, semithick]
(axis cs:-4,-1.08)
--(axis cs:-4,-1.04);

\addplot [semithick, black]
table {%
1 0.249999999998364
2 0.499999999996728
3 0.749999999995093
4 0.999999999993457
-1 -0.249999999998364
-2 -0.499999999996728
-3 -0.749999999995093
-4 -0.999999999993457
};
\addplot [semithick, red, mark=square*, mark size=3, mark options={solid}, only marks]
table {%
1 0.16
2 0.5
3 0.72
4 1.04
-1 -0.16
-2 -0.48
-3 -0.72
-4 -1.06
};
\end{axis}

\end{tikzpicture}

					\caption{$\upsilon_3=\cos{\beta}$}
				\end{center}
			\end{minipage}
		\end{figure}
		\par В ходе анализа экспериментальных данных была выявлена возможная ошибка при проведении эксперимента. Начиная с шестого
		измерения был изменён нудевой уровень, что сильно повлияло на графики, поэтому было принято решение при построении графиков
		вычесть разность нулевых уровней из подозрительных измерений. На рисунках 4 и 5 представленны графики с учётом исправлений.
		\par Как видно из графиков, зависимость лучше аппроксимируется квадратом косинуса.

		\subsection{Исследование зависимости видимости от разности хода лучей}
		\par Исследуем зависимость видности интерференционной картины от разности хода лучей при фиксированном угле поляризации,
		соответствующего максимальной видимости ($\beta\approx95^\circ$). Для этого будем передвигать блок Б$_2$ вдоль штанги и
		и записывать значения величин $h_1$, $h_2$, $h_3$, $h_4$. Учтём, что разность хода $l=2(x-x_0)$, где x --- значение координаты
		блока на штанге, а $x_0$ --- начальное положение, соответсвующее нулевой разности хода. В районе максимумов будем проводить
		измерения через 1-2 см, а в минимумах --- через 5 см. Полученные данные занесём в таблицу 2. Используя формулы (5), (14) и (16)
		рассчиатем коэффициент $\upsilon_2$. Результаты расчётов занесём в таблицу 2.
		\begin{table}[H]
			\captionsetup{font={small}, labelformat=fullparents, labelsep=fill, labelfont=bf, justification=raggedleft,
					singlelinecheck=false, skip=-0.2cm}
			\caption{Результаты измерений $\upsilon_2$ в зависимости от $x$}
				\begin{center}
					\begin{tabular}{|l|l|l|l|l|l|l|l|l|}\hline$x$, дел&8&25&36&52&-8&-24&-36&-53
\\\hline$m $&1&2&3&4&-1&-2&-3&-4
\\\hline\end{tabular}
				\end{center}    
		\end{table}
		\par Построим график зависимости $\upsilon_2$ от $l$ и аппроксимируем его многочлено шестой степени (рис. 6). В областях
		максимумов из-за большой погрешности данные слишком сильно смешиваются, попробуем их отчистить выкидывая близкие точки 
		с одинаковыми значениями (рис. 7).
		\begin{figure}[H]
			\begin{minipage}{0.49\textwidth}
				\begin{center}
					% This file was created with tikzplotlib v0.9.16.
\begin{tikzpicture}

\definecolor{color0}{rgb}{0.12156862745098,0.466666666666667,0.705882352941177}

\begin{axis}[
    height=7.2cm,
    tick align=inside,
    major tick length=0.2cm,
    minor tick length=0.1cm,
    tick pos=left,
    x grid style={white!69.0196078431373!black},
    xmin=-10, xmax=140,
    xtick={-40, 0, 40, 80, 120, 160},
    minor x tick num=3,
    xmajorgrids,
    minor x grid style={dotted,black},
    xminorgrids,
    xtick style={color=black},
    xlabel={$l$, \text{см}},
    ymin = -0.2, ymax = 0.8,
    ytick = {-0.4, 0, 0.4, 0.8},
    ytick style={color=black},
    minor y tick num=3,
    ymajorgrids,
    minor y grid style={dotted,black},
    yminorgrids,
    ytick style={color=black},
    ylabel={$\upsilon_2$}
]
\path [draw=red, semithick]
(axis cs:-6,0.225806451612903)
--(axis cs:-2,0.225806451612903);

\path [draw=red, semithick]
(axis cs:-2,0.6)
--(axis cs:2,0.6);

\path [draw=red, semithick]
(axis cs:0,0.133333333333333)
--(axis cs:4,0.133333333333333);

\path [draw=red, semithick]
(axis cs:2,0.214285714285714)
--(axis cs:6,0.214285714285714);

\path [draw=red, semithick]
(axis cs:4,0.2)
--(axis cs:8,0.2);

\path [draw=red, semithick]
(axis cs:6,0.161290322580645)
--(axis cs:10,0.161290322580645);

\path [draw=red, semithick]
(axis cs:8,0.225806451612903)
--(axis cs:12,0.225806451612903);

\path [draw=red, semithick]
(axis cs:10,0.225806451612903)
--(axis cs:14,0.225806451612903);

\path [draw=red, semithick]
(axis cs:14,0.161290322580645)
--(axis cs:18,0.161290322580645);

\path [draw=red, semithick]
(axis cs:26,0.0967741935483871)
--(axis cs:30,0.0967741935483871);

\path [draw=red, semithick]
(axis cs:36,0.0322580645161291)
--(axis cs:40,0.0322580645161291);

\path [draw=red, semithick]
(axis cs:46,0.0666666666666667)
--(axis cs:50,0.0666666666666667);

\path [draw=red, semithick]
(axis cs:56,0.0322580645161291)
--(axis cs:60,0.0322580645161291);

\path [draw=red, semithick]
(axis cs:66,0.0322580645161291)
--(axis cs:70,0.0322580645161291);

\path [draw=red, semithick]
(axis cs:76,0.0666666666666667)
--(axis cs:80,0.0666666666666667);

\path [draw=red, semithick]
(axis cs:86,0.0322580645161291)
--(axis cs:90,0.0322580645161291);

\path [draw=red, semithick]
(axis cs:96,0.0666666666666667)
--(axis cs:100,0.0666666666666667);

\path [draw=red, semithick]
(axis cs:106,0.0967741935483871)
--(axis cs:110,0.0967741935483871);

\path [draw=red, semithick]
(axis cs:110,0.225806451612903)
--(axis cs:114,0.225806451612903);

\path [draw=red, semithick]
(axis cs:114,0.290322580645161)
--(axis cs:118,0.290322580645161);

\path [draw=red, semithick]
(axis cs:116,0.290322580645161)
--(axis cs:120,0.290322580645161);

\path [draw=red, semithick]
(axis cs:118,0.533333333333333)
--(axis cs:122,0.533333333333333);

\path [draw=red, semithick]
(axis cs:120,0.548387096774194)
--(axis cs:124,0.548387096774194);

\path [draw=red, semithick]
(axis cs:122,0.161290322580645)
--(axis cs:126,0.161290322580645);

\path [draw=red, semithick]
(axis cs:124,0.161290322580645)
--(axis cs:128,0.161290322580645);

\path [draw=red, semithick]
(axis cs:126,0.0967741935483871)
--(axis cs:130,0.0967741935483871);

\path [draw=red, semithick]
(axis cs:128,0.0967741935483871)
--(axis cs:132,0.0967741935483871);

\path [draw=red, semithick]
(axis cs:-4,0.16988541584789)
--(axis cs:-4,0.281727487377916);

\path [draw=red, semithick]
(axis cs:0,0.524575276673435)
--(axis cs:0,0.675424723326565);

\path [draw=red, semithick]
(axis cs:2,0.0799074876436831)
--(axis cs:2,0.186759179022984);

\path [draw=red, semithick]
(axis cs:4,0.152955024080759)
--(axis cs:4,0.27561640449067);

\path [draw=red, semithick]
(axis cs:6,0.143431457505076)
--(axis cs:6,0.256568542494924);

\path [draw=red, semithick]
(axis cs:8,0.108312499224317)
--(axis cs:8,0.214268145936973);

\path [draw=red, semithick]
(axis cs:10,0.16988541584789)
--(axis cs:10,0.281727487377916);

\path [draw=red, semithick]
(axis cs:12,0.16988541584789)
--(axis cs:12,0.281727487377916);

\path [draw=red, semithick]
(axis cs:16,0.108312499224317)
--(axis cs:16,0.214268145936973);

\path [draw=red, semithick]
(axis cs:28,0.0467395826007438)
--(axis cs:28,0.14680880449603);

\path [draw=red, semithick]
(axis cs:38,-0.0148333340228294)
--(axis cs:38,0.0793494630550875);

\path [draw=red, semithick]
(axis cs:48,0.01638351778229)
--(axis cs:48,0.116949815551043);

\path [draw=red, semithick]
(axis cs:58,-0.0148333340228294)
--(axis cs:58,0.0793494630550875);

\path [draw=red, semithick]
(axis cs:68,-0.0148333340228294)
--(axis cs:68,0.0793494630550875);

\path [draw=red, semithick]
(axis cs:78,0.01638351778229)
--(axis cs:78,0.116949815551043);

\path [draw=red, semithick]
(axis cs:88,-0.0148333340228294)
--(axis cs:88,0.0793494630550875);

\path [draw=red, semithick]
(axis cs:98,0.01638351778229)
--(axis cs:98,0.116949815551043);

\path [draw=red, semithick]
(axis cs:108,0.0467395826007438)
--(axis cs:108,0.14680880449603);

\path [draw=red, semithick]
(axis cs:112,0.16988541584789)
--(axis cs:112,0.281727487377916);

\path [draw=red, semithick]
(axis cs:116,0.231458332471463)
--(axis cs:116,0.349186828818859);

\path [draw=red, semithick]
(axis cs:118,0.231458332471463)
--(axis cs:118,0.349186828818859);

\path [draw=red, semithick]
(axis cs:120,0.461051306812042)
--(axis cs:120,0.605615359854625);

\path [draw=red, semithick]
(axis cs:122,0.477749998965756)
--(axis cs:122,0.619024194582631);

\path [draw=red, semithick]
(axis cs:124,0.108312499224317)
--(axis cs:124,0.214268145936973);

\path [draw=red, semithick]
(axis cs:126,0.108312499224317)
--(axis cs:126,0.214268145936973);

\path [draw=red, semithick]
(axis cs:128,0.0467395826007438)
--(axis cs:128,0.14680880449603);

\path [draw=red, semithick]
(axis cs:130,0.0467395826007438)
--(axis cs:130,0.14680880449603);

\addplot [semithick, color0]
table {%
-4 0.289700909532296
-2.64646464646465 0.299561221223985
-1.29292929292929 0.303200008671337
0.0606060606060606 0.301611090044221
1.41414141414141 0.295698522381247
2.76767676767677 0.286281224103273
4.12121212121212 0.274097490756272
5.47474747474748 0.259809403983547
6.82828282828283 0.244007133727306
8.18181818181818 0.227213133659582
9.53535353535354 0.209886229842507
10.8888888888889 0.192425602617948
12.2424242424242 0.175174661726482
13.5959595959596 0.158424814655734
14.949494949495 0.142419128218066
16.3030303030303 0.127355883357614
17.6565656565657 0.113392023186687
19.010101010101 0.100646494251508
20.3636363636364 0.0892034810273152
21.7171717171717 0.079115533642817
23.0707070707071 0.0704065888339942
24.4242424242424 0.0630748841272588
25.7777777777778 0.0570957652519653
27.1313131313131 0.0524243867822743
28.4848484848485 0.0489983060083702
29.8383838383838 0.0467399700370295
31.1919191919192 0.0455590961215448
32.5454545454545 0.0453549452209991
33.8989898989899 0.0460184887888953
35.2525252525253 0.0474344687911365
36.6060606060606 0.0494833509533611
37.959595959596 0.0520431712376296
39.3131313131313 0.0549912755484642
40.6666666666667 0.0582059526682422
42.020202020202 0.0615679604219419
43.3737373737374 0.064961945071241
44.7272727272727 0.0682777539379683
46.0808080808081 0.0714116412569081
47.4343434343434 0.0742673672579583
48.7878787878788 0.0767571904776398
50.1414141414141 0.0788027532999603
51.4949494949495 0.0803358607266301
52.8484848484849 0.0812991523766312
54.2020202020202 0.0816466677151401
55.5555555555556 0.0813443045118011
56.9090909090909 0.0803701705283548
58.2626262626263 0.0787148284356191
59.6161616161616 0.076381433959822
60.969696969697 0.0733857672582888
62.3232323232323 0.0697561575244805
63.6767676767677 0.0655333008223871
65.030303030303 0.0607699711502716
66.3838383838384 0.0555306247337693
67.7373737373737 0.0498908975483373
69.0909090909091 0.0439369960710599
70.4444444444444 0.0377649812618038
71.7979797979798 0.0314799457737291
73.1515151515152 0.0251950843931509
74.5050505050505 0.0190306577087552
75.8585858585859 0.0131128490101669
77.2121212121212 0.00757251441587132
78.5656565656566 0.00254382623048871
79.9191919191919 -0.00183719046859988
81.2727272727273 -0.00543422801527125
82.6262626262626 -0.00811237310932827
83.979797979798 -0.00973989054181345
85.3333333333333 -0.0101901136909708
86.6868686868687 -0.00934344178885385
88.040404040404 -0.00708944395858157
89.3939393939394 -0.00332907002224061
90.7474747474748 0.00202303192056456
92.1010101010101 0.00903609114351488
93.4545454545455 0.0177606831737331
94.8080808080808 0.0282260919012925
96.1616161616162 0.0404375649180784
97.5151515151515 0.0543734620860032
98.8686868686869 0.0699822973345749
100.222222222222 0.0871796736878155
101.575757575758 0.105845111520537
102.929292929293 0.125818770043966
104.282828282828 0.146898062020724
105.636363636364 0.168834161709158
106.989898989899 0.191328406037028
108.343434343434 0.214028589004543
109.69696969697 0.236525149316752
111.050505050505 0.258347251245289
112.40404040404 0.278958758719469
113.757575757576 0.297754102646736
115.111111111111 0.314054041462469
116.464646464646 0.327101314909134
117.818181818182 0.336056191044795
119.171717171717 0.339991906480971
120.525252525253 0.337889999849856
121.878787878788 0.328635538500881
123.232323232323 0.311012238426636
124.585858585859 0.283697477418142
125.939393939394 0.24525720144948
127.292929292929 0.194140724291761
128.646464646465 0.128675420356469
130 0.047061310768137
};
\addplot [semithick, red, mark=*, mark size=3, mark options={solid}, only marks]
table {%
-4 0.225806451612903
0 0.6
2 0.133333333333333
4 0.214285714285714
6 0.2
8 0.161290322580645
10 0.225806451612903
12 0.225806451612903
16 0.161290322580645
28 0.0967741935483871
38 0.0322580645161291
48 0.0666666666666667
58 0.0322580645161291
68 0.0322580645161291
78 0.0666666666666667
88 0.0322580645161291
98 0.0666666666666667
108 0.0967741935483871
112 0.225806451612903
116 0.290322580645161
118 0.290322580645161
120 0.533333333333333
122 0.548387096774194
124 0.161290322580645
126 0.161290322580645
128 0.0967741935483871
130 0.0967741935483871
};
\end{axis}

\end{tikzpicture}

					\caption{$\upsilon_3=\cos{\beta}$}
				\end{center}
			\end{minipage}
			\hfill
			\begin{minipage}{0.49\textwidth}
				\begin{center}
					% This file was created with tikzplotlib v0.9.16.
\begin{tikzpicture}

\definecolor{color0}{rgb}{0.12156862745098,0.466666666666667,0.705882352941177}

\begin{axis}[
    height=7.2cm,
    tick align=inside,
    major tick length=0.2cm,
    minor tick length=0.1cm,
    tick pos=left,
    x grid style={white!69.0196078431373!black},
    xmin=-10, xmax=140,
    xtick={-40, 0, 40, 80, 120, 160},
    minor x tick num=3,
    xmajorgrids,
    minor x grid style={dotted,black},
    xminorgrids,
    xtick style={color=black},
    xlabel={$l$, \text{см}},
    ymin = -0.2, ymax = 0.8,
    ytick = {-0.4, 0, 0.4, 0.8},
    ytick style={color=black},
    minor y tick num=3,
    ymajorgrids,
    minor y grid style={dotted,black},
    yminorgrids,
    ytick style={color=black},
    ylabel={$\upsilon_2$}
]
\path [draw=red, semithick]
(axis cs:-6,0.225806451612903)
--(axis cs:-2,0.225806451612903);

\path [draw=red, semithick]
(axis cs:-2,0.6)
--(axis cs:2,0.6);

\path [draw=red, semithick]
(axis cs:10,0.225806451612903)
--(axis cs:14,0.225806451612903);

\path [draw=red, semithick]
(axis cs:14,0.161290322580645)
--(axis cs:18,0.161290322580645);

\path [draw=red, semithick]
(axis cs:26,0.0967741935483871)
--(axis cs:30,0.0967741935483871);

\path [draw=red, semithick]
(axis cs:36,0.0322580645161291)
--(axis cs:40,0.0322580645161291);

\path [draw=red, semithick]
(axis cs:46,0.0666666666666667)
--(axis cs:50,0.0666666666666667);

\path [draw=red, semithick]
(axis cs:56,0.0322580645161291)
--(axis cs:60,0.0322580645161291);

\path [draw=red, semithick]
(axis cs:66,0.0322580645161291)
--(axis cs:70,0.0322580645161291);

\path [draw=red, semithick]
(axis cs:76,0.0666666666666667)
--(axis cs:80,0.0666666666666667);

\path [draw=red, semithick]
(axis cs:86,0.0322580645161291)
--(axis cs:90,0.0322580645161291);

\path [draw=red, semithick]
(axis cs:96,0.0666666666666667)
--(axis cs:100,0.0666666666666667);

\path [draw=red, semithick]
(axis cs:106,0.0967741935483871)
--(axis cs:110,0.0967741935483871);

\path [draw=red, semithick]
(axis cs:110,0.225806451612903)
--(axis cs:114,0.225806451612903);

\path [draw=red, semithick]
(axis cs:114,0.290322580645161)
--(axis cs:118,0.290322580645161);

\path [draw=red, semithick]
(axis cs:118,0.533333333333333)
--(axis cs:122,0.533333333333333);

\path [draw=red, semithick]
(axis cs:120,0.548387096774194)
--(axis cs:124,0.548387096774194);

\path [draw=red, semithick]
(axis cs:124,0.161290322580645)
--(axis cs:128,0.161290322580645);

\path [draw=red, semithick]
(axis cs:128,0.0967741935483871)
--(axis cs:132,0.0967741935483871);

\path [draw=red, semithick]
(axis cs:-4,0.146722164412071)
--(axis cs:-4,0.304890738813736);

\path [draw=red, semithick]
(axis cs:0,0.493333333333333)
--(axis cs:0,0.706666666666667);

\path [draw=red, semithick]
(axis cs:12,0.146722164412071)
--(axis cs:12,0.304890738813736);

\path [draw=red, semithick]
(axis cs:16,0.0863683662851197)
--(axis cs:16,0.236212278876171);

\path [draw=red, semithick]
(axis cs:28,0.0260145681581686)
--(axis cs:28,0.167533818938606);

\path [draw=red, semithick]
(axis cs:38,-0.0343392299687825)
--(axis cs:38,0.0988553590010406);

\path [draw=red, semithick]
(axis cs:48,-0.00444444444444439)
--(axis cs:48,0.137777777777778);

\path [draw=red, semithick]
(axis cs:58,-0.0343392299687825)
--(axis cs:58,0.0988553590010406);

\path [draw=red, semithick]
(axis cs:68,-0.0343392299687825)
--(axis cs:68,0.0988553590010406);

\path [draw=red, semithick]
(axis cs:78,-0.00444444444444439)
--(axis cs:78,0.137777777777778);

\path [draw=red, semithick]
(axis cs:88,-0.0343392299687825)
--(axis cs:88,0.0988553590010406);

\path [draw=red, semithick]
(axis cs:98,-0.00444444444444439)
--(axis cs:98,0.137777777777778);

\path [draw=red, semithick]
(axis cs:108,0.0260145681581686)
--(axis cs:108,0.167533818938606);

\path [draw=red, semithick]
(axis cs:112,0.146722164412071)
--(axis cs:112,0.304890738813736);

\path [draw=red, semithick]
(axis cs:116,0.207075962539022)
--(axis cs:116,0.373569198751301);

\path [draw=red, semithick]
(axis cs:120,0.431111111111111)
--(axis cs:120,0.635555555555556);

\path [draw=red, semithick]
(axis cs:122,0.448491155046826)
--(axis cs:122,0.648283038501561);

\path [draw=red, semithick]
(axis cs:126,0.0863683662851197)
--(axis cs:126,0.236212278876171);

\path [draw=red, semithick]
(axis cs:130,0.0260145681581686)
--(axis cs:130,0.167533818938606);

\addplot [semithick, color0]
table {%
-4 0.313814118861075
-2.64646464646465 0.372719374858457
-1.29292929292929 0.414429546476477
0.0606060606060606 0.441252184268248
1.41414141414141 0.455307668066297
2.76767676767677 0.458538104134772
4.12121212121212 0.452716029232126
5.47474747474748 0.439452921584311
6.82828282828283 0.420207518768442
8.18181818181818 0.396293942506964
9.53535353535354 0.368889630372301
10.8888888888889 0.339043074402003
12.2424242424242 0.307681366624373
13.5959595959596 0.275617551494593
14.949494949495 0.243557785241332
16.3030303030303 0.21210830212385
17.6565656565657 0.181782187599588
19.010101010101 0.153005958402251
20.3636363636364 0.126125949530377
21.7171717171717 0.101414508146396
23.0707070707071 0.0790759943861873
24.4242424242424 0.059252589079113
25.7777777777778 0.0420299083785511
27.1313131313131 0.0274424253029159
28.4848484848485 0.0154786981871685
29.8383838383838 0.00608640604481389
31.1919191919192 -0.000822809159606586
32.5454545454545 -0.0053686933275373
33.8989898989899 -0.00769792413293865
35.2525252525253 -0.00797984801832828
36.6060606060606 -0.0064024102803318
37.959595959596 -0.00316827824474411
39.3131313131313 0.001508842468899
40.6666666666667 0.0074076985932389
42.020202020202 0.0143027867695034
43.3737373737374 0.0219674580144038
44.7272727272727 0.0301768290975233
46.0808080808081 0.0387105008291939
47.4343434343434 0.0473550832588656
48.7878787878788 0.0559065277839617
50.1414141414141 0.0641722661692277
51.4949494949495 0.0719731564765673
52.8484848484849 0.0791452359053693
54.2020202020202 0.0855412805433256
55.5555555555556 0.0910321720277353
56.9090909090909 0.0955080711173033
58.2626262626263 0.0988793981744253
59.6161616161616 0.101077620557964
60.969696969697 0.102055846926514
62.3232323232323 0.10178922845216
63.6767676767677 0.10027516694472
65.030303030303 0.0975333298864813
66.3838383838384 0.0936054723774249
67.7373737373737 0.088555065990942
69.0909090909091 0.0824667345400373
70.4444444444444 0.0754454967540248
71.7979797979798 0.0676158158657112
73.1515151515152 0.059120456109071
74.5050505050505 0.0501191461274099
75.8585858585859 0.0407870492920191
77.2121212121212 0.031313040931319
78.5656565656566 0.0218977924704934
79.9191919191919 0.0127516624816113
81.2727272727273 0.00409239464424182
82.6262626262626 -0.00385737738344372
83.979797979798 -0.0108728181830784
85.3333333333333 -0.0167297718840175
86.6868686868687 -0.0212078365607651
88.040404040404 -0.024093631719913
89.3939393939394 -0.025184258876587
90.7474747474748 -0.0242909552204062
92.1010101010101 -0.0212429403709502
93.4545454545455 -0.0158914562227379
94.8080808080808 -0.00811399987971576
96.1616161616162 0.00218125032074329
97.5151515151515 0.015050815694331
98.8686868686869 0.030511105006786
100.222222222222 0.048533409181362
101.575757575758 0.0690387029167914
102.929292929293 0.091892253215734
104.282828282828 0.116898034823718
105.636363636364 0.143792952578569
106.989898989899 0.172240870670332
108.343434343434 0.20182644881168
109.69696969697 0.232048785318812
111.050505050505 0.262314867102847
112.40404040404 0.291932826571698
113.757575757576 0.320105005442445
115.111111111111 0.345920825464192
116.464646464646 0.368349466051415
117.818181818182 0.386232348827805
119.171717171717 0.39827542908059
120.525252525253 0.403041294125359
121.878787878788 0.398941068581365
123.232323232323 0.384226126557329
124.585858585859 0.35697961074772
125.939393939394 0.315107758439542
127.292929292929 0.25633103442959
128.646464646465 0.178175070852219
130 0.0779614139175848
};
\addplot [semithick, red, mark=*, mark size=3, mark options={solid}, only marks]
table {%
-4 0.225806451612903
0 0.6
12 0.225806451612903
16 0.161290322580645
28 0.0967741935483871
38 0.0322580645161291
48 0.0666666666666667
58 0.0322580645161291
68 0.0322580645161291
78 0.0666666666666667
88 0.0322580645161291
98 0.0666666666666667
108 0.0967741935483871
112 0.225806451612903
116 0.290322580645161
120 0.533333333333333
122 0.548387096774194
126 0.161290322580645
130 0.0967741935483871
};
\end{axis}

\end{tikzpicture}

					\caption{$\upsilon_3=\cos{\beta}$}
				\end{center}
			\end{minipage}
		\end{figure}
		\par Используя график с более чистыми данными определим, используя формулу (1), расстояние $L$ между зеркалами оптического
		резонатора и межмодовое расстояние $\nu_m$. Погрешность определения максимуумов оценим как среднеквадратичное погрешностей
		коэффтнтов аппроксимации (около 10\%). Учтём погрешность измерений  В итоге получаем $L=57\pm7$ см. Откуда $\nu_m=260\pm30$
		МГц.
		\par Определим задержку $l_{1/2}$ на половине высоты главного максимума и рассчитаем по формуле (12) диапазон чистот
		$\Delta F$, в котором происходит генерация продольных мод. Погрешность задержкии оценим в 10\%. В итоге
		$l_{1/2}=16\pm2$ см, $\Delta F=1.1\pm0.1$ ГГц.
		\par Оценим число генерируемых лазером мод по формуле (13). Получим, что $n=10$.

	\section{Обсуждение результатов}

		\par В данной работе мы изучали зависимость видности интерфереционной картины от параметров лучей, образующих её. 
		\par Первый опыт выявлял зависимость от угла поляризации между двумя лучами. Мы аппроксимировали опытные данные косинусом и
		квадратом косинуса. Более подходящей аппроксимацией оказалась аппроксимация квадратом косинуса, значит, источник генерирует
		излучение с линейной поляризацией, но направление поляризации хаотически меняется о $0$ до $\pi$. Однако при провелении опыта
		была допущена ошибка (менялся нулевой уровень), что могло повлиять на неверную интерпритацию результатов.
		\par Целью второго опыта бало выявление зависимости видности интерференционной картины от разности хода лучей. Из-за
		довольно грубых измерений данные этого эксперимента пришлось дополнительно обработать, отсеяв лишнее. Значение расстояния
		между зеркалами оптического резонатора совпало с данным на установке, также в значение межмодового расстояния лежит в
		диапазоне, указанном в теории. Однако полноя ширина спектра оказалсь меньше, чем было казанно в описании, а число мод наоборот
		--- больше.
		\par Погрешности основных результатов составляют около 10\%. Во время измеерения наибольшую погрешнрость давал осцилограф.
		
	\section{Вывод}
		\par В целом, результаты работы неплохо согласуются с описанными в теории. Однако есть несколько подозрительных моментов.
		Во-первых, во время измерений не менялись значения величин $h_1$ и $h_2$. Во-вторых, была допущена ошибка при изучении
		влияния поляризации.
		\par Возможно, при более аккуратном проведении работы, данные проблемы будут решены.

\end{document}

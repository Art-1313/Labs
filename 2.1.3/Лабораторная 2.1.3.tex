\documentclass[12pt,a4paper]{article}
\usepackage[utf8]{inputenc}
\usepackage[english,russian]{babel}
\usepackage{indentfirst}
\usepackage{misccorr}
\usepackage{graphicx}
\usepackage{amsmath}
\usepackage{amssymb}
\usepackage{circuitikz}
\usepackage[font={small}]{caption}
\usepackage[left=20mm, top=20mm, right=20mm, bottom=20mm, nohead]{geometry}
\usepackage{float}
\usepackage{tabularx}
\usepackage{array}
\usepackage{longtable}
\usepackage{pstool}
\usepackage{pgfplots}
\usepackage{hhline}
\usepackage{multirow}

\DeclareCaptionLabelSeparator{fill}{.\\}
\DeclareCaptionLabelFormat{fullparents}{\bothIfFirst{#1}{~}#2}

\begin{document}

	\begin{titlepage}
		\begin{center}
			{\LARGE Отчёт по лабораторной работе 2.1.1.\\}
			\vspace*{11cm}
				\textbf{\LARGE Определение $C_p/C_v$ по скорости звука.}	
			\vspace*{6.5cm}
		\end{center}
		\hfill\begin{minipage}{0.37\textwidth}
				Работу выполнил Громов Артём\\
				ЛФИ Б02-006
		\end{minipage}
		\vspace{5cm}
		\begin{center}
 			Долгопрудный, 2021 г.
		\end{center}   
	\end{titlepage}
	
	\section{Аннотация}
		\noindent\textbf{Цель работы: }
		\begin{enumerate}
			\item измерение частоты колебаний и длины волны при резонансе звуковых колебаний в газе, заполняющем трубу;
			\item определение показателя адиабаты с помощью уравнения состояния идеального газа.
		\end{enumerate}
		\noindent\textbf{В работе используются: }  звуковой генератор ГЗ; электронный осциллограф ЭО; микрофон; телефон;
		раздвижная труба; теплоизолированная труба, обогреваемая водой из термостата; баллон со сжатым углекислым газом;
		газгольдер.
		\vspace{0.5 cm}
		\par Скорость распространения звуковой волны в газах зависит от показателя адиабаты $\gamma$. На измерении
		скорости звука основан один из наиболее точных методов определения показателя адиабаты.
		\par Скорость звука в газах определяется формулой:
		\begin{equation}
			c=\sqrt{\gamma\frac{RT}{\mu}},
		\end{equation}
		\noindent где $R$ --- газовая постоянная, $T$ --- температа газа, а $\mu$ --- его молярная масса. Преабразуя формулу,
		найдём
		\begin{equation}
			\gamma=\frac{mu}{RT}c^2.
		\end{equation}
		\noindent Таким образом, для определения показателя адиабаты достаточчно измерить температуру газа и скорость
		распространения звука (молярная масса газа преполагается известной).
		\par Звуковая волна, рапространяющаяся вдоль трубы, испытывает многократные отражения от торцов. Звуковые
		колебания в трубе является наложением всех отражённых волн и, вообще говоря, очень сложны. Картина упрощается,
		если длинна трубы равна целому числу полуволн, то есть когда
		\begin{equation}
			L=n\frac{\lambda}{2},
		\end{equation}
		\noindent где $\lambda$ --- длина волны звука в трубе, а $n$ -- любое целое число. Если условие (2) выполнено, то
		волна, отражённая от торца трубы, вернувшаяся к её началу и вновь отражённая, совпадает по фазе с падающей.
		Совпадающие по фазе волны усиливают друг друга. Амплитуда звуковых колебаний при этом резко возрастает.
		\par При звуковых колебаниях слои газа, прилегающие к торцам трубы, на испытывают смещения 
		(\textit{узел смещения}). Узлы смещения находятся максимум смещения (\textit{пучности}).
		\par Скорость звука $c$ связана с его частотой $f$ и длиной волны $\lambda$ соотношением
		\begin{equation}
			c=\lambda f.
		\end{equation}
		\par Подбор условий, при которых возникает резонанс, будем производить, изменяя частоту звуковых колебаний при
		постоянной длине трубы. В этом случае следует плавно изменять частоту $f$ звукового генератора, а следовательно,
		и длину звуковой волны $\lambda$. Для последовательных резонансов получим
		\begin{equation}
			L=\frac{\lambda_1}{2}n=\frac{\lambda_2}{2}(n+1)=...=\frac{\lambda_{k+1}}{2}(n+k)
		\end{equation}
		\par Из (4) и (5) имеем
		\begin{equation}
			f_{k+1}=\frac{c}{\lambda_{k+1}}=\frac{c}{2L}(n+k)=f_1+\frac{c}{2L}k.
		\end{equation}
		\par Скорость звука, делённая на $2L$, определяется, таким образом, по угловому коэффициеенту графика зависимости
		частоты от номера резонанса.
	\section{Экспериментальная установка}
		\par В установке (рис. 1) звуковые колебания в трубе возбуждаются телефоном Т и улавливаются миерофоном М.
		Мембрана телефона приводится в движение переменным током звуковой частоты; в качестве источника переменной
		ЭДС используется звуковой генератор ГЗ. Возникающий в микрофоне сигнал наблюдается на осцилографе ЭО.
		\par Микрофон и телефон присоеденены к установке через тонкие резиновые трубки. Такая связь достаточна для
		возбуждения и обнаружения звуковых колебаний в трубе и в то де время мало возмущает эти колебания: при расчётаах
		оба торца можно ссчитать неподвижными, а влиянием соеденительных отверстий пренебречь.
		\par Установка содержит теплоизолированную трубу постоянной длины ($L=800\pm1$ мм). Воздух в трубе нагревается
		водой из термотата. Температура газа принимается равной темпратуре воды, омывающей трубу. На данной установке
		определяется зависимость скорости звука от температуры.
		\begin{figure}[H]
			\begin{center}
				\includegraphics[width=0.8\textwidth]{C:/Users/gromo/Desktop/Установка213.png}
				\caption{Установка для изучения зависимости скорости звука от температуры}
			\end{center}
		\end{figure}
	\section{Результаты измерений и обработка данных}
		\subsection{Определение резонасных частот}
		\par Измерим скорость звука в трубе. Для этого будем плавно менять частоту на звуковом генераторе отлавливая резкое
		увеличенеи амплитуды с помощью осцилографа. Чтобы улучшить точность будем проводить две серии опытов: сначала
		увеличивая частоту, а затем уменьшая. Погрешность измерения частоты составила $\Delta_{\text{ген}}=\pm1$, Гц.
		Занесём полученные данные в таблицу 1.
		\begin{table}[H]
			\captionsetup{font={small}, labelformat=fullparents, labelsep=fill, labelfont=bf, justification=raggedleft,
			singlelinecheck=false, skip=-0.2cm}
			\caption{Значения резонансных частот}
			\begin{center}
				\begin{tabular}{|p{2.5cm}|p{1.4cm}|p{1.4cm}|p{1.4cm}|p{1.4cm}|p{1.4cm}|p{1.4cm}|p{1.4cm}|}
				\hline
				\multicolumn{8}{|c|}{$T=27$ °C} \\
				\hline
				$k$ & 0 & 1 & 2 & 3 & 4 & 5 & 6 \\
				\hline
				\end{tabular}
			\end{center}
		\end{table}
		\begin{table}[H]
			\begin{center}
				\begin{tabular}{|p{2.5cm}|p{1.4cm}|p{1.4cm}|p{1.4cm}|p{1.4cm}|p{1.4cm}|p{1.4cm}|p{1.4cm}|}
				\hline
				$f^{+}$, Гц & 449 & 663 & 880 & 1096 & 1315 & 1535 & 1750 \\
				\hline
				$f^{-}$, Гц & 450 & 661 & 881 & 1098 & 1313 & 1535 & 1751 \\
				\hline
				$f_{\text{ср}}$, Гц & 449.5 & 662.0 & 880.5 & 1097.0 & 1314.0 & 1535.0 & 1751.5 \\
				\hline
				$\sigma_{\text{сл}}$ & 0.5 & 1 & 0.5 & 1 & 1 & 0 & 0.5 \\
				\hline
				$\sigma_{\text{пол}}$ & 1.1 & 1.4 & 1.1 & 1.4 & 1.4 & 1.0 & 1.1 \\
				\hline 
				$f_{k+1}-f_1$, Гц & 0.0 & 212.5 & 431.0 & 647.5 & 864.5 & 1085.5 & 1301.0 \\
				\hline 
				\multicolumn{8}{|c|}{$T=35$ °C} \\
				\hline
				$k$ & 0 & 1 & 2 & 3 & 4 & 5 & 6 \\
				\hline
				$f^{+}$, Гц & 456 & 672 & 888 & 1107 & 1326 & 1550 & 1772 \\
				\hline
				$f^{-}$, Гц & 457 & 671 & 888 & 1106 & 1327 & 1549 & 1774 \\
				\hline
				$f_{\text{ср}}$, Гц & 456.5 & 671.5 & 888 & 1106.5 & 1326.5 & 1549.5 & 1773.0 \\
				\hline
				$\sigma_{\text{сл}}$ & 0.5 & 0.5 & 0 & 0.5 & 0.5 & 0.5 & 1 \\
				\hline
				$\sigma_{\text{пол}}$ & 1.1 & 1.1 & 1.0 & 1.1 & 1.1 & 1.1 & 1.4 \\
				\hline
				$f_{k+1}-f_1$, Гц & 0.0 & 215.0 & 431.5 & 650.0 & 870.0 & 1093.0 & 1316.5 \\
				\hline
				\multicolumn{8}{|c|}{$T=45$ °C} \\
				\hline
				$k$ & 0 & 1 & 2 & 3 & 4 & 5 & 6 \\
				\hline
				$f^{+}$, Гц & 462 & 685 & 904 & 1127 & 1352 & 1575 & 1800 \\
				\hline
				$f^{-}$, Гц & 463 & 684 & 902 & 1128 & 1352 & 1576 & 1801 \\
				\hline
				$f_{\text{ср}}$, Гц & 462.5 & 684.5 & 903.0 & 1127.5 & 1352.0 & 1575.5 & 1800.5 \\
				\hline
				$\sigma_{\text{сл}}$ & 0.5 & 0.5 & 1 & 0.5 & 0 & 0.5 & 0.5 \\
				\hline
				$\sigma_{\text{пол}}$ & 1.1 & 1.1 & 1.4 & 1.1 & 1.0 & 1.1 & 1.1 \\
				\hline
				$f_{k+1}-f_1$, Гц & 0.0 & 222.0 & 440.5 & 665.0 & 889.5 & 1113.0 & 1338 \\
				\hline
				\multicolumn{8}{|c|}{$T=55$ °C} \\
				\hline
				$k$ & 0 & 1 & 2 & 3 & 4 & 5 & 6 \\
				\hline
				$f^{+}$, Гц & 468 & 693 & 920 & 1145 & 1372 & 1600 & 1827 \\
				\hline
				$f^{-}$, Гц & 469 & 693 & 922 & 1144 & 1373 & 1599 & 1826 \\
				\hline
				$f_{\text{ср}}$, Гц & 468.5 & 693.0 & 921.0 & 1144.5 & 1372.5 & 1599.5 & 1826.5 \\
				\hline
				$\sigma_{\text{сл}}$ & 0.5 & 0 & 1 & 0.5 & 0.5 & 0.5 & 0.5 \\
				\hline
				$\sigma_{\text{пол}}$ & 1.1 & 1.0 & 1.4 & 1.1 & 1.1 & 1.1 & 1.1 \\
				\hline
				$f_{k+1}-f_1$, Гц & 0.0 & 224.5 & 452.5 & 676.0 & 904.0 & 1131.0 & 1358.0 \\
				\hline
				\end{tabular}
			\end{center}
		\end{table}
		\par Случайная и полная погрешности были оценены по формулам
		\begin{equation}
			\sigma_{\text{сл}}=\sqrt{\frac{1}{n(n-1)}\sum{(f_i-f_{\text{ср}})^2}}, \ \ \ \ \ 
			\sigma_{\text{пол}}=\sqrt{\sigma_{\text{сл}}^2+\Delta_{\text{ген}}^2}
		\end{equation}
		\subsection{Определение скорости звука и $C_p/C_v$ при разных температурах}
		\par Полученные результаты изобразим на графике, откладывая по оси абсцисс номер резонанса $k$, а по оси
		ординат --- разность между частотой последующих резонансов и частотой первого резонанса: $f_{k+1}-f_1$. Через
		полученные точки проведём наилучшую прямую, используя метод наименьших квадратов (фомула (8)). Угловой
		коэффициент прямой определяет величину $c/2L$ (см. формулу (6)). Вычислите значения скорости звука при разных
		температурах. Погрешности вычислений оценим по формуле (9). Результаты изобразим на рис. 2.
		\begin{equation}
			k=\frac{\langle xy \rangle}{\langle x^2 \rangle}
		\end{equation}
		\begin{equation}
			\sigma_k=\frac{1}{\sqrt{n}}\sqrt{\frac{\langle y^2 \rangle}{\langle x^2 \rangle}-k^2}
		\end{equation}
		\begin{figure}[H]
			\begin{minipage}[h]{0.49\linewidth}
				\begin{center}
					\input{C:/Users/gromo/PycharmProjects/plottest/plot15.tex}\\
				\end{center}
			\end{minipage}
			\begin{minipage}[h]{0.49\linewidth}	
				\begin{center}
					\input{C:/Users/gromo/PycharmProjects/plottest/plot16.tex}\\ 
				\end{center}
			\end{minipage}
			\hfill
			\begin{minipage}[h]{0.49\linewidth}
				\begin{center}
					\input{C:/Users/gromo/PycharmProjects/plottest/plot17.tex} \\ 
				\end{center}
			\end{minipage}	
			\begin{minipage}[h]{0.49\linewidth}
				\begin{center}
					\input{C:/Users/gromo/PycharmProjects/plottest/plot18.tex} \\
				\end{center}
			\end{minipage}
			\captionsetup{font={small}, justification=justified}
			\caption[figure]{Графики зависимости $f_{k+1}-f_1$ от $k$ при разных температурах}
		\end{figure}
		\par Вычислим теперь скорость звука и отношение $C_p/C_v$. Примем молярную массу воздуха равной $\mu=0.029$
		кг/моль. Результаты вычислений занесём в таблицу 2.
		\begin{table}[H]
			\captionsetup{font={small}, labelformat=fullparents, labelsep=fill, labelfont=bf, justification=raggedleft,
			singlelinecheck=false, skip=-0.2cm}
			\caption{Значения резонансных частот}
			\begin{center}
				\begin{tabular}{|p{1.5cm}|p{2.4cm}|p{2.4cm}|p{2.4cm}|p{2.4cm}|}
				\hline
				$T$, °C & 27 & 35 & 45 & 55 \\
				\hline
				$c$, м/с & $346.5\pm0.4$ & $349.4\pm0.7$ & $356.0\pm0.4$ & $361.8\pm0.2$ \\
				\hline
				$C_p/C_v$ & $1.396\pm0.003$ & $1.383\pm0.006$ & $1.390\pm0.003$ & $1.392\pm0.002$ \\
				\hline
				\end{tabular}
			\end{center}
		\end{table}
		\par Погрешность определения $C_p/C_v$ определяется по формуле
		\begin{equation}
			\sigma_{C_p/C_v}=C_p/C_v\cdot2\mathcal{E}_{c}
		\end{equation}
	\section{Обсуждение результатов}
		\par В данной лабораторной работе мы исследовали величину скорости звука в воздухе и определяли отношение
		$C_p/C_v$. Проведенные нами измерения были очень точными (погрешность составляла порядка 0.1\%). Погрешности
		велечин, измеряемых косвенным методом, обладают тем же порядком. Все полученные результаты с хорошей точностью
		согласуются с табличными значениями. Главным источником неточностей может быть звукой генератор, так как он
		доавльн сложен в испльзовании и некоторые измерения производились с большим трудом.
	\section{Вывод}
		\par Все цели работы выполнены и достигнуты ожидаемые результаты. Единственным улучшением в работе может стать
		обновление звукого генератора. Дальнейшим развитием работы может стать определение $\gamma$ при более высоких
		температурах и наблюдение активации дополнительных степеней свободы.
		
		
\end{document}
\documentclass[12pt,a4paper]{article}
\usepackage[T2A]{fontenc}
\usepackage[utf8]{inputenc}
\usepackage[english, russian]{babel}
\usepackage{indentfirst}
\usepackage{misccorr}
\usepackage{graphicx}
\usepackage{amsmath}
\usepackage{amssymb}
\usepackage{circuitikz}
\usepackage[font={small}]{caption}
\usepackage[left=20mm, top=20mm, right=20mm, bottom=20mm, nohead]{geometry}
\usepackage{float}
\usepackage{tabularx}
\usepackage{array}
\usepackage{longtable}
\usepackage{pstool}
\usepackage{pgfplots}
\usepackage{hhline}
\usepackage{multirow}
\usepackage{wrapfig}
\usepackage{pdfpages}

\DeclareCaptionLabelSeparator{fill}{.\\}

\DeclareCaptionLabelFormat{fullparents}{\bothIfFirst{#1}{~}#2}

\pgfplotsset{compat=1.17}

\begin{document}

	\begin{titlepage}
    	\begin{center}
            {\LARGE Отчёт по лабораторной работе 4.2.1.\\}
                \vspace*{11cm}
                        \textbf{\LARGE Кольца Ньютона.}
                \vspace*{6.5cm}
        \end{center}
        \hfill\begin{minipage}{0.37\textwidth}
            	Работу выполнил Громов Артём
                \\
                ЛФИ Б02-006
        \end{minipage}
        \vspace{4.8cm}
        \begin{center}
                Долгопрудный, 2022 г.
        \end{center}
	\end{titlepage}

	\section{Аннотация}

		\begin{flushleft}
        	\textbf{Цель работы:} познакомиться с явлением интерференции в тонких плёнках (полосы равной толщины) на примере колец
			Ньютона и с методикой интерференционных измерений кривизны стеклянной поверхности.
        \end{flushleft}
        \begin{flushleft}
                \textbf{В работе используется:} измерительный микроскоп с опак-иллюминатором; плосковыпуклая линза; пластинка из чёрного
				стекла; ртутная лампа ДРШ; щель; линзы; призма прямого зрения; объектная шкала.
        \end{flushleft}
		\begin{wrapfigure}{r}{0.5\textwidth}
			\centering
			\includegraphics[width=0.5\textwidth]{images/pic1.pdf}
			\caption{Схема наблюдения колец Ньютона.}
		\end{wrapfigure}
		\par Этот классический опыт используется для определения радиуса кривизны сферических поверхностей линз. В нём наблюдается
		интерференция волн, отражённых от границ тонкой воздушной прослойки, образованной сферической поверхностью линзы и плоской
		стеклянной пластиной. При нормальном падении света (рис. 1) интерференционные полосы локализованы на сферической поверхности и
		являются полосами равной толщины.
		\par Геометрическая разность хода между интерферирующими лучами равна удвоенной толщине воздушного зазора $2d$ в данном месте. Для
		точки на сферической поверхности, находящейся на расстоянии $r$ от оси системы, имеем $r^2=R^2-(R-d)^2=2Rd-d^2$, где $R$ --- радиус
		кривизны сферической поверхности.
		\par При $R\gg d$ получим $d=r^2/(2R)$. С учётом изменения фазы на $\pi$ при отражении волны от оптически более плотной среды (на
		границе воздух—стекло) получим оптическую разность хода интерферирующих лучей:
		$$
			\Delta=2d+\frac{\lambda}{2}=\frac{r^2}{R}+\frac{\lambda}{2}.
		$$
		\par Условие интерференционного минимума $\Delta=(2m+1)\frac{\lambda}{2}$ $(m=0,1,2,...)$,откуда получаем для радиусов тёмных колец
		\begin{equation}
			r_m=\sqrt{m\lambda R}.
		\end{equation}
		Аналогично для радиусов $r'_m$ светлых колец
		\begin{equation}
			r'_m=\sqrt{(2m-1)m\lambda R/2}.
		\end{equation}
		\par Для протяжённого источника линии равной толщины локализованы на поверхности линзы, если пластинка лежит на линзе, и вблизи
		поверхности линзы, если линза лежит на пластинке, как в нашем случае. Наблюдение ведётся в отражённом свете.

	\section{Экспериментальная установка}
		
		\par Схема экспериментальной установки приведена на рис. 2. Опыт выполняется с помощью измерительного микроскопа. На столике
		микроскопа помещается держатель с пластинкой чёрного стекла. На пластинке лежит исследуемая линза.
		\begin{figure}[H]
			\begin{center}
				\includegraphics[width=0.9\textwidth]{images/pic2.pdf}
				\caption{Схема установки для наблюдения колец Ньютона.}
			\end{center}
		\end{figure}
		\par Источником света служит ртутная лампа, находящаяся в защитном кожухе. Для получения монохроматического света применяется
		призменный монохроматор, состоящий из конденсора $K$, коллиматора (щель $S$ и объектив $O$) и призмы прямого зрения П. Эти
		устройства с помощью рейтеров располагаются на оптической скамье. Свет от монохроматора попадает на опак-иллюминатор (ОИ),
		расположенный между окуляром и объективом микроскопа --- специальное устройство для освещения объекта при работе в отражённом свете.
		Внутри опак- иллюминатора находится полупрозрачная пластинка $P$, наклоненная под углом 45$^\circ$ к оптической оси микроскопа. Свет
		частично отражается от этой пластинки, проходит через объектив микроскопа и попадает на исследуемый объект. Пластинка может
		поворачиваться вокруг горизонтальной оси $x$, а опак-иллюминатор --- вокруг вертикальной оси.
		\par Столик микроскопа может перемещаться в двух взаимно перпендикулярных направлениях с помощью винтов препаратоводителя. Отсчётный
		крест окулярной шкалы перемещается перпендикулярно оптической оси микроскопа с помощью микрометрического винта.
		\par Оптическая схема монохроматора позволяет получить в плоскости входного окна опак-иллюминатора достаточно хорошо разделённые линии
		спектра ртутной лампы. Изображение щели $S$ фокусируется на поверхность линзы объективом микроскопа, и в том же месте находится
		плоскость наблюдения микроскопа, т.е. точка источника и точка наблюдения интерференции совпадают. Картина интерференции как и в
		случае расположения пластинки сверху, так и в данном случае не зависит от коэффициента преломления линзы и определяется величиной
		зазора между нижней поверхностью линзы и стеклянной пластинкой.
		\par Сначала микроскоп настраивается на кольца Ньютона в белом свете, а затем из спектра ртутной лампы
		выделяется зелёная линия и проводятся измерения в монохроматическом свете.
		\par\textbf{Наблюдение «биений»}. При освещении системы светом, содержа-щим две спектральные компоненты, наблюдается характерная
		картина биений: чёткость интерференционных колец периодически изменяется. Это объясняется наложением двух систем интерференционных
		колец, возникающих для разных длин волн $\lambda_1$ и $\lambda_2$. Чёткие кольца в результирующей картине образуются при наложении
		светлых колец на светлые и тёмных на тёмные. Размытые кольца получаются при наложении светлых колец одной картины на тёмные кольца
		другой.
		\par Нетрудно рассчитать период возникающих биений. Пусть в промежутке между двумя центрами соседних чётких участков укладывается
		$\Delta m$ колец для спектральной линии с длиной волны $\lambda_1$. Тогда в этом промежутке должно располагаться $\Delta m+1$ колец
		для спектральной линии с длиной волны $\lambda_2$ (при $\lambda_2$ < $\lambda_1$). В итоге получим формулу для разности частот:
		\begin{equation}
			\Delta \lambda=\frac{\lambda_1}{m+1}.
		\end{equation}

	\section{Результаты измерений и обработка данных}

	\section{Обсуждение результатов}

	\section{Вывод}

\end{document}

\documentclass[12pt,a4paper]{article}
\usepackage[utf8]{inputenc}
\usepackage[english,russian]{babel}
\usepackage{indentfirst}
\usepackage{misccorr}
\usepackage{graphicx}
\usepackage{amsmath}
\usepackage{amssymb}
\usepackage{circuitikz}
\usepackage[font={small}]{caption}
\usepackage[left=20mm, top=20mm, right=20mm, bottom=20mm, nohead]{geometry}
\usepackage{float}
\usepackage{tabularx}
\usepackage{array}
\usepackage{longtable}
\usepackage{pstool}
\usepackage{pgfplots}
\usepackage{hhline}
\usepackage{multirow}
\usepackage{wrapfig}

\DeclareCaptionLabelSeparator{fill}{.\\}
\DeclareCaptionLabelFormat{fullparents}{\bothIfFirst{#1}{~}#2}

\begin{document}

	\begin{titlepage}
		\begin{center}
			{\LARGE Отчёт по лабораторной работе 2.3.1.\\}
			\vspace*{11cm}
				\textbf{\LARGE Эффект Холла в металлах.}	
			\vspace*{6.5cm}
		\end{center}
		\hfill\begin{minipage}{0.37\textwidth}
				Работу выполнил Громов Артём\\
				ЛФИ Б02-006
		\end{minipage}
		\vspace{5cm}
		\begin{center}
 			Долгопрудный, 2021 г.
		\end{center}   
	\end{titlepage}
	
	\section{Аннотация}
		\noindent\textbf{Цель работы:} измерение подвижности и концентрации носителей заряда в металлах.
		\noindent\textbf{В работе используются: } электромагнит с источником питания, источник постоянного тока,
		микровольтметр, амперметры, цифровой магнитометр, образцы из серебра и цинка.
		\vspace{0.5 cm}
		\par В работе изучаются особенности проводимости металлов в геометрии мостика Холла. Ток пропускается по
		плоской прямоугольной металлической пластинке, помещённой в перпендикулярное пластинке магнитное
		поле. Измеряется разность потенциалов между краями пластинки в поперечном к току направлении. По
		измерениям определяется константа Холла, тип проводимости (электронный или дырочный) и на основе
		соотношения $R_{H}=\frac{1}{nq}$ вычисляется концентрация основных носителей заряда.
	\section{Экспериментальная установка}
		\begin{wrapfigure}{l}{0.6\textwidth} % l - слева, r - справа, с - центр
			\centering %выравнивание по центру
			\includegraphics[width=0.59\textwidth]{C:/Users/gromo/Documents/TeX/Лабораторная 3.3.5/Ресурсы/Установка.jpg} %scale масштаб
			\caption{Схема установки для исследования эффекта Холла в металла} % подрисуночная подпись
			\label{pic:my} %метка для ссылки по тексту
		\end{wrapfigure}
		\par Электрическая схема установки для измерения ЭДС Холла представлена на рис. 1. В зазоре электромагнита
		(рис. 1а) создаётся постоянное магнитное поле, величину которого можно менять с помощью источника питания
		электромагнита. Разъём $\text{К}_{1}$ позволяет менять направление тока в обмотках электромагнита. Ток
		питания электромагнита измеряется амперметром $\text{А}_{1}$.
		\par Градуировка электромагнита (связь тока с индукцией поля) проводится при помощи миллитесламетра на
		основе датчика Холла.
		\par Металлические образцы в форме тонких пластинок, смонтированные в специальных держателях,
		подключаются к блоку питания через разъём (рис. 1б). Ток через образец регулируется реостатом $R_2$ и
		измеряется амперметром $\text{А}_{2}$.
		\par Для измерений ЭДС Холла используется микровольтметр, в котором высокая чувствительность по
		напряжению сочетается с малой величиной тока, потребляемого измерительной схемой (например, для
		микровольтметра Ф116/1 минимальный предел измерения напряжения составляет 1 мкВ, а потребляемый ток ---
		всего $10^{-8}$ А).
		\par В образце с током, помещённом в зазор электромагнита, между контактами 2 и 4 возникает холловская
		разность потенциалов $U_{\perp}$, которая измеряется с помощью микровольтметра, если переключатель
		$\text{К}_{3}$  подключён к точке 2 образца. При подключении $\text{К}_{3}$ к точке 3 микровольтметр
		измеряет омическое падение напряжения $U_{34}$, вызванное током через образец. При нейтральном положении
		ключа входная цепь микровольтметра разомкнута.
		\par Ключ $\text{К}_{2}$ позволяет менять полярность напряжения, поступающего на вход микровольтметра.
		\par Контакты 2 и 4 вследствие неточности подпайки могут лежать не на одной эквипотенциали. Тогда
		напряжение между ними связано не только с эффектом Холла, но и с омическим падением напряжения вдоль
		пластинки. Исключить этот эффект можно, изменяя направление магнитного поля, пронизывающего образец. При
		обращении поля ЭДС Холла меняет знак, а омическое падение напряжения остаётся неизменным. Поэтому ЭДС
		Холла $U_{\perp}$ может быть определена как половина алгебраической разности показаний вольтметра,
		полученных для двух противоположных направлений магнитного поля в зазоре: $U_{\perp}=\frac{1}{2}\left(
		U^{(+)}_{34}-U^{(-)}_{34}\right)$.
		\par Альтернативно можно исключить влияние омического падения напряжения, если при каждом значении тока
		через образец измерять напряжение между точками 3 и 4 в отсутствие магнитного поля. При фиксированном токе
		через образец это дополнительное к ЭДС Холла напряжение $U_0$ остаётся неизменным. От него следует (с
		учётом знака) отсчитывать величину ЭДС Холла:
		\begin{equation}
			U_{\perp}=U_{24}-U_{0}.
		\end{equation}
		\noindent При таком способе измерения нет необходимости проводить повторные измерения с противоположным
		направлением магнитного поля.
		\par По знаку $U_{\perp}$ можно определить характер проводимости --- электронный или дырочный. Для этого
		необходимо знать направление тока в образце и направление магнитного поля.
		\par Измерив ток $I$ в образце и напряжение $U_34$ между контактами 3 и 4 в отсутствие магнитного поля,
		можно, зная параметры образца, рассчитать удельное сопротивление $\rho_0$ и проводимость $\sigma_0$
		материала образца по формуле
		\begin{equation}
			\rho_0=\frac{U_{34}ah}{Il},
		\end{equation}
		\noindent где $l$ --- расстояние между контактами 3 и 4, $a$ --- ширина образца, $h$ --- его толщина.
	\section{Результаты измерений и обработка данных}
		\subsection{Определение калибровочной кривой электромагнита}
		\par Измерим с помощью миллитесламетра магнитное поле в зазоре при разных токах с шагом в 0.1 А.
		Полученные результаты запишем в таблицу 1. Также построим график зависимости $B$ от $I_{\text{М}}$
		(рис. 2).
		\par Как видно из графика при $I_{\text{М}}>0.7\text{ А}$ зависимость перестаёт быть линейной. Поэтому
		рассмотрим диапазон токов до 0.7 А и с помощью метода наименьших квадратов определим значение
		коэффициента k, связывающего силу тока и магнитную индукцию:
		\begin{equation}
			k=\frac{\langle I_{\text{М}}B \rangle}{\langle I_{\text{М}}^{2} \rangle},
		\end{equation}
		\begin{equation} 
			\sigma_{k}=\sqrt{\frac{1}{n-1}\left(\frac{\langle B^2 \rangle}{\langle I_{\text{М}}^2 \rangle}-k^{2}\right)}.
		\end{equation}
		\par Произведя все вычисления, приходим к следующим результатам: $k=1185\pm7$ мТл/А.
		\begin{figure}[H]
		\begin{minipage}[h]{0.3\textwidth}
			\begin{table}[H]
			\captionsetup{font={small}, labelformat=fullparents, labelsep=fill, labelfont=bf, justification=raggedright,
			singlelinecheck=false, skip=0.3cm}
			\caption{Значения $I_{\text{М}}$ и $B$}
				\begin{tabular}{|p{1.3cm}|p{1.4cm}|}
				\hline
				$I_{\text{М}}$, А & $B$, мТл \\
				\hline
				0.1 & 119.3 \\
				\hline
				0.2 & 237.3 \\
				\hline
				0.3 & 361.1 \\
				\hline
				0.4 & 485.1 \\
				\hline
				0.5 & 603.7 \\
				\hline
				0.6 & 709.9 \\
				\hline
				0.7 & 814.6 \\
				\hline
				0.8 & 883.9 \\
				\hline
				0.9 & 930.2 \\
				\hline
				1.0 & 970.1 \\
				\hline
				1.1 & 1006.8 \\
				\hline
				1.26 & 1038.5 \\
				\hline
				\end{tabular}
			\end{table}
		\end{minipage}
		\begin{minipage}[h]{0.69\textwidth}
			\begin{center}
				% This file was created with tikzplotlib v0.9.16.
\begin{tikzpicture}

\begin{axis}[
legend cell align={left},
legend style={
  fill opacity=0.8,
  draw opacity=1,
  text opacity=1,
  at={(0.03,0.97)},
  anchor=north west,
  draw=white!80!black
},
height=11.2cm,
tick align=inside,
major tick length=0.2cm,
minor tick length=0.1cm,
tick pos=left,
xmin=0, xmax=600,
xtick={0, 50, 100, 150, 200, 250, 300, 350, 400, 450, 500, 550, 600},
minor x tick num=4,
xmajorgrids,
minor x grid style={dotted,black},
xminorgrids,
xtick style={color=black},
xlabel={$D_{\text{винта}}$, мкм},
ymin=0, ymax=500,
ytick={0, 50, 100, 150, 200, 250, 300, 350, 400, 450, 500},
ytick style={color=black},
minor y tick num=4,
ymajorgrids,
minor y grid style={dotted,black},
yminorgrids,
ytick style={color=black},
ylabel={$D_{\text{расчёт}}$, мкм}
]
\path [draw=red, semithick]
(axis cs:40,32.9861111111111)
--(axis cs:60,32.9861111111111);

\path [draw=red, semithick]
(axis cs:90,98.9583333333333)
--(axis cs:110,98.9583333333333);

\path [draw=red, semithick]
(axis cs:140,131.944444444444)
--(axis cs:160,131.944444444444);

\path [draw=red, semithick]
(axis cs:190,164.930555555556)
--(axis cs:210,164.930555555556);

\path [draw=red, semithick]
(axis cs:240,197.916666666667)
--(axis cs:260,197.916666666667);

\path [draw=red, semithick]
(axis cs:290,230.902777777778)
--(axis cs:310,230.902777777778);

\path [draw=red, semithick]
(axis cs:340,263.888888888889)
--(axis cs:360,263.888888888889);

\path [draw=red, semithick]
(axis cs:390,296.875)
--(axis cs:410,296.875);

\path [draw=red, semithick]
(axis cs:440,362.847222222222)
--(axis cs:460,362.847222222222);

\path [draw=red, semithick]
(axis cs:490,395.833333333333)
--(axis cs:510,395.833333333333);

\path [draw=red, semithick]
(axis cs:50,-0.00334608800363867)
--(axis cs:50,65.9755683102259);

\path [draw=red, semithick]
(axis cs:100,65.9421196383399)
--(axis cs:100,131.974547028327);

\path [draw=red, semithick]
(axis cs:150,98.9048365902785)
--(axis cs:150,164.98405229861);

\path [draw=red, semithick]
(axis cs:200,131.860893814377)
--(axis cs:200,198.000217296734);

\path [draw=red, semithick]
(axis cs:250,164.810309447695)
--(axis cs:250,231.023023885639);

\path [draw=red, semithick]
(axis cs:300,197.75310554622)
--(axis cs:300,264.052450009336);

\path [draw=red, semithick]
(axis cs:350,230.689308019177)
--(axis cs:350,297.088469758601);

\path [draw=red, semithick]
(axis cs:400,263.618946552474)
--(axis cs:400,330.131053447526);

\path [draw=red, semithick]
(axis cs:450,329.458668896105)
--(axis cs:450,396.235775548339);

\path [draw=red, semithick]
(axis cs:500,362.368830131627)
--(axis cs:500,429.29783653504);

\addplot [semithick, red, forget plot]
table {%
0 0
500 392.40620488787
};
\path [draw=green!50.1960784313725!black, semithick]
(axis cs:90,53.0442477876106)
--(axis cs:110,53.0442477876106);

\path [draw=green!50.1960784313725!black, semithick]
(axis cs:140,95.1428571428572)
--(axis cs:160,95.1428571428572);

\path [draw=green!50.1960784313725!black, semithick]
(axis cs:190,148)
--(axis cs:210,148);

\path [draw=green!50.1960784313725!black, semithick]
(axis cs:240,188.913461538462)
--(axis cs:260,188.913461538462);

\path [draw=green!50.1960784313725!black, semithick]
(axis cs:290,233.740384615385)
--(axis cs:310,233.740384615385);

\path [draw=green!50.1960784313725!black, semithick]
(axis cs:340,270.260869565217)
--(axis cs:360,270.260869565217);

\path [draw=green!50.1960784313725!black, semithick]
(axis cs:390,326.204081632653)
--(axis cs:410,326.204081632653);

\path [draw=green!50.1960784313725!black, semithick]
(axis cs:440,360)
--(axis cs:460,360);

\path [draw=green!50.1960784313725!black, semithick]
(axis cs:490,396)
--(axis cs:510,396);

\path [draw=green!50.1960784313725!black, semithick]
(axis cs:540,468)
--(axis cs:560,468);

\path [draw=green!50.1960784313725!black, semithick]
(axis cs:100,52.4114526583863)
--(axis cs:100,53.6770429168349);

\path [draw=green!50.1960784313725!black, semithick]
(axis cs:150,91.0666302830339)
--(axis cs:150,99.2190840026804);

\path [draw=green!50.1960784313725!black, semithick]
(axis cs:200,141.683425571105)
--(axis cs:200,154.316574428895);

\path [draw=green!50.1960784313725!black, semithick]
(axis cs:250,182.271308845487)
--(axis cs:250,195.555614231436);

\path [draw=green!50.1960784313725!black, semithick]
(axis cs:300,223.88999208276)
--(axis cs:300,243.590777148009);

\path [draw=green!50.1960784313725!black, semithick]
(axis cs:350,257.145804132173)
--(axis cs:350,283.375934998262);

\path [draw=green!50.1960784313725!black, semithick]
(axis cs:400,311.969339856676)
--(axis cs:400,340.438823408631);

\path [draw=green!50.1960784313725!black, semithick]
(axis cs:450,341.133063042629)
--(axis cs:450,378.866936957371);

\path [draw=green!50.1960784313725!black, semithick]
(axis cs:500,376.956019710008)
--(axis cs:500,415.043980289992);

\path [draw=green!50.1960784313725!black, semithick]
(axis cs:550,448.557402072788)
--(axis cs:550,487.442597927212);

\addplot [semithick, green!50.1960784313725!black, forget plot]
table {%
0 0
550 439.564038903656
};
\addplot [semithick, red, mark=*, mark size=3, mark options={solid}, only marks]
table {%
50 32.9861111111111
100 98.9583333333333
150 131.944444444444
200 164.930555555556
250 197.916666666667
300 230.902777777778
350 263.888888888889
400 296.875
450 362.847222222222
500 395.833333333333
};
\addlegendentry{Линза}
\addplot [semithick, green!50.1960784313725!black, mark=*, mark size=3, mark options={solid}, only marks]
table {%
100 53.0442477876106
150 95.1428571428572
200 148
250 188.913461538462
300 233.740384615385
350 270.260869565217
400 326.204081632653
450 360
500 396
550 468
};
\addlegendentry{Спектр}
\end{axis}

\end{tikzpicture}

				\caption{График зависимости $B$ от $I_{\text{М}}$} 
			\end{center}
		\end{minipage}
		\end{figure}
		\par В силу проблем с интерполированием в области высысоких токов будем проводить эксперимент при токах
		меньше 0.7 А.
		\subsection{Исследование серебренного образца}
		\par Проведём исследование серебрянного образца. Для этого будем при фиксированном токе через образец
		исследованить зависимость поперечной разницы потенциалов от тока через обмотку электромагнита. Данный
		опыт повторим при нескольких значениях тока через образец. При максимальном токе проведём опыт в обратном
		порядке (уменьшая ток в обмотке). Испульзуя формулу (1) и формулу $B=kI_{\text{М}}$ вычислим $B$ и
		$U_{\perp}$. Данные занесём в таблицу 2.
		\begin{table}[H]
			\captionsetup{font={small}, labelformat=fullparents, labelsep=fill, labelfont=bf, justification=raggedleft,
			singlelinecheck=false, skip=-0.2cm}
			\caption{Измерение зависимости $U_{\perp}$ от $B$ при разных токах}
			\begin{center}
				\begin{tabular}{|p{1.7cm}|p{1.4cm}|p{1.4cm}|p{1.4cm}|p{1.4cm}|p{1.4cm}|p{1.4cm}|p{1.4cm}|}
					\hline
					\multicolumn{8}{|c|}{$I=0.2$ А, $U_0=-0.08$ мкВ} \\
					\hline
					$I_{\text{М}}$, А & 0.1 & 0.2 & 0.3 & 0.4 & 0.5 & 0.6 & 0.7 \\
					\hline
					$B$, мТл & 118.5 & 237.1 & 355.6 & 474.2 & 592.7 & 711.3 & 829.8 \\
					\hline
					$U_{24}$, мкВ & -0.08 & -0.08 & -0.08 & -0.08 & -0.04 & 0.0 & 0.0 \\
					\hline
					$U_{\perp}$, мкВ & 0.0 & 0.0 & 0.0 & 0.0 & 0.04 & 0.08 & 0.08 \\
					\hline
					\multicolumn{8}{|c|}{$I=0.4$ А, $U_0=-0.2$ мкВ} \\
					\hline
					$I_{\text{М}}$, А & 0.1 & 0.2 & 0.3 & 0.4 & 0.5 & 0.6 & 0.7 \\
					\hline
					$B$, мТл & 118.5 & 237.1 & 355.6 & 474.2 & 592.7 & 711.3 & 829.8 \\
					\hline
					$U_{24}$, мкВ & -0.16 & -0.14 & -0.1 & -0.06 & -0.02 & 0.0 & 0.04 \\
					\hline
					$U_{\perp}$, мкВ & 0.04 & 0.06 & 0.1 & 0.14 & 0.18 & 0.2 & 0.24 \\
					\hline
					\multicolumn{8}{|c|}{$I=0.6$ А, $U_0=-0.2$ мкВ} \\
					\hline
					$I_{\text{М}}$, А & 0.1 & 0.2 & 0.3 & 0.4 & 0.5 & 0.6 & 0.7 \\
					\hline
					$B$, мТл & 118.5 & 237.1 & 355.6 & 474.2 & 592.7 & 711.3 & 829.8 \\
					\hline
					$U_{24}$, мкВ & -0.16 & -0.1 & -0.04 & 0.02 & 0.1 & 0.16 & 0.22 \\
					\hline
				\end{tabular}
			\end{center}
		\end{table}
		\begin{table}[H]
			\captionsetup{font={small}, labelformat=fullparents, labelsep=fill, labelfont=bf, justification=raggedleft,
			singlelinecheck=false, skip=-0.2cm}
			\begin{center}
				\begin{tabular}{|p{1.7cm}|p{1.4cm}|p{1.4cm}|p{1.4cm}|p{1.4cm}|p{1.4cm}|p{1.4cm}|p{1.4cm}|}
					\hline
					$U_{\perp}$, мкВ & 0.04 & 0.1 & 0.16 & 0.22 & 0.3 & 0.36 & 0.42 \\
					\hline
					\multicolumn{8}{|c|}{$I=0.8$ А, $U_0=-0.28$ мкВ} \\
					\hline
					$I_{\text{М}}$, А & 0.1 & 0.2 & 0.3 & 0.4 & 0.5 & 0.6 & 0.7 \\
					\hline
					$B$, мТл & 118.5 & 237.1 & 355.6 & 474.2 & 592.7 & 711.3 & 829.8 \\
					\hline
					$U_{24}$, мкВ  & -0.2 & -0.12 & -0.04 & 0.04 & 0.12 & 0.2 & 0.28 \\
					\hline
					$U_{\perp}$, мкВ & 0.08 & 0.16 & 0.24 & 0.32 & 0.4 & 0.48 & 0.56 \\
					\hline
					\multicolumn{8}{|c|}{$I=1.0$ А, $U_0=-0.08$ мкВ} \\
					\hline
					$I_{\text{М}}$, А & 0.1 & 0.2 & 0.3 & 0.4 & 0.5 & 0.6 & 0.7 \\
					\hline
					$B$, мТл & 118.5 & 237.1 & 355.6 & 474.2 & 592.7 & 711.3 & 829.8 \\
					\hline
					$U_{24}$, мкВ & 0.04 & 0.14 & 0.24 & 0.34 & 0.44 & 0.54 & 0.64 \\
					\hline
					$U_{\perp}$, мкВ & 0.12 & 0.22 & 0.32 & 0.42 & 0.52 & 0.62 & 0.72 \\
					\hline
					\multicolumn{8}{|c|}{$I=1$ А, $U_0=-0.04$ мкВ} \\
					\hline
					$I_{\text{М}}$, А & -0.1 & -0.2 & -0.3 & -0.4 & -0.5 & -0.6 & -0.7 \\
					\hline
					$B$, мТл & -118.5 & -237.1 & -355.6 & -474.2 & -592.7 & -711.3 & -829.8 \\
					\hline
					$U_{24}$, мкВ & -0.16 & -0.28 & -0.36 & -0.48 & -0.58 & -0.7 & -0.82 \\
					\hline
					$U_{\perp}$, мкВ & -0.12 & -0.24 & -0.32 & -0.44 & -0.54 & -0.66 & -0.78 \\
					\hline
				\end{tabular}
			\end{center}
		\end{table}
		\par С помощью МНК для каждого опыта рассчитаем величину $g=dU_{\perp}/dB$ и
		оценим её погрешность, данные занесём в таблицу 3. Построим графики $U_{\perp}(B)$ (рис. 3).
		\begin{figure}[H]
		\begin{center}
				% This file was created with tikzplotlib v0.9.16.
\begin{tikzpicture}

\begin{axis}[
    height=11cm,
    tick align=inside,
    major tick length=0.2cm,
    minor tick length=0.1cm,
    tick pos=left,
    xmin=-4.1, xmax=4.1,
    xtick={-5, -4, -3, -2, -1, 0, 1, 2, 3, 4, 5},
    minor x tick num=4,
    xmajorgrids,
    minor x grid style={dotted,black},
    xminorgrids,
    xtick style={color=black},
    xlabel={$m$},
    ymin=-1.2, ymax=1.2,
    ytick={-1.5, -1, -0.5, 0, 0.5, 1, 1.5},
    minor y tick num=4,
    ymajorgrids,
    minor y grid style={dotted,black},
    yminorgrids,
    ytick style={color=black},
    ylabel={$x_m$, мм}
]
\path [draw=red, semithick]
(axis cs:1,0.14)
--(axis cs:1,0.18);

\path [draw=red, semithick]
(axis cs:2,0.48)
--(axis cs:2,0.52);

\path [draw=red, semithick]
(axis cs:3,0.7)
--(axis cs:3,0.74);

\path [draw=red, semithick]
(axis cs:4,1.02)
--(axis cs:4,1.06);

\path [draw=red, semithick]
(axis cs:-1,-0.18)
--(axis cs:-1,-0.14);

\path [draw=red, semithick]
(axis cs:-2,-0.5)
--(axis cs:-2,-0.46);

\path [draw=red, semithick]
(axis cs:-3,-0.74)
--(axis cs:-3,-0.7);

\path [draw=red, semithick]
(axis cs:-4,-1.08)
--(axis cs:-4,-1.04);

\addplot [semithick, black]
table {%
1 0.249999999998364
2 0.499999999996728
3 0.749999999995093
4 0.999999999993457
-1 -0.249999999998364
-2 -0.499999999996728
-3 -0.749999999995093
-4 -0.999999999993457
};
\addplot [semithick, red, mark=square*, mark size=3, mark options={solid}, only marks]
table {%
1 0.16
2 0.5
3 0.72
4 1.04
-1 -0.16
-2 -0.48
-3 -0.72
-4 -1.06
};
\end{axis}

\end{tikzpicture}

				\caption{График зависимости $U_{\perp}$ от $B$} 
			\end{center}
		\end{figure}
		\par Построим график зависимости $g(I)$. Для $I=1$ А возьмём среднее значение $g$. Рассчитаем постоянную
		Холла по формуле $R_{\text{H}}=h(dg/dI)$, где $h=0.09$ мм --- толщина пластины.
		\begin{figure}[H]
		\begin{minipage}[h]{0.3\textwidth}
			\begin{table}[H]
			\captionsetup{font={small}, labelformat=fullparents, labelsep=fill, labelfont=bf, justification=raggedright,
			singlelinecheck=false, skip=0.3cm}
			\caption{Значение $g$ при разных токах}
			\begin{tabular}{|p{2cm}|p{2cm}|}
				\hline
				$I$, А & $g$,  мкВ/Тл \\
				\hline 
				0.2 & $0.07\pm0.02$ \\
				\hline
				0.4 & $0.38\pm0.01$ \\
				\hline 
				0.6 & $0.49\pm0.01$ \\
				\hline
				0.8 & $0.67\pm0.0$ \\
				\hline
				1.0 & $0.91\pm0.01$ \\
				\hline
			\end{tabular}
		\end{table}
		\end{minipage}
		\begin{minipage}[h]{0.69\textwidth}
			\begin{center}
				% This file was created with tikzplotlib v0.9.16.
\begin{tikzpicture}

\definecolor{color0}{rgb}{0.12156862745098,0.466666666666667,0.705882352941177}

\begin{axis}[
    height=7.2cm,
    tick align=inside,
    major tick length=0.2cm,
    minor tick length=0.1cm,
    tick pos=left,
    x grid style={white!69.0196078431373!black},
    xmin=-10, xmax=140,
    xtick={-40, 0, 40, 80, 120, 160},
    minor x tick num=3,
    xmajorgrids,
    minor x grid style={dotted,black},
    xminorgrids,
    xtick style={color=black},
    xlabel={$l$, \text{см}},
    ymin = -0.2, ymax = 0.8,
    ytick = {-0.4, 0, 0.4, 0.8},
    ytick style={color=black},
    minor y tick num=3,
    ymajorgrids,
    minor y grid style={dotted,black},
    yminorgrids,
    ytick style={color=black},
    ylabel={$\upsilon_2$}
]
\path [draw=red, semithick]
(axis cs:-6,0.225806451612903)
--(axis cs:-2,0.225806451612903);

\path [draw=red, semithick]
(axis cs:-2,0.6)
--(axis cs:2,0.6);

\path [draw=red, semithick]
(axis cs:0,0.133333333333333)
--(axis cs:4,0.133333333333333);

\path [draw=red, semithick]
(axis cs:2,0.214285714285714)
--(axis cs:6,0.214285714285714);

\path [draw=red, semithick]
(axis cs:4,0.2)
--(axis cs:8,0.2);

\path [draw=red, semithick]
(axis cs:6,0.161290322580645)
--(axis cs:10,0.161290322580645);

\path [draw=red, semithick]
(axis cs:8,0.225806451612903)
--(axis cs:12,0.225806451612903);

\path [draw=red, semithick]
(axis cs:10,0.225806451612903)
--(axis cs:14,0.225806451612903);

\path [draw=red, semithick]
(axis cs:14,0.161290322580645)
--(axis cs:18,0.161290322580645);

\path [draw=red, semithick]
(axis cs:26,0.0967741935483871)
--(axis cs:30,0.0967741935483871);

\path [draw=red, semithick]
(axis cs:36,0.0322580645161291)
--(axis cs:40,0.0322580645161291);

\path [draw=red, semithick]
(axis cs:46,0.0666666666666667)
--(axis cs:50,0.0666666666666667);

\path [draw=red, semithick]
(axis cs:56,0.0322580645161291)
--(axis cs:60,0.0322580645161291);

\path [draw=red, semithick]
(axis cs:66,0.0322580645161291)
--(axis cs:70,0.0322580645161291);

\path [draw=red, semithick]
(axis cs:76,0.0666666666666667)
--(axis cs:80,0.0666666666666667);

\path [draw=red, semithick]
(axis cs:86,0.0322580645161291)
--(axis cs:90,0.0322580645161291);

\path [draw=red, semithick]
(axis cs:96,0.0666666666666667)
--(axis cs:100,0.0666666666666667);

\path [draw=red, semithick]
(axis cs:106,0.0967741935483871)
--(axis cs:110,0.0967741935483871);

\path [draw=red, semithick]
(axis cs:110,0.225806451612903)
--(axis cs:114,0.225806451612903);

\path [draw=red, semithick]
(axis cs:114,0.290322580645161)
--(axis cs:118,0.290322580645161);

\path [draw=red, semithick]
(axis cs:116,0.290322580645161)
--(axis cs:120,0.290322580645161);

\path [draw=red, semithick]
(axis cs:118,0.533333333333333)
--(axis cs:122,0.533333333333333);

\path [draw=red, semithick]
(axis cs:120,0.548387096774194)
--(axis cs:124,0.548387096774194);

\path [draw=red, semithick]
(axis cs:122,0.161290322580645)
--(axis cs:126,0.161290322580645);

\path [draw=red, semithick]
(axis cs:124,0.161290322580645)
--(axis cs:128,0.161290322580645);

\path [draw=red, semithick]
(axis cs:126,0.0967741935483871)
--(axis cs:130,0.0967741935483871);

\path [draw=red, semithick]
(axis cs:128,0.0967741935483871)
--(axis cs:132,0.0967741935483871);

\path [draw=red, semithick]
(axis cs:-4,0.16988541584789)
--(axis cs:-4,0.281727487377916);

\path [draw=red, semithick]
(axis cs:0,0.524575276673435)
--(axis cs:0,0.675424723326565);

\path [draw=red, semithick]
(axis cs:2,0.0799074876436831)
--(axis cs:2,0.186759179022984);

\path [draw=red, semithick]
(axis cs:4,0.152955024080759)
--(axis cs:4,0.27561640449067);

\path [draw=red, semithick]
(axis cs:6,0.143431457505076)
--(axis cs:6,0.256568542494924);

\path [draw=red, semithick]
(axis cs:8,0.108312499224317)
--(axis cs:8,0.214268145936973);

\path [draw=red, semithick]
(axis cs:10,0.16988541584789)
--(axis cs:10,0.281727487377916);

\path [draw=red, semithick]
(axis cs:12,0.16988541584789)
--(axis cs:12,0.281727487377916);

\path [draw=red, semithick]
(axis cs:16,0.108312499224317)
--(axis cs:16,0.214268145936973);

\path [draw=red, semithick]
(axis cs:28,0.0467395826007438)
--(axis cs:28,0.14680880449603);

\path [draw=red, semithick]
(axis cs:38,-0.0148333340228294)
--(axis cs:38,0.0793494630550875);

\path [draw=red, semithick]
(axis cs:48,0.01638351778229)
--(axis cs:48,0.116949815551043);

\path [draw=red, semithick]
(axis cs:58,-0.0148333340228294)
--(axis cs:58,0.0793494630550875);

\path [draw=red, semithick]
(axis cs:68,-0.0148333340228294)
--(axis cs:68,0.0793494630550875);

\path [draw=red, semithick]
(axis cs:78,0.01638351778229)
--(axis cs:78,0.116949815551043);

\path [draw=red, semithick]
(axis cs:88,-0.0148333340228294)
--(axis cs:88,0.0793494630550875);

\path [draw=red, semithick]
(axis cs:98,0.01638351778229)
--(axis cs:98,0.116949815551043);

\path [draw=red, semithick]
(axis cs:108,0.0467395826007438)
--(axis cs:108,0.14680880449603);

\path [draw=red, semithick]
(axis cs:112,0.16988541584789)
--(axis cs:112,0.281727487377916);

\path [draw=red, semithick]
(axis cs:116,0.231458332471463)
--(axis cs:116,0.349186828818859);

\path [draw=red, semithick]
(axis cs:118,0.231458332471463)
--(axis cs:118,0.349186828818859);

\path [draw=red, semithick]
(axis cs:120,0.461051306812042)
--(axis cs:120,0.605615359854625);

\path [draw=red, semithick]
(axis cs:122,0.477749998965756)
--(axis cs:122,0.619024194582631);

\path [draw=red, semithick]
(axis cs:124,0.108312499224317)
--(axis cs:124,0.214268145936973);

\path [draw=red, semithick]
(axis cs:126,0.108312499224317)
--(axis cs:126,0.214268145936973);

\path [draw=red, semithick]
(axis cs:128,0.0467395826007438)
--(axis cs:128,0.14680880449603);

\path [draw=red, semithick]
(axis cs:130,0.0467395826007438)
--(axis cs:130,0.14680880449603);

\addplot [semithick, color0]
table {%
-4 0.289700909532296
-2.64646464646465 0.299561221223985
-1.29292929292929 0.303200008671337
0.0606060606060606 0.301611090044221
1.41414141414141 0.295698522381247
2.76767676767677 0.286281224103273
4.12121212121212 0.274097490756272
5.47474747474748 0.259809403983547
6.82828282828283 0.244007133727306
8.18181818181818 0.227213133659582
9.53535353535354 0.209886229842507
10.8888888888889 0.192425602617948
12.2424242424242 0.175174661726482
13.5959595959596 0.158424814655734
14.949494949495 0.142419128218066
16.3030303030303 0.127355883357614
17.6565656565657 0.113392023186687
19.010101010101 0.100646494251508
20.3636363636364 0.0892034810273152
21.7171717171717 0.079115533642817
23.0707070707071 0.0704065888339942
24.4242424242424 0.0630748841272588
25.7777777777778 0.0570957652519653
27.1313131313131 0.0524243867822743
28.4848484848485 0.0489983060083702
29.8383838383838 0.0467399700370295
31.1919191919192 0.0455590961215448
32.5454545454545 0.0453549452209991
33.8989898989899 0.0460184887888953
35.2525252525253 0.0474344687911365
36.6060606060606 0.0494833509533611
37.959595959596 0.0520431712376296
39.3131313131313 0.0549912755484642
40.6666666666667 0.0582059526682422
42.020202020202 0.0615679604219419
43.3737373737374 0.064961945071241
44.7272727272727 0.0682777539379683
46.0808080808081 0.0714116412569081
47.4343434343434 0.0742673672579583
48.7878787878788 0.0767571904776398
50.1414141414141 0.0788027532999603
51.4949494949495 0.0803358607266301
52.8484848484849 0.0812991523766312
54.2020202020202 0.0816466677151401
55.5555555555556 0.0813443045118011
56.9090909090909 0.0803701705283548
58.2626262626263 0.0787148284356191
59.6161616161616 0.076381433959822
60.969696969697 0.0733857672582888
62.3232323232323 0.0697561575244805
63.6767676767677 0.0655333008223871
65.030303030303 0.0607699711502716
66.3838383838384 0.0555306247337693
67.7373737373737 0.0498908975483373
69.0909090909091 0.0439369960710599
70.4444444444444 0.0377649812618038
71.7979797979798 0.0314799457737291
73.1515151515152 0.0251950843931509
74.5050505050505 0.0190306577087552
75.8585858585859 0.0131128490101669
77.2121212121212 0.00757251441587132
78.5656565656566 0.00254382623048871
79.9191919191919 -0.00183719046859988
81.2727272727273 -0.00543422801527125
82.6262626262626 -0.00811237310932827
83.979797979798 -0.00973989054181345
85.3333333333333 -0.0101901136909708
86.6868686868687 -0.00934344178885385
88.040404040404 -0.00708944395858157
89.3939393939394 -0.00332907002224061
90.7474747474748 0.00202303192056456
92.1010101010101 0.00903609114351488
93.4545454545455 0.0177606831737331
94.8080808080808 0.0282260919012925
96.1616161616162 0.0404375649180784
97.5151515151515 0.0543734620860032
98.8686868686869 0.0699822973345749
100.222222222222 0.0871796736878155
101.575757575758 0.105845111520537
102.929292929293 0.125818770043966
104.282828282828 0.146898062020724
105.636363636364 0.168834161709158
106.989898989899 0.191328406037028
108.343434343434 0.214028589004543
109.69696969697 0.236525149316752
111.050505050505 0.258347251245289
112.40404040404 0.278958758719469
113.757575757576 0.297754102646736
115.111111111111 0.314054041462469
116.464646464646 0.327101314909134
117.818181818182 0.336056191044795
119.171717171717 0.339991906480971
120.525252525253 0.337889999849856
121.878787878788 0.328635538500881
123.232323232323 0.311012238426636
124.585858585859 0.283697477418142
125.939393939394 0.24525720144948
127.292929292929 0.194140724291761
128.646464646465 0.128675420356469
130 0.047061310768137
};
\addplot [semithick, red, mark=*, mark size=3, mark options={solid}, only marks]
table {%
-4 0.225806451612903
0 0.6
2 0.133333333333333
4 0.214285714285714
6 0.2
8 0.161290322580645
10 0.225806451612903
12 0.225806451612903
16 0.161290322580645
28 0.0967741935483871
38 0.0322580645161291
48 0.0666666666666667
58 0.0322580645161291
68 0.0322580645161291
78 0.0666666666666667
88 0.0322580645161291
98 0.0666666666666667
108 0.0967741935483871
112 0.225806451612903
116 0.290322580645161
118 0.290322580645161
120 0.533333333333333
122 0.548387096774194
124 0.161290322580645
126 0.161290322580645
128 0.0967741935483871
130 0.0967741935483871
};
\end{axis}

\end{tikzpicture}

				\caption{График зависимости $g$ от $I$} 
			\end{center}
		\end{minipage}
		\end{figure}
		\par В итоге получим $R_{\text{H}}=(0.92\pm0.02)\cdot10^{-10}\text{ м}^3/\text{Кл}$.
		\par Используя формулу (2) вычислим удельное сопротивление серебра. Для этого на максимальном токе
		$I=0.97$ А измерим падение напряжение между контактами 3 и 4 $U_{34} = 340$ мкВ. Зная размеры образца
		$a=10$ мм, $l=14.5$ мм, $h=0.09$ мм, получим $\rho_{\text{с}}=0.022$
		$\frac{\text{Ом} \cdot \text{мм}^{2}}{\text{м}}$. Значит проводимость серебра равна
		$\sigma_{\text{с}}=1/\rho_{\text{с}}=4.6\cdot10^7$ См/м.
		\par Рассчитаем концентрацию насителей заряда. Так как носителями в серебре являются электроны, то
		$n_{\text{с}}=1/(R_{\text{H}}e)=(6.8\pm0.1)\cdot10^{28}\text{ м}^{-3}$.
		\par Теперь определим подвижность носителей заряда в серебре. Из формулы $\sigma=nq\mu$ получаем, что
		$\mu_{\text{с}}=42\pm1 \text{ см}^{2}/(\text{В}\cdot\text{с})$.
		\subsection{Исследование цинкового образца}
		\par Проведём то же исследование для образца из цинка. Данные измерений занесём в таблицу 4 и построим по
		ним график (рис. 5). Используя МНК определим коэффицент $g$.
		\begin{table}[H]
			\captionsetup{font={small}, labelformat=fullparents, labelsep=fill, labelfont=bf, justification=raggedleft,
			singlelinecheck=false, skip=-0.2cm}
			\caption{Значения $I_{\text{М}}$ и $B$}
			\begin{center}
				\begin{tabular}{|p{1.7cm}|p{1.4cm}|p{1.4cm}|p{1.4cm}|p{1.4cm}|p{1.4cm}|p{1.4cm}|p{1.4cm}|}
					\hline
					\multicolumn{8}{|c|}{$I=1$ А, $U_0=0.92$ мкВ} \\
					\hline
					$I_{\text{М}}$, А & 0.1 & 0.2 & 0.3 & 0.4 & 0.5 & 0.6 & 0.7 \\
					\hline
					$B$, мТл & 118.5 & 237.1 & 355.6 & 474.2 & 592.7 & 711.3 & 829.8 \\
					\hline
					$U_{24}$, мкВ & 0.84 & 0.74 & 0.64 & 0.56 & 0.44 & 0.36 & 0.28 \\
					\hline
					$U_{\perp}$, мкВ & -0.08 & 0.18 & -0.28 & -0.36 & -0.48 & -0.56 & -0.64 \\
					\hline
				\end{tabular}
			\end{center}
		\end{table}
		\par В итоге получим значение $g=-0.78\pm0.02$ мкВ/Тл. Зная толщину образца $h=0.08$ мм, найдём значение
		постоянной Холла $R_{\text{H}}=(-0.62\pm0.02)\cdot10^{-10}\text{ м}^3/\text{Кл}$.
		\begin{figure}[H]
			\begin{center}
				% This file was created with tikzplotlib v0.9.16.
\begin{tikzpicture}

\definecolor{color0}{rgb}{0.12156862745098,0.466666666666667,0.705882352941177}

\begin{axis}[
    height=7.2cm,
    tick align=inside,
    major tick length=0.2cm,
    minor tick length=0.1cm,
    tick pos=left,
    x grid style={white!69.0196078431373!black},
    xmin=-10, xmax=140,
    xtick={-40, 0, 40, 80, 120, 160},
    minor x tick num=3,
    xmajorgrids,
    minor x grid style={dotted,black},
    xminorgrids,
    xtick style={color=black},
    xlabel={$l$, \text{см}},
    ymin = -0.2, ymax = 0.8,
    ytick = {-0.4, 0, 0.4, 0.8},
    ytick style={color=black},
    minor y tick num=3,
    ymajorgrids,
    minor y grid style={dotted,black},
    yminorgrids,
    ytick style={color=black},
    ylabel={$\upsilon_2$}
]
\path [draw=red, semithick]
(axis cs:-6,0.225806451612903)
--(axis cs:-2,0.225806451612903);

\path [draw=red, semithick]
(axis cs:-2,0.6)
--(axis cs:2,0.6);

\path [draw=red, semithick]
(axis cs:10,0.225806451612903)
--(axis cs:14,0.225806451612903);

\path [draw=red, semithick]
(axis cs:14,0.161290322580645)
--(axis cs:18,0.161290322580645);

\path [draw=red, semithick]
(axis cs:26,0.0967741935483871)
--(axis cs:30,0.0967741935483871);

\path [draw=red, semithick]
(axis cs:36,0.0322580645161291)
--(axis cs:40,0.0322580645161291);

\path [draw=red, semithick]
(axis cs:46,0.0666666666666667)
--(axis cs:50,0.0666666666666667);

\path [draw=red, semithick]
(axis cs:56,0.0322580645161291)
--(axis cs:60,0.0322580645161291);

\path [draw=red, semithick]
(axis cs:66,0.0322580645161291)
--(axis cs:70,0.0322580645161291);

\path [draw=red, semithick]
(axis cs:76,0.0666666666666667)
--(axis cs:80,0.0666666666666667);

\path [draw=red, semithick]
(axis cs:86,0.0322580645161291)
--(axis cs:90,0.0322580645161291);

\path [draw=red, semithick]
(axis cs:96,0.0666666666666667)
--(axis cs:100,0.0666666666666667);

\path [draw=red, semithick]
(axis cs:106,0.0967741935483871)
--(axis cs:110,0.0967741935483871);

\path [draw=red, semithick]
(axis cs:110,0.225806451612903)
--(axis cs:114,0.225806451612903);

\path [draw=red, semithick]
(axis cs:114,0.290322580645161)
--(axis cs:118,0.290322580645161);

\path [draw=red, semithick]
(axis cs:118,0.533333333333333)
--(axis cs:122,0.533333333333333);

\path [draw=red, semithick]
(axis cs:120,0.548387096774194)
--(axis cs:124,0.548387096774194);

\path [draw=red, semithick]
(axis cs:124,0.161290322580645)
--(axis cs:128,0.161290322580645);

\path [draw=red, semithick]
(axis cs:128,0.0967741935483871)
--(axis cs:132,0.0967741935483871);

\path [draw=red, semithick]
(axis cs:-4,0.146722164412071)
--(axis cs:-4,0.304890738813736);

\path [draw=red, semithick]
(axis cs:0,0.493333333333333)
--(axis cs:0,0.706666666666667);

\path [draw=red, semithick]
(axis cs:12,0.146722164412071)
--(axis cs:12,0.304890738813736);

\path [draw=red, semithick]
(axis cs:16,0.0863683662851197)
--(axis cs:16,0.236212278876171);

\path [draw=red, semithick]
(axis cs:28,0.0260145681581686)
--(axis cs:28,0.167533818938606);

\path [draw=red, semithick]
(axis cs:38,-0.0343392299687825)
--(axis cs:38,0.0988553590010406);

\path [draw=red, semithick]
(axis cs:48,-0.00444444444444439)
--(axis cs:48,0.137777777777778);

\path [draw=red, semithick]
(axis cs:58,-0.0343392299687825)
--(axis cs:58,0.0988553590010406);

\path [draw=red, semithick]
(axis cs:68,-0.0343392299687825)
--(axis cs:68,0.0988553590010406);

\path [draw=red, semithick]
(axis cs:78,-0.00444444444444439)
--(axis cs:78,0.137777777777778);

\path [draw=red, semithick]
(axis cs:88,-0.0343392299687825)
--(axis cs:88,0.0988553590010406);

\path [draw=red, semithick]
(axis cs:98,-0.00444444444444439)
--(axis cs:98,0.137777777777778);

\path [draw=red, semithick]
(axis cs:108,0.0260145681581686)
--(axis cs:108,0.167533818938606);

\path [draw=red, semithick]
(axis cs:112,0.146722164412071)
--(axis cs:112,0.304890738813736);

\path [draw=red, semithick]
(axis cs:116,0.207075962539022)
--(axis cs:116,0.373569198751301);

\path [draw=red, semithick]
(axis cs:120,0.431111111111111)
--(axis cs:120,0.635555555555556);

\path [draw=red, semithick]
(axis cs:122,0.448491155046826)
--(axis cs:122,0.648283038501561);

\path [draw=red, semithick]
(axis cs:126,0.0863683662851197)
--(axis cs:126,0.236212278876171);

\path [draw=red, semithick]
(axis cs:130,0.0260145681581686)
--(axis cs:130,0.167533818938606);

\addplot [semithick, color0]
table {%
-4 0.313814118861075
-2.64646464646465 0.372719374858457
-1.29292929292929 0.414429546476477
0.0606060606060606 0.441252184268248
1.41414141414141 0.455307668066297
2.76767676767677 0.458538104134772
4.12121212121212 0.452716029232126
5.47474747474748 0.439452921584311
6.82828282828283 0.420207518768442
8.18181818181818 0.396293942506964
9.53535353535354 0.368889630372301
10.8888888888889 0.339043074402003
12.2424242424242 0.307681366624373
13.5959595959596 0.275617551494593
14.949494949495 0.243557785241332
16.3030303030303 0.21210830212385
17.6565656565657 0.181782187599588
19.010101010101 0.153005958402251
20.3636363636364 0.126125949530377
21.7171717171717 0.101414508146396
23.0707070707071 0.0790759943861873
24.4242424242424 0.059252589079113
25.7777777777778 0.0420299083785511
27.1313131313131 0.0274424253029159
28.4848484848485 0.0154786981871685
29.8383838383838 0.00608640604481389
31.1919191919192 -0.000822809159606586
32.5454545454545 -0.0053686933275373
33.8989898989899 -0.00769792413293865
35.2525252525253 -0.00797984801832828
36.6060606060606 -0.0064024102803318
37.959595959596 -0.00316827824474411
39.3131313131313 0.001508842468899
40.6666666666667 0.0074076985932389
42.020202020202 0.0143027867695034
43.3737373737374 0.0219674580144038
44.7272727272727 0.0301768290975233
46.0808080808081 0.0387105008291939
47.4343434343434 0.0473550832588656
48.7878787878788 0.0559065277839617
50.1414141414141 0.0641722661692277
51.4949494949495 0.0719731564765673
52.8484848484849 0.0791452359053693
54.2020202020202 0.0855412805433256
55.5555555555556 0.0910321720277353
56.9090909090909 0.0955080711173033
58.2626262626263 0.0988793981744253
59.6161616161616 0.101077620557964
60.969696969697 0.102055846926514
62.3232323232323 0.10178922845216
63.6767676767677 0.10027516694472
65.030303030303 0.0975333298864813
66.3838383838384 0.0936054723774249
67.7373737373737 0.088555065990942
69.0909090909091 0.0824667345400373
70.4444444444444 0.0754454967540248
71.7979797979798 0.0676158158657112
73.1515151515152 0.059120456109071
74.5050505050505 0.0501191461274099
75.8585858585859 0.0407870492920191
77.2121212121212 0.031313040931319
78.5656565656566 0.0218977924704934
79.9191919191919 0.0127516624816113
81.2727272727273 0.00409239464424182
82.6262626262626 -0.00385737738344372
83.979797979798 -0.0108728181830784
85.3333333333333 -0.0167297718840175
86.6868686868687 -0.0212078365607651
88.040404040404 -0.024093631719913
89.3939393939394 -0.025184258876587
90.7474747474748 -0.0242909552204062
92.1010101010101 -0.0212429403709502
93.4545454545455 -0.0158914562227379
94.8080808080808 -0.00811399987971576
96.1616161616162 0.00218125032074329
97.5151515151515 0.015050815694331
98.8686868686869 0.030511105006786
100.222222222222 0.048533409181362
101.575757575758 0.0690387029167914
102.929292929293 0.091892253215734
104.282828282828 0.116898034823718
105.636363636364 0.143792952578569
106.989898989899 0.172240870670332
108.343434343434 0.20182644881168
109.69696969697 0.232048785318812
111.050505050505 0.262314867102847
112.40404040404 0.291932826571698
113.757575757576 0.320105005442445
115.111111111111 0.345920825464192
116.464646464646 0.368349466051415
117.818181818182 0.386232348827805
119.171717171717 0.39827542908059
120.525252525253 0.403041294125359
121.878787878788 0.398941068581365
123.232323232323 0.384226126557329
124.585858585859 0.35697961074772
125.939393939394 0.315107758439542
127.292929292929 0.25633103442959
128.646464646465 0.178175070852219
130 0.0779614139175848
};
\addplot [semithick, red, mark=*, mark size=3, mark options={solid}, only marks]
table {%
-4 0.225806451612903
0 0.6
12 0.225806451612903
16 0.161290322580645
28 0.0967741935483871
38 0.0322580645161291
48 0.0666666666666667
58 0.0322580645161291
68 0.0322580645161291
78 0.0666666666666667
88 0.0322580645161291
98 0.0666666666666667
108 0.0967741935483871
112 0.225806451612903
116 0.290322580645161
120 0.533333333333333
122 0.548387096774194
126 0.161290322580645
130 0.0967741935483871
};
\end{axis}

\end{tikzpicture}

				\caption{График зависимости $U_{\perp}$ от $B$} 
			\end{center}
		\end{figure}
		\par Используя формулу (2) вычислим удельное сопротивление цинка. Для этого на максимальном токе
		$I=1.01$ А измерим падение напряжение между контактами 3 и 4 $U_{34} = 290$ мкВ. Зная размеры образца
		$a=10$ мм, $l=4$ мм, $h=0.08$ мм, получим $\rho_{\text{ц}}=0.92$
		$\frac{\text{Ом} \cdot \text{мм}^{2}}{\text{м}}$. Значит проводимость цинка равна
		$\sigma_{\text{ц}}=1/\rho_{\text{ц}}=1.09\cdot10^7$ См/м.
		\par Рассчитаем концентрацию носителей заряда. Так как носителями в цинке являются дырки с зарядом равным
		по модулю заряду электрона (каждый атом в среднем отдаёт один электрон), то
		$n_{\text{ц}}=1/(R_{\text{H}}e)=(7.6\pm0.1)\cdot10^{28}\text{ м}^{-3}$.
		\par Теперь определим подвижность носителей заряда в цинке. Из формулы $\sigma=nq\mu$ получаем, что
		$\mu_{\text{ц}}=8.5\pm0.2 \text{ см}^{2}/(\text{В}\cdot\text{с})$.
		\par Отрицательность постоянной Холла для цинка объясняется тем, что в цинке носителями заряда являются
		дырки.
	\section{Обсуждение результатов}
		\par В данной работе мы исследовали эффект Холла в металлах. Нашей задачей было определить постоянную
		Холла, концентрацию носителей, удельные проводимость и сопротивление. С табличными данными хорошо
		согласуется только постоянная Холла для серебра. Остальные велечины расходятся более, чем на две
		погрешности. Причинами таких результатов могли стать микровольтметр (недостаточно чувствителен при малых
		токах, стрелка слишком сильно дёргается), и примеси в металлах. Но все велечины сошлишь с табличными по
		порядку.
	\section{Вывод}
		Рассхождения со справочными данными составляют порядка 10\%. Это позволяет предположить, что при более
		аккуратном проведении работы, можно достичь результатов, более похожих на табличные.
		
\end{document}

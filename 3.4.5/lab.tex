\documentclass[12pt,a4paper]{article}
\usepackage[T2A]{fontenc}
\usepackage[utf8]{inputenc}
\usepackage[english, russian]{babel}
\usepackage{indentfirst}
\usepackage{misccorr}
\usepackage{graphicx}
\usepackage{amsmath}
\usepackage{amssymb}
\usepackage{circuitikz}
\usepackage[font={small}]{caption}
\usepackage[left=20mm, top=20mm, right=20mm, bottom=20mm, nohead]{geometry}
\usepackage{float}
\usepackage{tabularx}
\usepackage{array}
\usepackage{longtable}
\usepackage{pstool}
\usepackage{pgfplots}
\usepackage{hhline}
\usepackage{multirow}
\usepackage{wrapfig}

\DeclareCaptionLabelSeparator{fill}{.\\}
\DeclareCaptionLabelFormat{fullparents}{\bothIfFirst{#1}{~}#2}

\pgfplotsset{compat=1.17}

\begin{document}

	\begin{titlepage}
		\begin{center}
			{\LARGE Отчёт по лабораторной работе 3.6.1.\\}
			\vspace*{11cm}
				\textbf{\LARGE Петля гистерезиса (динамический метод).}	
			\vspace*{6.5cm}
		\end{center}
		\hfill\begin{minipage}{0.37\textwidth}
				Работу выполнил Громов Артём
                \\
				ЛФИ Б02-006
		\end{minipage}
		\vspace{5cm}
		\begin{center}
 			Долгопрудный, 2021 г.
		\end{center}   
	\end{titlepage}
	
	\section{Аннотация}
	    \noindent\textbf{Цель работы:} изучение петель гистерезиса различных ферромагнитных материалов в переменных полях.
        \\
		\noindent\textbf{В работе используются:} автотрансформатор, понижающий транс­форматор, интегрирующая цепочка,
        амперметр, вольтметр, электронный осциллограф, делитель напряжения, тороидальные образцы с двумя об­мотками.
		\vspace{0.5 cm}
        \par Ферромагнитные материалы часто применяются в трансформато­рах, дросселях, машинах переменного тока, то есть в
        устройствах, где они подвергаются периодическому перемагничиванию, --- поэтому изу­чение магнитных характеристик
        ферромагнетиков в переменных полях представляет большой практический интерес. Основные характеристики ферромагнетиков
        --- их коэрцитивное поле $H_{c}$, магнитная проницаемость $\mu$, рассеиваемая в виде тепла при перемагничивании
        мощность --- зависят от частоты перемагничивающего поля. В данной работе кривые гистере­зиса ферромагнитных материалов
        изучаются в поле частоты $\nu_{0}=50$ Гц с помощью электронного осциллографа.
        \begin{figure}[H]
			\begin{center}
				\includegraphics[width=0.8\textwidth]{images/petlya.png}
				\caption{Петля гистерезиса ферромагнетика.}
			\end{center}
		\end{figure}
        \par Магнитная индукция $B$ и напряжённость поля $H$ в ферромагнитном материале неоднозначно связаны между собой:
        индукция зависит не только от напряжённости, но и от предыстории образца. Связь меж­ду $B$ и $H$ для типичного
        ферромагнетика иллюстрирует рис. 1.
        \par Если к ферромагнитному образцу прикладывать переменное внешнее магнитное поле, то его состояние на плоскости
        $H-B$ будет изменяться по замкнутой кривой --- \textit{петле гистерезиса}. Размер петли определяется максимальным
        значением напряжённости $H$ в цикле (напр., петля AA', обозначенная пунктиром на рис. 1). Если амплитуда
        напряжённости до­статочно велика, то образец будет периодически достигать насыщения, что на рисунке соответствует
        кривой CEFC'E'F'C (предельная петля гистерезиса). Пересечение предельной петли с вертикальной осью соот­ветствует
        остаточной индукции $B_r$, пересечение с горизонтальной осью --- коэрцитивному полю $H_c$. Крайние точки петель,
        соответствующие ам­плитудным значениям $H$ (например, точка A на рис. 1), лежат на на­чальной кривой намагничивания
        (OAC).
        \par \textbf{Измерение магнитной индукции.} Магнитную индукцию $B$ удобно определять с помощью ЭДС, возникающей при
        изменении магнитного потока $\Phi$ в катушке, намотанной на образец. Пусть катушка c $N$ вит­ками плотно охватывает
        образец сечением $S$, и индукция $B$ в образце однородна. Тогда имеем
        \begin{equation}
            \left|B\right|=\frac{1}{SN}\int{\varepsilon}\,dt.
        \end{equation} 
        Таким образом, для определения $B$ нужно проинтегрировать сигнал, наведённый меняющимся магнитным полем в 
        измерительной катушке, намотанной на образец.
        \begin{wrapfigure}{r}{0.4\textwidth}
            \centering
            \includegraphics[width=0.4\textwidth]{images/cell.png}
            \caption{Интегрирующая ячейка.}
        \end{wrapfigure}
        \par Для интегрирования в работе использу­ется интегрирующая $RC$-цепочка (рис. 2). «Входное» напряжение от источника
        $U_{\text{вх}}(t)$ подаётся на последовательно соединённые $U_{\text{вх}}$ резистор $R_{\text{и}}$ и конденсатор
        $C_{\text{и}}$. «Выходное» напряжение $U_{\text{вых}}(t)$ снимается с конденсато­ра. Предположим, что 1) сопротивление
        ис­точника мало по сравнению с $R_{\text{и}}$, 2) выход­ное сопротивление (сопротивление на входе осциллографа),
        напротив, велико: $R_{\text{вых}} \gg R_{\text{и}}$ и, наконец, 3) сопротив­ление $R_{\text{и}}$ достаточно велико,
        так что почти всё падение напряжения приходится на него, а $U_{\text{вых}} \ll U_{\text{вх}}$. В таком случае
        ток цепи равен $I=\left(U_{\text{вых}}-U_{\text{вх}}\right)/R_{\text{и}}\approx U_{\text{вх}}/R_{\text{и}}$, и
        входное и выходное сопротивление связаны соотношением
        \begin{equation}
            U_{\text{вых}}=\frac{q}{C_{\text{и}}}=\frac{1}{C_{\text{и}}}\int_{0}^{t}I\,dt\approx\frac{1}{\tau_{\text{и}}}
            \int_{0}^{t}U_{\text{вх}}\,dt.
        \end{equation}
        где $\tau_{\text{и}}=R_{\text{и}}C_{\text{и}}$ --- постоянная времени $RC$-цепочки. Для индукции поля из (1) получаем
        \begin{equation}
            \left|B\right|=\frac{1}{SN}\int U_{\text{вх}}\,dt=\frac{\tau_{\text{и}}}{SN}U_{\text{вых}},
        \end{equation}
        \par \textbf{Замечание.} Уточним критерий применимости соотношения (2). Пусть на вход интегрирующей ячейки подан
        синусоидальный сигнал с часто­той $\omega_0$. Тогда, пользуясь методом комплексных амплитуд, нетрудно най­ти отношение
        амплитуд входного и выходного напряжений:
        $$
            \frac{U_{\text{вых}}}{U_{\text{вх}}}=\frac{1/(\omega_0 C)}{\sqrt{R^2+1/(\omega_0 C)^2}}.
        $$
        Видно, неравенство $U_{\text{вых}} \ll U_{\text{вх}}$ реализуется, если
        \begin{equation}
            \tau \equiv RC \gg \frac{1}{\omega_0}
        \end{equation}
        (импеданс конденсатора мал по сравнению сопротивлением резистора). В таком случае для синусоидального сигнала имеем
        \begin{equation}
            \frac{U_{\text{вых}}}{U_{\text{вх}}}\approx\frac{1}{\omega_0\tau}.
        \end{equation}
        \par В общем случае, если $\omega_0$ --- частота самой низкой гармоники в спектре произвольного входного сигнала,
        то при $\omega_0\tau \gg 1$ неравенство $U_{\text{вых}} \ll U_{\text{вх}}$ выполняется на любой частоте
        $\omega > \omega_0$.

    \section{Экспериментальная установка}
        \par Схема установки изображена на рис. 3. Напряжение сети (220 В, 50 Гц) с помощью трансформаторного блока Т,
        состоящего из регули­ровочного автотрансформатора и разделительного понижающего транс­форматора, подаётся на
        намагничивающую обмотку $N_0$ исследуемого образца.
        \begin{figure}[H]
			\begin{center}
				\includegraphics[width=\textwidth]{images/stand.png}
				\caption{Схема установки для исследования намагничивания образцов.}
			\end{center}
		\end{figure}
        \par В цепь намагничивающей катушки, на которую подаётся некоторое напряжение $U_0$, последовательно включено
        сопротивление $R_0$. Напряже­ние на $R_0$, равное $U_R=R_0I_0$, где $I_0$ --- ток в намагничивающей обмот­ ке $N_0$,
        подаётся на канал $X$ осциллографа. Связь напряжённости $H$ в образце и тока $I_0$ рассчитывается по теореме о
        циркуляции. \textit{Действующее} значение переменного тока в обмотке $N_0$ измеряется ам­перметром $A$.
        \par Для измерения магнитной индукции $B$ с измерительной обмотки $N_{\text{и}}$ на вход $RC$-цепочки подаётся
        напряжение $U_{\text{и}}$ ($U_{\text{вх}}$), пропорциональное производной $dB/dt$. С интегрирующей ёмкости
        $C_{\text{и}}$ снимается напряже­ ние $U_C$ ($U_{\text{вых}}$), пропорциональное величине $B$, и подаётся на вход
        $Y$ осциллографа. Значение индукции поля $B$ рассчитывается по форму­ле (3).
        \par Замкнутая кривая, возникающая на экране, воспроизводит в неко­тором масштабе (различном для осей $X$ и $Y$)
        петлю гистерезиса. Что­бы придать этой кривой количественный смысл, необходимо установить масштабы изображения,
        т. е. провести калибровку каналов $X$ и $Y$ осцил­лографа.
        \par \textbf{Калибровка осциллографа.} Калибровка канала $X$ осциллографа производится с помощью амперметра $A$.
        Предварительно необходимо за­коротить обмотку $N_0$ (так как катушка с ферромагнитным образцом яв­ляется нелинейным
        элементом, ток в ней не имеет синусоидальной фор­мы, поэтому связать амплитуду тока с показаниями амперметра мож­но
        лишь с довольно большой погрешностью). При закороченной обмот­ке $N_0$ показания эффективного тока, умноженные на
        $2\sqrt{2}$, дадут значе­ние удвоенной амплитуды тока, подаваемого на ось $X$, соответствующе­го ширине горизонтальной
        развёртки на экране (осциллограф должен работать в режиме $X-Y$).
        \par Калибровка вертикальной оси $Y$, как правило, не нужна (переклю­чатель масштабов осциллографа откалиброван
        при изготовлении --- при условии, что ручка плавной регулировки находится в положении калиб­ровки). Тем не менее,
        она может проводиться с помощью сигнала, снимаемого через делитель напряжения со второй катушки понижающего
        трансформатора (рис. 3). Вольтметр $V$ может достаточно точно изме­рить эффективное напряжение, подаваемое на вход
        осциллографа. По­сле этого можно сравнить показания осциллографа и вольтметра.
        \par \textbf{Измерение параметров интегрирующей ячейки.} Постоянную вре­мени $RC$-цепочки можно определить
        экспериментально. С обмотки $U_0$ на вход интегрирующей цепочки подаётся синусоидальное напряжение с частотой цепи
        $\nu_0=\omega_0/2\pi=50$ Гц. На вход $Y$ осциллографа или цифрового вольтметра поочерёдно подаются сигналы со входа
        ($U_{\text{вх}}=U_0$) и выхода ($U_{\text{вых}}=U_C$) $RC$-цепочки. Измерив амплитуды этих сигналов, можно
        рассчитать постоянную времени $\tau_{\text{и}}=R_{\text{и}}C_{\text{и}}$ по формуле (5). Кроме того, сопротивление
        и ёмкость можно независимым образом измерить цифро­ым мультиметром.
    
    \section{Результаты измерений и обработка данных}
        \par Запишем параметры установки: $R_0=0.3$ Ом, $R_{\text{и}}=20$ кОм, $C_{\text{и}}=20$ мкФ. 
        \subsection{Исследования образца феррита.}
            \par Исследуемый образец обладает следующими параметрами: $N_0=40$ витков, $N_{\text{и}}=400$ витков,
            $S=3.0\text{ см}^2$, $2\pi R=25$ см.
            \par Определим значения коэффициентов для преобразования показаний осцилографа. Для этого воспользуемся
            следующими формулами:
            \begin{equation}
                K_x=2\sqrt{2}R_0I/(2x),
            \end{equation}
            \begin{equation}
                K_y=2\sqrt{2}U_{\text{вых}}/(2y).
            \end{equation}
            \par Необходимые данные и результаты занесём в таблицу 1.
            \begin{table}[H]
                \captionsetup{font={small}, labelformat=fullparents, labelsep=fill, labelfont=bf, justification=raggedleft,
                singlelinecheck=false, skip=-0.2cm}
                \caption{Определение коэффициентов калибровки}
                \begin{center}
                    \begin{tabular}{|p{2cm}|p{2cm}|p{2.2cm}|p{2cm}|p{2cm}|p{2.2cm}|}
                        \hline
                        $I$, мА & $2x$, дел & $K_x$, мВ/дел & $U_{\text{вых}}$, мВ & $2y$, дел & $K_y$, мВ/дел \\
                        \hline
                        $179.4\pm0.1$ & $8\pm0.1$ & $19.0\pm0.2$ & $34.3\pm0.1$ & $5\pm0.1$ & $19.4\pm0.4$ \\
                        \hline
                    \end{tabular}
                \end{center}
            \end{table}
            \par Используя полученные коэффициенты, найдём значения $H_s$, $B_s$, $H_c$, $B_r$. Для этого влспользуемся
            следующими формулами:
            \begin{equation}
                H=X\cdot\frac{K_xN_0}{2\pi RR_0},
            \end{equation}
            \begin{equation}
                B=Y\cdot\frac{K_yR_{\text{и}}C_{\text{и}}}{SN_{\text{и}}}.
            \end{equation}
            \par Необходимы данные и полученные результаты занесём в таблицу 2.
            \begin{table}[H]
                \captionsetup{font={small}, labelformat=fullparents, labelsep=fill, labelfont=bf, justification=raggedleft,
                singlelinecheck=false, skip=-0.2cm}
                \caption{Определение коэффициентов калибровки}
                \begin{center}
                    \begin{tabular}{|p{1.6cm}|p{1.6cm}|p{1.6cm}|p{1.6cm}|p{1.6cm}|p{1.6cm}|p{1.6cm}|p{1.6cm}|}
                        \hline
                        $2X_s$, дел & $H_s$, А/м & $2Y_s$, дел & $B_s$, мТ & $2X_c$, дел & $H_c$, А/м &
                        $2Y_r$, дел & $B_r$, мТ \\
                        \hline
                        $8\pm0.1$ & $40.6\pm0.5$ & $4.5\pm0.1$ & $142\pm3$ & $1.3\pm0.1$ & $6.6\pm0.1$ &
                        $1.9\pm0.1$ & $61\pm1$ \\
                        \hline
                    \end{tabular}
                \end{center}
            \end{table}
            \par При разных значениях токов запишем значения координат точек A и A' и построим график в координатах $X$ и $Y$.
            \begin{figure}[H]
		        \begin{center}
				    % This file was created with tikzplotlib v0.9.16.
\begin{tikzpicture}

\begin{axis}[
    height=13.5cm,
    tick align=inside,
    major tick length=0.2cm,
    minor tick length=0.1cm,
    tick pos=left,
    x grid style={white!69.0196078431373!black},
    xmin=0, xmax=2.1,
    xtick={0, 0.5, 1, 1.5, 2, 2.5},
    minor x tick num=4,
    xmajorgrids,
    minor x grid style={dotted,black},
    xminorgrids,
    xtick style={color=black},
    xlabel={$T_{\text{эксп}}$, \text{мс}},
    ymin = 0, ymax = 2.1,
    ytick = {0, 0.5, 1, 1.5, 2, 2.5},
    ytick style={color=black},
    minor y tick num=4,
    ymajorgrids,
    minor y grid style={dotted,black},
    yminorgrids,
    ytick style={color=black},
    ylabel={$T_{\text{теор}}$, \text{мс}}
]
\addplot [semithick, black]
table {%
0 0
2 1.98554049274599
};
\path [draw=red, semithick]
(axis cs:0.2,0.4)
--(axis cs:0.6,0.4);

\path [draw=red, semithick]
(axis cs:0.4,0.7)
--(axis cs:0.8,0.7);

\path [draw=red, semithick]
(axis cs:0.76,0.9)
--(axis cs:0.96,0.9);

\path [draw=red, semithick]
(axis cs:0.95,1.2)
--(axis cs:1.55,1.2);

\path [draw=red, semithick]
(axis cs:0.9,1.3)
--(axis cs:1.9,1.3);

\path [draw=red, semithick]
(axis cs:1.1,1.6)
--(axis cs:2.1,1.6);

\path [draw=red, semithick]
(axis cs:1.4,1.9)
--(axis cs:2.4,1.9);

\path [draw=red, semithick]
(axis cs:1.4,2)
--(axis cs:2.6,2);

\path [draw=red, semithick]
(axis cs:0.4,0.398)
--(axis cs:0.4,0.402);

\path [draw=red, semithick]
(axis cs:0.6,0.698)
--(axis cs:0.6,0.702);

\path [draw=red, semithick]
(axis cs:0.86,0.898)
--(axis cs:0.86,0.902);

\path [draw=red, semithick]
(axis cs:1.25,1.198)
--(axis cs:1.25,1.202);

\path [draw=red, semithick]
(axis cs:1.4,1.298)
--(axis cs:1.4,1.302);

\path [draw=red, semithick]
(axis cs:1.6,1.598)
--(axis cs:1.6,1.602);

\path [draw=red, semithick]
(axis cs:1.9,1.898)
--(axis cs:1.9,1.902);

\path [draw=red, semithick]
(axis cs:2,1.998)
--(axis cs:2,2.002);

\addplot [semithick, red, mark=*, mark size=3, mark options={solid}, only marks]
table {%
0.4 0.4
0.6 0.7
0.86 0.9
1.25 1.2
1.4 1.3
1.6 1.6
1.9 1.9
2 2
};
\end{axis}

\end{tikzpicture}

				    \caption{Кривая начального намагничивания феррита.}
    			\end{center}
	    	\end{figure}
            \par Для аппроксимации воспользуемся сследующей функцией: $y=A\tanh(Bx)$. В результате получим следующие
            значения: $A=2.47 \pm 0.02$ дел, $B=0.369 \pm 0.001$ дел$^{-1}$.
            \par Из графика видно, что наиюольшее значение намгначиваемости и начальное совпадают, поэтому
            $\mu_{\text{нач}}=\mu_{max}=(4.6\pm0.2)\cdot10^{3}$.
        
        \subsection{Исследования образца пермаллоя.}
            \par Исследуемый образец обладает следующими параметрами: $N_0=35$ витков, $N_U=220$ витков, $S=3.8\text{ см}^2$,
            $2\pi R=24$ см.
            \par Определим значения коэффициентов для преобразования показаний осцилографа. Для этого воспользуемся
            формулами (6)-(7):
            \par Необходимые данные и результаты занесём в таблицу 3.
            \begin{table}[H]
                \captionsetup{font={small}, labelformat=fullparents, labelsep=fill, labelfont=bf, justification=raggedleft,
                singlelinecheck=false, skip=-0.2cm}
                \caption{Определение коэффициентов калибровки}
                \begin{center}
                    \begin{tabular}{|p{2cm}|p{2cm}|p{2.2cm}|p{2cm}|p{2cm}|p{2.2cm}|}
                        \hline
                        $I$, мА & $2x$, дел & $K_x$, мВ/дел & $U_{\text{вых}}$, мВ & $2y$, дел & $K_y$, мВ/дел \\
                        \hline
                        $452.0\pm0.1$ & $8\pm0.1$ & $48\pm1$ & $160.0\pm0.1$ & $4.8\pm0.1$ & $94\pm2$ \\
                        \hline
                    \end{tabular}
                \end{center}
            \end{table}
            \par Используя полученные коэффициенты, найдём значения $H_s$, $B_s$, $H_c$, $B_r$. Для этого влспользуемся
            формулами (8)-(9):
            \par Необходимы данные и полученные результаты занесём в таблицу 4.
            \begin{table}[H]
                \captionsetup{font={small}, labelformat=fullparents, labelsep=fill, labelfont=bf, justification=raggedleft,
                singlelinecheck=false, skip=-0.2cm}
                \caption{Определение коэффициентов калибровки}
                \begin{center}
                    \begin{tabular}{|p{1.6cm}|p{2.1cm}|p{1.6cm}|p{1.6cm}|p{1.6cm}|p{2.1cm}|p{1.6cm}|p{1.6cm}|}
                        \hline
                        $2X_s$, дел & $H_s$, А/м & $2Y_s$, дел & $B_s$, мТ & $2X_c$, дел & $H_c$, А/м &
                        $2Y_r$, дел & $B_r$, мТ \\
                        \hline
                        $5\pm0.1$ & $58.26\pm0.03$ & $4.6\pm0.1$ & $1038\pm2$ & $2.2\pm0.1$ & $25.64\pm0.02$ &
                        $4.4\pm0.1$ & $992\pm2$ \\
                        \hline
                    \end{tabular}
                \end{center}
            \end{table}
            \par При разных значениях токов запишем значения координат точек A и A' и построим график в координатах $X$ и $Y$.
            \begin{figure}[H]
		        \begin{center}
				    % This file was created with tikzplotlib v0.9.16.
\begin{tikzpicture}

\begin{axis}[
    height=11cm,
    tick align=inside,
    major tick length=0.2cm,
    minor tick length=0.1cm,
    tick pos=left,
    x grid style={white!69.0196078431373!black},
    xmin=-3, xmax=3,
    xtick={-3, -2, -1, 0, 1, 2, 3},
    minor x tick num=4,
    xmajorgrids,
    minor x grid style={dotted,black},
    xminorgrids,
    xtick style={color=black},
    xlabel={$X$, \text{дел}},
    ymin = -2.5, ymax = 2.5,
    ytick = {-3, -2, -1, 0, 1, 2, 3},
    ytick style={color=black},
    minor y tick num=4,
    ymajorgrids,
    minor y grid style={dotted,black},
    yminorgrids,
    ytick style={color=black},
    ylabel={$Y$, \text{дел}}
]
\addplot [semithick, black]
table {%
-2.5 -2.30145809018127
-2.44949494949495 -2.29064142380035
-2.3989898989899 -2.27982475741944
-2.34848484848485 -2.26900809103852
-2.2979797979798 -2.2581914246576
-2.24747474747475 -2.24737475827668
-2.1969696969697 -2.23655809189577
-2.14646464646465 -2.22574142551485
-2.0959595959596 -2.21492475913393
-2.04545454545455 -2.20410809275301
-1.99494949494949 -2.1932914263721
-1.94444444444444 -2.18247475999118
-1.89393939393939 -2.17165809361026
-1.84343434343434 -2.16084142722934
-1.79292929292929 -2.15002476084843
-1.74242424242424 -2.13920809446751
-1.69191919191919 -2.12839142808654
-1.64141414141414 -2.11757476170082
-1.59090909090909 -2.10675809507228
-1.54040404040404 -2.09594142074769
-1.48989898989899 -2.08512458907725
-1.43939393939394 -2.07430560244053
-1.38888888888889 -2.06346614361605
-1.33838383838384 -2.0524873769541
-1.28787878787879 -2.0408093727904
-1.23737373737374 -2.02647386077297
-1.18686868686869 -2.0043230120107
-1.13636363636364 -1.96412155253623
-1.08585858585859 -1.8910169030002
-1.03535353535354 -1.77095324052716
-0.984848484848485 -1.59973063565835
-0.934343434343434 -1.38860879874098
-0.883838383838384 -1.16035294831325
-0.833333333333333 -0.938904861511976
-0.782828282828283 -0.741196211021896
-0.732323232323232 -0.575090237577399
-0.681818181818182 -0.441476097649093
-0.631313131313131 -0.337306275336167
-0.580808080808081 -0.257921404489649
-0.53030303030303 -0.198401319750097
-0.47979797979798 -0.154211669547385
-0.42929292929293 -0.121451899154615
-0.378787878787879 -0.0969088577561488
-0.328282828282828 -0.078025935340532
-0.277777777777778 -0.0628392483878507
-0.227272727272728 -0.0499023841434633
-0.176767676767677 -0.0382076806704075
-0.126262626262626 -0.0271066218377871
-0.0757575757575757 -0.0162300505410345
-0.0252525252525251 -0.00540835397634201
0.0252525252525251 0.00540835397634208
0.0757575757575757 0.0162300505410345
0.126262626262626 0.0271066218377871
0.176767676767677 0.0382076806704076
0.227272727272727 0.0499023841434632
0.277777777777778 0.0628392483878508
0.328282828282828 0.0780259353405321
0.378787878787879 0.0969088577561489
0.429292929292929 0.121451899154614
0.47979797979798 0.154211669547385
0.53030303030303 0.198401319750097
0.580808080808081 0.257921404489649
0.631313131313131 0.337306275336167
0.681818181818182 0.441476097649092
0.732323232323232 0.575090237577399
0.782828282828283 0.741196211021896
0.833333333333333 0.938904861511977
0.883838383838384 1.16035294831325
0.934343434343434 1.38860879874098
0.984848484848485 1.59973063565835
1.03535353535354 1.77095324052716
1.08585858585859 1.89101690300019
1.13636363636364 1.96412155253623
1.18686868686869 2.0043230120107
1.23737373737374 2.02647386077297
1.28787878787879 2.0408093727904
1.33838383838384 2.0524873769541
1.38888888888889 2.06346614361605
1.43939393939394 2.07430560244053
1.48989898989899 2.08512458907725
1.54040404040404 2.09594142074769
1.59090909090909 2.10675809507228
1.64141414141414 2.11757476170082
1.69191919191919 2.12839142808654
1.74242424242424 2.13920809446751
1.79292929292929 2.15002476084843
1.84343434343434 2.16084142722934
1.89393939393939 2.17165809361026
1.94444444444444 2.18247475999118
1.99494949494949 2.1932914263721
2.04545454545454 2.20410809275301
2.0959595959596 2.21492475913393
2.14646464646465 2.22574142551485
2.1969696969697 2.23655809189577
2.24747474747475 2.24737475827668
2.2979797979798 2.2581914246576
2.34848484848485 2.26900809103852
2.3989898989899 2.27982475741944
2.44949494949495 2.29064142380035
2.5 2.30145809018127
};
\path [draw=red, semithick]
(axis cs:-2.6,-2.4)
--(axis cs:-2.4,-2.4);

\path [draw=red, semithick]
(axis cs:-1.9,-2.3)
--(axis cs:-1.7,-2.3);

\path [draw=red, semithick]
(axis cs:-1.5,-2.1)
--(axis cs:-1.3,-2.1);

\path [draw=red, semithick]
(axis cs:-1.1,-1.7)
--(axis cs:-0.9,-1.7);

\path [draw=red, semithick]
(axis cs:-0.9,-1)
--(axis cs:-0.7,-1);

\path [draw=red, semithick]
(axis cs:-0.7,-0.3)
--(axis cs:-0.5,-0.3);

\path [draw=red, semithick]
(axis cs:-0.5,-0.1)
--(axis cs:-0.3,-0.1);

\path [draw=red, semithick]
(axis cs:-0.1,0)
--(axis cs:0.1,0);

\path [draw=red, semithick]
(axis cs:0.3,0.1)
--(axis cs:0.5,0.1);

\path [draw=red, semithick]
(axis cs:0.6,0.2)
--(axis cs:0.8,0.2);

\path [draw=red, semithick]
(axis cs:0.7,0.9)
--(axis cs:0.9,0.9);

\path [draw=red, semithick]
(axis cs:0.9,1.5)
--(axis cs:1.1,1.5);

\path [draw=red, semithick]
(axis cs:1.3,2)
--(axis cs:1.5,2);

\path [draw=red, semithick]
(axis cs:1.8,2.1)
--(axis cs:2,2.1);

\path [draw=red, semithick]
(axis cs:2.4,2.2)
--(axis cs:2.6,2.2);

\path [draw=red, semithick]
(axis cs:-2.5,-2.5)
--(axis cs:-2.5,-2.3);

\path [draw=red, semithick]
(axis cs:-1.8,-2.4)
--(axis cs:-1.8,-2.2);

\path [draw=red, semithick]
(axis cs:-1.4,-2.2)
--(axis cs:-1.4,-2);

\path [draw=red, semithick]
(axis cs:-1,-1.8)
--(axis cs:-1,-1.6);

\path [draw=red, semithick]
(axis cs:-0.8,-1.1)
--(axis cs:-0.8,-0.9);

\path [draw=red, semithick]
(axis cs:-0.6,-0.4)
--(axis cs:-0.6,-0.2);

\path [draw=red, semithick]
(axis cs:-0.4,-0.2)
--(axis cs:-0.4,0);

\path [draw=red, semithick]
(axis cs:0,-0.1)
--(axis cs:0,0.1);

\path [draw=red, semithick]
(axis cs:0.4,0)
--(axis cs:0.4,0.2);

\path [draw=red, semithick]
(axis cs:0.7,0.1)
--(axis cs:0.7,0.3);

\path [draw=red, semithick]
(axis cs:0.8,0.8)
--(axis cs:0.8,1);

\path [draw=red, semithick]
(axis cs:1,1.4)
--(axis cs:1,1.6);

\path [draw=red, semithick]
(axis cs:1.4,1.9)
--(axis cs:1.4,2.1);

\path [draw=red, semithick]
(axis cs:1.9,2)
--(axis cs:1.9,2.2);

\path [draw=red, semithick]
(axis cs:2.5,2.1)
--(axis cs:2.5,2.3);

\addplot [semithick, red, mark=*, mark size=0, mark options={solid}, only marks]
table {%
-2.5 -2.4
-1.8 -2.3
-1.4 -2.1
-1 -1.7
-0.8 -1
-0.6 -0.3
-0.4 -0.1
0 0
0.4 0.1
0.7 0.2
0.8 0.9
1 1.5
1.4 2
1.9 2.1
2.5 2.2
};
\end{axis}

\end{tikzpicture}

				    \caption{Кривая начального намагничивания пермаллоя.}
    			\end{center}
	    	\end{figure}
            \par Для аппроксимации воспользуемся сследующей функцией: $y=A\tanh(Bx^5)+Cx$. В результате получим следующие
            значения: $A=1.77 \pm 0.03$ дел, $B=1.15 \pm 0.02$ дел$^{-1}$, $C=0.54 \pm 0.05$ дел$^{-1}$.
            \par Дифференцируя, находим, что $\mu_{\text{нач}}=(3.2\pm0.3)\cdot10^{3}$ и $\mu_{max}=(6.8\pm0.7)\cdot10^{4}$.

        \subsection{Исследования образца кремнистого железа.}
            \par Исследуемый образец обладает следующими параметрами: $N_0=40$ витков, $N_U=400$ витков, $S=1.2\text{ см}^2$,
            $2\pi R=10$ см.
            \par Определим значения коэффициентов для преобразования показаний осцилографа. Для этого воспользуемся
            формулами (6)-(7):
            \par Необходимые данные и результаты занесём в таблицу 3.
            \begin{table}[H]
                \captionsetup{font={small}, labelformat=fullparents, labelsep=fill, labelfont=bf, justification=raggedleft,
                singlelinecheck=false, skip=-0.2cm}
                \caption{Определение коэффициентов калибровки}
                \begin{center}
                    \begin{tabular}{|p{2.1cm}|p{2cm}|p{2.4cm}|p{2cm}|p{2cm}|p{2.2cm}|}
                        \hline
                        $I$, мА & $2x$, дел & $K_x$, мВ/дел & $U_{\text{вых}}$, мВ & $2y$, дел & $K_y$, В/дел \\
                        \hline
                        $1327.0\pm0.1$ & $6\pm0.1$ & $0.188 \pm 0.003$ & $7.2\pm0.1$ & $0.2\pm0.1$ & $0.10 \pm 0.05$ \\
                        \hline
                    \end{tabular}
                \end{center}
            \end{table}
            \par Используя полученные коэффициенты, найдём значения $H_s$, $B_s$, $H_c$, $B_r$. Для этого влспользуемся
            формулами (8)-(9):
            \par Необходимы данные и полученные результаты занесём в таблицу 4.
            \begin{table}[H]
                \captionsetup{font={small}, labelformat=fullparents, labelsep=fill, labelfont=bf, justification=raggedleft,
                singlelinecheck=false, skip=-0.2cm}
                \caption{Определение коэффициентов калибровки}
                \begin{center}
                    \begin{tabular}{|p{1.6cm}|p{1.6cm}|p{1.6cm}|p{1.6cm}|p{1.6cm}|p{1.6cm}|p{1.6cm}|p{1.8cm}|}
                        \hline
                        $2X_s$, дел & $H_s$, А/м & $2Y_s$, дел & $B_s$, Т & $2X_c$, дел & $H_c$, А/м &
                        $2Y_r$, дел & $B_r$, Т \\
                        \hline
                        $7\pm0.1$ & $876\pm3$ & $5.8\pm0.1$ & $2.5 \pm 0.1$ & $0.6\pm0.1$ & $75.1\pm0.2$ &
                        $1.9\pm0.1$ & $0.89\pm0.05$ \\
                        \hline
                    \end{tabular}
                \end{center}
            \end{table}
            \par При разных значениях токов запишем значения координат точек A и A' и построим график в координатах $X$ и $Y$.
            \begin{figure}[H]
		        \begin{center}
				    % This file was created with tikzplotlib v0.9.16.
\begin{tikzpicture}

\begin{axis}[
    height=11cm,
    tick align=inside,
    major tick length=0.2cm,
    minor tick length=0.1cm,
    tick pos=left,
    x grid style={white!69.0196078431373!black},
    xmin=-4, xmax=4,
    xtick={-4, -3, -2, -1, 0, 1, 2, 3, 4},
    minor x tick num=4,
    xmajorgrids,
    minor x grid style={dotted,black},
    xminorgrids,
    xtick style={color=black},
    xlabel={$X$, \text{дел}},
    ymin = -3, ymax = 3,
    ytick = {-3, -2, -1, 0, 1, 2, 3},
    ytick style={color=black},
    minor y tick num=4,
    ymajorgrids,
    minor y grid style={dotted,black},
    yminorgrids,
    ytick style={color=black},
    ylabel={$Y$, \text{дел}}
]
\addplot [semithick, black]
table {%
-3.5 -2.90595620260445
-3.42929292929293 -2.87831200727241
-3.35858585858586 -2.85066552779701
-3.28787878787879 -2.82301617186554
-3.21717171717172 -2.79536319358342
-3.14646464646465 -2.76770565365975
-3.07575757575758 -2.74004236927603
-3.00505050505051 -2.71237185096763
-2.93434343434343 -2.68469222315771
-2.86363636363636 -2.65700112411674
-2.79292929292929 -2.62929558003151
-2.72222222222222 -2.60157184650091
-2.65151515151515 -2.5738252090605
-2.58080808080808 -2.54604973218962
-2.51010101010101 -2.51823794356522
-2.43939393939394 -2.4903804369675
-2.36868686868687 -2.46246537305387
-2.2979797979798 -2.43447785201092
-2.22727272727273 -2.40639912564314
-2.15656565656566 -2.3782056085012
-2.08585858585859 -2.34986763789973
-2.01515151515152 -2.32134792081054
-1.94444444444444 -2.29259959133569
-1.87373737373737 -2.26356378551628
-1.8030303030303 -2.23416662051139
-1.73232323232323 -2.20431544288389
-1.66161616161616 -2.17389418659733
-1.59090909090909 -2.1427576570514
-1.52020202020202 -2.11072453631711
-1.44949494949495 -2.07756889242325
-1.37878787878788 -2.04300998169079
-1.30808080808081 -2.00670017303322
-1.23737373737374 -1.9682109203496
-1.16666666666667 -1.92701689819582
-1.0959595959596 -1.88247874513296
-1.02525252525253 -1.83382539153454
-0.954545454545455 -1.78013775802558
-0.883838383838384 -1.72033676734844
-0.813131313131313 -1.6531801489909
-0.742424242424243 -1.57727436273085
-0.671717171717172 -1.49110984774936
-0.601010101010101 -1.39312909691909
-0.53030303030303 -1.28183667286652
-0.45959595959596 -1.15595668326086
-0.388888888888889 -1.0146347897732
-0.318181818181818 -0.857667759459597
-0.247474747474747 -0.685725446682524
-0.176767676767677 -0.500513248622106
-0.106060606060606 -0.304816461071112
-0.0353535353535355 -0.102380831536402
0.035353535353535 0.102380831536401
0.106060606060606 0.304816461071111
0.176767676767676 0.500513248622105
0.247474747474747 0.685725446682524
0.318181818181818 0.857667759459595
0.388888888888889 1.0146347897732
0.459595959595959 1.15595668326085
0.53030303030303 1.28183667286652
0.601010101010101 1.39312909691909
0.671717171717171 1.49110984774936
0.742424242424242 1.57727436273085
0.813131313131313 1.6531801489909
0.883838383838384 1.72033676734844
0.954545454545454 1.78013775802558
1.02525252525253 1.83382539153454
1.0959595959596 1.88247874513296
1.16666666666667 1.92701689819582
1.23737373737374 1.9682109203496
1.30808080808081 2.00670017303322
1.37878787878788 2.04300998169079
1.44949494949495 2.07756889242325
1.52020202020202 2.11072453631711
1.59090909090909 2.1427576570514
1.66161616161616 2.17389418659733
1.73232323232323 2.20431544288389
1.8030303030303 2.23416662051139
1.87373737373737 2.26356378551628
1.94444444444444 2.29259959133569
2.01515151515151 2.32134792081053
2.08585858585859 2.34986763789973
2.15656565656566 2.3782056085012
2.22727272727273 2.40639912564314
2.2979797979798 2.43447785201092
2.36868686868687 2.46246537305387
2.43939393939394 2.4903804369675
2.51010101010101 2.51823794356522
2.58080808080808 2.54604973218962
2.65151515151515 2.5738252090605
2.72222222222222 2.60157184650091
2.79292929292929 2.62929558003151
2.86363636363636 2.65700112411674
2.93434343434343 2.68469222315771
3.00505050505051 2.71237185096763
3.07575757575758 2.74004236927603
3.14646464646465 2.76770565365975
3.21717171717172 2.79536319358342
3.28787878787879 2.82301617186554
3.35858585858586 2.85066552779701
3.42929292929293 2.87831200727241
3.5 2.90595620260445
};
\path [draw=red, semithick]
(axis cs:-3.6,-2.9)
--(axis cs:-3.4,-2.9);

\path [draw=red, semithick]
(axis cs:-3.1,-2.7)
--(axis cs:-2.9,-2.7);

\path [draw=red, semithick]
(axis cs:-2.8,-2.6)
--(axis cs:-2.6,-2.6);

\path [draw=red, semithick]
(axis cs:-2.5,-2.5)
--(axis cs:-2.3,-2.5);

\path [draw=red, semithick]
(axis cs:-2.1,-2.3)
--(axis cs:-1.9,-2.3);

\path [draw=red, semithick]
(axis cs:-1.8,-2.2)
--(axis cs:-1.6,-2.2);

\path [draw=red, semithick]
(axis cs:-1.4,-2)
--(axis cs:-1.2,-2);

\path [draw=red, semithick]
(axis cs:-1.1,-1.8)
--(axis cs:-0.9,-1.8);

\path [draw=red, semithick]
(axis cs:-0.7,-1.4)
--(axis cs:-0.5,-1.4);

\path [draw=red, semithick]
(axis cs:-0.1,0)
--(axis cs:0.1,0);

\path [draw=red, semithick]
(axis cs:0.5,1.4)
--(axis cs:0.7,1.4);

\path [draw=red, semithick]
(axis cs:0.9,1.8)
--(axis cs:1.1,1.8);

\path [draw=red, semithick]
(axis cs:1.2,2)
--(axis cs:1.4,2);

\path [draw=red, semithick]
(axis cs:1.6,2.2)
--(axis cs:1.8,2.2);

\path [draw=red, semithick]
(axis cs:1.9,2.3)
--(axis cs:2.1,2.3);

\path [draw=red, semithick]
(axis cs:2.3,2.5)
--(axis cs:2.5,2.5);

\path [draw=red, semithick]
(axis cs:2.6,2.6)
--(axis cs:2.8,2.6);

\path [draw=red, semithick]
(axis cs:2.9,2.7)
--(axis cs:3.1,2.7);

\path [draw=red, semithick]
(axis cs:3.4,2.9)
--(axis cs:3.6,2.9);

\path [draw=red, semithick]
(axis cs:-3.5,-3)
--(axis cs:-3.5,-2.8);

\path [draw=red, semithick]
(axis cs:-3,-2.8)
--(axis cs:-3,-2.6);

\path [draw=red, semithick]
(axis cs:-2.7,-2.7)
--(axis cs:-2.7,-2.5);

\path [draw=red, semithick]
(axis cs:-2.4,-2.6)
--(axis cs:-2.4,-2.4);

\path [draw=red, semithick]
(axis cs:-2,-2.4)
--(axis cs:-2,-2.2);

\path [draw=red, semithick]
(axis cs:-1.7,-2.3)
--(axis cs:-1.7,-2.1);

\path [draw=red, semithick]
(axis cs:-1.3,-2.1)
--(axis cs:-1.3,-1.9);

\path [draw=red, semithick]
(axis cs:-1,-1.9)
--(axis cs:-1,-1.7);

\path [draw=red, semithick]
(axis cs:-0.6,-1.5)
--(axis cs:-0.6,-1.3);

\path [draw=red, semithick]
(axis cs:0,-0.1)
--(axis cs:0,0.1);

\path [draw=red, semithick]
(axis cs:0.6,1.3)
--(axis cs:0.6,1.5);

\path [draw=red, semithick]
(axis cs:1,1.7)
--(axis cs:1,1.9);

\path [draw=red, semithick]
(axis cs:1.3,1.9)
--(axis cs:1.3,2.1);

\path [draw=red, semithick]
(axis cs:1.7,2.1)
--(axis cs:1.7,2.3);

\path [draw=red, semithick]
(axis cs:2,2.2)
--(axis cs:2,2.4);

\path [draw=red, semithick]
(axis cs:2.4,2.4)
--(axis cs:2.4,2.6);

\path [draw=red, semithick]
(axis cs:2.7,2.5)
--(axis cs:2.7,2.7);

\path [draw=red, semithick]
(axis cs:3,2.6)
--(axis cs:3,2.8);

\path [draw=red, semithick]
(axis cs:3.5,2.8)
--(axis cs:3.5,3);

\addplot [semithick, red, mark=*, mark size=0, mark options={solid}, only marks]
table {%
-3.5 -2.9
-3 -2.7
-2.7 -2.6
-2.4 -2.5
-2 -2.3
-1.7 -2.2
-1.3 -2
-1 -1.8
-0.6 -1.4
0 0
0.6 1.4
1 1.8
1.3 2
1.7 2.2
2 2.3
2.4 2.5
2.7 2.6
3 2.7
3.5 2.9
};
\end{axis}

\end{tikzpicture}

				    \caption{Кривая начального намагничивания пермаллоя.}
    			\end{center}
	    	\end{figure}
            \par Для аппроксимации воспользуемся сследующей функцией: $y=A\tanh(Bx) + Cx$. В результате получим следующие
            значения: $A=1.538 \pm 0.0$ дел, $B=1.631 \pm 0.001$ дел$^{-1}$, $C=1.3677 \pm 0.0004$ дел$^{-1}$.
            \par Из графика видно, что наиюольшее значение намгначиваемости и начальное совпадают, поэтому
            $\mu_{\text{нач}}=\mu_{max}=(7.8\pm0.8)\cdot10^{3}$.
        
        \subsection{Вычисление временной константы}
            Вычислим временную константу и сравним со значением полученным из характеристик установки. Результаты занесём в
            таблицу 7.
            \begin{table}[H]
                \captionsetup{font={small}, labelformat=fullparents, labelsep=fill, labelfont=bf, justification=raggedleft,
                singlelinecheck=false, skip=-0.2cm}
                \caption{Определение коэффициентов калибровки}
                \begin{center}
                    \begin{tabular}{|p{2.1cm}|p{2.1cm}|p{2.1cm}|p{2.1cm}|p{2.1cm}|p{2.1cm}|}
                        \hline
                        $2Y_{\text{вх}}$, дел & $K_{\text{вх}}$, В/дел & $U_{\text{вх}}$, В & $2Y_{\text{вых}}$, дел &
                        $K_{\text{вх}}$, В/дел & $U_{\text{вых}}$, мВ \\
                        \hline
                        $6.0\pm0.1$ & 1 & $6.0\pm0.1$ & $4.8\pm0.1$ & 0.01 & $48\pm1$ \\
                        \hline
                    \end{tabular}
                \end{center}
            \end{table}
            \par Вычислим значение временной константы по следующей формуле:
            \begin{equation}
                \tau=\frac{U_{\text{вх}}}{2\pi \nu U_{\text{вых}}}
            \end{equation}
            где $\nu=50$ Гц. В итоге получим $\tau=0.40\pm0.01$ c, что совпадает со значением из параметров установки
            $\tau=0.4$ c.
            
    \section{Обсуждение результатов}
        \par В данной работе исследовались магнитные характеристики нескольких веществ. Результаты не очень сошлись с
        табличными (только по порядку велечины). Наиболее трудной частью работы стало снятие показаний с осцилографа,
        так как прибор является довольно грубым (погрешность порядка 4\%).

    \section{Вывод}
        \par В целом лабораторную работу стоит считать проделанной качественно, одноко можно улучшить результаты при
        использовании более точного оборудования. Также особый интерес представляет исследования кривой начальной
        намагниченности кремнистого железа, так как в данной работе не был обнаружен максимум, но на него указывают
        справочные материалы.

\end{document}
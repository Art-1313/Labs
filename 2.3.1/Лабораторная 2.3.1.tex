\documentclass[12pt,a4paper]{article}
\usepackage[utf8]{inputenc}
\usepackage[english,russian]{babel}
\usepackage{indentfirst}
\usepackage{misccorr}
\usepackage{graphicx}
\usepackage{amsmath}
\usepackage{amssymb}
\usepackage{circuitikz}
\usepackage[font={small}]{caption}
\usepackage[left=20mm, top=20mm, right=20mm, bottom=20mm, nohead]{geometry}
\usepackage{float}
\usepackage{tabularx}
\usepackage{array}
\usepackage{longtable}
\usepackage{pstool}
\usepackage{pgfplots}
\usepackage{hhline}
\usepackage{multirow}
\usepackage{wrapfig}

\DeclareCaptionLabelSeparator{fill}{.\\}
\DeclareCaptionLabelFormat{fullparents}{\bothIfFirst{#1}{~}#2}

\begin{document}

	\begin{titlepage}
		\begin{center}
			{\LARGE Отчёт по лабораторной работе 2.3.1.\\}
			\vspace*{11cm}
				\textbf{\LARGE Получение и измерение вакуума.}	
			\vspace*{6.5cm}
		\end{center}
		\hfill\begin{minipage}{0.37\textwidth}
				Работу выполнил Громов Артём\\
				ЛФИ Б02-006
		\end{minipage}
		\vspace{5cm}
		\begin{center}
 			Долгопрудный, 2021 г.
		\end{center}   
	\end{titlepage}
	
	\section{Аннотация}
		\noindent\textbf{Цель работы: }
		\begin{enumerate}
			\item измерение объёмов форвакуумной и высоковакуумной частей установки;
			\item определение скорости откачки системы в стационарном режиме, а также по ухудшению и по улучшению
			вакуума.
		\end{enumerate}
		\noindent\textbf{В работе используются:} вакуумная установка с манометрами: масляным, термопарным и
		ионизационным.
		\vspace{0.5 cm}
		\par По степени разрежения вакуумные установки принято делить на три класса:
		\begin{enumerate}
			\item низковакуумные --- до $10^{-2}\div10^{-3}$ торр; 
			\item высоковакуумные --- до $10^{-4}\div10^{-7}$ торр;
			\item установки сверхвысокого вакуума --- до $10^{-8}\div10^{-11}$ торр.
		\end{enumerate}
		\par С физической точки зрения низкий вакуум переходит в высокий, когда длина свободного пробега молекул газа
		оказывается сравнима с размерами установки (а течение газа становится сугубо молекулярным); сверхвысокий вакуум
		характерен крайней важностью процессов адсорбции и десорбции частиц на поверхности вакуумной камеры.
		\par В данной работе изучаются традиционные методы откачки механическим форвакуумным насосом до давления
		$10^{-2}$ торр и диффузионным масляным насосом до давления $10^{-5}$ торр, а также методы измерения
		вакуума в этом диапазоне.
	\section{Экспериментальная установка}
	\begin{figure}[H]
		\begin{center}
			\includegraphics[width=0.9\textwidth]{C:/Users/gromo/Desktop/Установка231_1.png}
			\caption{Установка для изучения зависимости скорости звука от температуры}
		\end{center}
	\end{figure}
	\par Установка изготовлена из стекла и состоит из форвакуумного баллона (ФБ), высоковакуумного диффузионного насоса
	(ВН), высоковакуумного баллона (ВБ), масляного (М) и ионизационного (И) манометров, термопарных манометров
	($\text{М}_1$ и $\text{М}_2$), форвакуумного насоса (ФН) и соединительных кранов $\text{К}_1$, $\text{К}_2$, ...,
	$\text{К}_6$ (рис. 1). Кроме того, в состав установки входят: вариатор (автотрансформатор с регулируемым выходным
	напряжением), или реостат, и амперметр для регулирования тока нагревателя диффузионного насоса.
	\par \textit{Краны.} Все краны вакуумной установки --- стеклянные. Стенки кранов тонкие, пробки кранов --- полые и
	составляют одно целое с рукоятками. Пробки кранов притёрты к корпусам. Для герметизации используется вакуумная
	смазка. Через стенки кранов видны отверстия в пробке крана, так что всегда можно понять, как он работает. Если на
	поверхности шлифа видны круговые полосы, то кран либо плохо притёрт, либо неправильно смазан и может пропускать
	воздух. За кранами нужно внимательно следить. Краны работают лишь в том случае, если давление внутри крана меньше
	атмосферного. При этом пробка вдавливается внутрь крана.	
	\begin{figure}[H]
		\begin{center}
			\includegraphics[width=0.9\textwidth]{C:/Users/gromo/Desktop/Установка231_2.png}
			\caption{Схема действия ротационного двухпластинчатого форвакуумного насоса. В положениях «а» и «б»
			пластина «А» засасывает разреженный воздух из откачиваемого объёма, а пластина «Б» вытесняет ранее
			захваченный воздух в атмосферу. В положениях «в» и «г» пластины поменялись ролями}
		\end{center}
	\end{figure}
	\par Кран $\text{К}_1$ используется для заполнения форвакуумного насоса и вакуумной установки атмосферным
	воздухом. Во время работы установки он должен быть закрыт. Трёхходовой кран $\text{К}_2$ служит для соединения
	форвакуумного насоса с установкой или атмосферой. Кран $\text{К}_3$ отделяет высоковакуумную часть установки от
	форвакуумной. Кран $\text{К}_4$ соединяет между собой колена масляного манометра. Он должен быть открыт во все
	время работы установки и закрывается лишь при измерении давления в форвакуумной части. Краны $\text{К}_5$ и
	$\text{К}_6$ стоят по концам капилляра и соединяют его с форвакуумной и высоковакуумной частями установки.
	Суммарный объём обоих кранов 50 $\text{см}^3$. Диаметр капилляра 0.9 мм. Его длина 300 мм. На каждой установке
	указан точный объём кранов.
	\par \text{Форвакуумный насос.} Устройство и принцип действия ротационного пластинчатого форвакуумного насоса
	схематически показаны на рис. 2.
	\par В цилиндрической полости массивного корпуса размещен эксцентрично ротор так, что он постоянно соприкасается
	своей верхней частью с корпусом. В диаметральный разрез ротора вставлены две пластины, раздвигаемые пружиной и
	плотно прижимаемые к поверхности полости. Они разделяют объём между ротором и корпусом на две части.
	\par Действие насоса ясно из изображённых на рис. 2 последовательных положений пластин при вращении ротора по
	часовой стрелке. В положении «а» газ из откачиваемого объёма поступает в пространство между пластиной «А» и
	линией соприкосновения корпуса и ротора. По мере вращения это пространство увеличивается (рис. 2б), пока вход в
	него не перекроет другая пластина «Б» (рис. 2в). После того как пластина «А» пройдёт выходное отверстие и линию
	соприкосновения (рис. 2г), лопасть «Б» будет сжимать следующую порцию газа и вытеснять его через клапан в
	атмосферу.
	\par При работе с насосом следует помнить, что после остановки насоса в него обязательно нужно впускать воздух. Если
	этого не делать, то атмосферное давление может выдавить масло из насоса в патрубки и в вакуумную систему.
	Соединять насос с атмосферой следует при помощи кранов $\text{К}_1$ или $\text{К}_2$.
	\par После включения насоса его присоединяют к установке не сразу, а через некоторое время, когда насос откачает
	собственный объём и пространство, расположенное до крана $\text{К}_2$. Об этом можно судить по звуку насоса.
	Вначале насос сильно шумит, затем его звук делается мягким, и, наконец, в насосе возникает сухой стук --- это
	происходит, когда достигается хорошее разрежение.
	\par Диффузионный насос. Откачивающее действие диффузионного насоса основано на диффузии (внедрении) молекул
	разреженного воздуха в струю паров масла. 
	\begin{wrapfigure}{l}{0.5\linewidth} % l - слева, r - справа, с - центр
		\centering %выравнивание по центру
		\includegraphics[width=0.5\textwidth]{C:/Users/gromo/Desktop/Установка231_3.png} %scale масштаб
		\caption{Схема работы диффузионного насоса} % подрисуночная подпись
		\label{pic:my} %метка для ссылки по тексту
	\end{wrapfigure}
	\par Устройство одной ступени масляного диффузионного насоса схематически показано на рис. 3 (в лабораторной
	установке используется несколько откачивающих ступеней). Масло, налитое в сосуд А, подогревается электрической
	печкой. Пары масла поднимаются по трубе Б и вырываются из сопла В. Струя паров увлекает молекулы газа, которые
	поступают из откачиваемого сосуда через трубку ВВ. Дальше смесь попадает в вертикальную трубу Г. Здесь масло
	осаждается на стенках трубы и маслосборников и стекает вниз, а оставшийся газ через трубу ФВ откачивается
	форвакуумным насосом. Диффузионный насос работает наиболее эффективно при давлении, когда длина свободного
	пробега молекул воздуха примерно равна ширине кольцевого зазора между соплом В и стенками трубы ВВ. В этом случае
	пары масла увлекают молекулы воздуха из всего сечения зазора.
	\par Давление насыщенных паров масла при рабочей температуре, создаваемой обогревателем сосуда А, много больше
	$5\cdot10^{-2}$ торр. Именно поэтому пары масла создают плотную струю, которая и увлекает с собой молекулы газа.
	Если диффузионный насос включить при давлении, сравнимом с давлением насыщенного пара масла, то последнее
	никакой струи не создаст и масло будет просто окисляться и угорать.
	\par Граничное давление, выше которого диффузионный насос включать нельзя, на вакуумметрах термопарных ламп
	отмечено красной линией (около 1.2 мВ).
	\par Диффузионный насос, используемый в нашей установке (рис. 1), имеет две ступени и соответственно два сопла.
	Одно сопло вертикальное (первая ступень), второе сопло горизонтальное (вторая ступень). За второй ступенью имеется
	ещё одна печь, но пар из этой печи поступает не в сопло, а по тонкой трубке подводится ближе к печке первой ступени.
	Эта печь осуществляет фракционирование масла. Легколетучие фракции масла, испаряясь, поступают в первую ступень,
	обогащая её легколетучей фракцией масла. По этой причине плотность струи первой ступени выше, и эта ступень
	начинает откачивать газ при более высоком давлении в форвакуумной части установки. Вторая ступень обогащается
	малолетучими фракциями. Плотность струи второй ступени меньше, но меньше и давление насыщенных паров масла в
	этой ступени. Соответственно в откачиваемый объём поступает меньше паров масла, и его удаётся откачать до более
	высокого вакуума, чем если бы мы работали только с одной ступенью.
	\par Спираль, опущенная в масло, подогревается переменным током ($U\approx12$ В). Ток регулируется
	автотрансформатором в пределах от 1 до 1.5 А. При включении подогрева давление в системе сначала возрастает
	вследствие выделения растворенного в масле воздуха. Минут через 10 после начала подогрева начинается интенсивное
	испарение масла, заметное по появлению на стенках насоса плёнки конденсирующихся паров. Не следует допускать
	слишком интенсивного кипения масла.\\
	\begin{wrapfigure}{r}{0.3\linewidth} % l - слева, r - справа, с - центр
		\centering %выравнивание по центру
		\includegraphics[width=0.3\textwidth]{C:/Users/gromo/Desktop/Установка231_4.png} %scale масштаб
		\caption{Схема термопарного манометра с лампой ЛТ-2} % подрисуночная подпись
		\label{pic:my} %метка для ссылки по тексту
	\end{wrapfigure}
	\par \textit{Масляный манометр} представляет собой U-образную трубку, до половины наполненную вязким маслом,
	обладающим весьма низким давлением насыщенных паров. Так как плотность масла $\rho=0.9$ г/$\text{см}^{3}$ мала,
	то при помощи манометра можно измерить только небольшие разности давлений (до нескольких торр). Во время откачки
	и заполнения установки атмосферным воздухом кран $\text{К}_4$, соединяющий оба колена манометра, должен быть
	открыт во избежание выброса масла и загрязнения установки. Кран $\text{К}_4$ закрывается только при измерении
	давления U-образным манометром.
	\par Из-за большой вязкости масла уровни в манометре устанавливаются не сразу.
	\par \textit{Термопарный манометр.} Чувствительным элементом манометра является платино-родиевая термопара,
	спаянная с никелевой нитью накала и заключённая в стеклянный баллон (лампа ЛТ-2 или ПМТ-2). Устройство термопары
	пояснено на рис. 4. По нити накала НН пропускается ток постоянной величины. Для установки тока служит потенциометр
	R --- «Рег. тока накала», расположенный на передней панели вакуумметра. Термопара ТТ присоединяется к
	милливольтметру, показания которого определяются температурой нити накала и зависят от отдачи тепла в окружающее
	пространство.
	\par Потери тепла определяются теплопроводностью нити и термопары, теплопроводностью газа, переносом тепла
	конвективными потоками газа внутри лампы и теплоизлучением нити (инфракрасное тепловое излучение). В обычном
	режиме лампы основную роль играет теплопроводность газа. При давлениях $\geqslant1$ торр теплопроводность газа, а
	вместе с ней и ЭДС термопары практически не зависят от давления газа, и прибор не работает.
	\par При улучшении вакуума средний свободный пробег молекул становится сравнимым с диаметром нити, теплоотвод
	падает и температура спая возрастает. При вакууме порядка $10^{-3}$ торр теплоотвод, осуществляемый газом, становится
	сравнимым с другими видами потерь тепла и температура нити становится практически постоянной.
	\par \textit{Ионизационный манометр.} Схема ионизационного манометра изображена на рис. 5. Он представляет собой
	трёхэлектродную лампу. Электроны испускаются накалённым катодом и увлекаются электрическим полем к аноду,
	имеющему вид редкой спирали. Проскакивая за её витки, электроны замедляются полем коллектора и возвращаются к
	катоду, а от него вновь увлекаются к аноду. Прежде чем осесть на аноде, они успеают много раз пересечь пространство
	между катодом и коллектором. На своём пути электроны ионизуют молекулы газа. Ионы, образовавшиеся между анодом и
	коллектором, притягиваются полем коллектора и определяют его ток.
	\begin{wrapfigure}{l}{0.4\linewidth} % l - слева, r - справа, с - центр
		\centering %выравнивание по центру
		\includegraphics[width=0.4\textwidth]{C:/Users/gromo/Desktop/Установка231_5.png} %scale масштаб
		\caption{Схема ионизационной лампы ЛМ-2} % подрисуночная подпись
		\label{pic:my} %метка для ссылки по тексту
	\end{wrapfigure}
	\par Ионный ток в цепи коллектора пропорционален плотности газа и поэтому может служить мерой давления. Вероятность
	ионизации зависит от рода газа, заполняющего лампу (а значит, и откачиваемый объём). Калибровка манометра верна,
	если остаточным газом является воздух.
	\par Накалённый катод ионизационного манометра перегорает, если давление в системе превышает $1\cdot10^{-3}$ торр.
	Поэтому включать ионизационный манометр можно, только убедившись по термопарному манометру, что давление в
	системе не превышает $10^{-3}$ торр.
	\par При измерении нить накала ионизационного манометра сильно греется. При этом она сама, окружающие её электроды
	и стенки стеклянного баллона могут десорбировать поглощённые ранее газы. Выделяющиеся газы изменяют давление в
	лампе и приводят к неверным показаниям. Поэтому перед измерениями ионизационный манометр прогревается
	(обезгаживается) в течение 10–15 мин. Для прогрева пропускается ток через спиральный анод лампы.
	\noindent \textbf{Процесс откачки.} Производительность насоса определяется скоростью откачки $W$ (л/с): $W$ --- это
	объём газа, удаляемого из сосуда при данном давлении за единицу времени. Скорость откачки форвакуумного насоса равна
	ёмкости воздухозаборной камеры, умноженной на число оборотов в секунду.
	\par Рассмотрим обычную схему откачки. Разделим вакуумную систему на две части: «откачиваемый объём» (в состав
	которого включим используемые для работы части установки) и «насос», к которому, кроме самого насоса, отнесём
	трубопроводы и краны, через которые производится откачка нашего объёма. Обозначим через $Q_{\text{д}}$ количество
	газа, десорбирующегося с поверхности откачиваемого объёма в единицу времени, через $Q_{\text{и}}$ --- количество газа,
	проникающего в единицу времени в этот объём извне --- через течи. Будем считать, что насос обладает скоростью откачки
	$W$ и в то же время сам является источником газа; пусть $Q_{\text{н}}$ --- поток газа, поступающего из насоса назад в
	откачиваемую систему. Будем измерять количество газа $Q_{\text{д}}$, $Q_{\text{н}}$ и $Q_{\text{и}}$ в единицах $PV$
	(легко видеть, что это произведение с точностью до множителя $RT/\mu$ равно массе газа). Основное уравнение,
	описывающее процесс откачки, имеет вид
	\begin{equation}
		-VdP=\left(PW-Q_{\text{д}}-Q_{\text{н}}-Q_{\text{и}}\right)dt.
	\end{equation}
	\par Левая часть этого уравнения равна убыли газа в откачиваемом объёме $V$ , а правая определяет количество газа,
	уносимого насосом, и количество прибывающего вследствие перечисленных выше причин за время dt. При достижении
	предельного вакуума (давление $P_{\text{пр}}$)
	$$
		\frac{dP}{dt}=0,
	$$
	\noindent так что
	\begin{equation}
		P_{\text{пр}}W=Q_{\text{д}}+Q_{\text{н}}+Q_{\text{и}}.
	\end{equation}
	\noindent Из этого уравнения найдём формулу, выражающую скорость откачки через предельный вакуум:
	$$
		W=\frac{\sum{Q_i}}{P_{\text{пр}}}
	$$
	Обычно $Q_{\text{и}}$ постоянно, a $Q_{\text{н}}$ и $Q_{\text{д}}$ слабо зависят от времени, поэтому в наших условиях
	все эти члены можно считать постоянными. Считая также постоянной скорость откачки $W$, уравнение (1) можно
	проинтегрировать и, используя (2),	получить.
	\begin{equation}
		P-P_{\text{пр}}=\left(P_0-P_{\text{пр}}\right)\text{exp}\left(-\frac{W}{V}t\right)
	\end{equation}
	где $P_0$ --- начальное давление. Оно обычно велико по сравнению с $P_{\text{пр}}$, поэтому можно записать, что
	\begin{equation}
		P=\left(P_0\right)\text{exp}\left(-\frac{W}{V}t\right)
	\end{equation}
	Постоянная времени откачки $\tau=V/W$ является мерой эффективности откачной системы.
	\par Рассмотрим теперь, чем определяется скорость откачки системы. По условию эта скорость характеризует действие всей
	откачивающей системы, которую мы пока обозначали общим понятием «насос». На самом деле эта система состоит из
	собственно насоса, а также из кранов и трубопроводов, соединяющих его с откачиваемым объёмом.
	\par При математическом описании откачивающей системы возникают уравнения, очень похожие на уравнения Кирхгофа,
	описывающие протекание тока в электрических цепях. Перепад давления $\Delta P$ заменяет разность электрических
	потенциалов, поток газа --- силу тока, а пропускная способность элементов вакуумной системы --- проводимость элементов
	цепи. Закон сложения пропускных способностей аналогичен закону сложения проводимостей. При последовательном
	соединении элементов.
	\begin{equation}
		\frac{1}{W}=\frac{1}{W_{\text{н}}}+\frac{1}{C_1}+\frac{1}{C_2}+...,
	\end{equation}
	где $W$ --- скорость откачки системы, $W_{\text{н}}$ --- скорость откачки собственно насоса, a $C_1$, $C_2$ и т. д. ---
	пропускные способности элементов вакуумной системы. Формула (5) показывает, что пропускная способность
	трубопроводов столь же сильно влияет на эффективность откачки, как и производительность насоса. Не имеет смысла
	ставить большой насос, если соединительные трубки недостаточно широки. Практическое правило заключается в том, что
	диаметры соединительных трубок не очень существенны в форвакуумной части установки и крайне важны в
	высоковакуумной. Диаметр трубок в этой части должен быть не меньше, чем диаметр самого насоса.
	\textbf{Течение газа через трубу.} Характер течения газа существенно зависит от соотношения между размерами системы
	и длиной свободного пробега молекул. При атмосферном давлении и даже при понижении давления до форвакуумного
	длина свободного пробега меньше диаметра трубок и течение откачиваемого газа определяется его вязкостью, т. е.
	взаимодействием его молекул. При переходе к высокому вакууму картина меняется. Столкновения молекул между собой
	начинают играть меньшую роль, чем соударения со стенками. Течение газа в трубе напоминает в этих условиях диффузию
	газа из области больших концентраций в области, где концентрация ниже, причём роль длины свободного пробега играет
	ширина трубы.
	\par Для количества газа, протекающего через трубу в условиях высокого вакуума, или, как говорят, в кнудсеновском
	режиме, справедлива формула:
	\begin{equation}
		\frac{d\left(PV\right)}{dt}=\frac{4}{3}r^3\sqrt{\frac{2\pi RT}{\mu}}\frac{P_2-P_1}{L}.
	\end{equation}
	Применим эту формулу к случаю, когда труба соединяет установку с насосом.
	\par Пренебрежём давлением $P_1$ у конца, обращённого к насосу. Будем измерять количество газа, покидающего
	установку при давлении $P=P_2$. Пропускная способность трубы
	\begin{equation}
		C_{\text{тр}}=\left(\frac{dV}{dt}\right)_{\text{тр}}=\frac{4}{3}\frac{r^3}{L}\sqrt{\frac{2\pi RT}{\mu}}.
	\end{equation}
	Мы видим, что пропускная способность зависит от радиуса трубы в третьей степени и обратно пропорциональна её длине.
	В вакуумных установках следует поэтому применять широкие короткие трубы.
	\par При расчёте вакуумных систем нужно принимать во внимание также пропускную способность отверстий, например, в
	кранах. Для них имеется формула
	\begin{equation}
		\nu=\frac{1}{4}Sn\overline{v},
	\end{equation}
	где $\nu$ --- число молекул, вылетающих из отверстия в вакуум в единицу времени, $S$ --- площадь отверстия, $n$ ---
	концентрация молекул перед отверстием, $\overline{v}$ --- средняя скорость молекул газа.
	\par С другой стороны, $\nu=dN/dt$, $N=PV/kT$, $n=P/kT$, и аналогично формуле (7) для количества газа, покидающего
	установку при давлении $P$, получается пропускная способность отверстия
	\begin{equation}
		C_{\text{отв}}=\left(\frac{dV}{dt}\right)_{\text{отв}}=S\frac{\overline{v}}{4}.
	\end{equation}
	Для воздуха при комнатной температуре $\overline{v}/4$ = 110 м/с = 11 $\text{л/с}\cdot\text{см}^2$.
	\par Формулу (9) можно получить непосредственно из условия
	$$
		v=C_{\text{отв}}n,
	$$
	выражающего эквивалентность откачивающего действия отверстия, открытого в высокий вакуум, и поршня, расширяющего
	объём со скоростью $C_{\text{отв}}$(л/с).
	\par Для диффузионного насоса можно считать, что каждая молекула воздуха, попавшая в кольцевой зазор между соплом и
	стенками насоса, увлекается струей пара и не возвращается обратно в откачиваемый объём. Скорость откачки такого насоса
	можно считать равной пропускной способности отверстия с площадью, равной площади кольцевого зазора, т. е. насос
	качает как кольцевой зазор, с одной стороны которого расположен откачиваемый объём, а с другой --- пустота.
	\section{Резульаьы измерений и обработка данных}
		\subsection{Определение объёма форвакуумной и высоковакуумной частей установки}
		\par Для определения объёмов откачаем установку до давления $10^{-2}$ торр. Затем выпустим из заранее
		подготовленной части установки $V_0=50\text{ см}^3$ воздуха при атмосферном давлении $P_{\text{атм}}=100.4$ кПа.
		По изменениям показаний маслянного манометра определим объёмы, пользуясь законом Бойля-Мариотта.
		\begin{equation}
			V=\frac{P_{\text{атм}}V_0}{\rho g(h_1-h_2)}-V_0
		\end{equation}
		где $\rho=0.885\text{ г/см}^{3}$ --- плотность масла, $V$ --- объём интересуещего нас балона, $h_1$ и $h_2$ --- уровни
		масла в коленах манометра. 
		\begin{table}[H]
			\captionsetup{font={small}, labelformat=fullparents, labelsep=fill, labelfont=bf, justification=raggedleft,
			singlelinecheck=false, skip=-0.2cm}
			\caption{Значения резонансных частот}
			\begin{center}
				\begin{tabular}{|p{1cm}|p{1.4cm}|p{1.4cm}|p{2.1cm}|}
				\hline
				Балон & $h_1$, см & $h_2$, см & $V$, л \\
				\hline
				ФБ & $8.6$ & $34.6$ & $2.17\pm0.01$\\
				\hline
				ВБ & $13.5$ & $30.0$ & $1.28\pm0.01$\\
				\hline
				\end{tabular}
			\end{center}
		\end{table}
		\noindent Погрешность измерений составила 1 мм.
		\subsection{Получение высокого вакуума и измерение скорости откачки}
		\par Найдём скорость откачки по улучшению вакуума во время откачки. Для этого построим графики зависимости
		ln$(P-P_{\text{пр}})$ от $t$. Для расчёта коэффициентов аппроксимирующей прямой воспользуемся МНК. Таблицы
		значений помещены в приложение. $P_{\text{пр}}=1.4\cdot10^{-4}$ торр.
		\begin{figure}[H]
			\begin{minipage}[h]{0.49\linewidth}
				\begin{center}
					% This file was created with tikzplotlib v0.9.16.
\begin{tikzpicture}

\begin{axis}[
legend cell align={left},
legend style={
  fill opacity=0.8,
  draw opacity=1,
  text opacity=1,
  at={(0.03,0.97)},
  anchor=north west,
  draw=white!80!black
},
height=11.2cm,
tick align=inside,
major tick length=0.2cm,
minor tick length=0.1cm,
tick pos=left,
xmin=0, xmax=600,
xtick={0, 50, 100, 150, 200, 250, 300, 350, 400, 450, 500, 550, 600},
minor x tick num=4,
xmajorgrids,
minor x grid style={dotted,black},
xminorgrids,
xtick style={color=black},
xlabel={$D_{\text{винта}}$, мкм},
ymin=0, ymax=500,
ytick={0, 50, 100, 150, 200, 250, 300, 350, 400, 450, 500},
ytick style={color=black},
minor y tick num=4,
ymajorgrids,
minor y grid style={dotted,black},
yminorgrids,
ytick style={color=black},
ylabel={$D_{\text{расчёт}}$, мкм}
]
\path [draw=red, semithick]
(axis cs:40,32.9861111111111)
--(axis cs:60,32.9861111111111);

\path [draw=red, semithick]
(axis cs:90,98.9583333333333)
--(axis cs:110,98.9583333333333);

\path [draw=red, semithick]
(axis cs:140,131.944444444444)
--(axis cs:160,131.944444444444);

\path [draw=red, semithick]
(axis cs:190,164.930555555556)
--(axis cs:210,164.930555555556);

\path [draw=red, semithick]
(axis cs:240,197.916666666667)
--(axis cs:260,197.916666666667);

\path [draw=red, semithick]
(axis cs:290,230.902777777778)
--(axis cs:310,230.902777777778);

\path [draw=red, semithick]
(axis cs:340,263.888888888889)
--(axis cs:360,263.888888888889);

\path [draw=red, semithick]
(axis cs:390,296.875)
--(axis cs:410,296.875);

\path [draw=red, semithick]
(axis cs:440,362.847222222222)
--(axis cs:460,362.847222222222);

\path [draw=red, semithick]
(axis cs:490,395.833333333333)
--(axis cs:510,395.833333333333);

\path [draw=red, semithick]
(axis cs:50,-0.00334608800363867)
--(axis cs:50,65.9755683102259);

\path [draw=red, semithick]
(axis cs:100,65.9421196383399)
--(axis cs:100,131.974547028327);

\path [draw=red, semithick]
(axis cs:150,98.9048365902785)
--(axis cs:150,164.98405229861);

\path [draw=red, semithick]
(axis cs:200,131.860893814377)
--(axis cs:200,198.000217296734);

\path [draw=red, semithick]
(axis cs:250,164.810309447695)
--(axis cs:250,231.023023885639);

\path [draw=red, semithick]
(axis cs:300,197.75310554622)
--(axis cs:300,264.052450009336);

\path [draw=red, semithick]
(axis cs:350,230.689308019177)
--(axis cs:350,297.088469758601);

\path [draw=red, semithick]
(axis cs:400,263.618946552474)
--(axis cs:400,330.131053447526);

\path [draw=red, semithick]
(axis cs:450,329.458668896105)
--(axis cs:450,396.235775548339);

\path [draw=red, semithick]
(axis cs:500,362.368830131627)
--(axis cs:500,429.29783653504);

\addplot [semithick, red, forget plot]
table {%
0 0
500 392.40620488787
};
\path [draw=green!50.1960784313725!black, semithick]
(axis cs:90,53.0442477876106)
--(axis cs:110,53.0442477876106);

\path [draw=green!50.1960784313725!black, semithick]
(axis cs:140,95.1428571428572)
--(axis cs:160,95.1428571428572);

\path [draw=green!50.1960784313725!black, semithick]
(axis cs:190,148)
--(axis cs:210,148);

\path [draw=green!50.1960784313725!black, semithick]
(axis cs:240,188.913461538462)
--(axis cs:260,188.913461538462);

\path [draw=green!50.1960784313725!black, semithick]
(axis cs:290,233.740384615385)
--(axis cs:310,233.740384615385);

\path [draw=green!50.1960784313725!black, semithick]
(axis cs:340,270.260869565217)
--(axis cs:360,270.260869565217);

\path [draw=green!50.1960784313725!black, semithick]
(axis cs:390,326.204081632653)
--(axis cs:410,326.204081632653);

\path [draw=green!50.1960784313725!black, semithick]
(axis cs:440,360)
--(axis cs:460,360);

\path [draw=green!50.1960784313725!black, semithick]
(axis cs:490,396)
--(axis cs:510,396);

\path [draw=green!50.1960784313725!black, semithick]
(axis cs:540,468)
--(axis cs:560,468);

\path [draw=green!50.1960784313725!black, semithick]
(axis cs:100,52.4114526583863)
--(axis cs:100,53.6770429168349);

\path [draw=green!50.1960784313725!black, semithick]
(axis cs:150,91.0666302830339)
--(axis cs:150,99.2190840026804);

\path [draw=green!50.1960784313725!black, semithick]
(axis cs:200,141.683425571105)
--(axis cs:200,154.316574428895);

\path [draw=green!50.1960784313725!black, semithick]
(axis cs:250,182.271308845487)
--(axis cs:250,195.555614231436);

\path [draw=green!50.1960784313725!black, semithick]
(axis cs:300,223.88999208276)
--(axis cs:300,243.590777148009);

\path [draw=green!50.1960784313725!black, semithick]
(axis cs:350,257.145804132173)
--(axis cs:350,283.375934998262);

\path [draw=green!50.1960784313725!black, semithick]
(axis cs:400,311.969339856676)
--(axis cs:400,340.438823408631);

\path [draw=green!50.1960784313725!black, semithick]
(axis cs:450,341.133063042629)
--(axis cs:450,378.866936957371);

\path [draw=green!50.1960784313725!black, semithick]
(axis cs:500,376.956019710008)
--(axis cs:500,415.043980289992);

\path [draw=green!50.1960784313725!black, semithick]
(axis cs:550,448.557402072788)
--(axis cs:550,487.442597927212);

\addplot [semithick, green!50.1960784313725!black, forget plot]
table {%
0 0
550 439.564038903656
};
\addplot [semithick, red, mark=*, mark size=3, mark options={solid}, only marks]
table {%
50 32.9861111111111
100 98.9583333333333
150 131.944444444444
200 164.930555555556
250 197.916666666667
300 230.902777777778
350 263.888888888889
400 296.875
450 362.847222222222
500 395.833333333333
};
\addlegendentry{Линза}
\addplot [semithick, green!50.1960784313725!black, mark=*, mark size=3, mark options={solid}, only marks]
table {%
100 53.0442477876106
150 95.1428571428572
200 148
250 188.913461538462
300 233.740384615385
350 270.260869565217
400 326.204081632653
450 360
500 396
550 468
};
\addlegendentry{Спектр}
\end{axis}

\end{tikzpicture}
\\
				\end{center}
			\end{minipage}
			\hfill
			\begin{minipage}[h]{0.49\linewidth}	
				\begin{center}
					% This file was created with tikzplotlib v0.9.16.
\begin{tikzpicture}

\begin{axis}[
    height=11cm,
    tick align=inside,
    major tick length=0.2cm,
    minor tick length=0.1cm,
    tick pos=left,
    xmin=-4.1, xmax=4.1,
    xtick={-5, -4, -3, -2, -1, 0, 1, 2, 3, 4, 5},
    minor x tick num=4,
    xmajorgrids,
    minor x grid style={dotted,black},
    xminorgrids,
    xtick style={color=black},
    xlabel={$m$},
    ymin=-1.2, ymax=1.2,
    ytick={-1.5, -1, -0.5, 0, 0.5, 1, 1.5},
    minor y tick num=4,
    ymajorgrids,
    minor y grid style={dotted,black},
    yminorgrids,
    ytick style={color=black},
    ylabel={$x_m$, мм}
]
\path [draw=red, semithick]
(axis cs:1,0.14)
--(axis cs:1,0.18);

\path [draw=red, semithick]
(axis cs:2,0.48)
--(axis cs:2,0.52);

\path [draw=red, semithick]
(axis cs:3,0.7)
--(axis cs:3,0.74);

\path [draw=red, semithick]
(axis cs:4,1.02)
--(axis cs:4,1.06);

\path [draw=red, semithick]
(axis cs:-1,-0.18)
--(axis cs:-1,-0.14);

\path [draw=red, semithick]
(axis cs:-2,-0.5)
--(axis cs:-2,-0.46);

\path [draw=red, semithick]
(axis cs:-3,-0.74)
--(axis cs:-3,-0.7);

\path [draw=red, semithick]
(axis cs:-4,-1.08)
--(axis cs:-4,-1.04);

\addplot [semithick, black]
table {%
1 0.249999999998364
2 0.499999999996728
3 0.749999999995093
4 0.999999999993457
-1 -0.249999999998364
-2 -0.499999999996728
-3 -0.749999999995093
-4 -0.999999999993457
};
\addplot [semithick, red, mark=square*, mark size=3, mark options={solid}, only marks]
table {%
1 0.16
2 0.5
3 0.72
4 1.04
-1 -0.16
-2 -0.48
-3 -0.72
-4 -1.06
};
\end{axis}

\end{tikzpicture}
\\ 
				\end{center}
			\end{minipage}
			\captionsetup{font={small}, justification=justified}
			\caption[figure]{Графики зависимости ln$(P-P_{\text{пр}})$ от $t$}
		\end{figure}
		\par Расчитаем $W$ по формуле $W=-\tau V_{\text{ВБ}}$:
		$$
			W_1=0.29\pm0.04\text{ л/с} \ \ \ \ \ W_2=0.27\pm0.06\text{ л/с}.
		$$ 
		\par Теперь определим велечину потока $Q_{\text{н}}$. Для этого построим графики зависимости $P$ от $t$ при
		ухудшении вакуума. Вновь воспользуемся методом наименьших квадратов, а затем вычисли занчение $Q_{\text{н}}$
		по следующей формуле:
		$$
			Q_{\text{н}}=P_{\text{пр}}W-\frac{dP}{dt}V_{\text{ВБ}}.
		$$
		$$
			Q_{\text{н1}}=(2.9\pm0.4)\cdot10^{-5}\text{ л}\cdot\text{торр}/\text{с}, 
			\ \ \ \ \ 
			Q_{\text{н2}}=(2.6\pm0.7)\cdot10^{-5}\text{ л}\cdot\text{торр}/\text{с}.
		$$
		\begin{figure}[H]
			\begin{minipage}[h]{0.49\linewidth}
				\begin{center}
					% This file was created with tikzplotlib v0.9.16.
\begin{tikzpicture}

\definecolor{color0}{rgb}{0.12156862745098,0.466666666666667,0.705882352941177}

\begin{axis}[
    height=7.2cm,
    tick align=inside,
    major tick length=0.2cm,
    minor tick length=0.1cm,
    tick pos=left,
    x grid style={white!69.0196078431373!black},
    xmin=-10, xmax=140,
    xtick={-40, 0, 40, 80, 120, 160},
    minor x tick num=3,
    xmajorgrids,
    minor x grid style={dotted,black},
    xminorgrids,
    xtick style={color=black},
    xlabel={$l$, \text{см}},
    ymin = -0.2, ymax = 0.8,
    ytick = {-0.4, 0, 0.4, 0.8},
    ytick style={color=black},
    minor y tick num=3,
    ymajorgrids,
    minor y grid style={dotted,black},
    yminorgrids,
    ytick style={color=black},
    ylabel={$\upsilon_2$}
]
\path [draw=red, semithick]
(axis cs:-6,0.225806451612903)
--(axis cs:-2,0.225806451612903);

\path [draw=red, semithick]
(axis cs:-2,0.6)
--(axis cs:2,0.6);

\path [draw=red, semithick]
(axis cs:0,0.133333333333333)
--(axis cs:4,0.133333333333333);

\path [draw=red, semithick]
(axis cs:2,0.214285714285714)
--(axis cs:6,0.214285714285714);

\path [draw=red, semithick]
(axis cs:4,0.2)
--(axis cs:8,0.2);

\path [draw=red, semithick]
(axis cs:6,0.161290322580645)
--(axis cs:10,0.161290322580645);

\path [draw=red, semithick]
(axis cs:8,0.225806451612903)
--(axis cs:12,0.225806451612903);

\path [draw=red, semithick]
(axis cs:10,0.225806451612903)
--(axis cs:14,0.225806451612903);

\path [draw=red, semithick]
(axis cs:14,0.161290322580645)
--(axis cs:18,0.161290322580645);

\path [draw=red, semithick]
(axis cs:26,0.0967741935483871)
--(axis cs:30,0.0967741935483871);

\path [draw=red, semithick]
(axis cs:36,0.0322580645161291)
--(axis cs:40,0.0322580645161291);

\path [draw=red, semithick]
(axis cs:46,0.0666666666666667)
--(axis cs:50,0.0666666666666667);

\path [draw=red, semithick]
(axis cs:56,0.0322580645161291)
--(axis cs:60,0.0322580645161291);

\path [draw=red, semithick]
(axis cs:66,0.0322580645161291)
--(axis cs:70,0.0322580645161291);

\path [draw=red, semithick]
(axis cs:76,0.0666666666666667)
--(axis cs:80,0.0666666666666667);

\path [draw=red, semithick]
(axis cs:86,0.0322580645161291)
--(axis cs:90,0.0322580645161291);

\path [draw=red, semithick]
(axis cs:96,0.0666666666666667)
--(axis cs:100,0.0666666666666667);

\path [draw=red, semithick]
(axis cs:106,0.0967741935483871)
--(axis cs:110,0.0967741935483871);

\path [draw=red, semithick]
(axis cs:110,0.225806451612903)
--(axis cs:114,0.225806451612903);

\path [draw=red, semithick]
(axis cs:114,0.290322580645161)
--(axis cs:118,0.290322580645161);

\path [draw=red, semithick]
(axis cs:116,0.290322580645161)
--(axis cs:120,0.290322580645161);

\path [draw=red, semithick]
(axis cs:118,0.533333333333333)
--(axis cs:122,0.533333333333333);

\path [draw=red, semithick]
(axis cs:120,0.548387096774194)
--(axis cs:124,0.548387096774194);

\path [draw=red, semithick]
(axis cs:122,0.161290322580645)
--(axis cs:126,0.161290322580645);

\path [draw=red, semithick]
(axis cs:124,0.161290322580645)
--(axis cs:128,0.161290322580645);

\path [draw=red, semithick]
(axis cs:126,0.0967741935483871)
--(axis cs:130,0.0967741935483871);

\path [draw=red, semithick]
(axis cs:128,0.0967741935483871)
--(axis cs:132,0.0967741935483871);

\path [draw=red, semithick]
(axis cs:-4,0.16988541584789)
--(axis cs:-4,0.281727487377916);

\path [draw=red, semithick]
(axis cs:0,0.524575276673435)
--(axis cs:0,0.675424723326565);

\path [draw=red, semithick]
(axis cs:2,0.0799074876436831)
--(axis cs:2,0.186759179022984);

\path [draw=red, semithick]
(axis cs:4,0.152955024080759)
--(axis cs:4,0.27561640449067);

\path [draw=red, semithick]
(axis cs:6,0.143431457505076)
--(axis cs:6,0.256568542494924);

\path [draw=red, semithick]
(axis cs:8,0.108312499224317)
--(axis cs:8,0.214268145936973);

\path [draw=red, semithick]
(axis cs:10,0.16988541584789)
--(axis cs:10,0.281727487377916);

\path [draw=red, semithick]
(axis cs:12,0.16988541584789)
--(axis cs:12,0.281727487377916);

\path [draw=red, semithick]
(axis cs:16,0.108312499224317)
--(axis cs:16,0.214268145936973);

\path [draw=red, semithick]
(axis cs:28,0.0467395826007438)
--(axis cs:28,0.14680880449603);

\path [draw=red, semithick]
(axis cs:38,-0.0148333340228294)
--(axis cs:38,0.0793494630550875);

\path [draw=red, semithick]
(axis cs:48,0.01638351778229)
--(axis cs:48,0.116949815551043);

\path [draw=red, semithick]
(axis cs:58,-0.0148333340228294)
--(axis cs:58,0.0793494630550875);

\path [draw=red, semithick]
(axis cs:68,-0.0148333340228294)
--(axis cs:68,0.0793494630550875);

\path [draw=red, semithick]
(axis cs:78,0.01638351778229)
--(axis cs:78,0.116949815551043);

\path [draw=red, semithick]
(axis cs:88,-0.0148333340228294)
--(axis cs:88,0.0793494630550875);

\path [draw=red, semithick]
(axis cs:98,0.01638351778229)
--(axis cs:98,0.116949815551043);

\path [draw=red, semithick]
(axis cs:108,0.0467395826007438)
--(axis cs:108,0.14680880449603);

\path [draw=red, semithick]
(axis cs:112,0.16988541584789)
--(axis cs:112,0.281727487377916);

\path [draw=red, semithick]
(axis cs:116,0.231458332471463)
--(axis cs:116,0.349186828818859);

\path [draw=red, semithick]
(axis cs:118,0.231458332471463)
--(axis cs:118,0.349186828818859);

\path [draw=red, semithick]
(axis cs:120,0.461051306812042)
--(axis cs:120,0.605615359854625);

\path [draw=red, semithick]
(axis cs:122,0.477749998965756)
--(axis cs:122,0.619024194582631);

\path [draw=red, semithick]
(axis cs:124,0.108312499224317)
--(axis cs:124,0.214268145936973);

\path [draw=red, semithick]
(axis cs:126,0.108312499224317)
--(axis cs:126,0.214268145936973);

\path [draw=red, semithick]
(axis cs:128,0.0467395826007438)
--(axis cs:128,0.14680880449603);

\path [draw=red, semithick]
(axis cs:130,0.0467395826007438)
--(axis cs:130,0.14680880449603);

\addplot [semithick, color0]
table {%
-4 0.289700909532296
-2.64646464646465 0.299561221223985
-1.29292929292929 0.303200008671337
0.0606060606060606 0.301611090044221
1.41414141414141 0.295698522381247
2.76767676767677 0.286281224103273
4.12121212121212 0.274097490756272
5.47474747474748 0.259809403983547
6.82828282828283 0.244007133727306
8.18181818181818 0.227213133659582
9.53535353535354 0.209886229842507
10.8888888888889 0.192425602617948
12.2424242424242 0.175174661726482
13.5959595959596 0.158424814655734
14.949494949495 0.142419128218066
16.3030303030303 0.127355883357614
17.6565656565657 0.113392023186687
19.010101010101 0.100646494251508
20.3636363636364 0.0892034810273152
21.7171717171717 0.079115533642817
23.0707070707071 0.0704065888339942
24.4242424242424 0.0630748841272588
25.7777777777778 0.0570957652519653
27.1313131313131 0.0524243867822743
28.4848484848485 0.0489983060083702
29.8383838383838 0.0467399700370295
31.1919191919192 0.0455590961215448
32.5454545454545 0.0453549452209991
33.8989898989899 0.0460184887888953
35.2525252525253 0.0474344687911365
36.6060606060606 0.0494833509533611
37.959595959596 0.0520431712376296
39.3131313131313 0.0549912755484642
40.6666666666667 0.0582059526682422
42.020202020202 0.0615679604219419
43.3737373737374 0.064961945071241
44.7272727272727 0.0682777539379683
46.0808080808081 0.0714116412569081
47.4343434343434 0.0742673672579583
48.7878787878788 0.0767571904776398
50.1414141414141 0.0788027532999603
51.4949494949495 0.0803358607266301
52.8484848484849 0.0812991523766312
54.2020202020202 0.0816466677151401
55.5555555555556 0.0813443045118011
56.9090909090909 0.0803701705283548
58.2626262626263 0.0787148284356191
59.6161616161616 0.076381433959822
60.969696969697 0.0733857672582888
62.3232323232323 0.0697561575244805
63.6767676767677 0.0655333008223871
65.030303030303 0.0607699711502716
66.3838383838384 0.0555306247337693
67.7373737373737 0.0498908975483373
69.0909090909091 0.0439369960710599
70.4444444444444 0.0377649812618038
71.7979797979798 0.0314799457737291
73.1515151515152 0.0251950843931509
74.5050505050505 0.0190306577087552
75.8585858585859 0.0131128490101669
77.2121212121212 0.00757251441587132
78.5656565656566 0.00254382623048871
79.9191919191919 -0.00183719046859988
81.2727272727273 -0.00543422801527125
82.6262626262626 -0.00811237310932827
83.979797979798 -0.00973989054181345
85.3333333333333 -0.0101901136909708
86.6868686868687 -0.00934344178885385
88.040404040404 -0.00708944395858157
89.3939393939394 -0.00332907002224061
90.7474747474748 0.00202303192056456
92.1010101010101 0.00903609114351488
93.4545454545455 0.0177606831737331
94.8080808080808 0.0282260919012925
96.1616161616162 0.0404375649180784
97.5151515151515 0.0543734620860032
98.8686868686869 0.0699822973345749
100.222222222222 0.0871796736878155
101.575757575758 0.105845111520537
102.929292929293 0.125818770043966
104.282828282828 0.146898062020724
105.636363636364 0.168834161709158
106.989898989899 0.191328406037028
108.343434343434 0.214028589004543
109.69696969697 0.236525149316752
111.050505050505 0.258347251245289
112.40404040404 0.278958758719469
113.757575757576 0.297754102646736
115.111111111111 0.314054041462469
116.464646464646 0.327101314909134
117.818181818182 0.336056191044795
119.171717171717 0.339991906480971
120.525252525253 0.337889999849856
121.878787878788 0.328635538500881
123.232323232323 0.311012238426636
124.585858585859 0.283697477418142
125.939393939394 0.24525720144948
127.292929292929 0.194140724291761
128.646464646465 0.128675420356469
130 0.047061310768137
};
\addplot [semithick, red, mark=*, mark size=3, mark options={solid}, only marks]
table {%
-4 0.225806451612903
0 0.6
2 0.133333333333333
4 0.214285714285714
6 0.2
8 0.161290322580645
10 0.225806451612903
12 0.225806451612903
16 0.161290322580645
28 0.0967741935483871
38 0.0322580645161291
48 0.0666666666666667
58 0.0322580645161291
68 0.0322580645161291
78 0.0666666666666667
88 0.0322580645161291
98 0.0666666666666667
108 0.0967741935483871
112 0.225806451612903
116 0.290322580645161
118 0.290322580645161
120 0.533333333333333
122 0.548387096774194
124 0.161290322580645
126 0.161290322580645
128 0.0967741935483871
130 0.0967741935483871
};
\end{axis}

\end{tikzpicture}
\\
				\end{center}
			\end{minipage}
			\hfill
			\begin{minipage}[h]{0.49\linewidth}	
				\begin{center}
					% This file was created with tikzplotlib v0.9.16.
\begin{tikzpicture}

\definecolor{color0}{rgb}{0.12156862745098,0.466666666666667,0.705882352941177}

\begin{axis}[
    height=7.2cm,
    tick align=inside,
    major tick length=0.2cm,
    minor tick length=0.1cm,
    tick pos=left,
    x grid style={white!69.0196078431373!black},
    xmin=-10, xmax=140,
    xtick={-40, 0, 40, 80, 120, 160},
    minor x tick num=3,
    xmajorgrids,
    minor x grid style={dotted,black},
    xminorgrids,
    xtick style={color=black},
    xlabel={$l$, \text{см}},
    ymin = -0.2, ymax = 0.8,
    ytick = {-0.4, 0, 0.4, 0.8},
    ytick style={color=black},
    minor y tick num=3,
    ymajorgrids,
    minor y grid style={dotted,black},
    yminorgrids,
    ytick style={color=black},
    ylabel={$\upsilon_2$}
]
\path [draw=red, semithick]
(axis cs:-6,0.225806451612903)
--(axis cs:-2,0.225806451612903);

\path [draw=red, semithick]
(axis cs:-2,0.6)
--(axis cs:2,0.6);

\path [draw=red, semithick]
(axis cs:10,0.225806451612903)
--(axis cs:14,0.225806451612903);

\path [draw=red, semithick]
(axis cs:14,0.161290322580645)
--(axis cs:18,0.161290322580645);

\path [draw=red, semithick]
(axis cs:26,0.0967741935483871)
--(axis cs:30,0.0967741935483871);

\path [draw=red, semithick]
(axis cs:36,0.0322580645161291)
--(axis cs:40,0.0322580645161291);

\path [draw=red, semithick]
(axis cs:46,0.0666666666666667)
--(axis cs:50,0.0666666666666667);

\path [draw=red, semithick]
(axis cs:56,0.0322580645161291)
--(axis cs:60,0.0322580645161291);

\path [draw=red, semithick]
(axis cs:66,0.0322580645161291)
--(axis cs:70,0.0322580645161291);

\path [draw=red, semithick]
(axis cs:76,0.0666666666666667)
--(axis cs:80,0.0666666666666667);

\path [draw=red, semithick]
(axis cs:86,0.0322580645161291)
--(axis cs:90,0.0322580645161291);

\path [draw=red, semithick]
(axis cs:96,0.0666666666666667)
--(axis cs:100,0.0666666666666667);

\path [draw=red, semithick]
(axis cs:106,0.0967741935483871)
--(axis cs:110,0.0967741935483871);

\path [draw=red, semithick]
(axis cs:110,0.225806451612903)
--(axis cs:114,0.225806451612903);

\path [draw=red, semithick]
(axis cs:114,0.290322580645161)
--(axis cs:118,0.290322580645161);

\path [draw=red, semithick]
(axis cs:118,0.533333333333333)
--(axis cs:122,0.533333333333333);

\path [draw=red, semithick]
(axis cs:120,0.548387096774194)
--(axis cs:124,0.548387096774194);

\path [draw=red, semithick]
(axis cs:124,0.161290322580645)
--(axis cs:128,0.161290322580645);

\path [draw=red, semithick]
(axis cs:128,0.0967741935483871)
--(axis cs:132,0.0967741935483871);

\path [draw=red, semithick]
(axis cs:-4,0.146722164412071)
--(axis cs:-4,0.304890738813736);

\path [draw=red, semithick]
(axis cs:0,0.493333333333333)
--(axis cs:0,0.706666666666667);

\path [draw=red, semithick]
(axis cs:12,0.146722164412071)
--(axis cs:12,0.304890738813736);

\path [draw=red, semithick]
(axis cs:16,0.0863683662851197)
--(axis cs:16,0.236212278876171);

\path [draw=red, semithick]
(axis cs:28,0.0260145681581686)
--(axis cs:28,0.167533818938606);

\path [draw=red, semithick]
(axis cs:38,-0.0343392299687825)
--(axis cs:38,0.0988553590010406);

\path [draw=red, semithick]
(axis cs:48,-0.00444444444444439)
--(axis cs:48,0.137777777777778);

\path [draw=red, semithick]
(axis cs:58,-0.0343392299687825)
--(axis cs:58,0.0988553590010406);

\path [draw=red, semithick]
(axis cs:68,-0.0343392299687825)
--(axis cs:68,0.0988553590010406);

\path [draw=red, semithick]
(axis cs:78,-0.00444444444444439)
--(axis cs:78,0.137777777777778);

\path [draw=red, semithick]
(axis cs:88,-0.0343392299687825)
--(axis cs:88,0.0988553590010406);

\path [draw=red, semithick]
(axis cs:98,-0.00444444444444439)
--(axis cs:98,0.137777777777778);

\path [draw=red, semithick]
(axis cs:108,0.0260145681581686)
--(axis cs:108,0.167533818938606);

\path [draw=red, semithick]
(axis cs:112,0.146722164412071)
--(axis cs:112,0.304890738813736);

\path [draw=red, semithick]
(axis cs:116,0.207075962539022)
--(axis cs:116,0.373569198751301);

\path [draw=red, semithick]
(axis cs:120,0.431111111111111)
--(axis cs:120,0.635555555555556);

\path [draw=red, semithick]
(axis cs:122,0.448491155046826)
--(axis cs:122,0.648283038501561);

\path [draw=red, semithick]
(axis cs:126,0.0863683662851197)
--(axis cs:126,0.236212278876171);

\path [draw=red, semithick]
(axis cs:130,0.0260145681581686)
--(axis cs:130,0.167533818938606);

\addplot [semithick, color0]
table {%
-4 0.313814118861075
-2.64646464646465 0.372719374858457
-1.29292929292929 0.414429546476477
0.0606060606060606 0.441252184268248
1.41414141414141 0.455307668066297
2.76767676767677 0.458538104134772
4.12121212121212 0.452716029232126
5.47474747474748 0.439452921584311
6.82828282828283 0.420207518768442
8.18181818181818 0.396293942506964
9.53535353535354 0.368889630372301
10.8888888888889 0.339043074402003
12.2424242424242 0.307681366624373
13.5959595959596 0.275617551494593
14.949494949495 0.243557785241332
16.3030303030303 0.21210830212385
17.6565656565657 0.181782187599588
19.010101010101 0.153005958402251
20.3636363636364 0.126125949530377
21.7171717171717 0.101414508146396
23.0707070707071 0.0790759943861873
24.4242424242424 0.059252589079113
25.7777777777778 0.0420299083785511
27.1313131313131 0.0274424253029159
28.4848484848485 0.0154786981871685
29.8383838383838 0.00608640604481389
31.1919191919192 -0.000822809159606586
32.5454545454545 -0.0053686933275373
33.8989898989899 -0.00769792413293865
35.2525252525253 -0.00797984801832828
36.6060606060606 -0.0064024102803318
37.959595959596 -0.00316827824474411
39.3131313131313 0.001508842468899
40.6666666666667 0.0074076985932389
42.020202020202 0.0143027867695034
43.3737373737374 0.0219674580144038
44.7272727272727 0.0301768290975233
46.0808080808081 0.0387105008291939
47.4343434343434 0.0473550832588656
48.7878787878788 0.0559065277839617
50.1414141414141 0.0641722661692277
51.4949494949495 0.0719731564765673
52.8484848484849 0.0791452359053693
54.2020202020202 0.0855412805433256
55.5555555555556 0.0910321720277353
56.9090909090909 0.0955080711173033
58.2626262626263 0.0988793981744253
59.6161616161616 0.101077620557964
60.969696969697 0.102055846926514
62.3232323232323 0.10178922845216
63.6767676767677 0.10027516694472
65.030303030303 0.0975333298864813
66.3838383838384 0.0936054723774249
67.7373737373737 0.088555065990942
69.0909090909091 0.0824667345400373
70.4444444444444 0.0754454967540248
71.7979797979798 0.0676158158657112
73.1515151515152 0.059120456109071
74.5050505050505 0.0501191461274099
75.8585858585859 0.0407870492920191
77.2121212121212 0.031313040931319
78.5656565656566 0.0218977924704934
79.9191919191919 0.0127516624816113
81.2727272727273 0.00409239464424182
82.6262626262626 -0.00385737738344372
83.979797979798 -0.0108728181830784
85.3333333333333 -0.0167297718840175
86.6868686868687 -0.0212078365607651
88.040404040404 -0.024093631719913
89.3939393939394 -0.025184258876587
90.7474747474748 -0.0242909552204062
92.1010101010101 -0.0212429403709502
93.4545454545455 -0.0158914562227379
94.8080808080808 -0.00811399987971576
96.1616161616162 0.00218125032074329
97.5151515151515 0.015050815694331
98.8686868686869 0.030511105006786
100.222222222222 0.048533409181362
101.575757575758 0.0690387029167914
102.929292929293 0.091892253215734
104.282828282828 0.116898034823718
105.636363636364 0.143792952578569
106.989898989899 0.172240870670332
108.343434343434 0.20182644881168
109.69696969697 0.232048785318812
111.050505050505 0.262314867102847
112.40404040404 0.291932826571698
113.757575757576 0.320105005442445
115.111111111111 0.345920825464192
116.464646464646 0.368349466051415
117.818181818182 0.386232348827805
119.171717171717 0.39827542908059
120.525252525253 0.403041294125359
121.878787878788 0.398941068581365
123.232323232323 0.384226126557329
124.585858585859 0.35697961074772
125.939393939394 0.315107758439542
127.292929292929 0.25633103442959
128.646464646465 0.178175070852219
130 0.0779614139175848
};
\addplot [semithick, red, mark=*, mark size=3, mark options={solid}, only marks]
table {%
-4 0.225806451612903
0 0.6
12 0.225806451612903
16 0.161290322580645
28 0.0967741935483871
38 0.0322580645161291
48 0.0666666666666667
58 0.0322580645161291
68 0.0322580645161291
78 0.0666666666666667
88 0.0322580645161291
98 0.0666666666666667
108 0.0967741935483871
112 0.225806451612903
116 0.290322580645161
120 0.533333333333333
122 0.548387096774194
126 0.161290322580645
130 0.0967741935483871
};
\end{axis}

\end{tikzpicture}
\\ 
				\end{center}
			\end{minipage}
			\captionsetup{font={small}, justification=justified}
			\caption[figure]{Графики зависимости $P$ от $t$}
		\end{figure}
		\par Теперь вычисли производительность нассоса с помощью создания искусственной течи. Для этого определим
		установившиеся после создания искусственной течи давления в форвакуумной и высоковауумнойчастях устновки
		$P_{\text{ВБ}}=1.7\cdot10^{-4}$ торр, $P_{\text{ВБ}}=6\cdot10^{-4}$ торр. Запишем формулу (2) для случаев когда
		капиляр перекрыт и когда он открыт:
		$$
			P_{\text{пр}}W=Q_{1}, \ \ \ \ \ P_{\text{ВБ}}W=Q_{1}+\frac{d(PV)_{\text{капилл}}}{dt}.
		$$
		В этих формулах символом $Q_{1}$ обозначена сумма всех натеканий, кроме натекания через искусственную течь.
		Исключая её из этих формул и используя формулу (6), получим выражения для рассчёта $W$
		$$
			W=\frac{d(PV)_{\text{капилл}}/dt}{P_{\text{ВБ}}-P_{\text{пр}}}=
			\frac{4}{3}r^3\sqrt{\frac{2\pi RT}{\mu}}\frac{P_{\text{ФБ}}-P_{\text{ВБ}}}{P_{\text{ВБ}}-P_{\text{пр}}}
		$$
		\par В итоге получим следущее значение $W=0.008\pm0.001$ л/с.
	\section{Обсуждение результатов}
		\par Первой частью работы было измерение объёмов форвакуумной и высоковакуумной чатсей установки. Погрешности
		определения поулчились небольшими (порядка 1\%). Результаты совпали с визуальными представлениями, что
		высоковакуумная часть имеет меньший объём, чем высоковакуумная.
		\par Второй частью работы было определение мощности диффузионного насоса. Два метода дали очень разные
		результаты отличающиеся на порядки. Результаты первого метода выглядят более адекватно. Возможно, ошибка
		заключается в неправильных вычислениях.
	\section{Вывод}
		\par Работа проведена неудачно, результаты двух методов очень сильно разняться. Для улучшения результотов можно
		поробовать провести работу на компьютерезированной установке, где все данные записываются точнее.
		
\end{document}
\documentclass[12pt,a4paper]{article}
\usepackage[utf8]{inputenc}
\usepackage[english,russian]{babel}
\usepackage{indentfirst}
\usepackage{misccorr}
\usepackage{graphicx}
\usepackage{amsmath}
\usepackage{amssymb}
\usepackage{circuitikz}
\usepackage[font={small}]{caption}
\usepackage[left=20mm, top=20mm, right=20mm, bottom=20mm, nohead]{geometry}
\usepackage{float}
\usepackage{tabularx}
\usepackage{array}
\usepackage{longtable}
\usepackage{pstool}
\usepackage{pgfplots}
\usepackage{hhline}
\usepackage{multirow}

\DeclareCaptionLabelSeparator{fill}{.\\}
\DeclareCaptionLabelFormat{fullparents}{\bothIfFirst{#1}{~}#2}

\begin{document}

	\begin{titlepage}
		\begin{center}
			{\LARGE Отчёт по лабораторной работе 2.2.3.\\}
			\vspace*{11cm}
				\textbf{\LARGE Измерение теплопроводности воздуха при атмосферном давлении.}	
			\vspace*{6cm}
		\end{center}
		\hfill\begin{minipage}{0.37\textwidth}
				Работу выполнил Громов Артём\\
				ЛФИ Б02-006
		\end{minipage}
		\vspace{5cm}
		\begin{center}
 			Долгопрудный, 2021 г.
		\end{center}   
	\end{titlepage}
	
	\section{Аннотация}
		\noindent\textbf{Цель работы: }измерить коэффициент теплопроводности воздуха при атмосферном
		давлении в зависимости от температуры.
		\par\noindent\textbf{В работе используются: } цилиндрическая колба с натянутой по оси нитью; термостат;
		вольтметр и амперметр (цифровые мультиметры); эталонное сопротивление; источник
		постоянного напряжения; реостат (или магазин сопротивлений).
		\vspace{0.5 cm}
		\par \textit{Теплопроводность} --- это процесс передачи тепловой энергии от нагретых частей системы к
		холодным за счёт хаотического движения частиц среды (молекул, атомов и т.п.). В газах теплопроводность
		осуществляется за счёт непосредственной передачи кинетической энергии от быстрых молекул к
		медленным при их столкновениях. Перенос тепла описывается законом Фурье, утверждающим, что
		плотность потока энергии $\vec{q}\text{ Вт}/\text{м}^2$ (количество теплоты, переносимое через
		единичную площадку в единицу времени) пропорциональна градиенту температуры $\nabla T$:
		\begin{equation}
			\vec{q}=-\kappa\nabla T,
		\end{equation}
		\noindent где $\kappa$ --- \textit{коэффициент теплопроводности}.
		\par Молекулярно-кинетическая теория даёт следующую оценку для коэффициента теплопроводности
		газов:
		\begin{equation}
			\kappa \sim \lambda \overline{v} n c_{V}
		\end{equation}
		\noindent где $\lambda$ --- длина свободного пробега молекул газа,
		$\overline{v}=\sqrt{\frac{8k_{\text{Б}}T}{\pi m}}$ --- средняя скорость их теплового движения, $n$ ---
		концентрация (объёмная плотность) газа, $c_{V}=\frac{i}{2}k_{\text{Б}}$ --- — его теплоёмкость при
		постоянном объёме в расчёте на одну молекулу ($i$ — эффективное число степеней свободы молекулы).
		\par Длина свободного пробега может быть оценена как $\lambda=1/n\sigma$, где $\sigma$ --- эффективно
		сечение столкновений молекул друг с другом. Тогда из (2) видно, что коэффициент теплопроводности газа
		не зависит от плотности газа и \textit{определяется только его температурой}. В простейшей модели
		твёрдых шариков $\sigma=\text{const}$, и коэффициент теплопроводности пропорционален корню
		абсолютной температуры: $\kappa \propto \overline{v} \propto \sqrt{T}$. На практике эффективное сечение
		$\sigma \left(T\right)$ следует считать медленно убывающей функцией $T$.
		\begin{figure}[H]
			\begin{minipage}[h]{0.69\linewidth}
				Рассмотрим стационарную теплопроводность в цилиндрической геометрии (см. рис. 1). Пусть
				тонкая нить радиусом $r_{1}$ и длиной $L$ помещена на оси цилиндра радиусом $r_{0}$.
				Температура стенок цилиндра $T_{0}$ поддерживается постоянной. Пусть в нити выделяется
				некоторая тепловая мощность $Q$. Если цилиндр длинный $\left(L\gg r_{0}\right)$, можно
				пренебречь теплоотводом через его торцы. Тогда все параметры газа можно считать зависящими
				только от расстояния до оси системы $r$. Вместо (1) имеем:
				\begin{equation}
					q=-\kappa\frac{dT}{dr}.
				\end{equation}
				В \textit{стационарном} состоянии полный поток тепла через любую 
				цилиндрическую поверхность радиуса $r$ площадью $S=2\pi rL$ должен быть одинаков и равен
				$Q=qS$:
				\begin{equation}
					Q=-2\pi rL \cdot \kappa \frac{dT}{dr}=\text{const}.
				\end{equation}
			\end{minipage}
			\hfill
			\begin{minipage}[h]{0.29\linewidth}
				\includegraphics[width=0.99\textwidth]{C:/Users/gromo/Desktop/Схема.png}
				\caption{Геометрия задачи}
			\end{minipage}
		\end{figure}
		\noindent Если перепад температуры $\Delta T=T_{1}-T_{2}$ между нитью и стенками цилиндра мал
		$\left(\Delta T \ll T_{0}\right)$, то в (4) можно пренебречь изменением теплопроводности от
		температуры в пределах системы, положив $\kappa \approx \kappa\left(T_{0}\right)$. Тогда разделяя
		переменные в (4) и интегрируя от радиуса нити до радиуса колбы, получим:
		\begin{equation}
			Q=\frac{2\pi L}{\text{ln}\frac{r_{0}}{r_{1}}}\kappa\cdot\Delta T.
		\end{equation}
		\noindent Видно, что поток тепла через систему пропорционален разности температур в ней (закон
		Ньютона).
		\par \textbf{Оценка времени установления равновесия.} При изменении параметровсистемы (температуры
		или мощности нагрева) система переходит в новое стационарное состояние не сразу, а в течение
		некоторого времени $\tau$. . Оценим значение $\tau$ по порядку величины. Рассмотрим для простоты
		плоский слой толщиной $a$ и сечением $S$, заполненный газом при постоянном давлении. Пусть
		температура одной из граней выросла на некоторую величину $\Delta T$.Это вызовет поток тепла через
		систему, который можно оценить по закону Фурье как $q \sim \kappa \frac{\Delta T}{a}$. Для того, чтобы
		весь слой прогрелся на $\Delta T$, в него должно поступить тепло $nSa \cdot c_{P} \Delta T$, где $c_{P}$ ---
		теплоёмкость при постоянном давлении (в расчёте на одну молекулу). С другой стороны, поступившее за
		время $\tau$ тепло можно вычислить как $qS\tau=\kappa \frac{\Delta T}{a}S\tau$. Приравнивая, находим
		искомую оценку времени перехода к стационарному состоянию:
		\begin{equation}
			\tau \sim \frac{a^2}{\chi} \text{,   где } \chi=\frac{\kappa}{nc_{P}}
		\end{equation}
		\par Коэффициент $\chi$, равный отношению теплопроводности $\kappa$ к теплоёмкости единицы объёма 
		$nc_{P}$ называют температуропроводностью среды. Он отвечает за скорость изменения температуры при
		теплопередаче. Для воздуха при нормальных условиях $\chi \approx 0.2\text{ см}^{2}/\text{с}$, так что при
		характерном размере $a\approx 1$ см имеем характерное время $\tau\approx 5$ с. Таким образом, можно
		ожидать, что в условиях опыта равновесие будет заведомо устанавливаться в течение нескольких десятков
		секунд. Более точная оценка потребовала бы решения уравнения теплопроводности с учётом геометрии
		задачи. В рамках данной работы необходимости прибегать к подобным расчётам нет.
		\par \textbf{Пределы применимости теории.} Укажем пределы применимости закона Фурье (1). В газах он
		может нарушаться, когда характерные масштабы задачи приближаются к длине свободного пробега
		молекул. Это, в частности, приводит к тому, что температура нити может отличаться от температуры
		окружающего её газа (\textit{температурный скачок}). В данной работе такого рода отклонениями можно
		пренебречь, поскольку при атмосферном давлении длина свободного пробега составляет порядка 
		$\lambda\sim10^{-5}$ см, что заведомо меньше наименьшего размера системы --- радиуса нити.
		\par Также возможны и другие механизмы теплопередачи: \textit{конвекция} и \textit{излучение}.
		Известно, что в поле тяжести конвекция возникает при достаточно большом вертикальном перепаде
		температур. Для её минимизации установка расположена вертикально (градиент температуры имеет место
		только в горизонтальном направлении).
		\par Вклад излучения может стать существенным при значительном перегреве нити относительно стенок.
		Оценить мощность излучения можно по закону Стефана–Больцмана:
		\begin{equation}
			Q_{изл}=\epsilon S\sigma_{S}\left(T^{4}_{1}-T^{4}_{0}\right)
			\approx4\epsilon S\sigma_{S}T^{3}_0\Delta T,
		\end{equation}
		\noindent где $S$ --- площладь поверхности нити, $\sigma_{S}=5.567\cdot10^{-8}
		\left(\text{ Вт}/\text{м}^{2}\text{К}^{4}\right)$ --- постоянная Стефана-Больцмана, $\epsilon$  ---
		безразмерный коэффициент ''черноты'',  зависящий от материала излучающей поверхности (для
		большинства металлов можно для оценки принять $\epsilon\sim0.1\div0.2$).
	\newpage
	\section{Экспериментальная установка}
		\begin{figure}[H]
			\begin{minipage}[h]{0.69\linewidth}
				\par Схема установки приведена на рис.2. На оси полой цилиндрической трубки с внутренним
				диаметром $2r_{0}=10\pm0.01$ мм размещена молибденовая нить диаметром 
				$2r_{0}1=0.05\pm0.005$ мм и длиной $L=365\pm2$ мм. Полость трубки заполнена воздухом
				(полость через небольшое отверстие сообщается с атмосферой). Стенки трубки помещены в кожух,
				через которых пропускается вода из термостата, так что их температура $t_{0}$ поддерживается
				постоянной. Для предотвращения конвекции трубка расположена вертикально.
				Молибденовая нить служит как источником тепла, так и датчиком температуры (термометром
				сопротивления). По пропускаемому через нить постоянному току $I$ и напряжению $U$ на ней
				вычисляется мощность нагрева по закону Джоуля–Ленца:
			\end{minipage}
			\hfill
			\begin{minipage}[h]{0.29\linewidth}
				\includegraphics[width=0.99\textwidth]{C:/Users/gromo/Desktop/Установка223.png}
				\caption{Схема утановки}
			\end{minipage}
		\end{figure}
		$$Q=UI,$$
		\noindent и сопротивление нити по закону Ома:
		$$R=\frac{U}{I}$$
		\par Сопротивление нити является однозначной функцией её температуры $R(t)$. Эта зависимость может
		быть измерена с помощью термостата по экстраполяции мощности нагрева к нулю $Q\rightarrow0$, когда
		температура нити и стенок совпадают $t_{1}\approx t_{0}$. Альтернативно, если материал нити известен,
		зависимость его удельного сопротивления от температуры может найдена по справочным данным.
		\par Для большинства металлов относительное изменение сопротивления из-за нагрева невелико: при
		изменении температуры на $\Delta t=1$ ℃ относительное изменение сопротивления нити
		$\frac{\Delta R}{R}$ может составлять приблизительно от 0.2\% до 0.6\% (в зависимости от её материала).
		Следовательно, измерение $R$ важно провести с высокой точностью. Желательно, чтобы методика
		измерений и чувствительность приборов обеспечивали измерение тока и напряжения с относительной
		погрешностью, не превышающей 0.1\% (т.е. необходимо уверенно измерять 4–5 значащих цифр, что
		вполне реально при использовании современных цифровых мультиметров).
		\begin{figure}[H]
			\begin{minipage}[h]{0.59\linewidth}
				\par Схема рис.3 предусматривает использование одного вольтметра и эталонного сопротивления
				$R_{\text{Э}}=10.000$Ом (класс точности 0.01), включённого последовательно с нитью. В
				положении переключателя 2 вольтметр измеряет напряжение на нити, а в положении 1 ---
				напряжение на $R_{\text{Э}}$, пропорциональное току через нить. Для исключения влияния
				контактов и подводящих проводов эталонное сопротивление $R_{\text{Э}}$ также необходимо
				подключать в цепь по четырёхпроводной схеме. Ток в цепи в обеих схемах регулируется с помощью
				реостата или магазина сопротивлений $R_{\text{м}}$, включённого последовательно с источником
				напряжения.
			\end{minipage}
			\hfill
			\begin{minipage}[h]{0.39\linewidth}
				\includegraphics[width=0.99\textwidth]{C:/Users/gromo/Desktop/Схема223.png}
				\caption{Схема для измерения сопротивления нити и мощности нагрева}
			\end{minipage}
		\end{figure}
		\par \textbf{Методика измерений.} Принципиально неустранимая систематическая ошибка измерения
		температуры с помощью термометра сопротивления возникает из-за необходимости пропускать через
		резистор (нить) измерительныйток. Чем этот ток выше, тем с большей точностью будет измерен как он
		сам, так и напряжение. Однако при этом квадратично возрастает выделяющаяся на резисторе мощность
		$Q=UI=I^{2}R$. Следовательно, температура резистора становится выше, чем у объекта, температуру
		которого надо измерить. Измерения же при малых токах не дают достаточной точности (в частности, из-за
		существенного вклада термоэлектрических явлений в проводниках и контактах). Эта проблема решается
		построением нагрузочной кривой — зависимости измеряемого сопротивления $R$ от выделяющейся в нём
		мощности $R(Q)$, с последующей экстраполяцией к нулевой мощности $Q\rightarrow0$ для определения
		сопротивления $R_{0}\equiv R(0)$, при котором его температура равна температуре измеряемого объекта.
		Кроме того, в данной работе измерение нагрузочных кривых позволяет в ходе эксперимента получить
		температурную зависимость сопротивления нити, так как при $Q\rightarrow0$ температура нити равна
		температуре термостата ($T\approx T_{0}$).
		\par В исследуемом интервале температур (20–70 ℃) зависимость сопротивления от температуры можно с
		хорошей точностью аппроксимировать линейной функцией:
		\begin{equation}
			R(t)=R_{273}\cdot\left(1+\alpha t\right),
		\end{equation}
		где $t$ --- температура в градусах Цельсия, $R_{273}$ --- сопротивление нити при температуре 0 ℃ и
		$\alpha=\frac{1}{R_{273}}\frac{dR}{dT}$ --- температурный коэффициент сопротивления материала.
		Измерение зависимости (8) по данным для $Q\rightarrow0$ позволит затем определять температуру нити
		$t$ по значению её сопротивления $R$ при произвольной мощности нагрева.
	\section{Результаты измерений и обработка данных}
		\subsection{Подготовка к эксперименту}
			\par Проведём предварительные расчёты параметров опыта. Приняв максимально допустимый
			перегрев нити относительно термостата равным $\Delta T_{max}=10$ ℃ , оценим максимальную
			мощность нагрева $Q_{max}$, которую следует подавать на нить. Для оценки коэффициент
			теплопроводности воздуха примем равным $\kappa\sim25\text{ мВт}/\text{м}\cdot\text{K}$. Пользуясь
			формулой (6) получим, что $Q\approx108$ мВт.
			\par Зная приближенное значение сопротивления нити $R\approx12$ Ом, определим соответствующие
			значения максимального тока $I_{max}$ и максимального напряжения $U_{max}$ в нити. Используя
			значение $Q_{max}$, получим, что $I_{max}\approx90$ мА, $U_{max}\approx1200$ мВ.
		\subsection{Измерение сопротивления проволоки при различных токах и температурах}
		\par Измерим зависимость сопротивления нити $R=U/I$ от подаваемой на неё мощности $Q=UI$ при
		различных температурах термомтата начиная с комнатной. Для каждого занчения температуры будем
		производить по 7 измерений для значений тока от 0 до $I_{max}$. Данные занесём в талицу 1. Так как
		ток мы измеряем с помощью вольтметра и эталонного сопротивления, то в таблицу будем заносить
		значение напряжение чере $R_{\text{Э}}$, а затем вычислять силу тока по формуле
		$I=U_{\text{Э}}/R_{\text{Э}}$. Погрешность вольтметра сотавляет около 0.004\% и может быть отброшена.
		Значит могут быть отброшены погрешности измерений силы тока и мощности, так как они совпадают по
		порядку величины с погрешностью измерения напряжения.
		\begin{table}[H]
      			\captionsetup{font={small}, labelformat=fullparents, labelsep=fill, labelfont=bf, justification=raggedleft,
              			singlelinecheck=false, skip=-0.2cm}
      			\caption{Результаты измерения напряжения и мощности на нити при разных температурах}
      			\begin{center}
          			\begin{tabular}{|>{\centering}p{1.7cm}|>{\centering}p{1.7cm}
                  						|>{\centering}p{1.7cm}|>{\centering}p{1.7cm}
                  						|>{\centering}p{1.7cm}|>{\centering}p{1.7cm}
                  						|>{\centering}p{1.7cm}|>{\centering}p{1.7cm}|}
			          \hline
			          \multicolumn{8}{|c|}{$T=24$ ℃} \\
			          \hline        
			          \multicolumn{1}{|p{1.7cm}|}{$U_{\text{н}}$, мВ} & \multicolumn{1}{p{1.7cm}|}{174.46} 
			          & \multicolumn{1}{p{1.7cm}|}{376.35} & \multicolumn{1}{p{1.7cm}|}{561.62} 
			          & \multicolumn{1}{p{1.7cm}|}{761.74} & \multicolumn{1}{p{1.7cm}|}{964.15} 
			          & \multicolumn{1}{p{1.7cm}|}{1167.00} & \multicolumn{1}{p{1.7cm}|}{1343.34} \\
			          \hline
			          \multicolumn{1}{|p{1.7cm}|}{$U_{\text{э}}$, мВ} & \multicolumn{1}{p{1.7cm}|}{120.16} 
			          & \multicolumn{1}{p{1.7cm}|}{258.74} & \multicolumn{1}{p{1.7cm}|}{384.99} 
			          & \multicolumn{1}{p{1.7cm}|}{521.92} & \multicolumn{1}{p{1.7cm}|}{654.34} 
			          & \multicolumn{1}{p{1.7cm}|}{786.65} & \multicolumn{1}{p{1.7cm}|}{899.45} \\
			          \hline
			          \multicolumn{1}{|p{1.7cm}|}{$I$, мА} & \multicolumn{1}{p{1.7cm}|}{12.016} 
			          & \multicolumn{1}{p{1.7cm}|}{25.874} & \multicolumn{1}{p{1.7cm}|}{38.499} 
			          & \multicolumn{1}{p{1.7cm}|}{52.192} & \multicolumn{1}{p{1.7cm}|}{65.434} 
			          & \multicolumn{1}{p{1.7cm}|}{78.665} & \multicolumn{1}{p{1.7cm}|}{89.945} \\
			          \hline
			          \multicolumn{1}{|p{1.7cm}|}{$R_{н}$, Ом} & \multicolumn{1}{p{1.7cm}|}{14.519} 
			          & \multicolumn{1}{p{1.7cm}|}{14.545} & \multicolumn{1}{p{1.7cm}|}{14.588} 
			          & \multicolumn{1}{p{1.7cm}|}{14.595} & \multicolumn{1}{p{1.7cm}|}{14.735} 
			          & \multicolumn{1}{p{1.7cm}|}{14.835} & \multicolumn{1}{p{1.7cm}|}{14.935} \\
			          \hline
			          \multicolumn{1}{|p{1.7cm}|}{$Q$, мВт} & \multicolumn{1}{p{1.7cm}|}{2.096} 
			          & \multicolumn{1}{p{1.7cm}|}{9.737} & \multicolumn{1}{p{1.7cm}|}{21.621} 
			          & \multicolumn{1}{p{1.7cm}|}{39.756} & \multicolumn{1}{p{1.7cm}|}{63.088} 
			          & \multicolumn{1}{p{1.7cm}|}{91.802} & \multicolumn{1}{p{1.7cm}|}{10.826} \\
			          \hline
			          \multicolumn{8}{|c|}{$T=30$ ℃} \\
			          \hline        
			          \multicolumn{1}{|p{1.7cm}|}{$U_{\text{н}}$, мВ} & \multicolumn{1}{p{1.7cm}|}{177.49} 
			          & \multicolumn{1}{p{1.7cm}|}{382.42} & \multicolumn{1}{p{1.7cm}|}{573.61} 
			          & \multicolumn{1}{p{1.7cm}|}{767.07} & \multicolumn{1}{p{1.7cm}|}{965.45} 
			          & \multicolumn{1}{p{1.7cm}|}{1164.5} & \multicolumn{1}{p{1.7cm}|}{1368.8} \\
			          \hline
			          \multicolumn{1}{|p{1.7cm}|}{$U_{\text{э}}$, мВ} & \multicolumn{1}{p{1.7cm}|}{120.02} 
			          & \multicolumn{1}{p{1.7cm}|}{258.15} & \multicolumn{1}{p{1.7cm}|}{386.12} 
			          & \multicolumn{1}{p{1.7cm}|}{514.27} & \multicolumn{1}{p{1.7cm}|}{643.86} 
			          & \multicolumn{1}{p{1.7cm}|}{771.74} & \multicolumn{1}{p{1.7cm}|}{900.31} \\
			          \hline
			          \multicolumn{1}{|p{1.7cm}|}{$I$, мА} & \multicolumn{1}{p{1.7cm}|}{12.002} 
			          & \multicolumn{1}{p{1.7cm}|}{25.815} & \multicolumn{1}{p{1.7cm}|}{38.612} 
			          & \multicolumn{1}{p{1.7cm}|}{51.427} & \multicolumn{1}{p{1.7cm}|}{64.386} 
			          & \multicolumn{1}{p{1.7cm}|}{77.174} & \multicolumn{1}{p{1.7cm}|}{90.031} \\
			          \hline
			          \multicolumn{1}{|p{1.7cm}|}{$R_{н}$, Ом} & \multicolumn{1}{p{1.7cm}|}{14.788} 
			          & \multicolumn{1}{p{1.7cm}|}{14.814} & \multicolumn{1}{p{1.7cm}|}{14.856} 
			          & \multicolumn{1}{p{1.7cm}|}{14.916} & \multicolumn{1}{p{1.7cm}|}{14.995} 
			          & \multicolumn{1}{p{1.7cm}|}{15.089} & \multicolumn{1}{p{1.7cm}|}{15.204} \\
			          \hline
			          \multicolumn{1}{|p{1.7cm}|}{$Q$, мВт} & \multicolumn{1}{p{1.7cm}|}{2.130} 
			          & \multicolumn{1}{p{1.7cm}|}{9.872} & \multicolumn{1}{p{1.7cm}|}{22.148} 
			          & \multicolumn{1}{p{1.7cm}|}{39.448} & \multicolumn{1}{p{1.7cm}|}{62.161}  
			          & \multicolumn{1}{p{1.7cm}|}{89.869} & \multicolumn{1}{p{1.7cm}|}{123.234} \\
			          \hline
			          \multicolumn{8}{|c|}{$T=40$ ℃} \\
			          \hline        
			          \multicolumn{1}{|p{1.7cm}|}{$U_{\text{н}}$, мВ} & \multicolumn{1}{p{1.7cm}|}{183.24} 
			          & \multicolumn{1}{p{1.7cm}|}{392.4} & \multicolumn{1}{p{1.7cm}|}{591.21} 
			          & \multicolumn{1}{p{1.7cm}|}{911.46} & \multicolumn{1}{p{1.7cm}|}{992.99} 
			          & \multicolumn{1}{p{1.7cm}|}{1200.3} & \multicolumn{1}{p{1.7cm}|}{1410.9} \\
			          \hline
			          \multicolumn{1}{|p{1.7cm}|}{$U_{\text{э}}$, мВ} & \multicolumn{1}{p{1.7cm}|}{120.28} 
			          & \multicolumn{1}{p{1.7cm}|}{257.14} & \multicolumn{1}{p{1.7cm}|}{386.33} 
			          & \multicolumn{1}{p{1.7cm}|}{591.52} & \multicolumn{1}{p{1.7cm}|}{643.03} 
			          & \multicolumn{1}{p{1.7cm}|}{772.42} & \multicolumn{1}{p{1.7cm}|}{901.29} \\
			          \hline
			          \multicolumn{1}{|p{1.7cm}|}{$I$, мА} & \multicolumn{1}{p{1.7cm}|}{12.028} 
			          & \multicolumn{1}{p{1.7cm}|}{25.714} & \multicolumn{1}{p{1.7cm}|}{38.633} 
			          & \multicolumn{1}{p{1.7cm}|}{59.152} & \multicolumn{1}{p{1.7cm}|}{64.303} 
			          & \multicolumn{1}{p{1.7cm}|}{77.242} & \multicolumn{1}{p{1.7cm}|}{90.129} \\
			          \hline
			          \multicolumn{1}{|p{1.7cm}|}{$R_{н}$, Ом} & \multicolumn{1}{p{1.7cm}|}{15.234} 
			          & \multicolumn{1}{p{1.7cm}|}{15.260} & \multicolumn{1}{p{1.7cm}|}{15.303} 
			          & \multicolumn{1}{p{1.7cm}|}{15.409} & \multicolumn{1}{p{1.7cm}|}{15.442} 
			          & \multicolumn{1}{p{1.7cm}|}{15.539} & \multicolumn{1}{p{1.7cm}|}{15.654} \\
			          \hline
			          \multicolumn{1}{|p{1.7cm}|}{$Q$, мВт} & \multicolumn{1}{p{1.7cm}|}{2.204} 
			          & \multicolumn{1}{p{1.7cm}|}{10.09} & \multicolumn{1}{p{1.7cm}|}{22.84} 
			          & \multicolumn{1}{p{1.7cm}|}{53.914} & \multicolumn{1}{p{1.7cm}|}{63.852} 
			          & \multicolumn{1}{p{1.7cm}|}{92.713} & \multicolumn{1}{p{1.7cm}|}{127.163} \\
			          \hline
			          \multicolumn{8}{|c|}{$T=49.9$ ℃} \\
			          \hline        
			          \multicolumn{1}{|p{1.7cm}|}{$U_{\text{н}}$, мВ} & \multicolumn{1}{p{1.7cm}|}{187.66} 
			          & \multicolumn{1}{p{1.7cm}|}{403.96} & \multicolumn{1}{p{1.7cm}|}{608.44} 
			          & \multicolumn{1}{p{1.7cm}|}{937.43} & \multicolumn{1}{p{1.7cm}|}{1021.24} 
			          & \multicolumn{1}{p{1.7cm}|}{1234.2} & \multicolumn{1}{p{1.7cm}|}{1445.7} \\
			          \hline
			          \multicolumn{1}{|p{1.7cm}|}{$U_{\text{э}}$, мВ} & \multicolumn{1}{p{1.7cm}|}{119.6} 
			          & \multicolumn{1}{p{1.7cm}|}{257.06} & \multicolumn{1}{p{1.7cm}|}{386.08} 
			          & \multicolumn{1}{p{1.7cm}|}{590.89} & \multicolumn{1}{p{1.7cm}|}{642.35} 
			          & \multicolumn{1}{p{1.7cm}|}{771.64} & \multicolumn{1}{p{1.7cm}|}{897.55} \\
			          \hline
			          \multicolumn{1}{|p{1.7cm}|}{$I$, мА} & \multicolumn{1}{p{1.7cm}|}{11.96} 
			          & \multicolumn{1}{p{1.7cm}|}{25.706} & \multicolumn{1}{p{1.7cm}|}{38.608} 
			          & \multicolumn{1}{p{1.7cm}|}{59.089} & \multicolumn{1}{p{1.7cm}|}{64.235} 
			          & \multicolumn{1}{p{1.7cm}|}{77.164} & \multicolumn{1}{p{1.7cm}|}{89.775} \\
			          \hline
			          \multicolumn{1}{|p{1.7cm}|}{$R_{н}$, Ом} & \multicolumn{1}{p{1.7cm}|}{15.691} 
			          & \multicolumn{1}{p{1.7cm}|}{15.715} & \multicolumn{1}{p{1.7cm}|}{15.759} 
			          & \multicolumn{1}{p{1.7cm}|}{15.865} & \multicolumn{1}{p{1.7cm}|}{15.898} 
			          & \multicolumn{1}{p{1.7cm}|}{15.995} & \multicolumn{1}{p{1.7cm}|}{16.107} \\
			          \hline
			          \multicolumn{1}{|p{1.7cm}|}{$Q$, мВт} & \multicolumn{1}{p{1.7cm}|}{2.244} 
			          & \multicolumn{1}{p{1.7cm}|}{10.384} & \multicolumn{1}{p{1.7cm}|}{23.49} 
			          & \multicolumn{1}{p{1.7cm}|}{55.391} & \multicolumn{1}{p{1.7cm}|}{65.599} 
			          & \multicolumn{1}{p{1.7cm}|}{95.235} & \multicolumn{1}{p{1.7cm}|}{129.758} \\
			          \hline
			          \multicolumn{8}{|c|}{$T=60$ ℃} \\
			          \hline        
			          \multicolumn{1}{|p{1.7cm}|}{$U_{\text{н}}$, мВ} & \multicolumn{1}{p{1.7cm}|}{194.22} 
			          & \multicolumn{1}{p{1.7cm}|}{416.05} & \multicolumn{1}{p{1.7cm}|}{626.95} 
			          & \multicolumn{1}{p{1.7cm}|}{965.05} & \multicolumn{1}{p{1.7cm}|}{1069.6}
			          &\multicolumn{1}{p{1.7cm}|}{1270.5} & \multicolumn{1}{p{1.7cm}|}{1488.6} \\
			          \hline
			          \multicolumn{1}{|p{1.7cm}|}{$U_{\text{э}}$, мВ} & \multicolumn{1}{p{1.7cm}|}{120.33} 
			          & \multicolumn{1}{p{1.7cm}|}{257.37} & \multicolumn{1}{p{1.7cm}|}{386.81} 
			          & \multicolumn{1}{p{1.7cm}|}{591.56} & \multicolumn{1}{p{1.7cm}|}{654.02} 
			          & \multicolumn{1}{p{1.7cm}|}{772.55} & \multicolumn{1}{p{1.7cm}|}{898.98} \\
			          \hline
			          \multicolumn{1}{|p{1.7cm}|}{$I$, мА} & \multicolumn{1}{p{1.7cm}|}{12.033} 
			          & \multicolumn{1}{p{1.7cm}|}{25.737} & \multicolumn{1}{p{1.7cm}|}{38.681} 
			          & \multicolumn{1}{p{1.7cm}|}{59.156} & \multicolumn{1}{p{1.7cm}|}{65.402} 
			          & \multicolumn{1}{p{1.7cm}|}{77.255} & \multicolumn{1}{p{1.7cm}|}{89.898} \\
			          \hline
			          \multicolumn{1}{|p{1.7cm}|}{$R_{н}$, Ом} & \multicolumn{1}{p{1.7cm}|}{16.141} 
			          & \multicolumn{1}{p{1.7cm}|}{16.165} & \multicolumn{1}{p{1.7cm}|}{16.208} 
			          & \multicolumn{1}{p{1.7cm}|}{16.314} & \multicolumn{1}{p{1.7cm}|}{16.354} 
			          & \multicolumn{1}{p{1.7cm}|}{16.446} & \multicolumn{1}{p{1.7cm}|}{16.559} \\
			          \hline
			          \multicolumn{1}{|p{1.7cm}|}{$Q$, мВт} & \multicolumn{1}{p{1.7cm}|}{2.337} 
			          	& \multicolumn{1}{p{1.7cm}|}{10.707} & \multicolumn{1}{p{1.7cm}|}{24.251} 
			         	 & \multicolumn{1}{p{1.7cm}|}{57.088} & \multicolumn{1}{p{1.7cm}|}{69.953} 
			         	 & \multicolumn{1}{p{1.7cm}|}{98.152} & \multicolumn{1}{p{1.7cm}|}{133.822} \\
			        	\hline
       			 \end{tabular}
      			\end{center}
    		\end{table}
		\par По данным таблицы построим график зависимости сопротивления нити $R_{\text{н}}$ от выделяемого на
		ней тепла $Q$. Так как погрешности малы (порядка 0.01\%) то их на график наносить не будем. Полученные в
		эксперименте точки хорошо ложаться на прямые, поэтому воспользуемся методом наименьших квадратов  и
		вычислим для каждого графика его угловой коэффициент, свободный член и их погрешности:
		\begin{equation}
			\frac{dR}{dQ}=\frac{\langle RQ \rangle
			-\langle R \rangle\langle Q \rangle}{\langle Q^{2} \rangle-\langle Q \rangle^{2}},
		\end{equation}
		\begin{equation}
			R_{0}=\langle R \rangle-k\langle Q \rangle,
		\end{equation}
		\begin{equation}
			\sigma_{\frac{dR}{dQ}}=\frac{1}{\sqrt{N}}\sqrt{\frac{\langle R^2 \rangle-\langle R \rangle^{2}}
						{\langle Q^2 \rangle-\langle Q \rangle^{2}}-\left(\frac{dR}{dQ}\right)^{2}},
		\end{equation}
		\begin{equation}
			\sigma_{R_{0}}=\sigma_{\frac{dR}{dQ}}\sqrt{\langle Q^{2} \rangle-\langle Q \rangle^{2}},
		\end{equation}
		\noindent где $\frac{dR}{dQ}$ --- угловой коэффициент, $R_{0}$ --- свободный член, $N$ --- количество опытов.
		\par Погрешность определения углового коэффициента по порядку величины равна 0.5\%, а погрешность
		определения свободного члена --- меньше 0.01\%. Занесём получившиеся данные в таблицу 2, и построим
		график рис.4.
		\begin{table}[H]
			\captionsetup{font={small}, labelformat=fullparents, labelsep=fill, labelfont=bf, justification=raggedleft,
              			singlelinecheck=false, skip=-0.2cm}
      			\caption{Сопротивление нити при разных температурах}
      			\begin{center}
      				\begin{tabular}{|p{2.2cm}|p{2cm}|p{2cm}|p{2cm}|p{2cm}|p{2cm}|}
					\hline
					$T$, ℃ & 24.0 & 30.0 & 40.0 & 49.9 & 60.0\\
					\hline
					$\frac{dR}{dQ}$, $\text{Ом}/\text{Вт}$ 
					& $3.51$ & $3.437$ & $3.367$ & $3.27$ & $3.19$ \\
					\hline
					$\sigma_{\frac{dR}{dQ}}$, $\text{Ом}/\text{Вт}$
					& $0.02$ & $0.012$ & $0.012$ & $0.02$ & $0.02$ \\
					\hline
					$R_{0}$, Ом
					& $14.5120$ & $14.7803$ & $15.2268$ & $15.6824$ & $16.1317$ \\
					\hline
					$\sigma_{R_{0}}$, Ом
					& $0.0007$ & $0.0005$ & $0.0005$ & $0.0009$ & $0.0008$ \\
					\hline
      				\end{tabular}
      			\end{center}
		\end{table}
		\begin{figure}[H]
			% This file was created with tikzplotlib v0.9.16.
\begin{tikzpicture}

\definecolor{color0}{rgb}{0.12156862745098,0.466666666666667,0.705882352941177}

\begin{axis}[
    height=7.2cm,
    tick align=inside,
    major tick length=0.2cm,
    minor tick length=0.1cm,
    tick pos=left,
    x grid style={white!69.0196078431373!black},
    xmin=-10, xmax=140,
    xtick={-40, 0, 40, 80, 120, 160},
    minor x tick num=3,
    xmajorgrids,
    minor x grid style={dotted,black},
    xminorgrids,
    xtick style={color=black},
    xlabel={$l$, \text{см}},
    ymin = -0.2, ymax = 0.8,
    ytick = {-0.4, 0, 0.4, 0.8},
    ytick style={color=black},
    minor y tick num=3,
    ymajorgrids,
    minor y grid style={dotted,black},
    yminorgrids,
    ytick style={color=black},
    ylabel={$\upsilon_2$}
]
\path [draw=red, semithick]
(axis cs:-6,0.225806451612903)
--(axis cs:-2,0.225806451612903);

\path [draw=red, semithick]
(axis cs:-2,0.6)
--(axis cs:2,0.6);

\path [draw=red, semithick]
(axis cs:0,0.133333333333333)
--(axis cs:4,0.133333333333333);

\path [draw=red, semithick]
(axis cs:2,0.214285714285714)
--(axis cs:6,0.214285714285714);

\path [draw=red, semithick]
(axis cs:4,0.2)
--(axis cs:8,0.2);

\path [draw=red, semithick]
(axis cs:6,0.161290322580645)
--(axis cs:10,0.161290322580645);

\path [draw=red, semithick]
(axis cs:8,0.225806451612903)
--(axis cs:12,0.225806451612903);

\path [draw=red, semithick]
(axis cs:10,0.225806451612903)
--(axis cs:14,0.225806451612903);

\path [draw=red, semithick]
(axis cs:14,0.161290322580645)
--(axis cs:18,0.161290322580645);

\path [draw=red, semithick]
(axis cs:26,0.0967741935483871)
--(axis cs:30,0.0967741935483871);

\path [draw=red, semithick]
(axis cs:36,0.0322580645161291)
--(axis cs:40,0.0322580645161291);

\path [draw=red, semithick]
(axis cs:46,0.0666666666666667)
--(axis cs:50,0.0666666666666667);

\path [draw=red, semithick]
(axis cs:56,0.0322580645161291)
--(axis cs:60,0.0322580645161291);

\path [draw=red, semithick]
(axis cs:66,0.0322580645161291)
--(axis cs:70,0.0322580645161291);

\path [draw=red, semithick]
(axis cs:76,0.0666666666666667)
--(axis cs:80,0.0666666666666667);

\path [draw=red, semithick]
(axis cs:86,0.0322580645161291)
--(axis cs:90,0.0322580645161291);

\path [draw=red, semithick]
(axis cs:96,0.0666666666666667)
--(axis cs:100,0.0666666666666667);

\path [draw=red, semithick]
(axis cs:106,0.0967741935483871)
--(axis cs:110,0.0967741935483871);

\path [draw=red, semithick]
(axis cs:110,0.225806451612903)
--(axis cs:114,0.225806451612903);

\path [draw=red, semithick]
(axis cs:114,0.290322580645161)
--(axis cs:118,0.290322580645161);

\path [draw=red, semithick]
(axis cs:116,0.290322580645161)
--(axis cs:120,0.290322580645161);

\path [draw=red, semithick]
(axis cs:118,0.533333333333333)
--(axis cs:122,0.533333333333333);

\path [draw=red, semithick]
(axis cs:120,0.548387096774194)
--(axis cs:124,0.548387096774194);

\path [draw=red, semithick]
(axis cs:122,0.161290322580645)
--(axis cs:126,0.161290322580645);

\path [draw=red, semithick]
(axis cs:124,0.161290322580645)
--(axis cs:128,0.161290322580645);

\path [draw=red, semithick]
(axis cs:126,0.0967741935483871)
--(axis cs:130,0.0967741935483871);

\path [draw=red, semithick]
(axis cs:128,0.0967741935483871)
--(axis cs:132,0.0967741935483871);

\path [draw=red, semithick]
(axis cs:-4,0.16988541584789)
--(axis cs:-4,0.281727487377916);

\path [draw=red, semithick]
(axis cs:0,0.524575276673435)
--(axis cs:0,0.675424723326565);

\path [draw=red, semithick]
(axis cs:2,0.0799074876436831)
--(axis cs:2,0.186759179022984);

\path [draw=red, semithick]
(axis cs:4,0.152955024080759)
--(axis cs:4,0.27561640449067);

\path [draw=red, semithick]
(axis cs:6,0.143431457505076)
--(axis cs:6,0.256568542494924);

\path [draw=red, semithick]
(axis cs:8,0.108312499224317)
--(axis cs:8,0.214268145936973);

\path [draw=red, semithick]
(axis cs:10,0.16988541584789)
--(axis cs:10,0.281727487377916);

\path [draw=red, semithick]
(axis cs:12,0.16988541584789)
--(axis cs:12,0.281727487377916);

\path [draw=red, semithick]
(axis cs:16,0.108312499224317)
--(axis cs:16,0.214268145936973);

\path [draw=red, semithick]
(axis cs:28,0.0467395826007438)
--(axis cs:28,0.14680880449603);

\path [draw=red, semithick]
(axis cs:38,-0.0148333340228294)
--(axis cs:38,0.0793494630550875);

\path [draw=red, semithick]
(axis cs:48,0.01638351778229)
--(axis cs:48,0.116949815551043);

\path [draw=red, semithick]
(axis cs:58,-0.0148333340228294)
--(axis cs:58,0.0793494630550875);

\path [draw=red, semithick]
(axis cs:68,-0.0148333340228294)
--(axis cs:68,0.0793494630550875);

\path [draw=red, semithick]
(axis cs:78,0.01638351778229)
--(axis cs:78,0.116949815551043);

\path [draw=red, semithick]
(axis cs:88,-0.0148333340228294)
--(axis cs:88,0.0793494630550875);

\path [draw=red, semithick]
(axis cs:98,0.01638351778229)
--(axis cs:98,0.116949815551043);

\path [draw=red, semithick]
(axis cs:108,0.0467395826007438)
--(axis cs:108,0.14680880449603);

\path [draw=red, semithick]
(axis cs:112,0.16988541584789)
--(axis cs:112,0.281727487377916);

\path [draw=red, semithick]
(axis cs:116,0.231458332471463)
--(axis cs:116,0.349186828818859);

\path [draw=red, semithick]
(axis cs:118,0.231458332471463)
--(axis cs:118,0.349186828818859);

\path [draw=red, semithick]
(axis cs:120,0.461051306812042)
--(axis cs:120,0.605615359854625);

\path [draw=red, semithick]
(axis cs:122,0.477749998965756)
--(axis cs:122,0.619024194582631);

\path [draw=red, semithick]
(axis cs:124,0.108312499224317)
--(axis cs:124,0.214268145936973);

\path [draw=red, semithick]
(axis cs:126,0.108312499224317)
--(axis cs:126,0.214268145936973);

\path [draw=red, semithick]
(axis cs:128,0.0467395826007438)
--(axis cs:128,0.14680880449603);

\path [draw=red, semithick]
(axis cs:130,0.0467395826007438)
--(axis cs:130,0.14680880449603);

\addplot [semithick, color0]
table {%
-4 0.289700909532296
-2.64646464646465 0.299561221223985
-1.29292929292929 0.303200008671337
0.0606060606060606 0.301611090044221
1.41414141414141 0.295698522381247
2.76767676767677 0.286281224103273
4.12121212121212 0.274097490756272
5.47474747474748 0.259809403983547
6.82828282828283 0.244007133727306
8.18181818181818 0.227213133659582
9.53535353535354 0.209886229842507
10.8888888888889 0.192425602617948
12.2424242424242 0.175174661726482
13.5959595959596 0.158424814655734
14.949494949495 0.142419128218066
16.3030303030303 0.127355883357614
17.6565656565657 0.113392023186687
19.010101010101 0.100646494251508
20.3636363636364 0.0892034810273152
21.7171717171717 0.079115533642817
23.0707070707071 0.0704065888339942
24.4242424242424 0.0630748841272588
25.7777777777778 0.0570957652519653
27.1313131313131 0.0524243867822743
28.4848484848485 0.0489983060083702
29.8383838383838 0.0467399700370295
31.1919191919192 0.0455590961215448
32.5454545454545 0.0453549452209991
33.8989898989899 0.0460184887888953
35.2525252525253 0.0474344687911365
36.6060606060606 0.0494833509533611
37.959595959596 0.0520431712376296
39.3131313131313 0.0549912755484642
40.6666666666667 0.0582059526682422
42.020202020202 0.0615679604219419
43.3737373737374 0.064961945071241
44.7272727272727 0.0682777539379683
46.0808080808081 0.0714116412569081
47.4343434343434 0.0742673672579583
48.7878787878788 0.0767571904776398
50.1414141414141 0.0788027532999603
51.4949494949495 0.0803358607266301
52.8484848484849 0.0812991523766312
54.2020202020202 0.0816466677151401
55.5555555555556 0.0813443045118011
56.9090909090909 0.0803701705283548
58.2626262626263 0.0787148284356191
59.6161616161616 0.076381433959822
60.969696969697 0.0733857672582888
62.3232323232323 0.0697561575244805
63.6767676767677 0.0655333008223871
65.030303030303 0.0607699711502716
66.3838383838384 0.0555306247337693
67.7373737373737 0.0498908975483373
69.0909090909091 0.0439369960710599
70.4444444444444 0.0377649812618038
71.7979797979798 0.0314799457737291
73.1515151515152 0.0251950843931509
74.5050505050505 0.0190306577087552
75.8585858585859 0.0131128490101669
77.2121212121212 0.00757251441587132
78.5656565656566 0.00254382623048871
79.9191919191919 -0.00183719046859988
81.2727272727273 -0.00543422801527125
82.6262626262626 -0.00811237310932827
83.979797979798 -0.00973989054181345
85.3333333333333 -0.0101901136909708
86.6868686868687 -0.00934344178885385
88.040404040404 -0.00708944395858157
89.3939393939394 -0.00332907002224061
90.7474747474748 0.00202303192056456
92.1010101010101 0.00903609114351488
93.4545454545455 0.0177606831737331
94.8080808080808 0.0282260919012925
96.1616161616162 0.0404375649180784
97.5151515151515 0.0543734620860032
98.8686868686869 0.0699822973345749
100.222222222222 0.0871796736878155
101.575757575758 0.105845111520537
102.929292929293 0.125818770043966
104.282828282828 0.146898062020724
105.636363636364 0.168834161709158
106.989898989899 0.191328406037028
108.343434343434 0.214028589004543
109.69696969697 0.236525149316752
111.050505050505 0.258347251245289
112.40404040404 0.278958758719469
113.757575757576 0.297754102646736
115.111111111111 0.314054041462469
116.464646464646 0.327101314909134
117.818181818182 0.336056191044795
119.171717171717 0.339991906480971
120.525252525253 0.337889999849856
121.878787878788 0.328635538500881
123.232323232323 0.311012238426636
124.585858585859 0.283697477418142
125.939393939394 0.24525720144948
127.292929292929 0.194140724291761
128.646464646465 0.128675420356469
130 0.047061310768137
};
\addplot [semithick, red, mark=*, mark size=3, mark options={solid}, only marks]
table {%
-4 0.225806451612903
0 0.6
2 0.133333333333333
4 0.214285714285714
6 0.2
8 0.161290322580645
10 0.225806451612903
12 0.225806451612903
16 0.161290322580645
28 0.0967741935483871
38 0.0322580645161291
48 0.0666666666666667
58 0.0322580645161291
68 0.0322580645161291
78 0.0666666666666667
88 0.0322580645161291
98 0.0666666666666667
108 0.0967741935483871
112 0.225806451612903
116 0.290322580645161
118 0.290322580645161
120 0.533333333333333
122 0.548387096774194
124 0.161290322580645
126 0.161290322580645
128 0.0967741935483871
130 0.0967741935483871
};
\end{axis}

\end{tikzpicture}

			\caption[figure]{График зависимости $R_{\text{н}}$ от $Q$}
		\end{figure}
		\subsection{Определение температурного коэффициента сопротивления молибдена}
		\par Пользуясь результами, полученными нами в предыдущем пункте, мы можем определить температурный
		коэффициэнт сопротивления материала нити. Для этого построим график зависимости сопротивления $R_{0}$
		от температуры $T$. Данные для построения графика возьмём из таблицы 2.
		\par Точки на графике хорошо аппроксимируются прямой, определим её угловой коэффициен и свободный
		член. Для этого опять воспользуемся методом наименьших квадратов:
		\begin{equation}
			\frac{dR}{dT}=\frac{\langle TR_{0} \rangle
			-\langle R_{0} \rangle\langle T \rangle}{\langle T^{2} \rangle-\langle T \rangle^{2}},
		\end{equation}
		\begin{equation}
			R_{273}=\langle R_{0} \rangle-k\langle T \rangle,
		\end{equation} 
		\noindent где $\frac{dR_{0}}{dT}$ --- угловой коэффициент, $R_{273}$ --- свободный член. Так же вычислим
		погрешности их определения:
		\begin{equation}
			\sigma_{\frac{dR_{0}}{dT}}=\frac{1}{\sqrt{N}}\sqrt{\frac{\langle R_{0}^2 \rangle-\langle R_{0} \rangle^{2}}
						{\langle T^2 \rangle-\langle T \rangle^{2}}-\left(\frac{dR_{0}}{dT}\right)^{2}},
		\end{equation}
		\begin{equation}
			\sigma_{R_{273}}=\sigma_{\frac{dR_{0}}{dT}}\sqrt{\langle T^{2} \rangle-\langle T \rangle^{2}}.
		\end{equation}
		\par После вычислений получим следующие значения: $\frac{dR_{0}}{dT}=0.0451\pm0.0002$ Ом/К,
		$R_{273}=\text{   =}13.429\pm0.003$ Ом. Теперь построим график рис.5.
		\begin{figure}[H]
			\begin{center}
				% This file was created with tikzplotlib v0.9.16.
\begin{tikzpicture}

\definecolor{color0}{rgb}{0.12156862745098,0.466666666666667,0.705882352941177}

\begin{axis}[
    height=7.2cm,
    tick align=inside,
    major tick length=0.2cm,
    minor tick length=0.1cm,
    tick pos=left,
    x grid style={white!69.0196078431373!black},
    xmin=-10, xmax=140,
    xtick={-40, 0, 40, 80, 120, 160},
    minor x tick num=3,
    xmajorgrids,
    minor x grid style={dotted,black},
    xminorgrids,
    xtick style={color=black},
    xlabel={$l$, \text{см}},
    ymin = -0.2, ymax = 0.8,
    ytick = {-0.4, 0, 0.4, 0.8},
    ytick style={color=black},
    minor y tick num=3,
    ymajorgrids,
    minor y grid style={dotted,black},
    yminorgrids,
    ytick style={color=black},
    ylabel={$\upsilon_2$}
]
\path [draw=red, semithick]
(axis cs:-6,0.225806451612903)
--(axis cs:-2,0.225806451612903);

\path [draw=red, semithick]
(axis cs:-2,0.6)
--(axis cs:2,0.6);

\path [draw=red, semithick]
(axis cs:10,0.225806451612903)
--(axis cs:14,0.225806451612903);

\path [draw=red, semithick]
(axis cs:14,0.161290322580645)
--(axis cs:18,0.161290322580645);

\path [draw=red, semithick]
(axis cs:26,0.0967741935483871)
--(axis cs:30,0.0967741935483871);

\path [draw=red, semithick]
(axis cs:36,0.0322580645161291)
--(axis cs:40,0.0322580645161291);

\path [draw=red, semithick]
(axis cs:46,0.0666666666666667)
--(axis cs:50,0.0666666666666667);

\path [draw=red, semithick]
(axis cs:56,0.0322580645161291)
--(axis cs:60,0.0322580645161291);

\path [draw=red, semithick]
(axis cs:66,0.0322580645161291)
--(axis cs:70,0.0322580645161291);

\path [draw=red, semithick]
(axis cs:76,0.0666666666666667)
--(axis cs:80,0.0666666666666667);

\path [draw=red, semithick]
(axis cs:86,0.0322580645161291)
--(axis cs:90,0.0322580645161291);

\path [draw=red, semithick]
(axis cs:96,0.0666666666666667)
--(axis cs:100,0.0666666666666667);

\path [draw=red, semithick]
(axis cs:106,0.0967741935483871)
--(axis cs:110,0.0967741935483871);

\path [draw=red, semithick]
(axis cs:110,0.225806451612903)
--(axis cs:114,0.225806451612903);

\path [draw=red, semithick]
(axis cs:114,0.290322580645161)
--(axis cs:118,0.290322580645161);

\path [draw=red, semithick]
(axis cs:118,0.533333333333333)
--(axis cs:122,0.533333333333333);

\path [draw=red, semithick]
(axis cs:120,0.548387096774194)
--(axis cs:124,0.548387096774194);

\path [draw=red, semithick]
(axis cs:124,0.161290322580645)
--(axis cs:128,0.161290322580645);

\path [draw=red, semithick]
(axis cs:128,0.0967741935483871)
--(axis cs:132,0.0967741935483871);

\path [draw=red, semithick]
(axis cs:-4,0.146722164412071)
--(axis cs:-4,0.304890738813736);

\path [draw=red, semithick]
(axis cs:0,0.493333333333333)
--(axis cs:0,0.706666666666667);

\path [draw=red, semithick]
(axis cs:12,0.146722164412071)
--(axis cs:12,0.304890738813736);

\path [draw=red, semithick]
(axis cs:16,0.0863683662851197)
--(axis cs:16,0.236212278876171);

\path [draw=red, semithick]
(axis cs:28,0.0260145681581686)
--(axis cs:28,0.167533818938606);

\path [draw=red, semithick]
(axis cs:38,-0.0343392299687825)
--(axis cs:38,0.0988553590010406);

\path [draw=red, semithick]
(axis cs:48,-0.00444444444444439)
--(axis cs:48,0.137777777777778);

\path [draw=red, semithick]
(axis cs:58,-0.0343392299687825)
--(axis cs:58,0.0988553590010406);

\path [draw=red, semithick]
(axis cs:68,-0.0343392299687825)
--(axis cs:68,0.0988553590010406);

\path [draw=red, semithick]
(axis cs:78,-0.00444444444444439)
--(axis cs:78,0.137777777777778);

\path [draw=red, semithick]
(axis cs:88,-0.0343392299687825)
--(axis cs:88,0.0988553590010406);

\path [draw=red, semithick]
(axis cs:98,-0.00444444444444439)
--(axis cs:98,0.137777777777778);

\path [draw=red, semithick]
(axis cs:108,0.0260145681581686)
--(axis cs:108,0.167533818938606);

\path [draw=red, semithick]
(axis cs:112,0.146722164412071)
--(axis cs:112,0.304890738813736);

\path [draw=red, semithick]
(axis cs:116,0.207075962539022)
--(axis cs:116,0.373569198751301);

\path [draw=red, semithick]
(axis cs:120,0.431111111111111)
--(axis cs:120,0.635555555555556);

\path [draw=red, semithick]
(axis cs:122,0.448491155046826)
--(axis cs:122,0.648283038501561);

\path [draw=red, semithick]
(axis cs:126,0.0863683662851197)
--(axis cs:126,0.236212278876171);

\path [draw=red, semithick]
(axis cs:130,0.0260145681581686)
--(axis cs:130,0.167533818938606);

\addplot [semithick, color0]
table {%
-4 0.313814118861075
-2.64646464646465 0.372719374858457
-1.29292929292929 0.414429546476477
0.0606060606060606 0.441252184268248
1.41414141414141 0.455307668066297
2.76767676767677 0.458538104134772
4.12121212121212 0.452716029232126
5.47474747474748 0.439452921584311
6.82828282828283 0.420207518768442
8.18181818181818 0.396293942506964
9.53535353535354 0.368889630372301
10.8888888888889 0.339043074402003
12.2424242424242 0.307681366624373
13.5959595959596 0.275617551494593
14.949494949495 0.243557785241332
16.3030303030303 0.21210830212385
17.6565656565657 0.181782187599588
19.010101010101 0.153005958402251
20.3636363636364 0.126125949530377
21.7171717171717 0.101414508146396
23.0707070707071 0.0790759943861873
24.4242424242424 0.059252589079113
25.7777777777778 0.0420299083785511
27.1313131313131 0.0274424253029159
28.4848484848485 0.0154786981871685
29.8383838383838 0.00608640604481389
31.1919191919192 -0.000822809159606586
32.5454545454545 -0.0053686933275373
33.8989898989899 -0.00769792413293865
35.2525252525253 -0.00797984801832828
36.6060606060606 -0.0064024102803318
37.959595959596 -0.00316827824474411
39.3131313131313 0.001508842468899
40.6666666666667 0.0074076985932389
42.020202020202 0.0143027867695034
43.3737373737374 0.0219674580144038
44.7272727272727 0.0301768290975233
46.0808080808081 0.0387105008291939
47.4343434343434 0.0473550832588656
48.7878787878788 0.0559065277839617
50.1414141414141 0.0641722661692277
51.4949494949495 0.0719731564765673
52.8484848484849 0.0791452359053693
54.2020202020202 0.0855412805433256
55.5555555555556 0.0910321720277353
56.9090909090909 0.0955080711173033
58.2626262626263 0.0988793981744253
59.6161616161616 0.101077620557964
60.969696969697 0.102055846926514
62.3232323232323 0.10178922845216
63.6767676767677 0.10027516694472
65.030303030303 0.0975333298864813
66.3838383838384 0.0936054723774249
67.7373737373737 0.088555065990942
69.0909090909091 0.0824667345400373
70.4444444444444 0.0754454967540248
71.7979797979798 0.0676158158657112
73.1515151515152 0.059120456109071
74.5050505050505 0.0501191461274099
75.8585858585859 0.0407870492920191
77.2121212121212 0.031313040931319
78.5656565656566 0.0218977924704934
79.9191919191919 0.0127516624816113
81.2727272727273 0.00409239464424182
82.6262626262626 -0.00385737738344372
83.979797979798 -0.0108728181830784
85.3333333333333 -0.0167297718840175
86.6868686868687 -0.0212078365607651
88.040404040404 -0.024093631719913
89.3939393939394 -0.025184258876587
90.7474747474748 -0.0242909552204062
92.1010101010101 -0.0212429403709502
93.4545454545455 -0.0158914562227379
94.8080808080808 -0.00811399987971576
96.1616161616162 0.00218125032074329
97.5151515151515 0.015050815694331
98.8686868686869 0.030511105006786
100.222222222222 0.048533409181362
101.575757575758 0.0690387029167914
102.929292929293 0.091892253215734
104.282828282828 0.116898034823718
105.636363636364 0.143792952578569
106.989898989899 0.172240870670332
108.343434343434 0.20182644881168
109.69696969697 0.232048785318812
111.050505050505 0.262314867102847
112.40404040404 0.291932826571698
113.757575757576 0.320105005442445
115.111111111111 0.345920825464192
116.464646464646 0.368349466051415
117.818181818182 0.386232348827805
119.171717171717 0.39827542908059
120.525252525253 0.403041294125359
121.878787878788 0.398941068581365
123.232323232323 0.384226126557329
124.585858585859 0.35697961074772
125.939393939394 0.315107758439542
127.292929292929 0.25633103442959
128.646464646465 0.178175070852219
130 0.0779614139175848
};
\addplot [semithick, red, mark=*, mark size=3, mark options={solid}, only marks]
table {%
-4 0.225806451612903
0 0.6
12 0.225806451612903
16 0.161290322580645
28 0.0967741935483871
38 0.0322580645161291
48 0.0666666666666667
58 0.0322580645161291
68 0.0322580645161291
78 0.0666666666666667
88 0.0322580645161291
98 0.0666666666666667
108 0.0967741935483871
112 0.225806451612903
116 0.290322580645161
120 0.533333333333333
122 0.548387096774194
126 0.161290322580645
130 0.0967741935483871
};
\end{axis}

\end{tikzpicture}

			\end{center}
			\caption[figure]{График зависимости $R_{0}$ от $T$}
		\end{figure}
		\par Зная значения $\frac{dR_{0}}{dT}$ и $R_{273}$ мы можем узнать чему равен температурный коэффициент
		сопротивления молибдена $\alpha$ при 0 ℃. Для этого воспользуемся формулой:
		\begin{equation}
			\alpha=\frac{1}{R_{273}}\frac{dR_{0}}{dT}.
		\end{equation}
		\noindent Так как погрешность измерения $R_{273}$ много меньше чем погрешность $\frac{dR}{dT}$, то
		$\sigma_\alpha \approx \alpha\mathcal{E}_{\frac{dR}{dT}}$. В итоге получаем:
		$\alpha=3.36\pm0.02\text{ K}^{-1}\cdot10^{-3}$. 
		\subsection{Определение зависимости коэффициента теплопроводности воздуха от температуры}
		\par Используя результаты предыдущих пунктов мы можем определить наклон зависимости выделяющейся на
		нити мощности $Q$ от её перегрева $\Delta T$ относительно стенок:
		\begin{equation}
			\frac{dQ}{d\left(\Delta T\right)}=\frac{dR_{0}}{dT}/\frac{dR}{dQ}.
		\end{equation}
		\noindent Погрешность определения $\frac{dQ}{d\left(\Delta T\right)}$ вычислим по формуле:
		\begin{equation}
			\sigma_{\frac{dQ}{d\left(\Delta T\right)}}=
			\frac{dQ}{d\left(\Delta T\right)}\sqrt{\mathcal{E}_{\frac{dR_{0}}{dT}}^2+\mathcal{E}_{\frac{dR}{dQ}}^2}.
		\end{equation}
		\par Зная значение $\frac{dQ}{d\left(\Delta T\right)}$ при каждой температуре термостата можно вычистлить
		теплопроводность воздуха используя формула (5):
		\begin{equation}
			\kappa=\frac{dQ}{d\left(\Delta T\right)}\cdot\frac{\text{ln}\frac{r_{0}}{r_{1}}}{2\pi L}.
		\end{equation}
		\noindent Наибольший вкалд в погрешность значения $\kappa$ вносит погрешность измерения диаметра
		нити (около 2\%). Полученные данные занесём в таблицу 3.
		\begin{table}[H]
			\captionsetup{font={small}, labelformat=fullparents, labelsep=fill, labelfont=bf, justification=raggedleft,
              			singlelinecheck=false, skip=-0.2cm}
      			\caption{Коэффициент теплопрроводности воздуха при разных температурах}
      			\begin{center}
      				\begin{tabular}{|l|p{2cm}|p{2cm}|p{2cm}|p{2cm}|p{2cm}|}
					\hline
					$T$, ℃ & 24.0 & 30.0 & 40.0 & 49.9 & 60.0\\
					\hline
					$\kappa\text{, Вт/м}\cdot\text{К}\cdot10^{3}$ 
					& $29.6$ & $30.3$ & $30.9$ & $31.8$ & $32.6$ \\
					\hline
					$\sigma_{\kappa}\text{, Вт/м}\cdot\text{К}\cdot10^{3}$ 
					& $0.6$ & $0.6$ & $0.6$ & $0.6$ & $0.7$ \\
					\hline
      				\end{tabular}
      			\end{center}
		\end{table}
		\par Исходя из теоретических соображений коэфициент теплопроводности воздуха должен быть пропорционален
		корню квадратному из температуры. Проверим это, построив график зависимости ln$\kappa$ от ln$T$. 
		Погрешности логарифмов вычислим пообщей формуле для погрешностей:
		\begin{equation}
			\sigma_{\text{ln}x}=\mathcal{E}_{x}.
		\end{equation} 
		\noindent Занесём полученные значения в таблицу 4.
		\begin{table}[H]
			\captionsetup{font={small}, labelformat=fullparents, labelsep=fill, labelfont=bf, justification=raggedleft,
              			singlelinecheck=false, skip=-0.2cm}
      			\caption{Данные для построения графика ln$\kappa$ от ln$T$}
      			\begin{center}
      				\begin{tabular}{|p{2cm}|p{2cm}|p{2cm}|p{2cm}|p{2cm}|p{2cm}|}
					\hline
					$T$, К & 297.15 & 303.15 & 313.15 & 323.05 & 333.15 \\
					\hline
					ln$T$ & 5.69 & 5.71 & 5.75 & 5.78 & 5.81 \\
					\hline
					ln$\kappa$ 
					& $3.39$ & $3.41$ & $3.43$ & $3.46$ & $3.49$ \\
					\hline
					$\sigma_{\text{ln}\kappa}$ 
					& $0.02$ & $0.02$ & $0.02$ & $0.02$ & $0.02$ \\
					\hline
      				\end{tabular}
      			\end{center}
		\end{table}
		\par По таблице 4 построим график рис.6. Угловой коэффициент наклона аппроксимирующей кривой и
		свободный член найдём с помощью метода наименьших квадратов.
		\begin{figure}[H]
			\begin{center}
				% This file was created with tikzplotlib v0.9.16.
\begin{tikzpicture}

\begin{axis}[
    height=6.5cm,
    tick align=inside,
    major tick length=0.2cm,
    minor tick length=0.1cm,
    tick pos=left,
    x grid style={white!69.0196078431373!black},
    xmin=0, xmax=20,
    xtick={0, 5, 10, 15, 20},
    minor x tick num=3,
    xmajorgrids,
    minor x grid style={dotted,black},
    xminorgrids,
    xtick style={color=black},
    xlabel={$t$, мин},
    ymin = 7.2, ymax = 8,
    ytick = {7.2, 7.4, 7.6, 7.8, 8.0},
    ytick style={color=black},
    minor y tick num=4,
    ymajorgrids,
    minor y grid style={dotted,black},
    yminorgrids,
    ytick style={color=black},
    ylabel={$\log(z)$}
]
\addplot [semithick, black]
table {%
0 7.7308375473148
0.202020202020202 7.72575882837657
0.404040404040404 7.72068010943834
0.606060606060606 7.71560139050011
0.808080808080808 7.71052267156188
1.01010101010101 7.70544395262365
1.21212121212121 7.70036523368542
1.41414141414141 7.69528651474719
1.61616161616162 7.69020779580896
1.81818181818182 7.68512907687073
2.02020202020202 7.6800503579325
2.22222222222222 7.67497163899426
2.42424242424242 7.66989292005603
2.62626262626263 7.6648142011178
2.82828282828283 7.65973548217957
3.03030303030303 7.65465676324134
3.23232323232323 7.64957804430311
3.43434343434343 7.64449932536488
3.63636363636364 7.63942060642665
3.83838383838384 7.63434188748842
4.04040404040404 7.62926316855019
4.24242424242424 7.62418444961196
4.44444444444444 7.61910573067373
4.64646464646465 7.6140270117355
4.84848484848485 7.60894829279727
5.05050505050505 7.60386957385904
5.25252525252525 7.59879085492081
5.45454545454545 7.59371213598258
5.65656565656566 7.58863341704435
5.85858585858586 7.58355469810612
6.06060606060606 7.57847597916788
6.26262626262626 7.57339726022965
6.46464646464646 7.56831854129142
6.66666666666667 7.56323982235319
6.86868686868687 7.55816110341496
7.07070707070707 7.55308238447673
7.27272727272727 7.5480036655385
7.47474747474747 7.54292494660027
7.67676767676768 7.53784622766204
7.87878787878788 7.53276750872381
8.08080808080808 7.52768878978558
8.28282828282828 7.52261007084735
8.48484848484848 7.51753135190912
8.68686868686869 7.51245263297089
8.88888888888889 7.50737391403266
9.09090909090909 7.50229519509443
9.29292929292929 7.4972164761562
9.49494949494949 7.49213775721797
9.6969696969697 7.48705903827973
9.8989898989899 7.4819803193415
10.1010101010101 7.47690160040327
10.3030303030303 7.47182288146504
10.5050505050505 7.46674416252681
10.7070707070707 7.46166544358858
10.9090909090909 7.45658672465035
11.1111111111111 7.45150800571212
11.3131313131313 7.44642928677389
11.5151515151515 7.44135056783566
11.7171717171717 7.43627184889743
11.9191919191919 7.4311931299592
12.1212121212121 7.42611441102097
12.3232323232323 7.42103569208274
12.5252525252525 7.41595697314451
12.7272727272727 7.41087825420628
12.9292929292929 7.40579953526805
13.1313131313131 7.40072081632982
13.3333333333333 7.39564209739159
13.5353535353535 7.39056337845335
13.7373737373737 7.38548465951512
13.9393939393939 7.38040594057689
14.1414141414141 7.37532722163866
14.3434343434343 7.37024850270043
14.5454545454545 7.3651697837622
14.7474747474747 7.36009106482397
14.9494949494949 7.35501234588574
15.1515151515152 7.34993362694751
15.3535353535354 7.34485490800928
15.5555555555556 7.33977618907105
15.7575757575758 7.33469747013282
15.959595959596 7.32961875119459
16.1616161616162 7.32454003225636
16.3636363636364 7.31946131331813
16.5656565656566 7.3143825943799
16.7676767676768 7.30930387544167
16.969696969697 7.30422515650344
17.1717171717172 7.2991464375652
17.3737373737374 7.29406771862697
17.5757575757576 7.28898899968874
17.7777777777778 7.28391028075051
17.979797979798 7.27883156181228
18.1818181818182 7.27375284287405
18.3838383838384 7.26867412393582
18.5858585858586 7.26359540499759
18.7878787878788 7.25851668605936
18.989898989899 7.25343796712113
19.1919191919192 7.2483592481829
19.3939393939394 7.24328052924467
19.5959595959596 7.23820181030644
19.7979797979798 7.23312309136821
20 7.22804437242998
};
\path [draw=red, semithick]
(axis cs:-0.0833333333333333,7.82164312623998)
--(axis cs:0.0833333333333333,7.82164312623998);

\path [draw=red, semithick]
(axis cs:0.916666666666667,7.78072088611792)
--(axis cs:1.08333333333333,7.78072088611792);

\path [draw=red, semithick]
(axis cs:1.91666666666667,7.68937110752969)
--(axis cs:2.08333333333333,7.68937110752969);

\path [draw=red, semithick]
(axis cs:2.91666666666667,7.65444322647011)
--(axis cs:3.08333333333333,7.65444322647011);

\path [draw=red, semithick]
(axis cs:3.91666666666667,7.61726781362835)
--(axis cs:4.08333333333333,7.61726781362835);

\path [draw=red, semithick]
(axis cs:4.91666666666667,7.58832367733522)
--(axis cs:5.08333333333333,7.58832367733522);

\path [draw=red, semithick]
(axis cs:5.91666666666667,7.55642796944025)
--(axis cs:6.08333333333333,7.55642796944025);

\path [draw=red, semithick]
(axis cs:6.91666666666667,7.51152464839087)
--(axis cs:7.08333333333333,7.51152464839087);

\path [draw=red, semithick]
(axis cs:7.91666666666667,7.48549160803075)
--(axis cs:8.08333333333333,7.48549160803075);

\path [draw=red, semithick]
(axis cs:8.91666666666667,7.46450983463653)
--(axis cs:9.08333333333333,7.46450983463653);

\path [draw=red, semithick]
(axis cs:9.91666666666667,7.44073370738926)
--(axis cs:10.0833333333333,7.44073370738926);

\path [draw=red, semithick]
(axis cs:10.9166666666667,7.41697962138115)
--(axis cs:11.0833333333333,7.41697962138115);

\path [draw=red, semithick]
(axis cs:11.9166666666667,7.39817409297047)
--(axis cs:12.0833333333333,7.39817409297047);

\path [draw=red, semithick]
(axis cs:12.9166666666667,7.37963215260955)
--(axis cs:13.0833333333333,7.37963215260955);

\path [draw=red, semithick]
(axis cs:13.9166666666667,7.36707705988101)
--(axis cs:14.0833333333333,7.36707705988101);

\path [draw=red, semithick]
(axis cs:14.9166666666667,7.34729970074316)
--(axis cs:15.0833333333333,7.34729970074316);

\path [draw=red, semithick]
(axis cs:15.9166666666667,7.33498187887181)
--(axis cs:16.0833333333333,7.33498187887181);

\path [draw=red, semithick]
(axis cs:16.9166666666667,7.32251043399739)
--(axis cs:17.0833333333333,7.32251043399739);

\path [draw=red, semithick]
(axis cs:17.9166666666667,7.30787278076371)
--(axis cs:18.0833333333333,7.30787278076371);

\path [draw=red, semithick]
(axis cs:18.9166666666667,7.29776828253138)
--(axis cs:19.0833333333333,7.29776828253138);

\path [draw=red, semithick]
(axis cs:19.9166666666667,7.28550654852279)
--(axis cs:20.0833333333333,7.28550654852279);

\path [draw=red, semithick]
(axis cs:0,7.81963831469227)
--(axis cs:0,7.8236479377877);

\path [draw=red, semithick]
(axis cs:1,7.77863233139778)
--(axis cs:1,7.78280944083805);

\path [draw=red, semithick]
(axis cs:2,7.68708277801024)
--(axis cs:2,7.69165943704914);

\path [draw=red, semithick]
(axis cs:3,7.65207355822367)
--(axis cs:3,7.65681289471656);

\path [draw=red, semithick]
(axis cs:4,7.61480839405137)
--(axis cs:4,7.61972723320533);

\path [draw=red, semithick]
(axis cs:5,7.5857920317656)
--(axis cs:5,7.59085532290484);

\path [draw=red, semithick]
(axis cs:6,7.55381427367444)
--(axis cs:6,7.55904166520607);

\path [draw=red, semithick]
(axis cs:7,7.50879091410984)
--(axis cs:7,7.51425838267189);

\path [draw=red, semithick]
(axis cs:8,7.48268577189158)
--(axis cs:8,7.48829744416992);

\path [draw=red, semithick]
(axis cs:9,7.46164450512363)
--(axis cs:9,7.46737516414942);

\path [draw=red, semithick]
(axis cs:10,7.43779943508879)
--(axis cs:10,7.44366797968973);

\path [draw=red, semithick]
(axis cs:11,7.41397481368885)
--(axis cs:11,7.41998442907346);

\path [draw=red, semithick]
(axis cs:12,7.39511224361345)
--(axis cs:12,7.40123594232748);

\path [draw=red, semithick]
(axis cs:13,7.37651300101879)
--(axis cs:13,7.38275130420032);

\path [draw=red, semithick]
(axis cs:14,7.36391850018423)
--(axis cs:14,7.37023561957779);

\path [draw=red, semithick]
(axis cs:15,7.34407805125863)
--(axis cs:15,7.3505213502277);

\path [draw=red, semithick]
(axis cs:16,7.33172030026777)
--(axis cs:16,7.33824345747586);

\path [draw=red, semithick]
(axis cs:17,7.31920792408986)
--(axis cs:17,7.32581294390492);

\path [draw=red, semithick]
(axis cs:18,7.30452157432939)
--(axis cs:18,7.31122398719802);

\path [draw=red, semithick]
(axis cs:19,7.29438304217931)
--(axis cs:19,7.30115352288345);

\path [draw=red, semithick]
(axis cs:20,7.28207954372498)
--(axis cs:20,7.28893355332059);

\addplot [semithick, red, mark=*, mark size=1, mark options={solid}, only marks]
table {%
0 7.82164312623998
1 7.78072088611792
2 7.68937110752969
3 7.65444322647011
4 7.61726781362835
5 7.58832367733522
6 7.55642796944025
7 7.51152464839087
8 7.48549160803075
9 7.46450983463653
10 7.44073370738926
11 7.41697962138115
12 7.39817409297047
13 7.37963215260955
14 7.36707705988101
15 7.34729970074316
16 7.33498187887181
17 7.32251043399739
18 7.30787278076371
19 7.29776828253138
20 7.28550654852279
};
\end{axis}

\end{tikzpicture}

			\end{center}
			\caption[figure]{График зависимости ln$\kappa$ от ln$T$}
		\end{figure}
		\par После всех вычислений получаем, что $\kappa$ зависит от $T^\beta$, где $\beta=0.82\pm0.02$.
	\section{Обсуждение результатов}
		\par В данной лабораторной работе было проведено два исследования: 1) определение коэффициента
		температурного сопротивления молибдена; 2) определение зависимости коэффициента теплопроводности
		воздуха от его температуры. К сожалению, ни одно из них не совпало с известными табличными данными и
		теоретическими моделями.
		\par Тот факт, что во втором исследование все значения коэффициента теплопроводности воздуха завышены, по
		сравнению с табличными может указывать на то, что виновниками данных результатов являются
		систематические погрешности приборов. Наибольшие подозрения вызывают показания термостата, так как по
		словам заведующей лаборатории: ''студенты с ним начудили''. По-мимо термостата возникают вопросы и к
		точности измерения диаметра нити, она слишком низкая в сравнение с остальными (2\% против $\sim$0.5\%).
		\par Утверждение о систематической погрешности приборов подтверждается ещё и тем, что все построенные
		линейные графики с очень высокой точностью ложатся на экспериментальные точки.
	\section{Вывод}
		\par Несмортя на столь разочаровывающие результаты, остаётся надежда, что при более качественном
		оборудовании мы можем получить подтверждение наших теоретических моделей.
		
\end{document}
\documentclass[12pt,a4paper]{article}
\usepackage[utf8]{inputenc}
\usepackage[english,russian]{babel}
\usepackage{indentfirst}
\usepackage{misccorr}
\usepackage{graphicx}
\usepackage{amsmath}
\usepackage{amssymb}
\usepackage{circuitikz}
\usepackage[font={small}]{caption}
\usepackage[left=20mm, top=20mm, right=20mm, bottom=20mm, nohead]{geometry}
\usepackage{float}
\usepackage{tabularx}
\usepackage{array}
\usepackage{longtable}
\usepackage{pstool}
\usepackage{pgfplots}
\usepackage{hhline}
\usepackage{multirow}

\DeclareCaptionLabelSeparator{fill}{.\\}
\DeclareCaptionLabelFormat{fullparents}{\bothIfFirst{#1}{~}#2}

\begin{document}

	\begin{titlepage}
		\begin{center}
			{\LARGE Отчёт по лабораторной работе 2.4.1.\\}
			\vspace*{11cm}
				\textbf{\LARGE Определение теплоты испарения жидкости.}	
			\vspace*{6.5cm}
		\end{center}
		\hfill\begin{minipage}{0.37\textwidth}
				Работу выполнил Громов Артём\\
				ЛФИ Б02-006
		\end{minipage}
		\vspace{5cm}
		\begin{center}
 			Долгопрудный, 2021 г.
		\end{center}   
	\end{titlepage}
	
	\section{Аннотация}
		\noindent\textbf{Цель работы: }
		\begin{enumerate}
			\item измерение давления насыщенного пара жидкости при разной температуре;
			\item вычисление по полученным данным теплоты испарения с помощью уравнения Клапейрона–Клаузиуса.
		\end{enumerate}
		\noindent\textbf{В работе используются: } термостат; герметический сосуд, заполненный исследуемой жидкостью;
		отсчётный микроскоп.
		\vspace{0.5 cm}
		\par Испарением называется переход вещества из жидкого в газообразное состояние. Оно происходит на свободной
		поверхности жидкости. При испарении с поверхности вылетают молекулы, образуя над ней пар. Для выхода из
		жидкости молекулы должны преодолеть силы молекулярного сцепления. Кроме того, при испарении совершается работа
		против внешнего давления $P$, поскольку объём жидкости меньше объёма пара. Не все молекулы жидкости способны
		совершить эту работу, а только те из них, которые обладают достаточной кинетической энергией. Поэтому переход
		части молекул в пар приводит к обеднению жидкости быстрыми молекулами, т. е. к её охлаждению. Чтобы испарение
		проходило без изменения температуры, к жидкости нужно подводить тепло. Количество теплоты, необходимое для
		изотермического испарения одного моля жидкости при внешнем давлении, равном упругости её насыщенных паров,
		называется молярной теплотой испарения (парообразования).
		\par Теплоту парообразования жидкостей можно измерить непосредственно при помощи калориметра. Такой метод,
		однако, не позволяет получить точных результатов из-за неконтролируемых потерь тепла, которые трудно сделать
		малыми. В настоящей работе для определения теплоты испарения применён косвенный метод, основанный на формуле
		Клапейрона–Клаузиуса:
		\begin{equation}
			\frac{dP}{dT} = \frac{L}{T\left(V_2-V_1\right)}.
		\end{equation}
		\noindent Здесь $P$ --- давление насыщенного пара жидкости при температуре $T$, $T$ --- абсолютная температура
		жидкости и пара, $L$ --- теплота испарения жидкости, $V_2$ --- объём пара, $V_1$ --- объём жидкости. Найдя из опыта
		$dP/dT$ , $T$ , $V_2$ и $V_1$, можно определить $L$ путём расчёта. Величины $L$, $V_2$ и $V_1$ в формуле (1)
		должны относиться к одному и тому же количеству вещества; мы будем относить их к одному молю.
		\par Для воды: $T_{\text{кип}}=373 \ \ K$, $V_1=18\cdot10^{-6} \ \ \frac{\text{м}^{3}}{\text{моль}}$,
		$V_2=31\cdot10^{-3} \ \ \frac{\text{м}^{3}}{\text{моль}}$, $b=26\cdot10^{-6} \ \ \frac{\text{м}^{3}}{\text{моль}}$,
		$a=0.4 \ \ \frac{\text{Па}\cdot\text{м}^{6}}{\text{моль}^{2}}$, $a/V_{2}^{2}=0.42 \ \ \text{кПа}$. Видно, что $V_1$ не
		превосходит 0.5\% от $V_2$. При нашей точности опытов величиной $V_1$ в (1) можно пренебречь.
		\par Обратимся теперь к $V_2$, которое в дальнейшем будем обозначать просто $V$ . Объём $V$ связан с давлением и
		температурой уравнением Ван-дер-Ваальса:
		\begin{equation}
			\left(P+\frac{a}{V_{2}^{2}}\right)\left(V-b\right)=RT.
		\end{equation}
		\noindent Из таблицы можно видеть, что $b$ одного порядка с $V_1$. В уравнении Ван-дер-Ваальса величиной $b$
		следует пренебречь. Пренебрежение членом $a/V^2$ по сравнению с $P$ вносит ошибку менее 3\%. При давлении
		ниже атмосферного ошибки становятся ещё меньше. Таким образом, при давлениях ниже атмосферного уравнение
		Ван-дер-Ваальса для насыщенного пара мало отличается от уравнения Клапейрона. Положим поэтому
		\begin{equation}
			V=\frac{RT}{P}.
		\end{equation}
		\noindent Подставляя (3) в (1), пренебрегая $V_1$ и разрешая уравнение относительно $L$, найдём
		\begin{equation}
			L=\frac{RT^{2}}{P}\frac{dP}{dT}=-R\frac{d\left(\text{ln}P\right)}{d\left(1/T\right)}.
		\end{equation}
		\par В нашем опыте температура жидкости измеряется термометром, давление пара определяется при помощи
		манометра, а производные $dP/dT$ или $d\left(\text{ln}P\right)/d\left(1/T\right)$ находятся графически как угловой
		коэффициент касательной к кривой $P(T)$ или соответственно к кривой, у которой по оси абсцисс отложено $1/T$, а
		по оси ординат $\text{ln}P$.
	\section{Экспериментальная установка}
		\begin{figure}[H]
			\begin{minipage}{0.55\textwidth}
				\hspace{0.5cm} Схема установки изображена на рис. 1. Резервуар с водой 1 играет роль термостата.
				Термостат нагревается спиралью 2, посредством электрического тока. Для охдаждения
				термостата через змеевик 3 пропускается водопроводная вода. Вода в термостате перемешивается воздухом,
				через трубку 4. Температура воды измеряется термометром 5. В термостат погружён запаянный
				прибор 6 с исследуемой жидкостью. Над ней находится насыщенный пар. Давление насыщенного пара
				определяется по ртутному манометру. Отсчёт показаний манометра производится при помощи микроскопа.
				\par\hspace{0.5cm} Описание прибора указывает на важное преимущество предложенного косвенного метода
				измерения $L$ перед прямым. При непосредственном измерении теплоты испарения опыты нужно производить
				при неизменном давлении, и прибор не может быть запаян. При этом невозможно обеспечить такую чистоту и
				неизменность экспериментальных условий, как при нашей постановке опыта. 
			\end{minipage}
			\hfill
			\begin{minipage}{0.40\textwidth}
				\begin{center}
					\includegraphics[width=\textwidth]{C:/Users/gromo/Desktop/Установка 241(1).png}
					\caption{Схема установки для определения теплоты испарения}
				\end{center}
			\end{minipage}
		\end{figure}
		\par Описываемый прибор обладает важным недостатком: термометр определяет температуру термостата, а не
		исследуемой жидкости (или её пара). Эти температуры близки друг к другу лишь в том случае, если нагревание
		происходит достаточно медленно. Убедиться в том, что темп нагревания не является слишком быстрым, можно,
		сравнивая результаты, полученные при нагревании и при остывании прибора. Такое сравнение необходимо сделать.
		Для ориентировки укажем, что температуру воды в калориметре следует менять не быстрее, чем на 1 °C в течение
		1---3 минут.
	\section{Резултаты измерений и обработка данных}
		\subsection{Измерения при повышении температуры}
		\par Для начала проведём измерния повышая температуру в термостате. Для этого будем пропускать ток через
		нагреватель. Через каждый градус будем записывать показания ртутного манометра: определим высоту столбцов ртути
		$h_1$ и $h_2$ относительно нулевого уровня. Погрешность опрделения высоты столбцов равна погрешности
		штангенциркуля $\Delta_{\text{шт}}=0.1$ мм. Приняв плотность ртути равной $\rho=13600 \ \ \text{кг/м}^{3}$, можно
		вычислиь давление по формуле
		\begin{equation}
			P=\rho g(h_2-h_1).
		\end{equation}
		\noindent Погрешность определения давления равна $\sigma_{P}=\rho g\sigma_{\Delta h}$, где $\Delta h=0.14$ мм. Для
		построения графиков вычислим так же ln$P$ и $1/T$. Погрешность $\sigma_{\text{ln}P}=\mathcal{E}_{P}$. Занесём
		полученные данные в таблицу 1.
		\begin{table}[H]
			\captionsetup{font={small}, labelformat=fullparents, labelsep=fill, labelfont=bf, justification=raggedleft,
			singlelinecheck=false, skip=-0.2cm}
			\caption{Измерения при растущей температуре}
			\begin{center}
				\begin{tabular}{|p{1.5cm}|p{1.5cm}|p{1.5cm}|p{1.5cm}|p{1.5cm}|l|p{1.5cm}|p{1.5cm}|}
				\hline
				$T$, °C & $h_1$, мм & $h_2$, мм & $P$, Па & $\sigma_{P}$, Па & $1/T$, 1/К $\cdot10^{-5}$ & ln$P$ &
				$\sigma_{\text{ln}P}$ \\
				\hline
				20.0 & 34.9 & 74.2 & 5240 & 20 & 34.1	 & 8.564 & 0,004 \\
				\hline
				21.0 & 33.1 & 77.7 & 5950 & 20 & 34.0	 & 8.691 & 0,003 \\
				\hline
				22.2 & 32.7 & 77.9 & 6030 & 20 & 33.9	 & 8.705 & 0,003 \\
				\hline
				23.2 & 30.9 & 78.3 & 6320 & 20 & 33.7	 & 8.753 & 0,003 \\
				\hline
				24.0 & 29.2 & 80.5 & 6844 & 20 & 33.7	 & 8.831 & 0,003 \\
				\hline
				25.0 & 28.2 & 81.4 & 7100 & 20 & 33.5	 & 8.877 & 0,003 \\
				\hline
				26.0 & 26.8 & 83.0 & 7500 & 20 & 33.4	 & 8.922 & 0,003 \\
				\hline
				27.0 & 25.2 & 85.8 & 8090 & 20 & 33.3	 & 8.998 & 0,002 \\
				\hline
				28.0 & 22.9 & 87.0 & 8550 & 20 & 33.2	 & 9.054 & 0,002 \\
				\hline
				29.0 & 22.0 & 88.5 & 8870 & 20 & 33.1	 & 9.091 & 0,002 \\
				\hline
				30.0 & 20.4 & 90.2 & 9310 & 20 & 33.0	 & 9.139 & 0,002 \\
				\hline
				31.0 & 16.8 & 92.9 & 10150 & 20 & 32.9 & 9.226 & 0,002 \\
				\hline
				32.0 & 14.9 & 95.5 & 10750 & 20 & 32.8 & 9.283 & 0,002 \\
				\hline
				33.0 & 12.4 & 97.5 & 11350 & 20 & 32.7 & 9.337 & 0,002 \\
				\hline
				34.0 & 10.2 & 99.6 & 11930 & 20 & 32.6 & 9.387 & 0,002 \\
				\hline
				35.0 & 7.7 & 101.7 & 12540 & 20 & 32.5 & 9.437 & 0,002 \\
				\hline
				36.0 & 5.0 & 105.8 & 13450 & 20 & 32.3 & 9.507 & 0,001 \\
				\hline
				37.0 & 2.8 & 107.9 & 14020 & 20 & 32.2 & 9.548 & 0,001 \\		
				\hline
				\end{tabular}
			\end{center}
		\end{table}
		\subsection{Измерения при понижении температуры}
		\par Для оценик правильности результатов полученных нами в предыдущем пункте выполним те же измерение, но
		понижая температуру термостата. Данные будем снимать черезкаждые два градуса. Результаты запишем в таблицу 2.
		\begin{table}[H]
			\captionsetup{font={small}, labelformat=fullparents, labelsep=fill, labelfont=bf, justification=raggedleft,
			singlelinecheck=false, skip=-0.2cm}
			\caption{Измерения при падающей температуре}
			\begin{center}
				\begin{tabular}{|p{1.5cm}|p{1.5cm}|p{1.5cm}|p{1.5cm}|p{1.5cm}|l|p{1.5cm}|p{1.5cm}|}
				\hline
				$T$, °C & $h_1$, мм & $h_2$, мм & $P$, Па & $\sigma_{P}$, Па & $1/T$, 1/К $\cdot10^{-5}$ & ln$P$ &
				$\sigma_{\text{ln}P}$ \\
				\hline
				35.0 & 4.4 & 104.0 & 13290 & 20 & 32.5 & 8.691 & 0,001 \\
				\hline
				33.0 & 9.5 & 99.0 & 11940 & 20 & 32.7	 & 8.705 & 0,002 \\
				\hline
				31.0 & 15.1 & 95.0 & 10660 & 20 & 32.9 & 8.753 & 0,002 \\
				\hline
				29.0 & 19.2 & 90.0 & 9450 & 20 & 33.71 & 8.831 & 0,002 \\
				\hline
				27.0 & 23.3 & 86.2 & 8390 & 20 & 33.3	 & 8.877 & 0,002 \\
				\hline
				25.0 & 26.4 & 83.0 & 7550 & 20 & 33.5	 & 8.922 & 0,003 \\
				\hline
				23.0 & 28.4 & 80.0 & 6880 & 20 & 33.8	 & 8.998 & 0,003 \\
				\hline		
				\end{tabular}
			\end{center}
		\end{table}
		\subsection{Построение графиков и вычисление удельной теплоты испарения}
		\par Займёмся построением графиков $P(T)$. Теория предсказывает, что зависимость должна быть экспонециальной.
		И правда, если аппроксимировать полученные значения прямой, то она плохо ложится на точки. Попробуем
		аппроксимировать полученные значения квадратичной функцией $P=aT^2+bT+c$. Для этого необходимо решить
		следующую систему уравнений
		\begin{equation}
			c\sum_{i=1}^{n}x_{i}^{4}+b\sum_{i=1}^{n}x_{i}^{3}+a\sum_{i=1}^{n}x_{i}^{2}=\sum_{i=1}^{n}y_{i}x_{i}^{2}
		\end{equation}
		\begin{equation}
			c\sum_{i=1}^{n}x_{i}^{3}+b\sum_{i=1}^{n}x_{i}^{2}+a\sum_{i=1}^{n}x_{i}=\sum_{i=1}^{n}y_{i}x_{i}
		\end{equation}
		\begin{equation}
			c\sum_{i=1}^{n}x_{i}^{2}+b\sum_{i=1}^{n}x_{i}+an=\sum_{i=1}^{n}y_{i}
		\end{equation}
		\par Решая её получим следующие значения для графика нагревания $a=14.1\pm0.1 \text{}$,
		$b=-7980\pm50$. Погрешность оценим в $3\mathcal{E}_{P}$.
		\par Для графика охлаждения значения будут следущими  $a=17.6\pm0.1 \text{}$, $b=-10100\pm100$.
		\begin{figure}[H]
			\begin{center}
				\input{C:/Users/gromo/PycharmProjects/plottest/plot12.tex}
			\end{center}
			\captionsetup{font={small}, justification=justified}
			\caption[figure]{Графики зависимости $P$ от $T$}
		\end{figure}
		\par Опираясь на полученные значения, вычисли значения $L$ для каждой температуры в обоиих слуаях. Для этого
		воспользуемся формулой (3). Заменим в ней $dP/dT$ на $2aT-b$. Так же полученные значения надо поделить на
		молярную масу воды $\mu=0.018$ кг/моль. Погрешность оценим в $4\mathcal{E}_{P}$. В итоге получим
		$\sigma_{L}=0.02$ МДж/кг. Результаты вычислений занесём в таблицы 3 и 4.
		\begin{table}[H]
			\captionsetup{font={small}, labelformat=fullparents, labelsep=fill, labelfont=bf, justification=raggedleft,
			singlelinecheck=false, skip=-0.2cm}
			\caption{Значения L при нагревании}
			\begin{center}
				\begin{tabular}{|p{1.4cm}|p{1.4cm}|p{1.4cm}|p{1.4cm}|p{1.4cm}|p{1.4cm}|p{1.4cm}|p{1.4cm}|p{1.4cm}|}
				\hline
				\multicolumn{9}{|c|}{$T$, °C} \\
				\hline
				20.0 & 21.0 & 22.2 & 23.2 & 24.0 & 25.0 & 26.0 & 27.0 & 28.0 \\
				\hline
				\multicolumn{9}{|c|}{$L$, МДж/кг} \\
				\hline
				2.04 & 2.00 & 2.21 & 2.31 & 2.28 & 2.37 & 2.42 & 2.40 & 2.42 \\
				\hline
				\multicolumn{9}{|c|}{$T$, °C} \\
				\hline
				29.0 & 30.0 & 31.0 & 32.0 & 33.0 & 34.0 & 35.0 & 36.0 & 37.0 \\ 
				\hline
				\multicolumn{9}{|c|}{$L$, МДж/кг} \\
				\hline
				2.48 & 2.51 & 2.44 & 2.43 & 2.42 & 2.42 & 2.42 & 2.36 & 2.37 \\
				\hline
				\end{tabular}
			\end{center}
		\end{table}
		\begin{table}[H]
			\captionsetup{font={small}, labelformat=fullparents, labelsep=fill, labelfont=bf, justification=raggedleft,
			singlelinecheck=false, skip=-0.2cm}
			\caption{Значения L при охлаждении}
			\begin{center}
				\begin{tabular}{|p{1.9cm}|p{1.9cm}|p{1.9cm}|p{1.9cm}|p{1.9cm}|p{1.9cm}|p{1.9cm}|}
				\hline
				\multicolumn{7}{|c|}{$T$, °C} \\
				\hline
				35.0 & 33.0 & 31.0 & 29.0 & 27.0 & 25.0 & 23.0 \\
				\hline
				\multicolumn{7}{|c|}{$L$, МДж/кг} \\
				\hline
				2.48 & 2.47 & 2.45 & 2.41 & 2.33 & 2.17 & 1.94 \\
				\hline
				\end{tabular}
			\end{center}
		\end{table}
		\par Теперь построим графики ln$P(1/T)$. Для этого обратимся к таблицам 1 и 2. Эти графики хорошо аппроксимирутся
		прямым.
		\begin{figure}[H]
			\begin{minipage}{0.49\textwidth}
				\begin{center}
					\input{C:/Users/gromo/PycharmProjects/plottest/plot13.tex}
				\end{center}
				\captionsetup{font={small}, justification=justified}
				\caption[figure]{Нагревание}
			\end{minipage}
			\hfill
			\begin{minipage}{0.49\textwidth}
				\begin{center}
					\input{C:/Users/gromo/PycharmProjects/plottest/plot14.tex}
				\end{center}
				\captionsetup{font={small}, justification=justified}
				\caption[figure]{Охлаждение}
			\end{minipage}
		\end{figure}
		\par Используем метод наименьших квадратов для вычисления коэффициентов
		\begin{equation}
			\frac{dy}{dx}=\frac{\langle xy  \rangle -\langle x \rangle \langle y \rangle}
			{\langle x^2 \rangle - \langle x \rangle^2}
		\end{equation}
		\begin{equation}
			b = \langle y \rangle - \left(\frac{dy}{dx}\right)\langle x \rangle
		\end{equation}
		\begin{equation}
			\sigma_{\frac{dy}{dx}} = \frac{1}{\sqrt{n}} \sqrt{\frac{\langle y^2 \rangle - \langle y \rangle^2 }
			{\langle x^2 \rangle - \langle x \rangle^2} - \left(\frac{dy}{dx}\right)^2}
		\end{equation}
		\par В результате получим, что при нагревании коэффициент $\left(\frac{d(\text{ln}P)}{d(1/T)}\right)_{н}=-5170\pm70$
		К, а при охладжении $\left(\frac{d(\text{ln}P)}{d(1/T)}\right)_{о}=-5069\pm80$ К. Теперь можно вычислить $L$
		пользуясь формулой (3). Получим $L_{\text{н}}=2.35\pm0.4$ МДж/кг, $L_{\text{о}}=2.35\pm0.4$ МДж/кг.
	\section{Обсуждение результатов}
		\par В данной работе мы измеряли зависимость давления насыщенного пара от температуры, а затем находили удельную
		теплоту парообразования. Погрешность определения давления составила около 0.3\%, что довольно точно. Однако,
		несмотря на точность измерений результаты определения теплоты испарения первым способом (с помощью $dP/dT$) не
		совпали с табличными. Результаты второго способо совпали в пределах $\sigma$. Возможные причины расхождений могут
		заключаться в неточности измерений высоты столба ртути (шкала микроскопа и столбик находились в разных плоскостях).
		А также в недостаточном времени ожидания теплового равновесия.
	\section{Вывод}
		\par Данное обурудование позволяет измерять необходимые велечины с очень хорошей точностью, поэтому ожидаем, что
		при более аккуратных измерниях мы получим результаты, сходящиеся с табличными.
		
\end{document}
\documentclass[12pt,a4paper]{article}
\usepackage[T2A]{fontenc}
\usepackage[utf8]{inputenc}
\usepackage[english, russian]{babel}
\usepackage{indentfirst}
\usepackage{misccorr}
\usepackage{graphicx}
\usepackage{amsmath}
\usepackage{amssymb}
\usepackage{circuitikz}
\usepackage[font={small}]{caption}
\usepackage[left=20mm, top=20mm, right=20mm, bottom=20mm, nohead]{geometry}
\usepackage{float}
\usepackage{tabularx}
\usepackage{array}
\usepackage{longtable}
\usepackage{pstool}
\usepackage{pgfplots}
\usepackage{hhline}
\usepackage{multirow}
\usepackage{wrapfig}
\usepackage{pdfpages}

\DeclareCaptionLabelSeparator{fill}{.\\}

\DeclareCaptionLabelFormat{fullparents}{\bothIfFirst{#1}{~}#2}

\pgfplotsset{compat=1.17}

\begin{document}

    \begin{titlepage}
        \begin{center}
            {\LARGE Отчёт по лабораторной работе 4.1.2\\}
            \vspace*{11cm}
                \textbf{\LARGE Моделирование оптических приборов и определение их увеличения.}
            \vspace*{6.5cm}
        \end{center}
        \hfill\begin{minipage}{0.37\textwidth}
            Работу выполнил Громов Артём
            \\
            ЛФИ Б02-006
        \end{minipage}
        \vspace{4.8cm}
        \begin{center}
            Долгопрудный, 2022 г.
        \end{center}
    \end{titlepage}

    \section{Аннотация}

        \begin{flushleft}
            \textbf{Цель работы:} изучить модели зрительных труб (астрономической трубы Кеплера и земной трубы Галилея) и микроскопа, определить их увеличения.
        \end{flushleft}
        \begin{flushleft}
            \textbf{В работе используется:} оптическая скамья, набор линз, экран, осветитель со шкалой, зрительная труба, диафрагма, линейка.
        \end{flushleft}

        \begin{wrapfigure}{r}{0.4\textwidth}
            \centering
            \includegraphics[width=0.4\textwidth]{images/pic1.pdf}
            \caption{Ход лучей в трубе Кеплера.}
            \includegraphics[width=0.4\textwidth]{images/pic2.pdf}
            \caption{Ход лучей в трубе Галилея.}
            \includegraphics[width=0.4\textwidth]{images/pic3.pdf}
            \caption{Ход лучей в микроскопе.}
        \end{wrapfigure}
        \par В настоящей работе изучаются модели зрительных труб (астрономической и земной) и микроскопа. Каждый из этих оптических приборов состоит из двух
        основных частей: объектива --- линзы, обращённой к объекту, и окуляра --- линзы, обращённой к наблюдателю. Объектив, в качестве которого используется
        положительная линза, создаёт действительное изображение предмета. Это изображение рассматривается глазом через окуляр. Ход лучей в астрономической и
        земной зрительных трубах и в микроскопе представлен на рис. 1–3.
        \par Поскольку зрительные трубы используются для наблюдения удалённых предметов, находящихся от объектива на расстояниях, значительно превышающих его
        фокусное расстояние, изображение $A$ предмета, даваемое объективом, находится практически в его фокальной плоскости. В случае микроскопа промежуточное
        изображение $A$ находится далеко за фокальной плоскостью объектива, так как предмет располагается вблизи переднего фокуса.
        \par Мнимое изображение $B$, даваемое окуляром, располагается на некотором расстоянии $d$ от окуляра. Наводя оптический инструмент на резкость,
        наблюдатель автоматически устанавливает такое расстояние $d$, которое удобно для аккомодации глаза. Поскольку глаз обладает значительной областью
        аккомодации, расстояние $d$ даже для одного и того же наблюдателя может существенно изменяться от опыта к опыту. При изменении аккомодации оптический
        прибор, вооружающий глаз, должен быть несколько перефокусирован. В зрительных трубах этого достигают перемещением окуляра, а в микроскопе ---
        перемещением всей оптической системы относительно предмета. Для того чтобы исключить в теории произвол, связанный с неопределённостью расстояния $d$,
        полагают обычно, что глаз наблюдателя аккомодирован на бесконечность. При этом мнимое изображение $B$ должно располагаться в бесконечности, и,
        следовательно, промежуточное изображение $A$ должно совпадать с фокальной плоскостью окуляра.
        \par При наблюдении предметов с помощью зрительной трубы или микроскопа угловой размер изображения, рассматриваемого глазом, оказывается существенно
        больше, чем угловой размер объекта при наблюдении невооружённым глазом. Отношение углового размера изображения объекта, рассматриваемого наблюдателем
        через окуляр прибора, к угловому размеру объекта, рассматриваемого невооружённым глазом, называется угловым увеличением оптического прибора. При этом
        в случае микроскопа полагают, что при непосредственном наблюдении расстояние между объектом и глазом равно расстоянию наилучшего зрения глаза, т.е.
        25 см. В случае зрительной трубы всегда предполагается, что расстояние между объектом и наблюдателем значительно превышает фокусное расстояние
        объектива.

        \subsection{Увеличение астрономической зрительной трубы}
        \par Как было выяснено, при наблюдении далёких предметов с помощью астрономической зрительной трубы (трубы Кеплера) глазом, аккомодированным на
        бесконечность, задний фокус объектива совпадает с передним фокусом окуляра. В этом случае труба является афокальной системой: параллельный пучок
        лучей, входящий в объектив, остаётся параллельным по выходе из окуляра. Такой ход лучей называют телескопическим.
        \begin{figure}[H]
            \begin{center}
                \includegraphics[width=0.9\textwidth]{images/pic4.pdf}
                \caption{К расчёту увеличения зрительной трубы Кеплера.}
            \end{center}
        \end{figure}
        \par Рассматривая параллельный пучок лучей, исходящий из бесконечно удалённой точки, лежащей в стороне от оптической оси, можно для простоты
        ограничиться лучом, проходящим через центр объектива (рис. 4а). На выходе из окуляра угол наклона пучка к оптической оси изменяется.
        \par Пусть пучок света, попадающий в объектив, составляет с оптической осью угол $\phi_1$, а пучок, выходящий из окуляра, --- угол $\phi_2$.
        Увеличение $\gamma$ зрительной трубы по определению равно
        \begin{equation}
            \gamma=\frac{\tg{\phi_2}}{\tg{\phi_1}}.
        \end{equation}
        \par Строго говоря, $\phi_1$ --- это угловой размер объекта, рассматриваемого невооружённым глазом, но при наблюдении бесконечно удалённого объекта с
        помощью зрительной трубы угол $\phi_1$ для объектива трубы и для невооружённого глаза одинаков.
        \par Как следует из рис. 4а, угловое увеличение телескопа равно отношению фокусов объектива и окуляра:
        \begin{equation}
            \gamma=\frac{\tg{\phi_2}}{\tg{\phi_1}}=\frac{f_1}{f_2}.
        \end{equation}
        \par Отношение фокусных расстояний равно отношению диаметров пучка, прошедшего объектив и окуляр (рис. 4б). Ширина пучка, прошедшего объектив,
        определяется диаметром $D_1$ его оправы; ширина пучка, выходящего из окуляра, --- диаметром $D_2$ изображения оправы объектива, даваемого окуляром:
        \begin{equation}
            \frac{f_1}{f_2}=\frac{D_1}{D_2}.
        \end{equation}
        Таким образом, угловое увеличение телескопа
        \begin{equation}
            \gamma=\frac{\tg{\phi_2}}{\tg{\phi_1}}=\frac{f_1}{f_2}=\frac{D_1}{D_2}.
        \end{equation}
        \par В том случае, когда диаметр $D_2$ пучка, выходящего из окуляра, равен диаметру $d_0$ зрачка наблюдателя ($d_0\approx5$ мм), увеличение телескопа
        называется нормальным.
        \par Соотношение (4) показывает, что увеличение трубы можно определить следующими тремя способами: путём измерения углов, под которыми предмет виден
        через трубу и без неё, путём измерения диаметров объектива и его изображения в окуляре, и наконец, путём измерения фокусных расстояний объектива и
        окуляра. В настоящей работе используются все три способа.

        \subsection{Увеличение галилеевой зрительной трубы}
        \par Если заменить положительный окуляр астрономической трубы отрицательным, получается галилеева (или земная) труба. При телескопическом ходе лучей
        в галилеевой трубе расстояние между объективом и окуляром равно разности (точнее --- алгебраической сумме) их фокусных расстояний (рис. 5а), а
        изображение оправы объектива, даваемое окуляром, оказывается мнимым. Это изображение располагается между объективом и окуляром. Легко показать, что
        формула (4), полученная для астрономической трубы, справедлива и для земной трубы.
        \begin{figure}[H]
            \begin{center}
                \includegraphics[width=0.9\textwidth]{images/pic5.pdf}
                \caption{К расчёту увеличения галилеевой зрительной трубы.}
            \end{center}
        \end{figure}
        \par Достоинством галилеевой трубы является то, что она даёт прямое изображение. Поэтому зрительные трубы, бинокли и т.д. делаются по схеме Галилея.

        \subsection{Увеличение микроскопа}
        \par Рассмотрим ход лучей в микроскопе в предположении, что глаз наблюдателя аккомодирован на бесконечность (рис. 6). Тангенс угла $\phi_2$, под
        которым видно изображение, определяется соотношением
        \begin{equation}
            \tg{\phi_2}=\frac{l'}{f_2}=\frac{l(\Delta-f_1-f_2)}{f_1f_2},
        \end{equation}
        где $l'$ --- размер промежуточного изображения, $l$ --- размер предмета, $\delta$ --- длина тубуса (расстояние между линзами).
        \par При наблюдении предмета невооружённым глазом с расстояния наилучшего зрения $L$ угловой размер предмета $l$ равен
        \begin{equation}
            \tg{\phi_1}=\frac{l}{L}.
        \end{equation}
        Увеличение микроскопа, следовательно, равно
        \begin{equation}
            \gamma=\frac{\tg{\phi_2}}{\tg{\phi_1}}=\frac{L(\Delta-f_1-f_2)}{f_1f_2}.
        \end{equation}
        \begin{figure}[H]
            \begin{center}
                \includegraphics[width=0.9\textwidth]{images/pic6.pdf}
                \caption{К расчёту увеличения микроскопа.}
            \end{center}
        \end{figure}
        \par У всех микроскопов, выпускаемых отечественной промышленностью, длина тубуса равна $\delta=16$ см. Следует ещё раз подчеркнуть, что формулы для
        расчёта увеличения оптических приборов основаны на предположении об аккомодации глаза наблюдателя на бесконечность. В этом предположении увеличение
        является объективной характеристикой оптического инструмента. Если глаз наблюдателя изменяет аккомодацию, то оптический инструмент должен быть
        соответственно перефокусирован, и его увеличение несколько изменится. В связи с этим часто говорят о субъективном увеличении прибора. Впрочем, как
        правило, разница между субъективным и объективным увеличениями оптического инструмента оказывается незначительной.
        \par Можно показать, что при аккомодации глаза на расстояние наилучшего зрения $L$ угловое увеличение микроскопа $\gamma$ равно линейному $\Gamma$:
        $$
            \gamma=\frac{\tg{\phi_2}}{\tg{\phi_1}}=\frac{l''/L}{l/L}=\frac{l''}{l}=\Gamma,
        $$
        где $l''$ --- размер окончательного изображения.
        \par Можно оценить увеличение микроскопа как произведение увеличений объектива $\Gamma_{\text{об}}$ и окуляра $\Gamma_{\text{ок}}$:
        $$
            \gamma=\Gamma=\frac{l''}{l}=\frac{l'}{l}\frac{l''}{l'}=\Gamma_{\text{об}}\cdot\Gamma_{\text{ок}}.
        $$
        \par С учётом того, что объектив и окуляр микроскопа --- короткофокусные линзы (предмет и промежуточное изображение лежат практически в фокальных
        плоскостях объектива и окуляра, а $\Delta-f_2\approx\Delta$), при аккомодации глаза на расстояние наилучшего зрения увеличение микроскопа
        \begin{equation}
            \gamma=\Gamma=\Gamma_{\text{об}}\cdot\Gamma_{\text{ок}}\approx\frac{\Delta-f_2}{f_1}\cdot\frac{L}{f_2}\approx\frac{\Delta}{f_2}\cdot\frac{L}{f_2}.
        \end{equation}
        \par Зная увеличение объектива (в стандартных микроскопах оно обычно указано на оправе) и длину тубуса (16 см), можно оценить расстояние от объектива
        до плоскости, в которой следует располагать предмет. Обычно это 1–3 см.

    \section{Экспериментальная установка}
        
        \par Набор линз, осветитель, экран, зрительная труба, необходимые для моделирования оптических приборов, устанавливаются при помощи рейтеров на
        оптической скамье. Предметом служит миллиметровая сетка, нанесённая на матовое стекло осветителя.
        \par\textbf{Центрирование линз.} При юстировке любых оптических приборов важно правильно центрировать входящие в систему линзы. Проходя через плохо
        отцентрированную систему линз, лучи света отклоняются в сторону и могут вообще не доходить до глаза наблюдателя. Центрировать линзы следует как по
        высоте, так и в поперечном направлении (для чего линзы крепятся на поперечных салазках).
        \par\textbf{Юстировка коллиматора.} При составлении моделей телескопических систем необходимо иметь удалённый объект. В качестве такого объекта обычно
        используется бесконечно удалённое изображение предмета (шкалы осветителя), установленного в фокальной плоскости положительной линзы. Лучи, выходящие
        из одной точки предмета, пройдя через линзу, образуют параллельный пучок. Устройство такого рода называется коллиматором. Для юстировки коллиматора
        удобно использовать вспомогательную зрительную трубу, предварительно настроенную на бесконечность. Передвигая линзу коллиматора вдоль скамьи,
        добиваются появления резкого изображения предмета в окуляре зрительной трубы.
        \par\textbf{Измерение фокусных расстояний линз.} Для того чтобы сознательно моделировать оптические инструменты, нужно знать фокусные расстояния линз,
        которые могут быть использованы в качестве объектива или окуляра модели. Фокусные расстояния положительных линз проще всего найти с помощью
        вспомогательной зрительной трубы, установленной на бесконечность.
        \par При определении фокусного расстояния отрицательной линзы предметом служит изображение сетки, которое даёт вспомогательная положительная линза.

    \section{Результаты измерений и обработка данных}

        \subsection{Определение фокусных расстояний тонких линз с помощью зрительной трубы}
        
        \begin{wrapfigure}{r}{0.35\textwidth}
            \centering
            \includegraphics[width=0.35\textwidth]{images/pic7.pdf}
            \caption{Определение фокусного расстояния собирающей линзы.}
        \end{wrapfigure}
        \par Для определения фокусных расстояний линз с помощью зрительной трубы (рис. 7) настроим трубу на бесконечность.
        \par Поставим положительную линзу на расстоянии от предмета примерно равном фокусному. На небольшом расстоянии от линзы закрепим трубу, настроенную
        на бесконечность (рис. 7), и отцентрируем её по высоте.
        \par Передвигая линзу вдоль скамьи, получим в окуляре зрительной трубы изображение предмета --- миллиметровой сетки. При этом расстояние между
        предметом и серединой тонкой линзы (между проточками на оправах) равно фокусному. Затем повернёи линзу другой стороной к источнику и повторим измерения
        для определения тонкости линзы. Погрешность измерения растоняния составила 1 мм. Данные занесём в таблицу 1.
        \begin{table}[H]
            \captionsetup{font={small}, labelformat=fullparents, labelsep=fill, labelfont=bf, justification=raggedleft,
                        singlelinecheck=false, skip=-0.2cm}
            \caption{Фокусные расстояния собирающих линз.}
            \begin{center}
                \begin{tabular}{|l|l|}\hline$m$&$z$, дел
\\\hline0.0&275.0
\\\hline1.0&311.0
\\\hline2.0&348.0
\\\hline3.0&381.0
\\\hline4.0&419.0
\\\hline5.0&453.0
\\\hline6.0&489.0
\\\hline7.0&529.0
\\\hline-1.0&250.0
\\\hline-2.0&214.0
\\\hline-3.0&178.0
\\\hline-4.0&148.0
\\\hline-5.0&109.0
\\\hline-6.0&73.0
\\\hline-7.0&35.0
\\\hline\end{tabular}
            \end{center}
        \end{table}
        \par Тонкими можно счиатать первую и четвёртую линзы, так как наши измерения не смогли найти отличий при их развороте.
        
        \begin{wrapfigure}{r}{0.45\textwidth}
            \centering
            \includegraphics[width=0.45\textwidth]{images/pic8.pdf}
            \caption{Определение фокусного расстояния рассеивающей линзы.}
        \end{wrapfigure}
        \par Для определения фокусного расстояния тонкой отрицательной линзы сначала получим на экране увеличенное изображение сетки при помощи одной
        короткофокусной положительной линзы. Измерим расстояние между линзой и экраном $a_0=275\pm1$ мм.
        \par Разместим сразу за экраном трубу, настроенную на бесконечность, и закрепим её. Уберём экран и поставим на его место исследуемую рассеивающую
        линзу (рис. 8). Перемещая рассеивающую линзу, найдём в окуляре зрительной трубы резкое изображение сетки.
        \par Измерив расстояние между линзами $l=205\pm1$ мм, рассчитаем фокусное расстояние рассеивающей линзы $f_5=a_0-l=70\pm2$ мм.
        \par Повернув линзу другой стороной источника, получили такие же результаты. Значит линзу можно считать тонкой.

        \subsection{Труба Кеплера}

        \begin{wrapfigure}{r}{0.45\textwidth}
            \centering
            \includegraphics[width=0.45\textwidth]{images/pic9.pdf}
            \caption{Модель телескопа.}
        \end{wrapfigure}
        \par Из имеющегося набора возьмём две собирающих линзы 4 и 2 для создания модели зрительной трубы Кеплера с увеличением 2-3 (рис. 9). В качестве
        коллиматора используем линзу 3.
        \par Для последующих расчётов увеличения определим размер изображения $l_1=0.9\pm0.1$ дел. одного миллиметра шкалы осветителя в делениях окулярной
        шкалы зрительной трубы. Очевидно, $l_1=k\tg\phi_1$, где $k$ --- некоторый коэффициент, характеризующий увеличение зрительной трубы, $\phi_1$ ---
        угловой размер изображения миллиметрового деления шкалы осветителя, наблюдаемого через коллиматор.
        \par Линзу с максимальным фокусным расстоянием --- объектив модели --- расположим почти вплотную к линзе коллиматора, окуляр --- на расстоянии,
        примерно равном сумме фокусных расстояний обеих линз трубы.
        \par Рассчитаем увеличение исследуемой модели по формуле (4) через отношение фокусов. Получим $\gamma=f_4/f_2=2.60\pm0.03$.
        \par Для определения увеличения телескопа через тангенсы углов, под которыми объект виден через трубу и без неё, определим размер $l_2=2.4$ дел.
        изображения миллиметрового деления шкалы осветителя в делениях окулярной шкалы вспомогательной трубы: $l_2=k\tg\phi_2$. Здесь $\phi_2$ --— угловой
        размер изображения миллиметрового деления шкалы, наблюдаемой через исследуемую трубу.
        \par Сравнивая угловые размеры изображения с телескопом и без него, определим увеличение телескопа по формуле (1). Получим $\gamma=2.7\pm0.3$.
        \par Определим увеличение телескопа, измерив диаметр оправы его объектива и диаметр изображения этой оправы в окуляре. Для этого отодвинем
        вспомогательную трубу и расположим экран за окуляром телескопа. Отодвигая экран от окуляра, получите на нём чёткое изображение оправы объектива.
        Поднеся к объективу какой-нибудь предмет (например, край линейки), убедимся, что наблюдается именно изображение оправы объектива. Измерим диаметр
        объектива и диаметр его изображения. Получим $D_1=34\pm1$ мм, $D_2=13\pm1$ мм. Тогда из формулы (4) имеем $\gamma=2.6\pm0.2$. Все результаты
        согласуются в пределах погрешностей.

        \subsection{Труба Галилея}

        \par Переход от трубы Кеплера к трубе Галилея легко осуществить, если, не трогая коллиматора и объектива, вместо собирающей окулярной линзы поставить
        рассеивающую на расстоянии от объектива, равном разности фокусов объектива и окуляра.
        \par Проведём для трубы Галилея исследования, аналогичный предыдущему пункту (кроме измерения диаметров). При определении увеличения через фокусные
        расстояния получим $\gamma=f_4/f_5=4.04\pm0.08$. Измерим $l_1=0.9\pm0.1$ дел. и $l_2=3.5\pm0.1$ дел. и по формуле (1) получим $\gamma=3.9\pm0.4$.
        Результаты согласуются в пределах погрешностей.

        \subsection{Модель микроскопа}

        \par Для создания модели микроскопа с пятикратным увеличением используем самые короткофокусные линзы из набора 1 и 2. Рассчитаем необходимую длину
        тубуса $\Delta$ по формуле (7). Она составит $\Delta\approx360$ мм.

        \begin{wrapfigure}{r}{0.4\textwidth}
            \centering
            \includegraphics[width=0.4\textwidth]{images/pic10.pdf}
            \caption{Модель микроскопа.}
        \end{wrapfigure}
        \par Расположим объектив и окуляр на соответствующем расстоянии $\Delta$ друг от друга (рис. 10) и закрепем рейтеры. Сфокусируем модель микроскопа на
        сетку осветителя. Для этого будем перемещать осветитель вдоль оптической скамьи до тех пор, пока в окуляре микроскопа не появится отчётливое
        увеличенное изображение сетки.
        \par Расположим за окуляром модели микроскопа зрительную трубу, настроенную на бесконечность. Слегка перемещая осветитель, получим в поле зрения трубы
        изображение миллиметровой шкалы осветителя. Чёткость изображения повысится, если надеть на объектив микроскопа диафрагму диаметром 1 см и уменьшить
        яркость осветителя.
        \par Для экспериментального определения увеличения микроскопа измерим величину изображения $l_2=3.7\pm0.1$ дел. миллиметрового деления предметной шкалы в
        делениях окулярной шкалы зрительной трубы. Используя результат аналогичных измерений с коллиматорной линзой $l_1=0.9\pm0.1$, фокус которой известен,
        рассчитаем увеличение микроскопа по формуле
        \begin{equation}
            \gamma=\frac{l_2}{l_1}\cdot\frac{L}{f},
        \end{equation}
        где $L=25$ см --- расстояние наилучшего зрения нормального глаза. Получим $\gamma=5.3\pm0.6$. Результат согласуется с теоретическим в пределах
        погрешности.

    \section{Обсуждение результатов}

        \par Все рещультаты в этой работе хорошоо согласуются между собой. Но стоит отметить, что у многих результатов довольно большая погрешность (порядка
        10\%). Главным источником погрешностей была шкала зрительной трубы.

    \section{Вывод}

        \par Работа выполнена хорошо, можно улучшить точность измерений более хорошей шкалой зрительной трубы.

    \newpage\pagenumbering{gobble}

    \begin{figure}[H]
        \begin{center}
            \includegraphics[width=1.4\linewidth, angle=90]{data/approved.pdf}
        \end{center}
    \end{figure}

\end{document}

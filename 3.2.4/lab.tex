\documentclass[12pt,a4paper]{article}
\usepackage[T2A]{fontenc}
\usepackage[utf8]{inputenc}
\usepackage[english, russian]{babel}
\usepackage{indentfirst}
\usepackage{misccorr}
\usepackage{graphicx}
\usepackage{amsmath}
\usepackage{amssymb}
\usepackage{circuitikz}
\usepackage[font={small}]{caption}
\usepackage[left=20mm, top=20mm, right=20mm, bottom=20mm, nohead]{geometry}
\usepackage{float}
\usepackage{tabularx}
\usepackage{array}
\usepackage{longtable}
\usepackage{pstool}
\usepackage{pgfplots}
\usepackage{hhline}
\usepackage{multirow}
\usepackage{wrapfig}
\usepackage{pdfpages}

\DeclareCaptionLabelSeparator{fill}{.\\}
\DeclareCaptionLabelFormat{fullparents}{\bothIfFirst{#1}{~}#2}

\pgfplotsset{compat=1.17}

\begin{document}

	\begin{titlepage}
		\begin{center}
			{\LARGE Отчёт по лабораторной работе 3.2.4.\\}
			\vspace*{11cm}
				\textbf{\LARGE Свободные колебания в электрическом контуре.}	
			\vspace*{6.5cm}
		\end{center}
		\hfill\begin{minipage}{0.37\textwidth}
				Работу выполнил Громов Артём
				\\
				ЛФИ Б02-006
		\end{minipage}
		\vspace{4.8cm}
		\begin{center}
 			Долгопрудный, 2021 г.
		\end{center}   
	\end{titlepage}

    \section{Аннотация}

        \noindent\textbf{Цель работы:} исследование свободных колебаний в электрическом колебательном контуре.
        \\
        \noindent\textbf{В работе используются:} генератор импульсов, электронное реле, магазин сопротивлений, магазин ёмкостей,
        катушка индуктивности, электронный осциллограф, универсальный измерительный мост.
        \vspace{0.5 cm}
        
        \par Основное уравнение колебательного контура:
        \begin{equation}
            \ddot{I} + 2\gamma\dot{I} + \omega_0^2I = 0,
        \end{equation}
        \noindent где $ \gamma = \dfrac{R}{2L} $ --- коэффициент затухания, $ \omega_0^2 = \dfrac{1}{LC} $ --- собственная
        частота контура. Решением этого уравнения являются затухающие колебания:
        \begin{equation}
            I = A e^{-\gamma t} \cos (\omega t - \theta).
        \end{equation}
        \noindent Здесь $ \omega = \sqrt{\omega_0^2 - \gamma^2} $. Можно записать решение \eqref{ddot I} и для напряжения:
        \begin{equation}
            U_C = U_0 \dfrac{\omega_0}{\omega} e^{-\gamma t}\cos (\omega t - \theta)
        \end{equation}
        \par В контуре со слабым затуханием $ (\omega \backsimeq \omega_0) $ верна \textbf{формула Томпсона} для периода: 
        \begin{equation}
            T = \dfrac{2\pi}{\omega_0} \backsimeq  \dfrac{2\pi}{\omega} = 2\pi\sqrt{LC}
        \end{equation}
        \par Режим работы контура, при котором $ \gamma = \omega_0 $, называется \textbf{критическим}. Его сопротивление равно 
        \begin{equation}
            R_{кр} = 2\sqrt{\dfrac{L}{C}}
        \end{equation}
        \par Потери затухающих колебаний принято характеризовать через \textbf{добротность} и
        \textbf{логарифмический декремент затухания}: 
        \begin{equation}
            Q = 2\pi \dfrac{W}{\Delta W} = \dfrac{1}{R} \sqrt{\dfrac{L}{C}} \quad - \quad \text{Добротность, потери энергии}
        \end{equation}
        \begin{equation}
            \Theta = \dfrac{1}{n} \gamma T = \dfrac{1}{n} \ln \dfrac{U_k}{U_{k+n}}  \quad - \quad \text{Лог. декремент, потери амплитуды}
        \end{equation}
	
	\section{Экспериментальна установка}

        \par В работе исследуются свободные колебания, возбуждаемые в колебательном RLC-контуре. Конденсатор контура заряжается
        поступающими от специального генератора короткими одиночными импульсами, после каждого из которых в контуре возникают
        свободные затухающие колебания. По картине колебаний, наблюдаемой на экране электронного осциллографа, можно определить
        период свободных колебаний в контуре и коэффициент затухания, и вычислить параметры колебательного контура.
        Характеристики контура можно также определить, рассматривая затухающие колебания на фазовой плоскости системы на экране
        осциллографа.
        \begin{figure}[H]
			\begin{center}
				\includegraphics[width=\textwidth]{images/stand.pdf}
				\caption{Схема установки для исследования свободных колебаний.}
			\end{center}
		\end{figure}
        \par На рисунке 1 приведена схема используемой установки. Для периодического возбуждения колебаний в контуре используется
        генератор импульсов Г5-54. Импульсы с частотой $\nu_0$ = 100 Гц ($T_0$ = 0,01 с) заряжают конденсатор. После каждого
        импульса генератор отключается от колебательного контура, и в контуре возникают свободные затухающие колебания.
        Осциллограф не оказывает влияния на цепь, имея сопротивление 1 МОм.

    \section{Результаты измерений и обработка данных}

    \subsection{Измерение периодов свободных колебаний}
        \par Проведем измерения при $ R = 0 $. Будем изменять емкость от 0.02 до 0.90 мкФ, проводя измерения периода по формуле:
        \begin{equation}\label{}
            T_{\text{эксп}} = T_0 \dfrac{x}{nx_0},
        \end{equation}
        \noindent где $ T_0 = 0,01 $ c, $ x_0 $ --- расстояние одного импульса, $ x $ --- расстояние $ n $ импульсов. Погрешность
        $ \sigma_x = \sigma_{x_0} = 0,1, \sigma_{T_0} = 0,001  $c. Тогда 
        \begin{equation}\label{}
            \sigma_{T_\text{э}} = T_{\text{э}} \sqrt{ \left( \dfrac{ \sigma_x}{x} \right)^2 + \left( \dfrac{ \sigma_{x_0}}{x_0} 
            \right)^2  + \left( \dfrac{ \sigma_{T_0}}{T_0} \right)^2}
        \end{equation}
        \par При рассчёте $T_{\text{теор}}$ возьмём значение $L=393\pm2$ мГн.
        Результаты сведем в таблицу \ref{resT} и построим график рис. 2. 
        \par В результате аппроксиации получим, что $k = 1.1598 \pm 0.0012$

        \subsection{Критическое сопротивление и декремент затухания}

        \par Приняв $ L = 400 $ мГн, вычислим емкость по формуле $ \nu_0 = 1/2\pi\sqrt{LC}$, где $\nu_0 = 5$ кГц .Значит, $ C = 5  $ нФ. Тогда 
         
         \begin{equation}\label{}
         R_ {\text{кр}} = 2 \sqrt{\dfrac{L}{C}} \approx 17.6 \text{кОм}
         \end{equation}

    \par Получим на осцилографе ситуацию соответсвующую критическому режиму. Критический режим реалезовался при $R=17.6$ кОм.
        
        
    
    \begin{table}[H]
        \captionsetup{font={small}, labelformat=fullparents, labelsep=fill, labelfont=bf, justification=raggedleft,
                singlelinecheck=false, skip=-0.2cm}
        \caption{Результаты измерений}
        \begin{center}
            \begin{tabular}{|c|c|c|c|c|c|c|c|}
            \hline
            $C$, мкФ & $x_0$ & $n$ & $x$ & $T_{\text{эксп}}$, мс & $\sigma_{T_\text{э}}$, мс
            & $T_{\text{теор}}$, мс & $\sigma_{T_\text{т}}$, мс \\
            \hline
            0.02 & 6.0 & 2 & 0.7 & 0.58 & 0.2 & 0.56 & 0.01 \\
            \hline
            0.09 & 3.0 & 2 & 0.6 & 1.00 & 0.39 & 1.18 & 0.02 \\
            \hline
            0.15 & 6.0 & 4 & 3.5 & 1.46 & 0.17 & 1.52 & 0.02 \\
            \hline
            0.22 & 6.0 & 3 & 3.0 & 1.67 & 0.21 & 1.84 & 0.02 \\
            \hline
            0.35 & 6.0 & 3 & 3.3 & 1.83 & 0.22 & 2.32 & 0.03 \\
            \hline
            0.48 & 6.0 & 3 & 4.0 & 2.22 & 0.26 & 2.72 & 0.04 \\
            \hline
            0.59 & 6.5 & 3 & 4.5 & 2.31 & 0.26 & 3.02 & 0.04 \\
            \hline
            0.75 & 6.5 & 2 & 3.4 & 2.62 & 0.32 & 3.4 & 0.04 \\
            \hline
            0.87 & 6.5 & 2 & 4.3 & 3.31 & 0.38 & 3.66 & 0.05 \\
            \hline
            0.90 & 6.5 & 2 & 4.6 & 3.54 & 0.4 & 3.72 & 0.05 \\
            \hline
            \end{tabular}
        \end{center}    
    \label{resT}
    \end{table} 
    \begin{figure}[H]
        \begin{center}
            % This file was created with tikzplotlib v0.9.16.
\begin{tikzpicture}

\begin{axis}[
    height=13.5cm,
    tick align=inside,
    major tick length=0.2cm,
    minor tick length=0.1cm,
    tick pos=left,
    x grid style={white!69.0196078431373!black},
    xmin=0, xmax=2.1,
    xtick={0, 0.5, 1, 1.5, 2, 2.5},
    minor x tick num=4,
    xmajorgrids,
    minor x grid style={dotted,black},
    xminorgrids,
    xtick style={color=black},
    xlabel={$T_{\text{эксп}}$, \text{мс}},
    ymin = 0, ymax = 2.1,
    ytick = {0, 0.5, 1, 1.5, 2, 2.5},
    ytick style={color=black},
    minor y tick num=4,
    ymajorgrids,
    minor y grid style={dotted,black},
    yminorgrids,
    ytick style={color=black},
    ylabel={$T_{\text{теор}}$, \text{мс}}
]
\addplot [semithick, black]
table {%
0 0
2 1.98554049274599
};
\path [draw=red, semithick]
(axis cs:0.2,0.4)
--(axis cs:0.6,0.4);

\path [draw=red, semithick]
(axis cs:0.4,0.7)
--(axis cs:0.8,0.7);

\path [draw=red, semithick]
(axis cs:0.76,0.9)
--(axis cs:0.96,0.9);

\path [draw=red, semithick]
(axis cs:0.95,1.2)
--(axis cs:1.55,1.2);

\path [draw=red, semithick]
(axis cs:0.9,1.3)
--(axis cs:1.9,1.3);

\path [draw=red, semithick]
(axis cs:1.1,1.6)
--(axis cs:2.1,1.6);

\path [draw=red, semithick]
(axis cs:1.4,1.9)
--(axis cs:2.4,1.9);

\path [draw=red, semithick]
(axis cs:1.4,2)
--(axis cs:2.6,2);

\path [draw=red, semithick]
(axis cs:0.4,0.398)
--(axis cs:0.4,0.402);

\path [draw=red, semithick]
(axis cs:0.6,0.698)
--(axis cs:0.6,0.702);

\path [draw=red, semithick]
(axis cs:0.86,0.898)
--(axis cs:0.86,0.902);

\path [draw=red, semithick]
(axis cs:1.25,1.198)
--(axis cs:1.25,1.202);

\path [draw=red, semithick]
(axis cs:1.4,1.298)
--(axis cs:1.4,1.302);

\path [draw=red, semithick]
(axis cs:1.6,1.598)
--(axis cs:1.6,1.602);

\path [draw=red, semithick]
(axis cs:1.9,1.898)
--(axis cs:1.9,1.902);

\path [draw=red, semithick]
(axis cs:2,1.998)
--(axis cs:2,2.002);

\addplot [semithick, red, mark=*, mark size=3, mark options={solid}, only marks]
table {%
0.4 0.4
0.6 0.7
0.86 0.9
1.25 1.2
1.4 1.3
1.6 1.6
1.9 1.9
2 2
};
\end{axis}

\end{tikzpicture}

            \caption{Зависимость $T_{\text{эксп}}$ от $T_{\text{теор}}$.}
        \end{center}
    \end{figure}

    Установим $ C $ на магазине емкостей, будем наблюдать картину затухающих колебаний, изменяя $ R $ от $ 0.1 R_{\text{кр}}$ до $ 0.3 R_{\text{кр}} $.
    \begin{table}[H]
        \captionsetup{font={small}, labelformat=fullparents, labelsep=fill, labelfont=bf, justification=raggedleft,
                singlelinecheck=false, skip=-0.2cm}
        \caption{Результаты измерений}
        \begin{center}
        
        \begin{tabular}{|c|c|c|c|c|c|c|c|}
            \hline
            $ R $, Ом & $ n $ & $ U_k$ & $ U_{k+n} $& $ \Theta $& $ \sigma_\Theta $ & $ R_\text{к} $, Ом & $ \sigma_{R_\text{к}} $, Ом \\
            \hline
            1100 & 3 & 3.0 & 0.9 & 0.40 & 0.05 & 1142 & 2 \\
            \hline
            1400 & 3 & 3.0 & 0.6 & 0.54 & 0.09 & 1442 & 3 \\
            \hline
            1800 & 2 & 2.9 & 0.8 & 0.64 & 0.08 & 1842 & 4 \\
            \hline
            2200 & 2 & 2.8 & 0.6 & 0.77 & 0.13 & 2242 & 4 \\
            \hline
            2500 & 1 & 2.7 & 1.0 & 0.99 & 0.11 & 2542 & 5 \\
            \hline
            2700 & 1 & 2.6 & 0.9 & 1.06 & 0.12 & 2742 & 5 \\
            \hline
            3300 & 1 & 2.5 & 0.8 & 1.14 & 0.15 & 3342 & 7 \\
            \hline
            3600 & 1 & 2.5 & 0.7 & 1.27 & 0.19 & 3642 & 7 \\
            \hline
        \end{tabular}% 
        \label{resR}% 
    \end{center}
    \end{table}% 

    Теперь, изменяя сопротивление от примерно $ 0.1 R_{\text{кр}} $ до $ 0,3 R_{\text{кр}} $, будем измерять амплитуды колебаний, разделенных на $ n $ частей. Погрешности амплитуд $  \sigma_{U_k} = \sigma_{U_{k+n}} = 0,1 $, т.е. 

\begin{equation}
\sigma_{\Theta} = \Theta \sqrt{ \left( \dfrac{ \sigma_{U_k}}{U_k} \right)^2 + \left( \dfrac{ \sigma_{U_{k+n}} }{U_{k+n}} \right)^2 }
\end{equation}

Измерив на универсальном мосте сопротивление катушки при нашей частоте 5 кГц, добавим его к сопротивлению магазина, получив сопротивление контура $R_\text{к}$. Результаты сведем в таблицу 2. Теперь построим график $ \dfrac{1}{\Theta^2} $ от $ \dfrac{1}{R^2} $, считая погрешность $ \sigma_{\frac{1}{\Theta^2}} = 2 \dfrac{1}{\Theta^2} \dfrac{\sigma_\Theta}{\Theta}$ Данные для графика рис. 2 сведены в таблице 3.

\begin{table}[H]
	\captionsetup{font={small}, labelformat=fullparents, labelsep=fill, labelfont=bf, justification=raggedleft,
                singlelinecheck=false, skip=-0.2cm}
        \caption{Результаты измерений}
        \begin{center}
	\begin{tabular}{|c|c|c|}
		\hline
	$ \dfrac{1}{\Theta^2} $& $ \sigma_{\frac{1}{\Theta^2}} $ & $ \dfrac{1}{R^2}, 10^{-6} \text{ Ом}^{-2}$  \\
		\hline
		6.21 & 0.75 & 0.77 \\
        \hline
		3.47 & 0.68 & 0.48 \\
        \hline
		2.41 & 0.63 & 0.29 \\
        \hline
		1.69 & 0.57 & 0.2 \\
        \hline
		1.01 & 0.22 & 0.15 \\
        \hline
		0.89 & 0.21 & 0.13 \\
        \hline
		0.77 & 0.2 & 0.09 \\
        \hline
		0.62 & 0.18 & 0.08 \\
		\hline
	\end{tabular}% 
	\label{res}%
\end{center} 
\end{table}% 
\par После аппроксимации получим следущее значение $k=7.86 \pm 0.03 \text{ Ом}^6$. Откуда $R_{\text{к}} = 17.62\pm0.07$ кОм.
\begin{table}[H]
	\captionsetup{font={small}, labelformat=fullparents, labelsep=fill, labelfont=bf, justification=raggedleft,
                singlelinecheck=false, skip=-0.2cm}
        \caption{Результаты исследования $R_{\text{кр}}$}
        \begin{center}
	\begin{tabular}{|c|c|c|}
		\hline
        $R_{\text{теор}}$, кОм & $R_{\text{подбор}}$, кОм & $R_{\text{эксп}}$, кОм \\
		\hline
        17.6 & 17.6 & $17.62\pm0.07$ \\
        \hline
	\end{tabular}% 
\end{center} 
\end{table}% 

\begin{figure}[H]
    \begin{center}
        % This file was created with tikzplotlib v0.9.16.
\begin{tikzpicture}

\begin{axis}[
    height=11cm,
    tick align=inside,
    major tick length=0.2cm,
    minor tick length=0.1cm,
    tick pos=left,
    xmin=-4.1, xmax=4.1,
    xtick={-5, -4, -3, -2, -1, 0, 1, 2, 3, 4, 5},
    minor x tick num=4,
    xmajorgrids,
    minor x grid style={dotted,black},
    xminorgrids,
    xtick style={color=black},
    xlabel={$m$},
    ymin=-1.2, ymax=1.2,
    ytick={-1.5, -1, -0.5, 0, 0.5, 1, 1.5},
    minor y tick num=4,
    ymajorgrids,
    minor y grid style={dotted,black},
    yminorgrids,
    ytick style={color=black},
    ylabel={$x_m$, мм}
]
\path [draw=red, semithick]
(axis cs:1,0.14)
--(axis cs:1,0.18);

\path [draw=red, semithick]
(axis cs:2,0.48)
--(axis cs:2,0.52);

\path [draw=red, semithick]
(axis cs:3,0.7)
--(axis cs:3,0.74);

\path [draw=red, semithick]
(axis cs:4,1.02)
--(axis cs:4,1.06);

\path [draw=red, semithick]
(axis cs:-1,-0.18)
--(axis cs:-1,-0.14);

\path [draw=red, semithick]
(axis cs:-2,-0.5)
--(axis cs:-2,-0.46);

\path [draw=red, semithick]
(axis cs:-3,-0.74)
--(axis cs:-3,-0.7);

\path [draw=red, semithick]
(axis cs:-4,-1.08)
--(axis cs:-4,-1.04);

\addplot [semithick, black]
table {%
1 0.249999999998364
2 0.499999999996728
3 0.749999999995093
4 0.999999999993457
-1 -0.249999999998364
-2 -0.499999999996728
-3 -0.749999999995093
-4 -0.999999999993457
};
\addplot [semithick, red, mark=square*, mark size=3, mark options={solid}, only marks]
table {%
1 0.16
2 0.5
3 0.72
4 1.04
-1 -0.16
-2 -0.48
-3 -0.72
-4 -1.06
};
\end{axis}

\end{tikzpicture}

        \caption{Зависимость $ \dfrac{1}{\Theta^2} $ от $ \dfrac{1}{R^2} $.}
    \end{center}
\end{figure}
\subsection{Исследование добротности}

Рассчитаем добротность системы тремя способами. Сначала воспользуемся формулой (6) для двух значений сопротивления.
А затем для них же воспользуемся формулой (7), измерив на спирали расстояния между крайними витками. Третий способ заключается в применении (6) и значений
$\Theta$, полученных из графика на рисунке 3.
\begin{table}[H]
	\captionsetup{font={small}, labelformat=fullparents, labelsep=fill, labelfont=bf, justification=raggedleft,
                singlelinecheck=false, skip=-0.2cm}
        \caption{Результаты расчёта добротности}
        \begin{center}
	\begin{tabular}{|c|c|c|c|}
		\hline
        $R$, Ом & $Q_{\text{теор}}$ & $Q_{\text{спир}}$ & $Q_{\text{граф}}$ \\
		\hline
        1000 & $7.7\pm0.2$ & $7.8\pm0.5$ &$8.8\pm0.2$ \\
        \hline
        3000 & $2.4\pm0.1$ & $2.5\pm0.4$ &$2.9\pm0.1$\\
        \hline
	\end{tabular}% 
\end{center} 
\end{table}% 

\section{Вывод}

\par При исследовании периодов свободных колебаний и критического сопротивления результаты согласуются с теорией. Но при изучении
добротности результаты расходятся.

\end{document}
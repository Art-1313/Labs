\documentclass[12pt,a4paper]{article}
\usepackage[T2A]{fontenc}
\usepackage[utf8]{inputenc}
\usepackage[english, russian]{babel}
\usepackage{indentfirst}
\usepackage{misccorr}
\usepackage{graphicx}
\usepackage{amsmath}
\usepackage{amssymb}
\usepackage{circuitikz}
\usepackage[font={small}]{caption}
\usepackage[left=20mm, top=20mm, right=20mm, bottom=20mm, nohead]{geometry}
\usepackage{float}
\usepackage{tabularx}
\usepackage{array}
\usepackage{longtable}
\usepackage{pstool}
\usepackage{pgfplots}
\usepackage{hhline}
\usepackage{multirow}
\usepackage{wrapfig}
\usepackage{pdfpages}

\DeclareCaptionLabelSeparator{fill}{.\\}
\DeclareCaptionLabelFormat{fullparents}{\bothIfFirst{#1}{~}#2}

\pgfplotsset{compat=1.17}

\begin{document}

	\begin{titlepage}
		\begin{center}
			{\LARGE Отчёт по лабораторной работе 3.1.3.\\}
			\vspace*{11cm}
				\textbf{\LARGE Измерение магнитного поля Земли.}	
			\vspace*{6.5cm}
		\end{center}
		\hfill\begin{minipage}{0.37\textwidth}
				Работу выполнил Громов Артём
                \\
				ЛФИ Б02-006
		\end{minipage}
		\vspace{4.8cm}
		\begin{center}
 			Долгопрудный, 2021 г.
		\end{center}   
	\end{titlepage}

	\section{Аннотация}

		\begin{flushleft}
			\textbf{Цель работы:} определить характеристики шарообразных неодимовых магнитов и, используя законы
			взаимодействия магнитных моментов с полем, измерить горизонтальную и вертикальную составляющие индукции
			магнитного поля Земли и магнитное наклонение.
		\end{flushleft}
		\begin{flushleft}
			\textbf{В работе используется:} 12 одинаковых неодимовых магнитных шариков, тонкая нить для изготовления
			крутильного маятника, медная проволока диаметром (0,5 - 0,6) мм, электронные весы, секундомер, измеритель
			магнитной индукции АТЕ-8702, штангенциркуль, брусок из немагнитного материала
			($25 \times 30 \times 60 \, \text{мм}^3$), деревянная линейка, штатив из немагнитного материала;
			дополнительные неодимовые магнитные шарики ($ \sim 20$ шт.) и неодимовые магниты в форме параллелепипедов
			(2 шт.), набор гирь и разновесов.
		\end{flushleft}
		\par\textbf{Точечный магнитный диполь} Простейший магнитный диполь может быть образован витком с током или
		постоянным магнитом. По определению, магнитный момент $P_m$ тонкого витка площадью $S$ с током $I$ равен:
		\begin{equation}
 			\vec{P}_{m}=(I / c) \vec{S}=(I / c) S \vec{n},
		\end{equation}
		где $c$ --- скорость света в вакууме, $S = S \vec{n}$ --- вектор площади контура, образующий с направлением
		тока правовинтовую систему,  $\vec{n}$ --- единичный вектор нормали к площадке $S$.
		\par Магнитное поле точечного диполя определяется по формуле, аналогичной формуле для поля элементарного
		электрического диполя:
		\begin{equation}
			\vec{B}=3\left(\vec{P}_{m} \vec{r}\right) \vec{r} / r^{5}-\vec{P}_{m} / r^{3}
		\end{equation}
		\par В магнитном поле с индукцией $\vec{B}$ на точечный магнитный диполь $\vec{P}_{m}$ действует механический
		момент сил:
		\begin{equation}
			\vec{M}=\vec{P}_{m} \times \vec{B}
		\end{equation}
		\par Магнитный диполь в магнитном поле обладает энергией:
		\begin{equation}
			W=-\left(\vec{P}_{m}, \vec{B}\right)
		\end{equation}
		\par В неоднородном поле на точечный магнитный диполь, кроме момента сил, действует ещё и сила:
		\begin{equation}
			\vec{F}=\left(\vec{P}_{m}, \vec{\nabla}\right) \vec{B}
		\end{equation}
		\par Если магнитные моменты $P_1 = P_2 =P_m$  двух одинаковых небольших магнитов направлены вдоль соединяющей
		их прямой, а расстояние между ними равно $r$, то магниты взаимодействуют с силой:
		\begin{equation}
			F=P_{m} \partial B / \partial r=P_{m} \partial\left(2 P_{m} / r^{3}\right) / \partial r=-6 P_{m}^{2} / r^{4}
		\end{equation}
		\par\textbf{Неодимовые магнитные шары} Для однородно намагниченного шара намагниченность равна:
		\begin{equation}
			\vec{p}_{m}=\vec{P}_{m} / V,
		\end{equation}
		где $V$ --- объём шара.
		\par Индукция магнитного поля $B_p$ на полюсах однородно намагниченного шара связана с величиной
		намагниченности $p_m$ и остаточной магнитной индукцией $B_r$ формулой:
		\begin{equation}
			\vec{B}_{p}=(8 \pi / 3) \vec{p}_{m}=(2 / 3) \vec{B}_{r}
		\end{equation}

	\section{Экспериментальная установка}

		\subsection{Определение величины магнитного момента шариков} 
		\noindent\textbf{Метод A.}

		\begin{wrapfigure}{r}{0.4\textwidth}
			\center{\includegraphics[width=0.9\linewidth]{ris1}}
			\caption{Метод A.}
		\end{wrapfigure}
		\par\normalsize Величину магнитного момента $P_m$ одинаковых шариков можно рассчитать, зная их массу $m$ и определив
		максимальное расстояние $r_{max}$, на котором они ещё удерживают друг друга в поле тяжести (см. рис. 1).
		При максимальном расстоянии сила тяжести шариков равна силе их магнитного притяжения:
		\begin{equation}
			6 P_{m}^{2} / r_{\max }^{4}=m g, \quad P_{m}=\sqrt{\frac{m g r_{\max }^{4}}{6}}
		\end{equation}
		\par По величине магнитного момента $P_m$ можно рассчитать величину индукции магнитного поля вблизи любой
		точки на поверхности шара радиуса $R$. Максимальная величина индукции наблюдаются на полюсах:
		\begin{equation}
			\vec{B}_{p}=2 \vec{P}_{m} / R^{3}
		\end{equation}
		\textbf{Метод B}.

		\begin{wrapfigure}{r}{0.25\textwidth}
			\center{\includegraphics[width=0.9\linewidth]{ris2}}
			\caption{Метод B.}
		\end{wrapfigure}
		\par\normalsize Величину магнитного момента шариков можно определить также по силе их сцепления. Она определяется как
		сила, необходимая для разрыва двух сцепившихся магнитных шариков. Сила сцепления максимальна, если шары
		соединяются своими противоположными полюсами.
		\par Максимальную силу сцепления можно определить по весу магнитной цепочки, которую способен удержать
		самый верхний магнитный шарик. Если цепь состоит из одинаковых магнитных шариков (см. рис. 2, а), то при
		определённой длине она отрывается от верхнего шарика. При этом, учитывая, что сила притяжения убывает как
		$r$ в четвёртой степени ($r$— расстояния между центрами шаров) $F \sim 1 / r^{4}$, для расчёта прочности
		цепочки достаточно учитывать силу взаимодействия верхнего шара с тремя-четырьмя ближайшими соседями.
		\par Если сила сцепления двух одинаковых шаров диаметром $d$ c магнитными моментами $P_m$ равна:
		\begin{equation}
			F_{0}=6 P_{m}^{2} / d^{4},
		\end{equation}
		то минимальный вес цепочки, при которой она оторвётся от верхнего шарика равен:
		\begin{equation}
			F=6 P_{m}^{2} / d^{4}+6 P_{m}^{2} /(2 d)^{4}+\ldots=
			F_{0}\left(1+1 / 2^{4}+1 / 3^{4}+1 / 4^{4}+\ldots\right) \approx 1,08 F_{0}.
		\end{equation}
		\par Таким образом, сила сцепления двух шаров равна:
		\begin{equation}
			F_{0}=F / 1.08.
		\end{equation}
		
		\subsection{Измерение горизонтальной составляющей индукции магнитного поля Земли} 
		\par "Магнитная стрелка" образована из сцепленных друг с другом противоположными полюсами шариков и с помощью
		$\Lambda$-образного подвеса подвешена в горизонтальном положении (см. рис. 3). Магнитные моменты шариков
		направлены в одну сторону вдоль оси "стрелки". Под действием вращательного момента
		$\vec{M}=\vec{P} \times \vec{B}$магнитный момент "стрелки" $\vec{P}$ выстроится вдоль горизонтальной
		составляющей магнитного поля g Земли $\vec{B}_n$ в направлении Юг $\rightarrow$ Север. При отклонении
		"стрелки" на угол $\theta$ от равновесного положения в горизонтальной плоскости возникают крутильные
		колебания вокруг вертикальной оси, проходящей через середину стрелки. Если пренебречь упругостью нити,
		то уравнение крутильных колебаний такого маятника определяется возвращающим моментом сил
		$M=-P_{0} B_{h} \sin \theta$, действующим на "стрелку" со стороны магнитного поля Земли, и моментом
		инерции $I_n$ "стрелки" относительно оси вращения.

		\begin{wrapfigure}{r}{0.27\textwidth}
			\center{\includegraphics[width=0.9\linewidth]{ris3}}
			\caption{Крутильный маятник.}
		\end{wrapfigure}
		\par При малых амплитудах ($\sin \theta \approx \theta$) уравнение колебаний "стрелки" имеет вид:
		\begin{equation}
			I_{n} d^{2} \theta / d t^{2}=-P_{0} B_{h} \theta, \quad 
			\text{ или } \quad I_{n} \ddot{\theta}+P_{0} B_{h} \theta=0,
		\end{equation}
		где $P_0 = n P_m$ --- полный магнитный момент магнитной "стрелки", составленной из $n$ шариков.
		\par Момент инерции «стрелки», состоящей из $n$ шариков с хорошей точностью равен моменту инерции тонкого
		однородного стержня массой $m_{\text{ст}}=n m$  и длиной $\ell_{\text{ст}}=n d$:
		\begin{equation}
			I_{n}=(1 / 12) m_{\text{ст}} l_{\text{ст}}^{2}=(1 / 12) n m(n d)^{2}=(1 / 12) n^{3} m d^{2}
		\end{equation}
		\par Даже для трёх шариков момент инерции, рассчитанный по приближённой формуле, отличается от точного
		результата примерно на $2 \%$, а для $n \geq 5$ — различие не превышает процента; если же учесть, что
		$T \sim \sqrt{I_{n}},$, то для всех $n \geq 3$ погрешность наших расчетов для периода колебаний $T$ не
		превысит процента, что освобождает нас от необходимости вводить поправочные коэффициенты.
		\par Таким образом, в нашем приближении период колебаний маятника оказывается пропорциональным числу шаров
		$n$, составляющих «стрелку»:
		\begin{equation}
			T(n)=2 \pi \sqrt{I_{n} / n P_{m} B_{h}}=
			\pi n \sqrt{m d^{2} / 3 P_{m} B_{h}}=k n,
		\end{equation}
		где $k=\pi \sqrt{m d^{2} / 3 P_{m} B_{h}}$
		\par При выводе этой формулы предполагалось, что магнитный момент --- величина аддитивная: полный магнитный
		момент системы магнитов («стрелки») равен векторной сумме магнитных моментов шариков, составляющих «стрелку».
		Экспериментальное подтверждение этой зависимости ($T \sim n$) будет являться косвенным доказательством наших
		предположений о магнитожёсткости материала магнитов и, соответственно, свойства аддитивности магнитных моментов
		шаров.
		\subsection{Измерение вертикальной составляющей индукции магнитного поля Земли. Магнитное наклонение.}
		\par Для измерения вертикальной $B_v$ составляющей вектора индукции поля Земли используется та же установка,
		что и для измерения горизонтальной составляющей с тем лишь отличием, что магнитная «стрелка» подвешивается
		на нити без $\Lambda$-образного подвеса. В этом случае магнитная «стрелка», составленная из чётного числа
		шариков и подвешенная

		\begin{wrapfigure}{r}{0.4\textwidth}
			\center{\includegraphics[width=0.9\linewidth]{ris4}}
			\caption{Определение вертикальной составляющей поля Земли.}
		\end{wrapfigure}
		\noindent на тонкой нити за середину, расположится не горизонтально, а под некоторым углом к
		горизонту (см. рис. 4, а). Это связано с тем, что вектор $\vec{B}$ индукции магнитного поля
		Земли в общем случае не горизонтален, а образует с горизонтом угол $\beta$, зависящим от географической
		широты $\phi$ места, где проводится опыт. Величина угла  $\beta$ называется магнитным наклонением.
		\par С помощью небольшого дополнительного грузика «стрелку» можно «выровнять», расположив её горизонтально
		(см. рис. 4, б): в этом случае момент силы тяжести груза относительно точки подвеса будет равен моменту
		сил, действующих на «стрелку» со стороны магнитного поля Земли. Если масса уравновешивающего груза равна
		$m_{\text{гр}}$ плечо силы тяжести $r_{\text{гр}}$, а полный магнитный момент «стрелки» $P_0 = n P_m$, то в
		равновесии:
		\begin{equation}
			m_{\Gamma p} g r_{\Gamma p}=P_{0} B_{v}=n P_{m} B_{v},
		\end{equation}
		где $B_v$ --- вертикальная составляющая поля Земли. Видно, что момент $M(n)$ силы тяжести уравновешивающего
		груза пропорционален числу $n$ шариков, образующих магнитную «стрелку»:
		\begin{equation}
			M(n)=A n \text {, где } A=P_{m} B_{v}
		\end{equation}
	
	\section{Результаты измерений и обработка данных}
	
		\subsection{Определение магнитного момента, намагниченности и остаточной магнитной индукции вещества шариков}
		\noindent\textbf{Метод A}
		\par Взвесим шарики на весах. Чтобы магнитное поле не повлияло, добавим между шариками и весами толстый слой
		бумаги. Чтобы убедиться, что магнитное поле не влияет, добавим еще бумаги, и увидим, что значение на весах не
		меняется. $M_\text{ш}$ = 39.43 г, $\Delta M_\text{ш} = 0.01 $ г, $L = 27.5$ см, $\Delta L = 0.2$ см, $N = 47$,
		$m_\text{ш} = \dfrac{M_\text{ш}}{N} = (0.839\pm0.001)$ г, $d=(0.585\pm0.004)$ см.
		\par Измеряя толщину прослойки и учтя диаметр шариков, получим $r_{max} = (2.49\pm0.01)$ см. Отсюда
		$$P_m = \sqrt{\dfrac{m_\text{ш} g r_{max}^4}{6}} = (72\pm2) \text{ эрг/Гс}$$
		\par Рассчитаем намагниченность материала шариков:
		$$p_m = P_m/V =3 P_m/(\pi d^3) = (346\pm11) \text{эрг/(Гс}\cdot \text{см}^3)$$
		\par Рассчитаем значение магнитного поля на полюсе:
		$$B_{p}=(8 \pi / 3) {p}_{m}=(2.9\pm0.1)\text{ кГс}$$
		\par Теперь измерим $B_p$ с помощью магнетометра. Получаем $B_p = (3.00\pm0.16)$ кГс. Результаты совпадают в
		пределах погрешности.
		\par Рассчитаем величину остаточной магнитной индукции материала, из которого изготовлен магнитный шарик.
		$$B_r = 4 \pi p_m = (4.35\pm0.15)\text{ кГс}$$
		\par Сравним наш результат с табличными значениями $B_r$ для соединения неодим-железо-бор. В справочнике
		оно лежит в пределах (9.4 - 14.0) кГс. Мы получили значение меньше, чем справочное.
		Это можно обьяснить постепенным размагничниванием, которое вызвано механическими воздействиями на шарики.
		\newline\newline
		\noindent\textbf{Метод B}
		\par Составим цепочку, в которой сначала идут все 47 шариков, потом неодимовые магниты в форме параллелепипедов
		подсоедините цепочку к гире и разновесам, так, чтобы общая масса системы составила ~ 500 г (рис. 2. б).
		Добавляя или удаляя шарики (шарики можно примагничивать непосредственно к гире), подберём минимальный вес
		$F$ системы цепочки с гирей, при котором она отрывается от верхнего шарика.
		\par Взвесим всю цепочку. Получаем $F = (273\pm5)$ кдин. Сила сцепления двух шаров: $F_0 = F/1.08 = (252\pm5)$
		кдин. Тогда из формулы $F_{0}=6 P_{m}^{2} / d^{4}$ получаем:
		$$P_m = \sqrt{F_0 d^4/6} = (70\pm3)\text{ эрг/Гс}.$$
		\noindent Здесь учтено, что погрешность поправочного коэффициента 1.08 не больше 1 процента.
		\par Поле на полюсах $B_{p}=(8 \pi / 3) {p}_{m}=(2.8\pm0.2)$ кГс, совпадает с измеренным магнитометром в
		пределах погрешности. Метод А дает более точный результат, т.к. его погрешность меньше

		\subsection{Определение горизонтальной составляющей магнитного поля Земли}
		\par Соберем крутильный маятник и, используя $\Lambda$-образный подвес, установим «магнитную стрелку» из 12
		магнитных шариков в горизонтальном положении (юстировка системы).
		\par Покажем, что упругость нити можно не учитывать. Измерим период колебаний кольца. $T \approx 60$ c.
		Введем эффективный коэффициент упругости нити$\chi$: $I \ddot{\phi} + \chi \phi = 0$. Момент инерции кольца
		можно оценить как 
		$$I=\frac{12 m R^{2}}{2}=6.3 \,\text{г} \cdot \text{см}/\text{с}^2.$$
		\noindent Так как $T = 2\pi \sqrt{\chi/I}$, то $\chi_{\text{нити}} = (2 \pi /T)^2 I \approx
		6 \cdot 10^{-2}  \, \text{г}\cdot \text{см}/\text{с}^4$ 
		\par Найдем диапазон $\chi$ у магнитной стрелы. $T_{max} = 3.21$ c, $T_{min} = 0.87$ c,
		$I_{max} = 41 \,\text{г} \cdot \text{см}/\text{с}^2$,  $I_{min} = 0.7 \,\text{г} \cdot \text{см}/\text{с}^2$
		Получаем $\chi_{\text{стрелы}} \in [33, 158] \, \text{г}\cdot \text{см}/\text{с}^4$, что много больше,
		чем $\chi_{\text{нити}}$, поэтому влияние нити можем не учитывать.
		\par Исследуем зависимость периода $T$ крутильных колебаний стрелки от количества магнитных шариков. Будем
		отсчитывать 20 колебаний, и записывать в таблицу только период. Время реакции человека $T_{\text{реак}} = 1$ c.
		$\Delta T = T_{\text{реак}}/20 = 0.05$ c. Данные отображены на таблице 1.
		\begin{table}[H]
			\captionsetup{font={small}, labelformat=fullparents, labelsep=fill, labelfont=bf, justification=raggedleft,
			singlelinecheck=false, skip=-0.2cm}
			\caption{Исследование зависимости $T$ от количества шариков}
			\begin{center}
				\begin{tabular}{|l|l|l|l|l|l|l|l|l|l|l|l|}
					\hline
					T, c & 3.21 & 2.89 & 2.55 & 2.38 & 1.9 & 2.13 & 0.87 & 1.6 & 1.4 & 1.16 \\
					\hline
					n & 12  & 11 & 10& 9 & 7 & 8 & 3 & 6 & 5 & 4 \\
					\hline
				\end{tabular}
			\end{center}
		\end{table}
		\par Построим график $T(n)$ (рис. 8).
		\begin{figure}[H]
			\center{% This file was created with tikzplotlib v0.9.16.
\begin{tikzpicture}

\begin{axis}[
legend cell align={left},
legend style={
  fill opacity=0.8,
  draw opacity=1,
  text opacity=1,
  at={(0.03,0.97)},
  anchor=north west,
  draw=white!80!black
},
height=11.2cm,
tick align=inside,
major tick length=0.2cm,
minor tick length=0.1cm,
tick pos=left,
xmin=0, xmax=600,
xtick={0, 50, 100, 150, 200, 250, 300, 350, 400, 450, 500, 550, 600},
minor x tick num=4,
xmajorgrids,
minor x grid style={dotted,black},
xminorgrids,
xtick style={color=black},
xlabel={$D_{\text{винта}}$, мкм},
ymin=0, ymax=500,
ytick={0, 50, 100, 150, 200, 250, 300, 350, 400, 450, 500},
ytick style={color=black},
minor y tick num=4,
ymajorgrids,
minor y grid style={dotted,black},
yminorgrids,
ytick style={color=black},
ylabel={$D_{\text{расчёт}}$, мкм}
]
\path [draw=red, semithick]
(axis cs:40,32.9861111111111)
--(axis cs:60,32.9861111111111);

\path [draw=red, semithick]
(axis cs:90,98.9583333333333)
--(axis cs:110,98.9583333333333);

\path [draw=red, semithick]
(axis cs:140,131.944444444444)
--(axis cs:160,131.944444444444);

\path [draw=red, semithick]
(axis cs:190,164.930555555556)
--(axis cs:210,164.930555555556);

\path [draw=red, semithick]
(axis cs:240,197.916666666667)
--(axis cs:260,197.916666666667);

\path [draw=red, semithick]
(axis cs:290,230.902777777778)
--(axis cs:310,230.902777777778);

\path [draw=red, semithick]
(axis cs:340,263.888888888889)
--(axis cs:360,263.888888888889);

\path [draw=red, semithick]
(axis cs:390,296.875)
--(axis cs:410,296.875);

\path [draw=red, semithick]
(axis cs:440,362.847222222222)
--(axis cs:460,362.847222222222);

\path [draw=red, semithick]
(axis cs:490,395.833333333333)
--(axis cs:510,395.833333333333);

\path [draw=red, semithick]
(axis cs:50,-0.00334608800363867)
--(axis cs:50,65.9755683102259);

\path [draw=red, semithick]
(axis cs:100,65.9421196383399)
--(axis cs:100,131.974547028327);

\path [draw=red, semithick]
(axis cs:150,98.9048365902785)
--(axis cs:150,164.98405229861);

\path [draw=red, semithick]
(axis cs:200,131.860893814377)
--(axis cs:200,198.000217296734);

\path [draw=red, semithick]
(axis cs:250,164.810309447695)
--(axis cs:250,231.023023885639);

\path [draw=red, semithick]
(axis cs:300,197.75310554622)
--(axis cs:300,264.052450009336);

\path [draw=red, semithick]
(axis cs:350,230.689308019177)
--(axis cs:350,297.088469758601);

\path [draw=red, semithick]
(axis cs:400,263.618946552474)
--(axis cs:400,330.131053447526);

\path [draw=red, semithick]
(axis cs:450,329.458668896105)
--(axis cs:450,396.235775548339);

\path [draw=red, semithick]
(axis cs:500,362.368830131627)
--(axis cs:500,429.29783653504);

\addplot [semithick, red, forget plot]
table {%
0 0
500 392.40620488787
};
\path [draw=green!50.1960784313725!black, semithick]
(axis cs:90,53.0442477876106)
--(axis cs:110,53.0442477876106);

\path [draw=green!50.1960784313725!black, semithick]
(axis cs:140,95.1428571428572)
--(axis cs:160,95.1428571428572);

\path [draw=green!50.1960784313725!black, semithick]
(axis cs:190,148)
--(axis cs:210,148);

\path [draw=green!50.1960784313725!black, semithick]
(axis cs:240,188.913461538462)
--(axis cs:260,188.913461538462);

\path [draw=green!50.1960784313725!black, semithick]
(axis cs:290,233.740384615385)
--(axis cs:310,233.740384615385);

\path [draw=green!50.1960784313725!black, semithick]
(axis cs:340,270.260869565217)
--(axis cs:360,270.260869565217);

\path [draw=green!50.1960784313725!black, semithick]
(axis cs:390,326.204081632653)
--(axis cs:410,326.204081632653);

\path [draw=green!50.1960784313725!black, semithick]
(axis cs:440,360)
--(axis cs:460,360);

\path [draw=green!50.1960784313725!black, semithick]
(axis cs:490,396)
--(axis cs:510,396);

\path [draw=green!50.1960784313725!black, semithick]
(axis cs:540,468)
--(axis cs:560,468);

\path [draw=green!50.1960784313725!black, semithick]
(axis cs:100,52.4114526583863)
--(axis cs:100,53.6770429168349);

\path [draw=green!50.1960784313725!black, semithick]
(axis cs:150,91.0666302830339)
--(axis cs:150,99.2190840026804);

\path [draw=green!50.1960784313725!black, semithick]
(axis cs:200,141.683425571105)
--(axis cs:200,154.316574428895);

\path [draw=green!50.1960784313725!black, semithick]
(axis cs:250,182.271308845487)
--(axis cs:250,195.555614231436);

\path [draw=green!50.1960784313725!black, semithick]
(axis cs:300,223.88999208276)
--(axis cs:300,243.590777148009);

\path [draw=green!50.1960784313725!black, semithick]
(axis cs:350,257.145804132173)
--(axis cs:350,283.375934998262);

\path [draw=green!50.1960784313725!black, semithick]
(axis cs:400,311.969339856676)
--(axis cs:400,340.438823408631);

\path [draw=green!50.1960784313725!black, semithick]
(axis cs:450,341.133063042629)
--(axis cs:450,378.866936957371);

\path [draw=green!50.1960784313725!black, semithick]
(axis cs:500,376.956019710008)
--(axis cs:500,415.043980289992);

\path [draw=green!50.1960784313725!black, semithick]
(axis cs:550,448.557402072788)
--(axis cs:550,487.442597927212);

\addplot [semithick, green!50.1960784313725!black, forget plot]
table {%
0 0
550 439.564038903656
};
\addplot [semithick, red, mark=*, mark size=3, mark options={solid}, only marks]
table {%
50 32.9861111111111
100 98.9583333333333
150 131.944444444444
200 164.930555555556
250 197.916666666667
300 230.902777777778
350 263.888888888889
400 296.875
450 362.847222222222
500 395.833333333333
};
\addlegendentry{Линза}
\addplot [semithick, green!50.1960784313725!black, mark=*, mark size=3, mark options={solid}, only marks]
table {%
100 53.0442477876106
150 95.1428571428572
200 148
250 188.913461538462
300 233.740384615385
350 270.260869565217
400 326.204081632653
450 360
500 396
550 468
};
\addlegendentry{Спектр}
\end{axis}

\end{tikzpicture}
}
			\caption{График зависимости $T$ от $n$.}
		\end{figure}
		\par Видим, что график это прямая пропорциональость $T=kn$. Из МНК получаем коэффициент $k = (0.251\pm0.007)$
		c,  $\chi^2/(10-2) $= 0.76.
		По значению углового коэффициента $k$ рассчитаем величину горизонтальной составляющей магнитного поля Земли
		по формуле: $B_{h}=\pi^{2} m d^{2} / (3 k^{2} P_{m}) = (0.21\pm0.01)$ Гс

		\subsection{Определение вертикальной составляющей магнитного поля Земли}
		\par Изготовим магнитную «стрелку» из n = 10 шариков и подвесим её за середину с помощью
		нити на штативе (см. рис. 4, а). Приведем магнитную стрелку в горизонтальное положение,
		уравновесив ее с помощью кусочков проволоки. Повторим это для нескольких чётных значений $n$ и
		занесём данные в таблицу 2, $\Delta M = 0.03$ г.
		\begin{table}[H]
			\captionsetup{font={small}, labelformat=fullparents, labelsep=fill, labelfont=bf, justification=raggedleft,
			singlelinecheck=false, skip=-0.2cm}
			\caption{Исследование зависимости $T$ от количества шариков}
			\begin{center}
				\begin{tabular}{|l|l|l|l|l|l|l|l|l|l|l|l|}
					\hline
					M, эрг & 114.6 & 229.3 & 343.0 & 458.6 & 478.2 \\ \hline
					l, см & 0.585 & 1.17 & 1.755 & 2.34 & 2.925 \\ \hline
					m, г & 0.20 & 0.20 & 0.20 & 0.20 & 0.16 \\ \hline
					n & 4 & 6 & 8 & 10 & 12 \\ \hline
					\end{tabular}
			\end{center}
		\end{table}
		\par Построим график M(n)
		\begin{figure}[H]
			\center{% This file was created with tikzplotlib v0.9.16.
\begin{tikzpicture}

\begin{axis}[
    height=11cm,
    tick align=inside,
    major tick length=0.2cm,
    minor tick length=0.1cm,
    tick pos=left,
    xmin=-4.1, xmax=4.1,
    xtick={-5, -4, -3, -2, -1, 0, 1, 2, 3, 4, 5},
    minor x tick num=4,
    xmajorgrids,
    minor x grid style={dotted,black},
    xminorgrids,
    xtick style={color=black},
    xlabel={$m$},
    ymin=-1.2, ymax=1.2,
    ytick={-1.5, -1, -0.5, 0, 0.5, 1, 1.5},
    minor y tick num=4,
    ymajorgrids,
    minor y grid style={dotted,black},
    yminorgrids,
    ytick style={color=black},
    ylabel={$x_m$, мм}
]
\path [draw=red, semithick]
(axis cs:1,0.14)
--(axis cs:1,0.18);

\path [draw=red, semithick]
(axis cs:2,0.48)
--(axis cs:2,0.52);

\path [draw=red, semithick]
(axis cs:3,0.7)
--(axis cs:3,0.74);

\path [draw=red, semithick]
(axis cs:4,1.02)
--(axis cs:4,1.06);

\path [draw=red, semithick]
(axis cs:-1,-0.18)
--(axis cs:-1,-0.14);

\path [draw=red, semithick]
(axis cs:-2,-0.5)
--(axis cs:-2,-0.46);

\path [draw=red, semithick]
(axis cs:-3,-0.74)
--(axis cs:-3,-0.7);

\path [draw=red, semithick]
(axis cs:-4,-1.08)
--(axis cs:-4,-1.04);

\addplot [semithick, black]
table {%
1 0.249999999998364
2 0.499999999996728
3 0.749999999995093
4 0.999999999993457
-1 -0.249999999998364
-2 -0.499999999996728
-3 -0.749999999995093
-4 -0.999999999993457
};
\addplot [semithick, red, mark=square*, mark size=3, mark options={solid}, only marks]
table {%
1 0.16
2 0.5
3 0.72
4 1.04
-1 -0.16
-2 -0.48
-3 -0.72
-4 -1.06
};
\end{axis}

\end{tikzpicture}
}
			\caption{График зависимости $M$ от $n$}
		\end{figure}
		Аппроксимируем график прямой $M = A n$. Из МНК получаем $A = (48\pm6)$ эрг, $\chi^2/(5-2) $= 0.6.
		Отсюда $B_v = A/P_m = (0.6\pm0.1) $Гс, $\beta = \arctan (B_v/B_h) = (73\pm3)^\circ$
	
	\section{Вывод}
		Согласно справочныч данным: $B_v=0.5$ Гс, $B_h=0.15-0.20$ Гс, $\beta=70^\circ$, результаты работы согласуются с
		ними.

\end{document}
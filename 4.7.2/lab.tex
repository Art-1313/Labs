\documentclass[12pt,a4paper]{article}
\usepackage[T2A]{fontenc}
\usepackage[utf8]{inputenc}
\usepackage[english, russian]{babel}
\usepackage{indentfirst}
\usepackage{misccorr}
\usepackage{graphicx}
\usepackage{amsmath}
\usepackage{amssymb}
\usepackage{circuitikz}
\usepackage[font={small}]{caption}
\usepackage[left=20mm, top=20mm, right=20mm, bottom=20mm, nohead]{geometry}
\usepackage{float}
\usepackage{tabularx}
\usepackage{array}
\usepackage{longtable}
\usepackage{pstool}
\usepackage{pgfplots}
\usepackage{hhline}
\usepackage{multirow}
\usepackage{wrapfig}
\usepackage{pdfpages}
\usepackage{subfigure}

\DeclareCaptionLabelSeparator{fill}{.\\}

\DeclareCaptionLabelFormat{fullparents}{\bothIfFirst{#1}{~}#2}

\pgfplotsset{compat=1.17}

\begin{document}

    \begin{titlepage}
        \begin{center}
            {\LARGE Отчёт по лабораторной работе 4.7.2.\\}
            \vspace*{11cm}
                \textbf{\LARGE Эффект Поккельса.}
            \vspace*{6.5cm}
        \end{center}
        \hfill\begin{minipage}{0.37\textwidth}
            Работу выполнил Громов Артём
            \\
            ЛФИ Б02-006
        \end{minipage}
        \vspace{4.8cm}
        \begin{center}
            Долгопрудный, 2022 г.
        \end{center}
    \end{titlepage}

    \section{Аннотация}

        \begin{flushleft}
            \textbf{Цель работы:} исследовать интерференцию рассеянного света, прошедшего кристалл; наблюдать изменение характера поляризации света при наложении на кристалл электрического поля.
        \end{flushleft}
        \begin{flushleft}
            \textbf{В работе используется:} гелий-неоновый лазер, поляризатор, кристалл ниобата лития, матовая пластинка, экран, источник высоковольтного переменного и постоянного напряжения, фотодиод, осциллограф, линейка.
        \end{flushleft}
        \par Эффектом Поккельса называется изменение показателя преломления света в кристалле под действием электрического поля, причём это изменение пропорционально напряжённости электрического поля. Как следствие эффекта Поккельса в кристалле появляется двойное лучепреломление или меняется его величина, если кристалл был двулучепреломляющим в отсутствие поля.
        \par Изменение показателя преломления кристаллов под действием внешнего электрического поля происходит исключительно за счёт анизотропных свойств кристаллов. Под действием постоянного электрического поля электроны смещаются в сторону того или иного иона (в случае кристалла ниобата лития LiNbO$_3$ --- это ион Li или Nb), при этом меняется поляризуемость среды и связанный с ней показатель преломления. В первом приближении это изменение линейно относительно внешнего электрического поля. Эффект Поккельса может наблюдаться только в кристаллах, не обладающих центром симметрии. Вследствие линейности эффекта относительно внешнего поля $E_{\text{эл}}$ при изменении направления поля на противоположное должен меняться на противоположный и знак изменения показателя преломления $\Delta_n$. Но в кристаллах с центром симметрии это невозможно, так как оба взаимно противоположных направления внешнего поля физически эквивалентны. Кристалл можно поместить между двумя скрещенными поляроидами таким образом, что в отсутствие внешнего электрического поля пропускание света системой будет равно нулю. При подаче на кристалл внешнего поля появится наведённое двулучепреломление, которое изменит поляризацию прошедшего через кристалл света, и такая система начнёт пропускать свет. На этом принципе основаны многочисленные применения эффекта Поккельса в лазерной технике для оптических модуляторов, затворов и других устройств, управляющих лазерным излучением. Поскольку эффект Поккельса связан с изменением электронной поляризуемости под действием электрического поля, то он практически безынерционен --- быстродействие устройств на его основе меньше $10^{-9}$ с.
        \par Рассмотрим сначала кристалл в отсутствие внешнего электрического поля. Кристалл ниобата лития является одноосным кристаллом, то есть кристаллом, оптические свойства которого обладают симметрией вращения относительно некоторого одного направления, называемого оптической осью $Z$ кристалла. Для световой волны, вектор электрического поля $\vec{E}$ которой перпендикулярен оси $Z$, показатель преломления равен $n_o$, а для волны, вектор $\vec{E}$ которой располагается вдоль оси $Z$, он равен $n_e$, причём $n_e<n_o$, т. е. LiNbO$_3$ --- «отрицательный кристалл». В общем случае, когда луч света распространяется под углом $\theta$ к оптической оси $Z$ (рис. 1), существуют два собственных значения показателя преломления $n_1$ и $n_2$: если световой вектор $\vec{E}$ перпендикулярен плоскости $(\vec{k}$, $\vec{Z})$, где $\vec{k}$ --- волновой вектор луча, то волна называется обыкновенной («o» — ординарная), а показатель преломления $n_1$ равен $n_o$ и не зависит от угла $\theta$; когда световой вектор $\vec{E}$ лежит в плоскости $\vec{k}$, $\vec{Z}$ --- это необыкновенная («e» — экстраординарная) волна, при этом показатель преломления $n_2$ зависит от угла $\theta$ и определяется уравнением
        \begin{equation}
            \frac{1}{n_2^2}=\frac{\cos^2{\theta}}{n_o^2}+\frac{\sin^2{\theta}}{n_e^2}.
        \end{equation}
        Нетрудно видеть, что при $\theta=0$ и 90$^{\circ}$ $n_2$ равен no и ne соответственно.
        \begin{figure}[H]
            \begin{center}
                \includegraphics[width=0.9\textwidth]{images/pic1.pdf}
                \caption{Схема для наблюдения интерференционной картины.}
            \end{center}
        \end{figure}
        \par Если перед кристаллом, помещённым между скрещенными поляроидами (рис. 1), расположить линзу или матовую пластинку, после которых лучи будут рассеиваться под различными углами, то на экране, расположенном за поляроидом, мы увидим тёмные концентрические окружности (коноскопическую картину) --- результат интерференции обыкновенной и необыкновенной волн, точнее, проекцию их электрических полей на разрешённое направление выходного поляроида. В нашем эксперименте используется лазер, излучение которого поляризовано, поэтому входной поляроид можно не ставить.
        \par Разность фаз между обыкновенной и необыкновенной волнами, приобретаемая при прохождении через кристалл длиной $l$, равна
        \begin{equation*}
            \Delta\phi=\frac{2\pi}{\lambda}\cdot l\cdot(n_1-n_2).
        \end{equation*}
        Для обыкновенного луча $n_1=n_o$ и не зависит от угла $\theta$ между направлением луча и осью $Z$. Для необыкновенного луча $n_2$ зависит от угла $\theta$ и определяется уравнением (1). Считая, что $n_e$ и no отличаются незначительно, для малых углов $(\sin{\theta}\approx\theta$, $\cos{\theta}\approx1-\theta^2/2)$ получаем $n_2=n_o-(n_o-n_e)\theta^2$. Таким образом,
        \begin{equation*}
            \delta=\frac{2\pi}{\lambda}\cdot l\cdot(n_o-n_e)\cdot\theta^2.
        \end{equation*}
        \par Направлениями постоянной разности фаз служат конусы $\theta=\text{const}$, поэтому интерференционная картина представляет собой концентрические окружности. Интерференционные кольца перерезаны тёмным «мальтийским крестом», который выделяет области, где интерференция отсутствует. В этих направлениях распространяется только одна поляризованная волна (обыкновенная или необыкновенная). При повороте выходного поляроида (анализатора) на 90$^{\circ}$ картина меняется с позитива на негатив: везде, где были светлые места, появляются тёмные и наоборот.
        \par Для случая, когда разрешённое направление анализатора перпендикулярно поляризации лазерного излучения (скрещенные поляризации), найдём радиус тёмного кольца с номером $m$. Для луча, идущего вдоль оси $Z$ $(m=0)$, показатели преломления для двух волн совпадают, сдвиг фаз между ними равен нулю, поляризация излучения на выходе остаётся такой же, как на входе, и луч не проходит через анализатор. Картина не изменится при сдвиге фаз между обыкновенной и необыкновенной волной, кратном $2\pi$. Поэтому для $m$-го тёмного кольца $\delta=2\pi m$ или $\delta=\frac{2\pi}{\lambda}l(n_o-n_e)\theta^2=2\pi m$. Если $L$ --- расстояние от центра кристалла до экрана, то, учитывая закон преломления (закон Снеллиуса) на границе кристалла, при малых углах $\theta_{\text{внешн}}=n_o\theta$ (рис. 1) получаем выражение для радиуса кольца:
        \begin{equation}
            r_m^2=\frac{\lambda}{l}\frac{(n_oL)^2}{(n_o-n_e)}m.
        \end{equation}
        Измеряя радиусы колец, можно найти разность $(n_o-n_e)$ --- двулучепреломление кристалла.

        \begin{wrapfigure}{r}{0.4\textwidth}
            \centering
            \includegraphics[width=0.4\textwidth]{images/pic2.pdf}
            \caption{Эффект Поккельса --- появление новых главных направлений при наложении электрического поля.}
        \end{wrapfigure}
        \par Представим теперь, что мы поместили кристалл в постоянное электрическое поле $E_{\text{эл}}$, направленное вдоль оси $X$, перпендикулярной оптической оси кристалла $Z$. Луч света распространяется вдоль оси $Z$, при этом для любой поляризации в отсутствие внешнего поля показатель преломления равен $n_o$. Свойства симметрии кристалла и его электрооптический тензор таковы, что в результате линейного электрооптического эффекта (эффекта Поккельса) в плоскости $(X$, $Y)$ возникают два главных направления $\xi$ и $\eta$ под углами 45$^{\circ}$ к осям $X$ и $Y$ (рис. 2) с показателями преломления $(n_o-\delta n)$ и $(n_o+\delta n)$, то есть появляются «медленная» и «быстрая» ось, причём $\delta n=A\cdot E_{\text{эл}}$ ($A$ --- некая константа, зависящая только от типа кристалла).
        \par Пусть свет на входе в кристалл поляризован вертикально, а на выходе стоит анализатор, пропускающий горизонтальную поляризацию. Разложим исходный световой вектор $E=E_0e^{i(\omega t-kz)}$ по осям $\xi$ и $\eta$: $E_{\xi}=E_{\eta}=E_0/\sqrt{2}$. После прохождения кристалла между векторами $E_{\xi}$ и $E_{\eta}$ появится разность фаз
        \begin{equation*}
            \delta=\frac{2\pi l}{\lambda}2\Delta n=\frac{2\pi l}{\lambda}AE_{\text{эл}}=\frac{2\pi l}{\lambda}\frac{l}{d}AU,
        \end{equation*}
        где $U=E_{\text{эл}}d$ --- напряжение на кристалле, $d$ --- размер кристалла в поперечном направлении. Результирующее поле после анализатора --- это сумма проекций $E_{\xi}$ и $E_{\eta}$ на направление $X$, т. е.
        \begin{equation*}
            E_{\text{вых}}=\frac{E_0}{2}e^{i(\omega t-kz)}(e^{i\delta/2}-e^{-i\delta/2})=E_0e^{i(\omega t-kz)}\sin{\left(\frac{\delta}{2}\right)}.
        \end{equation*}
        Интенсивность света пропорциональна квадрату модуля вектора электрического поля в волне:
        \begin{equation*}
            I_{\text{вых}}\sim EE^{*}=E^2_0\sin^2{\left(\frac{\delta}{2}\right)},
        \end{equation*}
        поэтому
        \begin{equation}
            I_{\text{вых}}=I_0\sin^2{\left(\frac{\delta}{2}\right)}=I_0\sin^2{\left(\frac{\pi}{2}\frac{U}{U_{\lambda/2}}\right)}.
        \end{equation}
        Здесь
        \begin{equation}
            U_{\lambda/2}=\frac{\lambda}{4A}\frac{d}{l}
        \end{equation}
        так называемое полуволновое напряжение --- имеет тот смысл, что при $U=U_{\lambda/2}$ сдвиг фаз между двумя волнами, соответствующими двум собственным поляризациям, $\delta=\pi$ (разность хода равна $\lambda/2$), и интенсивность света на выходе анализатора достигает максимума. Это следует из (3).
        \par Нетрудно показать, что при параллельных поляризациях лазера и анализатора
        \begin{equation}
            I_{\text{вых}}=I_0\cos^2{\left(\frac{\pi}{2}\frac{U}{U_{\lambda/2}}\right)}
        \end{equation}
        \par Напряжение $U=U_{\lambda/2}$ называют также управляющим напряжением. Оно уменьшается, как это видно из (4), с уменьшением длины волны света $\lambda$ и с увеличением отношения $\lambda/d$ кристалла (это справедливо для поперечного электрооптического эффекта, который используется в нашем опыте). Характерная величина полуволнового напряжения в ниобате лития для видимого света составляет несколько сотен вольт.

    \section{Экспериментальная установка}

        \begin{figure}[H]
            \begin{center}
                \includegraphics[width=0.7\textwidth]{images/pic3.pdf}
                \caption{Схема для изучения двойного лучепреломления в электрическом поле.}
            \end{center}
        \end{figure}
        \par Оптическая часть установки представлена на рис. 1. Свет гелий-неонового лазера, поляризованный в вертикальной плоскости, проходя сквозь матовую пластинку, рассеивается и падает на двоякопреломляющий кристалл под различными углами. Кристалл ниобата лития с размерами $3\times3\times26$ мм вырезан вдоль оптической оси $Z$. На экране, расположенном за скрещенным поляроидом, видна интерференционная картина.
        \par Для $\lambda=0.63$ мкм (длина волны гелий-неонового лазера) в ниобате лития $n_o=2.29$.
        \par Убрав рассеивающую пластинку и подавая на кристалл постоянное напряжение, можно величиной напряжения влиять на поляризацию луча, вышедшего из кристалла.
        \par Заменив экран фотодиодом (рис. 3) и подав на кристалл переменное напряжение, можно исследовать поляризацию луча с помощью осциллографа.

    \section{Результаты измерений и обработка данных}

        \par Установим середину кристалла на расстоянии $L=75.0 \pm 0.5$ см от экрана. Приведём анализатор в горизантальное разрешённое направление. Измерим радиусы тёмных колец $r(m)$ в каждом из квадрантов (где это возможно) и занесём их в таблицу 1. Приборная погрешность измерения радиуса составила 1 мм. Построим график $r^2=f(m)$ (рис. 4).
        \begin{table}[H]
            \captionsetup{font={small}, labelformat=fullparents, labelsep=fill, labelfont=bf, justification=raggedleft,
                        singlelinecheck=false, skip=-0.2cm}
            \caption{Радиусы в разных квадрантах их среднее и погрешность.}
            \begin{center}
                \begin{tabular}{|l|l|}\hline$m$&$z$, дел
\\\hline0.0&275.0
\\\hline1.0&311.0
\\\hline2.0&348.0
\\\hline3.0&381.0
\\\hline4.0&419.0
\\\hline5.0&453.0
\\\hline6.0&489.0
\\\hline7.0&529.0
\\\hline-1.0&250.0
\\\hline-2.0&214.0
\\\hline-3.0&178.0
\\\hline-4.0&148.0
\\\hline-5.0&109.0
\\\hline-6.0&73.0
\\\hline-7.0&35.0
\\\hline\end{tabular}
            \end{center}
        \end{table}
        \begin{figure}[H]
            \begin{center}
                % This file was created with tikzplotlib v0.9.16.
\begin{tikzpicture}

\begin{axis}[
legend cell align={left},
legend style={
  fill opacity=0.8,
  draw opacity=1,
  text opacity=1,
  at={(0.03,0.97)},
  anchor=north west,
  draw=white!80!black
},
height=11.2cm,
tick align=inside,
major tick length=0.2cm,
minor tick length=0.1cm,
tick pos=left,
xmin=0, xmax=600,
xtick={0, 50, 100, 150, 200, 250, 300, 350, 400, 450, 500, 550, 600},
minor x tick num=4,
xmajorgrids,
minor x grid style={dotted,black},
xminorgrids,
xtick style={color=black},
xlabel={$D_{\text{винта}}$, мкм},
ymin=0, ymax=500,
ytick={0, 50, 100, 150, 200, 250, 300, 350, 400, 450, 500},
ytick style={color=black},
minor y tick num=4,
ymajorgrids,
minor y grid style={dotted,black},
yminorgrids,
ytick style={color=black},
ylabel={$D_{\text{расчёт}}$, мкм}
]
\path [draw=red, semithick]
(axis cs:40,32.9861111111111)
--(axis cs:60,32.9861111111111);

\path [draw=red, semithick]
(axis cs:90,98.9583333333333)
--(axis cs:110,98.9583333333333);

\path [draw=red, semithick]
(axis cs:140,131.944444444444)
--(axis cs:160,131.944444444444);

\path [draw=red, semithick]
(axis cs:190,164.930555555556)
--(axis cs:210,164.930555555556);

\path [draw=red, semithick]
(axis cs:240,197.916666666667)
--(axis cs:260,197.916666666667);

\path [draw=red, semithick]
(axis cs:290,230.902777777778)
--(axis cs:310,230.902777777778);

\path [draw=red, semithick]
(axis cs:340,263.888888888889)
--(axis cs:360,263.888888888889);

\path [draw=red, semithick]
(axis cs:390,296.875)
--(axis cs:410,296.875);

\path [draw=red, semithick]
(axis cs:440,362.847222222222)
--(axis cs:460,362.847222222222);

\path [draw=red, semithick]
(axis cs:490,395.833333333333)
--(axis cs:510,395.833333333333);

\path [draw=red, semithick]
(axis cs:50,-0.00334608800363867)
--(axis cs:50,65.9755683102259);

\path [draw=red, semithick]
(axis cs:100,65.9421196383399)
--(axis cs:100,131.974547028327);

\path [draw=red, semithick]
(axis cs:150,98.9048365902785)
--(axis cs:150,164.98405229861);

\path [draw=red, semithick]
(axis cs:200,131.860893814377)
--(axis cs:200,198.000217296734);

\path [draw=red, semithick]
(axis cs:250,164.810309447695)
--(axis cs:250,231.023023885639);

\path [draw=red, semithick]
(axis cs:300,197.75310554622)
--(axis cs:300,264.052450009336);

\path [draw=red, semithick]
(axis cs:350,230.689308019177)
--(axis cs:350,297.088469758601);

\path [draw=red, semithick]
(axis cs:400,263.618946552474)
--(axis cs:400,330.131053447526);

\path [draw=red, semithick]
(axis cs:450,329.458668896105)
--(axis cs:450,396.235775548339);

\path [draw=red, semithick]
(axis cs:500,362.368830131627)
--(axis cs:500,429.29783653504);

\addplot [semithick, red, forget plot]
table {%
0 0
500 392.40620488787
};
\path [draw=green!50.1960784313725!black, semithick]
(axis cs:90,53.0442477876106)
--(axis cs:110,53.0442477876106);

\path [draw=green!50.1960784313725!black, semithick]
(axis cs:140,95.1428571428572)
--(axis cs:160,95.1428571428572);

\path [draw=green!50.1960784313725!black, semithick]
(axis cs:190,148)
--(axis cs:210,148);

\path [draw=green!50.1960784313725!black, semithick]
(axis cs:240,188.913461538462)
--(axis cs:260,188.913461538462);

\path [draw=green!50.1960784313725!black, semithick]
(axis cs:290,233.740384615385)
--(axis cs:310,233.740384615385);

\path [draw=green!50.1960784313725!black, semithick]
(axis cs:340,270.260869565217)
--(axis cs:360,270.260869565217);

\path [draw=green!50.1960784313725!black, semithick]
(axis cs:390,326.204081632653)
--(axis cs:410,326.204081632653);

\path [draw=green!50.1960784313725!black, semithick]
(axis cs:440,360)
--(axis cs:460,360);

\path [draw=green!50.1960784313725!black, semithick]
(axis cs:490,396)
--(axis cs:510,396);

\path [draw=green!50.1960784313725!black, semithick]
(axis cs:540,468)
--(axis cs:560,468);

\path [draw=green!50.1960784313725!black, semithick]
(axis cs:100,52.4114526583863)
--(axis cs:100,53.6770429168349);

\path [draw=green!50.1960784313725!black, semithick]
(axis cs:150,91.0666302830339)
--(axis cs:150,99.2190840026804);

\path [draw=green!50.1960784313725!black, semithick]
(axis cs:200,141.683425571105)
--(axis cs:200,154.316574428895);

\path [draw=green!50.1960784313725!black, semithick]
(axis cs:250,182.271308845487)
--(axis cs:250,195.555614231436);

\path [draw=green!50.1960784313725!black, semithick]
(axis cs:300,223.88999208276)
--(axis cs:300,243.590777148009);

\path [draw=green!50.1960784313725!black, semithick]
(axis cs:350,257.145804132173)
--(axis cs:350,283.375934998262);

\path [draw=green!50.1960784313725!black, semithick]
(axis cs:400,311.969339856676)
--(axis cs:400,340.438823408631);

\path [draw=green!50.1960784313725!black, semithick]
(axis cs:450,341.133063042629)
--(axis cs:450,378.866936957371);

\path [draw=green!50.1960784313725!black, semithick]
(axis cs:500,376.956019710008)
--(axis cs:500,415.043980289992);

\path [draw=green!50.1960784313725!black, semithick]
(axis cs:550,448.557402072788)
--(axis cs:550,487.442597927212);

\addplot [semithick, green!50.1960784313725!black, forget plot]
table {%
0 0
550 439.564038903656
};
\addplot [semithick, red, mark=*, mark size=3, mark options={solid}, only marks]
table {%
50 32.9861111111111
100 98.9583333333333
150 131.944444444444
200 164.930555555556
250 197.916666666667
300 230.902777777778
350 263.888888888889
400 296.875
450 362.847222222222
500 395.833333333333
};
\addlegendentry{Линза}
\addplot [semithick, green!50.1960784313725!black, mark=*, mark size=3, mark options={solid}, only marks]
table {%
100 53.0442477876106
150 95.1428571428572
200 148
250 188.913461538462
300 233.740384615385
350 270.260869565217
400 326.204081632653
450 360
500 396
550 468
};
\addlegendentry{Спектр}
\end{axis}

\end{tikzpicture}

                \caption{График зависимости $r^2(m)$.}
            \end{center}
        \end{figure}
        \par В итоге коэффициент наклона равен $k=800\pm10$ мм$^2$, откуда по формуле (2) имеем $n_e=2.198 \pm 0.002$. Характеристики установки: $n_o=2.29$, $l=26$ мм, $\lambda=630$ нм.
        \par Подадим на кристал напряжение и по изменению яркости определим полуволновое напряжение ниобаьа лития. Получим $U_{\lambda/2}=450\pm15$ В.
        \par Поставив вместо экрана фотодиод получим из фигур Лиссажу, что $U_{\lambda/2}=450\pm15$ В. Для скрещенной поляризации результаты представлены на рисунке 5. Для параллельной поляризации минимумы и максимумы меняются местами.
        \begin{figure}[H]  
            \centering
            \subfigure[]
            {
                \includegraphics[width=0.25\linewidth]{images/IMG_0082.jpg}
            }  
            \hspace{4ex}
            \subfigure[]
            {
                \includegraphics[width=0.25\linewidth]{images/IMG_0080.jpg}
            }
            \hspace{4ex}
            \subfigure[]
            {
                \includegraphics[width=0.24\linewidth]{images/IMG_0079.jpg}
            }  
            \caption{Фигуры Лиссажу для напряжения: (a) $U_{\lambda/2}$; (b) $U_{\lambda}$; (c) $U_{3\lambda/2}$.}
        \end{figure}

    \section{Обсуждение результатов}
        
        \par Погрещность определения экстроординорного показателя прломления составила 2\%, что является очень хорошим результатом. При этом значение совпало с табличным. Результаты определения полувонового напряжения двуми способами согласуются между собой.

    \section{Вывод}

        \par Лабораторная работа выполнена успешно.

\end{document}

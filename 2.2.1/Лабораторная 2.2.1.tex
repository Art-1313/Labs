\documentclass[12pt,a4paper]{article}
\usepackage[utf8]{inputenc}
\usepackage[english,russian]{babel}
\usepackage{indentfirst}
\usepackage{misccorr}
\usepackage{graphicx}
\usepackage{amsmath}
\usepackage{amssymb}
\usepackage{circuitikz}
\usepackage[font={small}]{caption}
\usepackage[left=20mm, top=20mm, right=20mm, bottom=20mm, nohead]{geometry}
\usepackage{float}
\usepackage{tabularx}
\usepackage{array}
\usepackage{longtable}
\usepackage{pstool}
\usepackage{pgfplots}
\usepackage{hhline}
\usepackage{multirow}

\DeclareCaptionLabelSeparator{fill}{.\\}
\DeclareCaptionLabelFormat{fullparents}{\bothIfFirst{#1}{~}#2}

\begin{document}

	\begin{titlepage}
		\begin{center}
			{\LARGE Отчёт по лабораторной работе 2.2.1.\\}
			\vspace*{11cm}
				\textbf{\LARGE Исследование взаимной диффузии газов.}	
			\vspace*{6.5cm}
		\end{center}
		\hfill\begin{minipage}{0.37\textwidth}
				Работу выполнил Громов Артём\\
				ЛФИ Б02-006
		\end{minipage}
		\vspace{5cm}
		\begin{center}
 			Долгопрудный, 2021 г.
		\end{center}   
	\end{titlepage}
	
	\section{Аннотация}
		\noindent\textbf{Цель работы: }
		\begin{enumerate}
			\item регистрация зависимости концентрации гелия в воздухе от времени с помощью датчиков теплопроводности
			при разных начальных давлениях смеси газов;
			\item определение коэффициента диффузии по результатам измерений.
		\end{enumerate}
		\noindent\textbf{В работе используются: } измерительная установка; форвакуумный насос; баллон с газом (гелий);
		манометр; источник питания; магазин сопротивлений; гальванометр; секундомер.
		\vspace{0.5 cm}
		\par Диффузией называется самопроизвольное перемешивание молекул, происходящее вследствие их хаотичного
		теплового движения. При перемешивании молекул разного сорта говорят о взаимной (или концентрационной)
		диффузии. Для наблюдения взаимной диффузии необходимо равенство давлений во всей системе (в противном случае
		возникнет гораздо более быстрое макроскопическое течение газа как сплошной среды).
		\par В системе, состоящей из двух компонетов $a$ и $b$, плотность потока вещества любого компонента в результате
		взаимной диффузии определяется законом Фика:
		\begin{equation}
			j_{a}=-D_{ab}\frac{\partial n_{a}}{\partial x}, \ \ \ \  j_{b}=-D_{ba}\frac{\partial n_{b}}{\partial x},
		\end{equation}
		\noindent где $D_{ab}=D_{ba}=D$ --- коэффициент взатмной диффузии компонентов, а $j_{a,b}$ --- плотности потока
		частиц соответствующего сорта (количество частиц, пересекающих единичную площадку в единицу времени).
		\par В данной работе исследуется диффузия примеси лёгкого газа (гелия) на фоне воздуха. Концентрация воздуха
		$n_{\text{в}}$ условиях опыта предполагается значительно большей, чем концентрация примеси $n_{\text{He}}$, и её
		относительное изменение в результате взаимной диффузии будет незначительным. Поэтому мы будем описывать
		только диффузию примеси (гелия) на стационарном фоне воздуха и в дальнейшем, если не оговорено особо, под $n$
		будем иметь в виду концентрацию примеси $n_{\text{He}}$.
		\par Для исследования взаимной диффузии газов и определения коэффициента диффузии используется установка,
		изображённая на рис. 1. Два сосуда с объёмами $V_1$ и $V_2$ соединены трубкой длины $l$ и сечения $S$. Сосуды
		заполнены смесью двух газов при одинаковом давлении, но с различной концентрацией компонентов. Вследствие
		взаимной диффузии концентрации каждого из компонентов в обоих сосудах с течением времени выравниваются.
		\par Рассмотрим процесс выравнивания концентрации. В общем случае концентрация зависит от координат и времени
		во всей установке. В наших условиях решение задачи упрощается, поскольку объём соединительной трубки мал по
		сравнению с объёмами сосудов. В связи этим концентрацию газов внутри каждого сосуда можно считать постоянной
		по всему объёму сосуда, и предположить, что процесс выравнивания концентраций происходит в основном благодаря
		диффузии в трубке.
		\par Если бы концентрация в сосудах $V_1$ и $V_2$ поддерживались постоянными и равными $n_1$ и $n_2$, то в
		трубке установился бы стационарный поток частиц $J=-DS\frac{\partial n}{\partial x}$, одинаковый в каждом сечение
		трубки. Следовательно, $n(x)$ была бы линейной функцией коорднаты и $dn/dx=\Delta n/l$, где $l$ --- длина тубки,
		откуда получаем
		\begin{equation}
			J=-DS\frac{n_1-n_2}{l}.
		\end{equation}
		\par Предположим, что процесс выравнивания концентраций в сосудах происходит достаточно медленно, так что всё
		время успевает установиться стационарный (линейный) профиль концентрации и в каждый момент времени
		справедливо соотношение (2). Исходя из этого получим зависимость концентраций в каждом сосуде $n_1$ и $n_2$ от
		времени. Такое приближение называют квазистационарным.
		\begin{figure}[H]
			\begin{minipage}[h]{0.58\linewidth}
				\begin{center}
					\includegraphics[width=\textwidth]{C:/Users/gromo/Desktop/Установка221_1.png}
					\caption{Схема установки для исследования взаимной диффузии газов}
				\end{center}
			\end{minipage}	
			\hfill
			\begin{minipage}[h]{0.39\linewidth}
				\begin{center}
					\includegraphics[width=\textwidth]{C:/Users/gromo/Desktop/Установка221_2.png}
					\caption{Мостовая схема с датчиками теплопроводности для измерения разности концентраций газов}
				\end{center}
			\end{minipage}	
		\end{figure}
		\par Обозначим через $\Delta n_1$ и $\Delta n_2$ измения концентрации в объёмах $V_1$ и $V_2$ за время $\Delta t$.
		Тогда $V_{1} \Delta n_{1}$ равно измению количества компонента в объёме $V_{1}$, а $V_{2}\Delta n_{2}$ ---
		изменению количества  этого компонента в $V_2$. Из закона сохранения вещества следует, что
		$V_{1}n_{1}+V_{2}n_{2}=\text{const}$ откуда $V_{1}\Delta n_{1}=-V_{2}\Delta n_{2}$. Эти изменеия происходят
		вследствии диффузии, поэтому
		\begin{equation}
			V_{1}\Delta n_{1}=-V_{2}\Delta n_{2}=J\Delta t=-DS\frac{n_1-n_2}{l}\Delta t.
		\end{equation}
		\noindent Деля это равенство на $\Delta t$, получим
		\begin{equation}
			V_{1}\frac{dn_1}{dt}=-DS\frac{n_1-n_2}{l}, \ \ \ \ V_{2}\frac{dn_2}{dt}=DS\frac{n_1-n_2}{l}.
		\end{equation}
		\noindent Разделив первое из этих уравнений на $V_1$, а второе на $V_2$ и вычтя эти равенства друг из друга, найдём
		\begin{equation}
			\frac{dn_1}{dt}-\frac{dn_2}{dt}=-DS\frac{n_1-n_2}{l}\left(\frac{1}{V_1}+\frac{1}{V_2}\right).
		\end{equation}
		\noindent Введём новую переменную $\Delta n=n_1-n_2$, после чего уравнение легко интегрируется:
		\begin{equation}
			\Delta n=\Delta n_{0}e^{-t/\tau},
		\end{equation}
		\noindent где $\Delta n_0$ --- разность концентраций в начальный момент времени,
		\begin{equation}
			\tau=\frac{V_{1}V_{2}}{V_1+V_2}\frac{l}{DS}.
		\end{equation}
		\noindent Формула (6) показывает, что разность концентраций убывает по экспоненциальному закону и тем быстрее,
		чем меньше $\tau$ (постоянная времени процесса). Величина $\tau$ определяется геометрическими размерами
		установки ($l, S, V_1, V_2$) и величиной коэффициента диффузии $D$.
		\par Для проверки применимости квазистационарного приближения необходимо убедиться, что время $\tau$ много
		больше характерного времени диффузии одной частицы вдоль трубки длиной $l$:
		$t_{\text{диф}}~\frac{l^2}{D}\ll\tau$.
		\par Для измерения концентраций в данной установке применяются датчики теплопроводности $D_1$, $D_2$ (рис. 1)
		и используется зависимость теплопроводности газовой смеси от её состава. Тонкая проволочка радиуса
		$r_{\text{пр}}$, протянутая вдоль оси стеклянного цилиндра радиуса $R_{\text{ц}}$, нагревается током. Тепло от
		проволочки к стенке цилиндра переходит главным образом вследствие теплопроводности газа, находящегося внутри
		цилиндра. Количество тепла, передающееся стенке в единицу времени, посчитаем с помощью следующей формулы:
		\begin{equation}
			Q=\kappa\frac{2\pi L}{\text{ln}\left(R_{\text{ц}}/r_{\text{пр}}\right)}\left(T_1-T_2\right),
		\end{equation}
		\noindent где $\kappa$ --- теплпроводность, $L$ --- длина нити, $T_1, T_2$ --- температы проволочки и стенки. При
		заданном режиме нагревания ($Q=$const) температура проволочки и соответственно её сопротивление опрелеляются
		теплопровдностью газа и, следовательно, его составом.
		\par Для измерения разности концентраций газов используется мостовая схема (рис. 2). Здесь $D_1$ и $D_2$ ---
		датчики теплопроводности, расположенные в сосудах $V_1$ и $V_2$. Сопротивления $R_1$, $R_2$ и $R$ служат для
		установки прибора на нуль (балансировка моста). В одну из диагоналей моста включён гальванометр, к другой
		подключается небольшое постоянное напряжение. Мост балансируется при заполнении сосудов (и датчиков)
		одной и той же смесью. При заполнении сосудов смесями различного состава возникает «разбаланс» моста,
		зависящий от разности концентраций.
		\par Зависимость теплопроводности смеси газов от её состава, вообще говоря, довольно сложна. Однако при
		достаточно малых изменениях концентраций можно ожидать, что величина тока, проходящего через гальванометр
		$G$, будет пропорциональна разности концентраций (первый член разложения функции в ряд Тейлора).
		Эксперименты показывают, что при разности концентраций, равной 15\%, поправка к линейному закону не
		превышает 0.5\%, что для наших целей вполне достаточно.
		\par В процессе диффузии разность концентраций убывает по закону (6). По тому же закону изменяются во времени
		показания гальванометра (например, в делениях шкалы), т. е.
		\begin{equation}
			N=N_{0}e^{-t/\tau},
		\end{equation}
		\noindent где $N_0$ --- показание в начальный момент времени.
		\par Отметим некоторые особенности методики, применённой в данной работе.
		\par Во-первых, для устранения тепловой конвекции датчик выполнен в виде длинной стеклянной трубки, внутри
		которой натянута нагреваемая током платиновая нить. Внутренняя полость датчика сообщается с объёмом камеры
		через специально сделанные отверстия. Размер отверстий и объём датчика таковы, что скорость диффузии газов из
		объёма сосуда в полость датчика значительно больше скорости диффузии из одного объёма в другой. Таким образом,
		состав газа в датчике практически совпадает с составом газа в объёме.
		\par Во-вторых, в силу неполного обмена энергией между молекулами газа и поверхностью нити её температура
		несколько выше, чем температура прилегающих слоёв газа --- на её поверхности возникает температурный скачок.
		Величина температурного скачка зависит от давления. Вследствие этого, а также потому, что датчики не абсолютно
		идентичны, баланс моста несколько зависит от давления. Для повышения точности опытов рекомендуется
		балансировать мост, заполнив установку воздухом при давлении, близком к рабочему.
	\section{Экспериментальная установка}
		\par Общий вид конструкции установки приведён на рис. 1. Схема электрических соединений показана на рис. 2.
		На рис. 3 изображена конструкция многоходового крана $\text{К}_{6}$.
		\begin{figure}[H]
			\begin{minipage}[h]{0.58\linewidth}
			\hspace{0.5cm} Установка состоит из двух сосудов $V_1$ и $V_2$, соединённых краном $\text{К}_{3}$,
			форвакуумного насоса Ф.Н. с выключателем Т, манометра M и системы напуска гелия, включающей в себя краны
			$\text{К}_{6}$ и $\text{К}_{7}$. Кран $\text{К}_{5}$ позволяет соединять форвакуумный насос либо с установкой,
			либо с атмосферой. Между форвакуумным насосом и краном $\text{К}_{5}$ вставлен предохранительный баллон
			П.Б., защищающий кран $\text{К}_{5}$ и установку при неправильной эксплуатации её от попадания
			форвакуумного масла из насоса Ф.Н. Сосуды $V_1$ и $V_2$ и порознь и вместе можно соединять как с системой
			напуска гелия, так и с форвакуумным насосом. Для этого служат краны $\text{К}_{1}$, $\text{К}_{2}$,
			$\text{К}_{4}$ и $\text{К}_{5}$. Манометр M регистрирует давление газа, до которого заполняют тот или другой
			сосуды.
			\end{minipage}	
			\hfill
			\begin{minipage}[h]{0.39\linewidth}
				\begin{center}
					\includegraphics[width=\textwidth]{C:/Users/gromo/Desktop/Установка221_3.png}
					\caption{Кран $\text{К}_{6}$}
				\end{center}
			\end{minipage}	
		\end{figure}
		\par Давление гелия в трубопроводе больше атмосферного. Это необходимо для того, чтобы из-за возможных
		неплотностей в трубопроводе гелий оставался бы в нём без примесей воздуха. А система напуска гелия, особенно
		многоходовой кран $\text{К}_{6}$, как правило, имеет утечки. Для сохранения гелия, а также для уменьшения
		неконтролированного попадания гелия в установку (по протечкам в кране $\text{К}_{6}$) между трубопроводом
		подачи гелия и краном $\text{К}_{6}$ поставлен металлический кран $\text{К}_{7}$. Его открывают только на время
		непосредственного заполнения установки гелием. Все остальное время он закрыт.
		\par В силу того, что в сосуд требуется подавать малое давление гелия, между кранами $\text{К}_{7}$ и $\text{К}_{4}$
		стоит кран $\text{К}_{6}$, снабженный дозатором. Дозатор --- это маленький объём, который заполняют до давления
		гелия в трубопроводе, а затем уже эту порцию гелия с помощью крана $\text{К}_{6}$ впускают в установку. Подробно
		разрез крана $\text{К}_{6}$ приведён на рис. 3, где показано соединение трубопровода гелия с дозатором. Рычажок Р
		(рис. 1 и 3), показанный пунктиром, находится в квадранте I. Для подачи гелия из дозатора в установку необходимо
		рычажок Р перевести из квадранта I в квадрант II. Если при этом окажется, что однократной подачи недостаточно, то
		процедуру надо повторить. Если же окажется, что гелия напущено больше необходимого, излишки можно откачать
		форвакуумным насосом. Устройство крана $\text{К}_{6}$ не на всех установках одинаково, но на всех установках
		одинаков его принцип --- гелий в установку подаётся малыми порциями.
		\par Кран $\text{К}_{4}$ обладает повышенной вакуумоплотностью. Как отмечалось выше, кран $\text{К}_{6}$ такой
		плотностью не обладает. Поэтому после заполнения сосудов $V_1$ и $V_2$ рабочей смесью кран $\text{К}_{4}$ надо
		обязательно закрыть, чтобы в рабочей части установки давление в процессе измерений сохранялось постоянным.
		\par На рис. 2 приведена схема электрического соединения. $D_1$ и $D_2$ --- сопротивления проволок датчиков
		парциального давления, которые составляют одно плечо моста. Второе плечо моста составляют сопротивления $r_1$,
		$R_1$ и $r_2$, $R_2$. $r_1 \ll R_1$, $r_2 \ll R_2$, $R_1$ и $R_2$ --- спаренные, их подвижные контакты находятся на
		общей оси. Оба они используются для грубой регулировки моста. Точная балансировка моста выполняется
		потенциометром $R$. Последовательно с гальванометром Г, стоящим в диагонали моста, поставлен магазин
		сопротивлений $M_R$. Когда мост балансируют, магазин сопротивлений выводят на ноль. В процессе же составления
		рабочей смеси в сосудах V1 и V2 мост разбалансирован. Чтобы не сжечь при этом гальванометр, магазин $M_R$ ставят
		на максимальное сопротивление.
	\section{Результаты измерений и обработка данных}
		\par В нашей работе установка компьтеризирована, поэтому данные снимались компьютером кждую секунду.
		Измерения были проведены при четырёх значениях давления. Для определения определения коэффициента
		диффузии построим графики, откладывая по оси абсцисс время, а по оси ординат --- логорифм от показаний
		гальванометра. Для определния угловых коэффициентов и их погрешностей воспользуемся методом наименьших
		квадратов.
		\begin{equation}
			k = \frac{\langle xy \rangle -\langle x \rangle \langle y \rangle}{\langle x^2 \rangle - \langle x \rangle^2}, \ \ \ \ 
			b = \langle y \rangle - k\langle x \rangle, \ \ \ \ 
			\sigma_{k} = \frac{1}{\sqrt{n}} \sqrt{\frac{\langle y^2 \rangle - \langle y \rangle^2 }
			{\langle x^2 \rangle - \langle x \rangle^2} - k^2},
		\end{equation}
		\par Из формулы (9) следует, что $k=-1/\tau$, значит мы можем получить выражение для $D$ через $k$
		\begin{equation}
			D=-k\frac{V_{1}V_{2}}{V_1+V_2}\frac{l}{S}, \ \ \ \ 
			\mathcal{E}_{D}=\sqrt{\mathcal{E}^{2}_{k}+\mathcal{E}^{2}_{l/S}+\mathcal{E}^{2}_{V_1}+\mathcal{E}^{2}_{V_2}+\mathcal{E}^{2}_{V_1+V_2}}.
		\end{equation}
		\par Характеристики установки следующие: $V_1=420\pm10\text{ см}^3$, $V_2=420\pm10\text{ см}^3\pm$,
		$l/S=9.0\pm0.1$ 1/см.
		\begin{figure}[H]
			\begin{minipage}[h]{0.49\linewidth}
				\begin{center}
					\input{C:/Users/gromo/PycharmProjects/plottest/plot19.tex}
				\end{center}
			\end{minipage}
			\hfill
			\begin{minipage}[h]{0.49\linewidth}	
				\begin{center}
					\input{C:/Users/gromo/PycharmProjects/plottest/plot20.tex}
				\end{center}
			\end{minipage}
			\begin{minipage}[h]{0.49\linewidth}
				\begin{center}
					\input{C:/Users/gromo/PycharmProjects/plottest/plot21.tex}
				\end{center}
			\end{minipage}
			\hfill
			\begin{minipage}[h]{0.49\linewidth}	
				\begin{center}
					\input{C:/Users/gromo/PycharmProjects/plottest/plot22.tex}
				\end{center}
			\end{minipage}
			\captionsetup{font={small}, justification=justified}
			\caption[figure]{Графики зависимости $\text{ln}U$ от $t$}
		\end{figure}
		\par Зная угловые коэффициенты, вычислим коэффициент взаимной дифузии при каждом давлении. Данные занесём
		в таблицу 1.
		\begin{table}[H]
			\captionsetup{font={small}, labelformat=fullparents, labelsep=fill, labelfont=bf, justification=raggedleft,
			singlelinecheck=false, skip=-0.2cm}
			\caption{Коэффициенты взаимной диффузии при разных давлениях}
			\begin{center}
				\begin{tabular}{|p{2cm}|p{3cm}|p{3cm}|p{3cm}|p{3cm}|}
				\hline
				$P$, торр & 40.5 & 77.2 & 154.5 & 187.7 \\
				\hline
				$D,\text{ см}^2$/с  & $9.56\pm0.05$ & $4.89\pm0.05$ & $2.70\pm0.05$ & $2.25\pm0.05$ \\
				\hline
				\end{tabular}
			\end{center}
		\end{table}
		\par Построим график зависимости D от 1/P.
		\begin{figure}[H]
			\begin{center}
				\input{C:/Users/gromo/PycharmProjects/plottest/plot23.tex}
			\end{center}
			\caption[figure]{Графики зависимости $D$ от времени $1/P$}
		\end{figure}
		\par Опредлим угловой коэффициент этого графика с помощью метода наименьших квадратов
		$k=377\pm7\text{ торр}\cdot\text{ см}^2$/с, $b=0.19\pm0.05\text{ см}^2$/с. Посчитаем коэффициент дифузии при
		атмосферном давлении $D_\text{атм}=0.69\pm0.05\text{ см}^2$/с.
	\section{Обусуждение результатов}
		\par Целью рабооты было определение коэффициента взаимной диффузии между гелием и воздухом. Итоговое
		значение, которое можно сравнить со справочным материалом --- значение коэффициента диффузии при
		атмосферном давлении. По результатом эксперимента он оказдся равен $D_\text{атм}=0.69\pm0.05\text{ см}^2$/с.
		Полученное значение хорошо совпадает с табличным при комнатной температуре. О погрешностях измерений судить
		сложно, так как они производились компьютером. Однако благодаря этому удалось получить очеь низкую случайную
		погрешность (порядка 0.01\%). Итоговая погрешость косвенных вычислений равнеятся 7\%, что является довольно
		хорошим результатом. В ходе проведения опыта была вявлена высокая чувствительность установки к внешним
		возмущениям. На рис. 6 изображены несколько примеров этого. 
		\begin{figure}[H]
			\begin{minipage}[h]{0.49\linewidth}
				\begin{center}
					\input{C:/Users/gromo/PycharmProjects/plottest/plot24.tex}
				\end{center}
			\end{minipage}
			\hfill
			\begin{minipage}[h]{0.49\linewidth}	
				\begin{center}
					\input{C:/Users/gromo/PycharmProjects/plottest/plot25.tex}
				\end{center}
			\end{minipage}
			\captionsetup{font={small}, justification=justified}
			\caption[figure]{Неудачные опыты}
		\end{figure}
	\section{Вывод}
		\par Проведённый эсперимент можно считать удачным. Он хорошо совпал с теоретическими предсказаниями и
		известными табличными значениями. Для повышения точности во время работы нужно быть предельно аккуратным
		и не оказывать силных воздействий на установку (удары по столу).
		
\end{document}
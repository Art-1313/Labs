\documentclass[12pt,a4paper]{article}
\usepackage[T2A]{fontenc}
\usepackage[utf8]{inputenc}
\usepackage[english, russian]{babel}
\usepackage{indentfirst}
\usepackage{misccorr}
\usepackage{graphicx}
\usepackage{amsmath}
\usepackage{amssymb}
\usepackage{circuitikz}
\usepackage[font={small}]{caption}
\usepackage[left=20mm, top=20mm, right=20mm, bottom=20mm, nohead]{geometry}
\usepackage{float}
\usepackage{tabularx}
\usepackage{array}
\usepackage{longtable}
\usepackage{pstool}
\usepackage{pgfplots}
\usepackage{hhline}
\usepackage{multirow}
\usepackage{wrapfig}
\usepackage{pdfpages}

\DeclareCaptionLabelSeparator{fill}{.\\}

\DeclareCaptionLabelFormat{fullparents}{\bothIfFirst{#1}{~}#2}

\pgfplotsset{compat=1.17}

\begin{document}

    \begin{titlepage}
        \begin{center}
            {\LARGE Отчёт по лабораторной работе 4.3.1.\\}
            \vspace*{11cm}
                \textbf{\LARGE Изучение дифракции света.}
            \vspace*{6.5cm}
        \end{center}
        \hfill\begin{minipage}{0.37\textwidth}
            Работу выполнил Громов Артём
            \\
            ЛФИ Б02-006
        \end{minipage}
        \vspace{4.8cm}
        \begin{center}
            Долгопрудный, 2022 г.
        \end{center}
    \end{titlepage}


    \begin{flushleft}
        \textbf{Цель работы:} исследовать явления дифракции Френеля и Фраунгофера на щели, изучить влияние дифракции на разрешающую способность оптических инструментов.
    \end{flushleft}
    \begin{flushleft}
        \textbf{В работе используется:} оптическая скамья, ртутная лампа, монохроматор, щели с регулируемой шириной, рамка с вертикальной итью, двойная щель, микроскоп на поперечных салазках с микрометрическим винтом, зрительная труба.
    \end{flushleft}
    
    \section{Дифракция Френеля}

        \subsection{Экспериментальная установка}
        \par Схема установки для наблюдения дифракции Френеля на щели представлена на рис. 1. Световые лучи освещают щель $S_2$ и испытывают на ней дифракцию. Дифракционная картина рассматривается с помощью микроскопа М, сфокусированного на некоторую плоскость наблюдения П.
        \begin{figure}[H]
            \begin{center}
                \includegraphics[width=0.9\textwidth]{images/pic1.pdf}
                \caption{Схема установки для наблюдения дифракции Френеля.}
            \end{center}
        \end{figure}

        \begin{wrapfigure}{r}{0.25\textwidth}
            \centering
            \includegraphics[width=0.25\textwidth]{images/pic2.pdf}
            \caption{Зоны Френеля в плоскости щели.}
        \end{wrapfigure}
        \par Щель $S_2$ освещается параллельным пучком монохроматического света с помощью коллиматора, образованного объективом $O_1$, и щелью $S_1$, находящейся в его фокусе. На щель $S_1$ сфокусировано изображение спектральной линии, выделенной из спектра ртутной лампы Л при помощи простого монохроматора C, в котором используется призма прямого зрения. Распределение интенсивности света в плоскости наблюдения П проще всего рассчитывать с помощью зон Френеля (для щели их иногда называют зонами Шустера). При освещении щели $S_2$ параллельным пучком лучей (плоская волна) зоны Френеля представляют собой полоски, параллельные краям щели (рис. 2). Результирующая амплитуда в точке наблюдения определяется суперпозицией колебаний от тех зон Френеля, которые не перекрыты створками щели. Графическое определение результирующей амплитуды производится с помощью векторной диаграммы --- спирали Корню. Суммарная ширина $n$ зон Френеля (Шустера) определяется соотношением:
        \begin{equation}
            \xi_n=\sqrt{zn\lambda},
        \end{equation}
        где $z$ --- расстояние от щели до плоскости наблюдения (рис. 1), а $\lambda$ --- длина волны.
        \par Вид наблюдаемой дифракционной картины на щели шириной $b$ определяется волновым параметром $p$ или числом Френеля $C$ (число открытых полных зон):
        \begin{equation*}
            p=\frac{\sqrt{z\lambda}}{b}, \ \ \ \ C=\frac{1}{p^2}.
        \end{equation*}
        \par Дифракционная картина отсутствует вблизи щели при $p\ll1$ ($C\gg1$, т. е. на щели укладывается огромное число зон), а распределение интенсивности света за щелью можно приближённо получить с помощью законов геометрической оптики. Дифракционная картина в этом случае наблюдается только в узкой области на границе света и тени у краёв экрана.
        \par При небольшом удалении от щели (или изменении ширины щели $S_2$) эти две группы дифракционных полос перемещаются практически независимо друг от друга. Каждая из этих групп образует картину дифракции Френеля на краю экрана. Распределение интенсивности при дифракции света на краю экрана может быть найдено с помощью спирали Корню.
        \par При дальнейшем увеличении расстояния $z$ (или уменьшении ширины щели $S_2$) обе системы дифракционных полос постепенно сближаются и, наконец, при $C\gtrsim1$ накладываются друг на друга. Распределение интенсивности в плоскости наблюдения в этом случае определяется числом зон Френеля, укладывающихся на полуширине щели $b/2$. Если это число равно $n$, то в поле зрения наблюдается $m=n-1$ тёмных полос. Таким образом, по виду дифракционной картины можно оценить число зон Френеля на полуширине щели.

        \subsection{Измерения и обработка результатов}
        \par Нйдём положение микроскопа, при котором наиболее чётка фидна щель. Затем отодвинем его в положение при котором видна одна дифракционная полосса. Постепенно придвигая микроскоп будем записывать его положение в моменты, когда видны дифракционные полосы. Результаты занесём в таблицу 1. Погрешность определения расстояния равана 1 мм.

        \begin{minipage}{0.45\textwidth}
            \begin{table}[H]
                \captionsetup{font={small}, labelformat=fullparents, labelsep=fill, labelfont=bf, justification=raggedright,
                            singlelinecheck=false, skip=-0.2cm}
                \caption{Дифаркциооные полосы при разных положениях микроскопа.}
                \begin{center}
                    \begin{tabular}{|l|l|}\hline$m$&$z$, дел
\\\hline0.0&275.0
\\\hline1.0&311.0
\\\hline2.0&348.0
\\\hline3.0&381.0
\\\hline4.0&419.0
\\\hline5.0&453.0
\\\hline6.0&489.0
\\\hline7.0&529.0
\\\hline-1.0&250.0
\\\hline-2.0&214.0
\\\hline-3.0&178.0
\\\hline-4.0&148.0
\\\hline-5.0&109.0
\\\hline-6.0&73.0
\\\hline-7.0&35.0
\\\hline\end{tabular}
                \end{center}
            \end{table}
        \end{minipage}
        \hfill
        \begin{minipage}{0.45\textwidth}
            \begin{figure}[H]
                \begin{center}
                    % This file was created with tikzplotlib v0.9.16.
\begin{tikzpicture}

\begin{axis}[
legend cell align={left},
legend style={
  fill opacity=0.8,
  draw opacity=1,
  text opacity=1,
  at={(0.03,0.97)},
  anchor=north west,
  draw=white!80!black
},
height=11.2cm,
tick align=inside,
major tick length=0.2cm,
minor tick length=0.1cm,
tick pos=left,
xmin=0, xmax=600,
xtick={0, 50, 100, 150, 200, 250, 300, 350, 400, 450, 500, 550, 600},
minor x tick num=4,
xmajorgrids,
minor x grid style={dotted,black},
xminorgrids,
xtick style={color=black},
xlabel={$D_{\text{винта}}$, мкм},
ymin=0, ymax=500,
ytick={0, 50, 100, 150, 200, 250, 300, 350, 400, 450, 500},
ytick style={color=black},
minor y tick num=4,
ymajorgrids,
minor y grid style={dotted,black},
yminorgrids,
ytick style={color=black},
ylabel={$D_{\text{расчёт}}$, мкм}
]
\path [draw=red, semithick]
(axis cs:40,32.9861111111111)
--(axis cs:60,32.9861111111111);

\path [draw=red, semithick]
(axis cs:90,98.9583333333333)
--(axis cs:110,98.9583333333333);

\path [draw=red, semithick]
(axis cs:140,131.944444444444)
--(axis cs:160,131.944444444444);

\path [draw=red, semithick]
(axis cs:190,164.930555555556)
--(axis cs:210,164.930555555556);

\path [draw=red, semithick]
(axis cs:240,197.916666666667)
--(axis cs:260,197.916666666667);

\path [draw=red, semithick]
(axis cs:290,230.902777777778)
--(axis cs:310,230.902777777778);

\path [draw=red, semithick]
(axis cs:340,263.888888888889)
--(axis cs:360,263.888888888889);

\path [draw=red, semithick]
(axis cs:390,296.875)
--(axis cs:410,296.875);

\path [draw=red, semithick]
(axis cs:440,362.847222222222)
--(axis cs:460,362.847222222222);

\path [draw=red, semithick]
(axis cs:490,395.833333333333)
--(axis cs:510,395.833333333333);

\path [draw=red, semithick]
(axis cs:50,-0.00334608800363867)
--(axis cs:50,65.9755683102259);

\path [draw=red, semithick]
(axis cs:100,65.9421196383399)
--(axis cs:100,131.974547028327);

\path [draw=red, semithick]
(axis cs:150,98.9048365902785)
--(axis cs:150,164.98405229861);

\path [draw=red, semithick]
(axis cs:200,131.860893814377)
--(axis cs:200,198.000217296734);

\path [draw=red, semithick]
(axis cs:250,164.810309447695)
--(axis cs:250,231.023023885639);

\path [draw=red, semithick]
(axis cs:300,197.75310554622)
--(axis cs:300,264.052450009336);

\path [draw=red, semithick]
(axis cs:350,230.689308019177)
--(axis cs:350,297.088469758601);

\path [draw=red, semithick]
(axis cs:400,263.618946552474)
--(axis cs:400,330.131053447526);

\path [draw=red, semithick]
(axis cs:450,329.458668896105)
--(axis cs:450,396.235775548339);

\path [draw=red, semithick]
(axis cs:500,362.368830131627)
--(axis cs:500,429.29783653504);

\addplot [semithick, red, forget plot]
table {%
0 0
500 392.40620488787
};
\path [draw=green!50.1960784313725!black, semithick]
(axis cs:90,53.0442477876106)
--(axis cs:110,53.0442477876106);

\path [draw=green!50.1960784313725!black, semithick]
(axis cs:140,95.1428571428572)
--(axis cs:160,95.1428571428572);

\path [draw=green!50.1960784313725!black, semithick]
(axis cs:190,148)
--(axis cs:210,148);

\path [draw=green!50.1960784313725!black, semithick]
(axis cs:240,188.913461538462)
--(axis cs:260,188.913461538462);

\path [draw=green!50.1960784313725!black, semithick]
(axis cs:290,233.740384615385)
--(axis cs:310,233.740384615385);

\path [draw=green!50.1960784313725!black, semithick]
(axis cs:340,270.260869565217)
--(axis cs:360,270.260869565217);

\path [draw=green!50.1960784313725!black, semithick]
(axis cs:390,326.204081632653)
--(axis cs:410,326.204081632653);

\path [draw=green!50.1960784313725!black, semithick]
(axis cs:440,360)
--(axis cs:460,360);

\path [draw=green!50.1960784313725!black, semithick]
(axis cs:490,396)
--(axis cs:510,396);

\path [draw=green!50.1960784313725!black, semithick]
(axis cs:540,468)
--(axis cs:560,468);

\path [draw=green!50.1960784313725!black, semithick]
(axis cs:100,52.4114526583863)
--(axis cs:100,53.6770429168349);

\path [draw=green!50.1960784313725!black, semithick]
(axis cs:150,91.0666302830339)
--(axis cs:150,99.2190840026804);

\path [draw=green!50.1960784313725!black, semithick]
(axis cs:200,141.683425571105)
--(axis cs:200,154.316574428895);

\path [draw=green!50.1960784313725!black, semithick]
(axis cs:250,182.271308845487)
--(axis cs:250,195.555614231436);

\path [draw=green!50.1960784313725!black, semithick]
(axis cs:300,223.88999208276)
--(axis cs:300,243.590777148009);

\path [draw=green!50.1960784313725!black, semithick]
(axis cs:350,257.145804132173)
--(axis cs:350,283.375934998262);

\path [draw=green!50.1960784313725!black, semithick]
(axis cs:400,311.969339856676)
--(axis cs:400,340.438823408631);

\path [draw=green!50.1960784313725!black, semithick]
(axis cs:450,341.133063042629)
--(axis cs:450,378.866936957371);

\path [draw=green!50.1960784313725!black, semithick]
(axis cs:500,376.956019710008)
--(axis cs:500,415.043980289992);

\path [draw=green!50.1960784313725!black, semithick]
(axis cs:550,448.557402072788)
--(axis cs:550,487.442597927212);

\addplot [semithick, green!50.1960784313725!black, forget plot]
table {%
0 0
550 439.564038903656
};
\addplot [semithick, red, mark=*, mark size=3, mark options={solid}, only marks]
table {%
50 32.9861111111111
100 98.9583333333333
150 131.944444444444
200 164.930555555556
250 197.916666666667
300 230.902777777778
350 263.888888888889
400 296.875
450 362.847222222222
500 395.833333333333
};
\addlegendentry{Линза}
\addplot [semithick, green!50.1960784313725!black, mark=*, mark size=3, mark options={solid}, only marks]
table {%
100 53.0442477876106
150 95.1428571428572
200 148
250 188.913461538462
300 233.740384615385
350 270.260869565217
400 326.204081632653
450 360
500 396
550 468
};
\addlegendentry{Спектр}
\end{axis}

\end{tikzpicture}

                    \caption{Сравнение измерений ширины щели.}
                \end{center}
            \end{figure}
        \end{minipage}
        \par С помощью микроскопа определим ширину щели $b=0.24 \pm 0.02$ мм.
        \par Сравним размер зон Френеля с измеренной шириной $b$ щели $S_2$. Для этого свяжем число тёмных полос $m$ в поле зрения с числом зон Френеля $n$ на полуширине щели, рассчитаем величину $2\xi_n$ по формуле (1) и построем график $2\xi_n=f(n)$ (рис. 3). Отложим на графике величину $b$. Длина волны $\lambda=578$ нм.
        \par Таким образом мы видим, что вычисленный размер щели совпадает с измеренным в предела погрешности.

    \section{Дифракция Фраунгофера на щели}

        \subsection{Экспериментальная установка}
        \par На значительном удалении от щели, когда выполнено условие $C\ll1$ (то есть ширина щели становится значительно меньше ширины первой зоны Френеля, $b\ll\lambda z$), изображение щели размывается и возникает дифракционная картина, называемая дифракцией Фраунгофера.
        \par Дифракцию Френеля и Фраунгофера можно наблюдать на одной и той же установке (рис. 1). Однако при обычных размерах установки дифракция Фраунгофера возникает только при очень узких щелях. Например, при $z\approx20-40$ см и $\lambda\approx5\cdot10^{-5}$ см получаем $b\ll0.3$ мм. Поскольку работать с такими тонкими щелями неудобно, для наблюдения дифракции Фраунгофера к схеме, изображённой на рис. 1, добавляется объектив $O_2$ (рис. 3).
        \begin{figure}[H]
            \begin{center}
                \includegraphics[width=0.9\textwidth]{images/pic4.pdf}
                \caption{Схема установки для наблюдения дифракции Фраунгофера на щели.}
            \end{center}
        \end{figure}
        \par Дифракционная картина наблюдается здесь в фокальной плоскости объектива $O_2$. Каждому значению угла $\theta$ соответствует в этой плоскости точка, отстоящая от оптической оси на расстоянии
        \begin{equation}
            x=f_2\tg{\theta}\approx f_2\theta.
        \end{equation}
        \par Поскольку объектив не вносит дополнительной разности хода между интерферирующими лучами (таутохронизм), в его фокальной плоскости наблюдается неискажённая дифракционная картина Фраунгофера. Эта картина соответствует бесконечно удалённой плоскости наблюдения.
        \par В центре поля зрения наблюдается дифракционный максимум (светлая полоса). При малых углах $\theta$ положение минимумов (тёмных полос) определяется соотношением
        \begin{equation}
            \theta_m=m\frac{\lambda}{b}.
        \end{equation}
        \par Расстояние $x_m$ от тёмной полосы до оптической оси объектива $O_2$ пропорционально фокусному расстоянию $f_2$. Из (2) и (3) следует
        \begin{equation}
            x_m=m\frac{\lambda}{b}f_2.
        \end{equation}
        \par Из (4) видно, что при малых углах минимумы эквидистантны, а расстояния $\delta x$ между минимумами обратно пропорциональны ширине $b$ щели $S_2$.

        \subsection{Измерения и обработка результатов}
        \par Подобрав ширину щели $S_2$ так, чтобы была видна дифракционная картина, измерим расстояние между минимумами и запишем их порядок. Данные занесём в таблицу 2. Ширина щели составила $b=312 \pm 1$ мкм. Цена деления микроскопа равна 0.02 мм, погрешность --- одно деление. Фокусное расстояние $O_1$ равно $f_1=110$ мм, а $O_2$ --- $f_2=155$ мм.
        \begin{table}[H]
            \captionsetup{font={small}, labelformat=fullparents, labelsep=fill, labelfont=bf, justification=raggedleft,
                        singlelinecheck=false, skip=-0.2cm}
            \caption{Положение минимумов дифракционной картины.}
            \begin{center}
                \begin{tabular}{|l|l|l|l|l|l|l|l|l|}\hline$x$, дел&8&25&36&52&-8&-24&-36&-53
\\\hline$m $&1&2&3&4&-1&-2&-3&-4
\\\hline\end{tabular}
            \end{center}
        \end{table}
        \par Построим график зависимости координаты минимума от его порядка (рис. 5). С помощью МНК определим угловой коэффициент аппроксимирующей прямой $k=250.00\pm0.06$ мкм. Значит из формулы (4) имеем $b_{\text{эксп}}=358.36 \pm 0.08$ мкм. Как видим, результаты в целом совпали, отклонение можно объянисть люфтом микрометрического винта и неточным определением нулевого положения.
        \begin{figure}[H]
            \begin{center}
                % This file was created with tikzplotlib v0.9.16.
\begin{tikzpicture}

\begin{axis}[
    height=11cm,
    tick align=inside,
    major tick length=0.2cm,
    minor tick length=0.1cm,
    tick pos=left,
    xmin=-4.1, xmax=4.1,
    xtick={-5, -4, -3, -2, -1, 0, 1, 2, 3, 4, 5},
    minor x tick num=4,
    xmajorgrids,
    minor x grid style={dotted,black},
    xminorgrids,
    xtick style={color=black},
    xlabel={$m$},
    ymin=-1.2, ymax=1.2,
    ytick={-1.5, -1, -0.5, 0, 0.5, 1, 1.5},
    minor y tick num=4,
    ymajorgrids,
    minor y grid style={dotted,black},
    yminorgrids,
    ytick style={color=black},
    ylabel={$x_m$, мм}
]
\path [draw=red, semithick]
(axis cs:1,0.14)
--(axis cs:1,0.18);

\path [draw=red, semithick]
(axis cs:2,0.48)
--(axis cs:2,0.52);

\path [draw=red, semithick]
(axis cs:3,0.7)
--(axis cs:3,0.74);

\path [draw=red, semithick]
(axis cs:4,1.02)
--(axis cs:4,1.06);

\path [draw=red, semithick]
(axis cs:-1,-0.18)
--(axis cs:-1,-0.14);

\path [draw=red, semithick]
(axis cs:-2,-0.5)
--(axis cs:-2,-0.46);

\path [draw=red, semithick]
(axis cs:-3,-0.74)
--(axis cs:-3,-0.7);

\path [draw=red, semithick]
(axis cs:-4,-1.08)
--(axis cs:-4,-1.04);

\addplot [semithick, black]
table {%
1 0.249999999998364
2 0.499999999996728
3 0.749999999995093
4 0.999999999993457
-1 -0.249999999998364
-2 -0.499999999996728
-3 -0.749999999995093
-4 -0.999999999993457
};
\addplot [semithick, red, mark=square*, mark size=3, mark options={solid}, only marks]
table {%
1 0.16
2 0.5
3 0.72
4 1.04
-1 -0.16
-2 -0.48
-3 -0.72
-4 -1.06
};
\end{axis}

\end{tikzpicture}

                \caption{График зависимости $x_m$ от $m$.}
            \end{center}
        \end{figure}

    \section{Дифракция Фраунгофера на двух щелях}

        \subsection{Экспериментальная установка}
        \par Для наблюдения дифракции Фраунгофера на двух щелях в установке (рис. 4) следует заменить щель $S_2$ экраном Э с двумя щелями (рис. 6). При этом для оценки влияния ширины входной щели на чёткость дифракционной картины вместо входной щели S1 следует поставить щель с микрометрическим винтом. Два дифракционных изображения входной щели, одно из которых образовано лучами, прошедшими через левую, а другое --- через правую щели, накладываются друг на друга.
        \begin{figure}[H]
            \begin{center}
                \includegraphics[width=0.9\textwidth]{images/pic6.pdf}
                \caption{Схема установки для наблюдения дифракции Фраунгофера на двух щелях.}
            \end{center}
        \end{figure}
        \par Если входная щель достаточно узка, то дифракционная картина в плоскости П (рис. 6) подобна той, что получалась при дифракции на одной щели (рис. 4), однако теперь вся картина испещрена рядом дополнительных узких полос.
        \par Угловая координата $\theta_m$ интерференционного максимума $m$-го порядка определяется соотношением
        \begin{equation}
            \theta_m=m\frac{\lambda}{d},
        \end{equation}
        где $d$ --- расстояние между щелями. Линейное расстояние $\delta x$ между соседними интерференционными полосами в плоскости П равно, поэтому
        \begin{equation}
            \delta x=f_2\frac{\lambda}{d}.
        \end{equation}
        \par На рис. 6 показано распределение интенсивности в фокальной плоскости объектива $O_2$. Штриховой линией (в увеличенном масштабе) изображено распределение интенсивности при дифракции света на одиночной щели. Нетрудно оценить число $n$ интерференционных полос, укладывающихся в области центрального дифракционного максимума. Согласно (4) полная ширина главного максимума равна $2f_2\lambda/b$, где $b$ --- ширина щели, отсюда
        \begin{equation}
            n=\frac{2\lambda f_2}{b}\frac{1}{\delta x}=\frac{2d}{b}.
        \end{equation}
        \par При дифракции света на двух щелях чёткая система интерференционных полос наблюдается только при достаточно узкой ширине входной щели $S$, которую можно рассматривать как протяжённый источник света размером $b$. Для наблюдения интерференции необходимо, чтобы расстояние $d$ между щелями не превышало радиуса когерентности
        \begin{equation}
            d\leq\frac{\lambda}{b}f_1.
        \end{equation}
        Здесь $b$ --- ширина входной щели $S$ и, следовательно, $b/f_1$ --- её угловая ширина. Таким образом, по размытию интерференционной картины можно оценить размер источника. Этот метод используется в звёздном интерферометре при измерении угловых размеров звёзд.

        \subsection{Измерения и обработка результатов}
        \par Определим с помощью микроскопа координаты самых удалённых друг от друга тёмных полос внутри центрального максимума и просчитаем число светлых промежутков между ними. Получим $\Delta X=1.44\pm0.02$ мм, $m=20$. Откуда $\delta x=0.072 \pm 0.001$ Ширина щели равна $b=121\pm1$ мкм.
        \par По формуле (6) рассчитаем величину $d=1.24 \pm 0.02$ мм. Отсюда и из (7) получим число полос внутри главного максимума $n=20$, что согласуется с наблюдением.
        \par Дифракционная картина размывается при ${b_0}_{\text{эксп}}=47\pm1$ мкм. Из теоретического расчёта получим $b_0=\lambda f_1/d=51\pm1$ мкм.

    \section{Влияние дифракции на разрешающую способность оптического инструмента}

        \subsection{Экспериментальная установка}
        \par Установка, представленная на рис. 7, позволяет исследовать влияние дифракции на разрешающую способность оптических инструментов.
        \begin{figure}[H]
            \begin{center}
                \includegraphics[width=0.9\textwidth]{images/pic7.pdf}
                \caption{Схема установки для исследования разрешающей способности оптического инструмента.}
            \end{center}
        \end{figure}
        \par Как уже было выяснено, линзы $O_1$ и $O_2$ в отсутствие щели $S_2$ создают в плоскости П изображение щели $S_1$, и это изображение рассматривается в микроскоп М. Таким образом, нашу установку можно рассматривать как оптический инструмент, предназначенный для получения изображения предмета. При этом коллиматор (щель $S_1$ и объектив $O_1$) является моделью далёкого предмета, а объектив $O_2$ и микроскоп М составляют зрительную трубу, наведённую на этот предмет.
        \par Щель $S_2$, установленная непосредственно перед объективом $O_2$, позволяет изменять эффективный размер объектива и, следовательно, разрешающую способность оптической системы.
        \par Поместим вместо щели $S_1$ экран Э с двумя узкими щелями, расстояние между которыми равно $d$ (рис. 7). Тогда расстояние $l$ между изображениями щелей в плоскости П равно
        \begin{equation}
            l=\phi f_2=\frac{f_1}{f_2}d,
        \end{equation}
        а ширина каждого изображения
        \begin{equation*}
            \delta x\approx\frac{\lambda}{b}f_2
        \end{equation*}
        определяется дифракцией света на щели $S_2$. Когда полуширина дифракционного изображения превышает расстояние между изображениями, то по виду дифракционной картины трудно определить, представляет собой источник двойную или одиночную щель.
        \par Условия, при которых ещё можно различить, имеем мы дело с одной или двумя щелями, для разных наблюдателей различны. Для того чтобы исключить связанный с этим произвол, пользуются обычно критерием Рэлея, который приблизительно соответствует возможностям визуального наблюдения: изображения считаются различимыми, когда максимум одного дифракционного пятна совпадает с минимумом другого, а в условиях нашей задачи --- когда полуширина дифракционного изображения $\delta x$ совпадает с расстоянием $l$ между изображениями отдельных щелей:
        \begin{equation}
            \delta x\sim l\rightarrow\frac{\lambda}{b}\sim\frac{d}{f_1}.
        \end{equation}

        \subsection{Измерения и обработка результатов}
        \par С помощью микроскопа измерим к ширину каждой из двух щелей и расстояние между ними. Получим $b_1=0.16\pm0.02$ мм, $b_1=0.26\pm0.02$ мм и $d=1.26\pm0.02$ мм. Отсюда получим $b_0=\lambda f_1/d=51\pm1$ мкм.

    \section{Обсуждение результатов}
        \par Результат неплохо согласуются между собой и с теорией. Основным источником погрешностей можно считать микрометрический винт, так как у него может присутсвовать люфт и неправильно быть определён нулевой уровень.

    \section{Вывод}
        \par Работу можно выполнить точнее, если использовать более точный способ определение ширины щели (можно мерить её каждый раз микроскпом).

    \newpage\pagenumbering{gobble}

    \begin{figure}[H]
        \begin{center}
            \includegraphics[width=1.5\linewidth, angle=90]{data/approved.pdf}
        \end{center}
    \end{figure}

\end{document}

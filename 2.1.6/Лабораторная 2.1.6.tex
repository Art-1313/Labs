\documentclass[12pt,a4paper]{article}
\usepackage[utf8]{inputenc}
\usepackage[english,russian]{babel}
\usepackage{indentfirst}
\usepackage{misccorr}
\usepackage{graphicx}
\usepackage{amsmath}
\usepackage{amssymb}
\usepackage{circuitikz}
\usepackage[font={small}]{caption}
\usepackage[left=20mm, top=20mm, right=20mm, bottom=20mm, nohead]{geometry}
\usepackage{float}
\usepackage{tabularx}
\usepackage{array}
\usepackage{longtable}
\usepackage{pstool}
\usepackage{pgfplots}
\usepackage{hhline}
\usepackage{multirow}

\DeclareCaptionLabelSeparator{fill}{.\\}
\DeclareCaptionLabelFormat{fullparents}{\bothIfFirst{#1}{~}#2}

\begin{document}

	\begin{titlepage}
		\begin{center}
			{\LARGE Отчёт по лабораторной работе 2.1.6.\\}
			\vspace*{11cm}
				\textbf{\LARGE Эффект Джоуля-Томпсона.}	
			\vspace*{6.5cm}
		\end{center}
		\hfill\begin{minipage}{0.37\textwidth}
				Работу выполнил Громов Артём\\
				ЛФИ Б02-006
		\end{minipage}
		\vspace{5cm}
		\begin{center}
 			Долгопрудный, 2021 г.
		\end{center}   
	\end{titlepage}
	
	\section{Аннотация}
		\noindent\textbf{Цель работы: }
		\begin{enumerate}
			\item определение изменения температуры углекислого газа при протекании через малопроницаемую перегородку
			при разных начальных значениях давления и температуры;
			\item вычисление по результатам опытов коэффициентов Ван-дер-Ваальса «$a$» и «$b$».
		\end{enumerate}
		\noindent\textbf{В работе используются: } трубка с пористой перегородкой, труба Дьюара, термостат, термометры,
		дифференциальная термопара, микровольтметр, балластный баллон, манометр.
		\vspace{0.5 cm}
		\par Эффектом Джоуля–Томсона называется изменение температуры газа, медленно протекающего из области высокого 
		в область низкого давления в условиях хорошей тепловой изоляции. В разреженных газах, которые приближаются по
		своим свойствам к идеальному газу, при таком течении температура газа не меняется. Эффект Джоуля–Томсона
		демонстрирует отличие исследуемого газа от идеального.
		\par В работе исследуется изменение температуры углекислого газа при медленном его течении по трубке с пористой
		перегородкой (рисунок 1). Трубка 1 хорошо теплоизолирована. Газ из области повышенного давления $P_1$ проходит
		через множество узких и длинных каналов пористой перегородки 2 в область с атмосферным давлением $P_{2}$.
		Перепад давления $\Delta P=P_{1}-P_{2}$ из-за большого сопротивления каналов может быть заметным даже при
		малой скорости течения газа в трубке. Величина эффекта Джоуля–Томсона определяется по разности температуры газа
		до и после перегородки.
		\par Рассмотрим стационарный поток газа между произвольными сечениями $I$ и $II$ трубки (до перегородки и после
		нее). Пусть, для определенности, через трубку прошел 1 моль углекислого газа; $\mu$ --- его молярная масса. Молярные
		объемы газа, его давления и отнесенные к молю внутренние энергии газа в сечениях $I$ и $II$ обозначим
		соответственно $V_1$, $P_1$, $U_1$ и $V_2$, $P_2$, $U_2$. Для того чтобы ввести в трубку объем $V_1$, над газом
		нужно совершить работу $A_{1}=P_{1}V_{1}$. Проходя через сечение $II$, газ сам совершает работу 
		$A_{2} = P_{2}V_{2}$. Так как через боковые стенки не происходит ни обмена теплом, ни передачи механической
		энергии, то:
		\begin{equation}
			A_{1}-A_{2}=\left(U_{2}+\frac{\mu v^{2}_{2}}{2}\right)-\left(U_{1}+\frac{\mu v^{2}_{1}}{2}\right).
		\end{equation}
		\noindent В уравнении (1) учтено изменение как внутренней (первые члены в скобках), так и кинетической (вторые
		члены в скобках) энергии газа. Подставляя в (1) написанные выражения для $A_{1}$ и $A_{2}$ и перегруппировывая
		члены, найдем:
		\begin{equation}
			H_{1}-H_{2}=\left(U_{1}+P_{1}V_{1}\right)-\left(U_{2}+P_{2}V_{2}\right)=
			\frac{1}{2}\mu\left(v^{2}_{2}-v^{2}_{1}\right).
		\end{equation}
		\par Сделаем несколько замечаний. Прежде всего отметим, что в процессе Джоуля–Томсона газ испытывает в пористой
		перегородке существенное трение, приводящее к ее нагреву. Потери энергии на нагрев трубки в начале процесса могут
		быть очень существенными и сильно искажают ход явления. После того как температура трубки установится и газ
		станет уносить с собой все выделенное им в пробке тепло, формула (1) становится точной, если, конечно,
		теплоизоляция трубки достаточно хороша и не происходит утечек тепла наружу через её стенки.
		\par Второе замечание связано с правой частью (2). Процесс Джоуля–Томсона в чистом виде осуществляется лишь в
		том случае, если правой частью можно пренебречь, т. е. если макроскопическая скорость газа с обеих сторон трубки
		достаточно мала. У нас сейчас нет критерия, который позволил бы установить, когда это можно сделать. Поэтому мы
		отложим на некоторое время обсуждение вопроса о правой части (2), а пока будем считать, что энтальпия газа не
		меняется.
		\par Воспользуемся следующей формулой:
		\begin{equation}
			\mu_{\text{д-т}}=\frac{2a/RT-b}{C_{p}}
		\end{equation}
		\par Из формулы (3) видно, что эффект Джоуля–Томсона для не очень плотного газа зависит от соотношения величин
		$a$ и $b$, которые оказывают противоположное влияние на знак эффекта. Если силы взаимодействия между
		молекулами велики, так что превалирует «поправка на давление», то основную роль играет член, содержащий $a$, и
		$$\frac{\Delta T}{\Delta P}<0,$$
		\noindent т. е. газ при расширении охлаждается ($\Delta T<0$, так как всегда $\Delta P<0$). В обратном случае (малые
		$a$)
		$$\frac{\Delta T}{\Delta P}>0,$$
		\noindent т. е. газ нагревается ($\Delta T>0$, так как по-прежнему $\Delta P<0$).
		\par Этот результат нетрудно понять из энергетических соображений. Как мы уже знаем, у идеального газа эффект
		Джоуля–Томсона отсутствует. Идеальный газ отличается от реального тем, что в нем можно пренебречь
		потенциальной энергией взаимодействия молекул. Наличие этой энергии приводит к охлаждению или нагреванию
		реальных газов при расширении. При больших $a$ велика энергия притяжения молекул. Это означает, что
		потенциальная энергия молекул при их сближении уменьшается, а при удалении --- при расширении газа --
		возрастает. Возрастание потенциальной энергии молекул происходит за счет их кинетической энергии --- температура
		газа при расширении падает. Аналогичные рассуждения позволяют понять, почему расширяющийся газ нагревается
		при больших значениях $b$.
		\par Как следует из формулы (3), при температуре $T_{\text{инв}}=\frac{2a}{Rb}$ коэффициент $\mu_{\text{д-т}}$
		обращается в нуль. Используя связь между коэффициентами $a$ и $b$ и критической температурой, получим:
		\begin{equation}
			T_{\text{инв}}=\frac{27}{4}T_{\text{кр}}.
		\end{equation}
		\noindent При температуре $T_{\text{инв}}$ эффект Джоуля–Томсона меняет знак: ниже температуры инверсии
		эффект положителен ($\mu_{\text{д-т}} > 0$, газ охлаждается), выше $T_{\text{инв}}$ эффект отрицателен
		($\mu_{\text{д-т}} < 0$, газ нагревается).
		\par Температура инверсии у всех газов лежит значительно выше критической. Для большинства газов 
		$T_{\text{инв}}/T_{\text{кр}}= 5--8$. Например, для гелия $T_{\text{инв}} = 46$ К, $T_{\text{кр}}= 5.2$ К; для
		водорода $T_{\text{инв}} = 205$ К, $T_{\text{кр}} = 33$ К; для азота $T_{\text{инв}} = 604$ К, $T_{\text{кр}}=126$ К;
		для воздуха $T_{\text{инв}}=650 К$, $T_{\text{кр}}=132.6$ К; для углекислого газа $T_{\text{инв}}= 2050$ К,
		$T_{\text{инв}}= 304$ К. Температура инверсии у гелия и водорода значительно ниже комнатной, поэтому при
		обычных температурах эти газы при расширении нагрeваются. Температура инверсии остальных газов выше
		комнатной, и при нормальных условиях температура при расширении газа падает.
		\par Сравнивая приведенные значения $T_{\text{инв}}$ и $T_{\text{кр}}$, можно убедиться в том, что предсказания,
		следующие из формулы Ван-дер-Ваальса, у реальных газов выполняются не очень хорошо. Правильно передавая
		качественную картину поведения реальных газов, формула Ван-дер-Ваальса не претендует на хорошее
		количественное описание этой картины.
		\par Вернемся к влиянию правой части уравнения (2) на изменение температуры расширяющегося газа. Для этого
		сравним изменение температуры, происходящее вследствие эффекта Джоуля–Томсона, с изменением температуры,
		возникающим из-за изменения кинетической энергии газа. Увеличение кинетической энергии газа вызывает заметное
		и приблизительно одинаковое понижение его температуры как у реальных, так и у идеальных газов. Поэтому при
		оценках нет смысла пользоваться сложными формулами для газа Ван-дер-Ваальса.
		\par Заменя в формуле (2) $U$ через $C_{V}T$ и $PV$ через $RT$, найдём:
		$$\left(R+C_{V}\right)\left(T_{1}+T_{2}\right)=\mu\left(v^{2}_{2}-v^{2}_{1}\right)/2,$$
		или
		$$\Delta T=\frac{\mu}{2C_{p}}\left(v^{2}_{2}-v^{2}_{1}\right).$$
		\noindent В условиях нашего опыта расход газа $Q$ на выходе из пористой перегородки не превышает 
		$10\text{ см}^{3}\text{с}$, а диаметр трубки равен 3 мм. Поэтому
		$$v_{2}\leqslant\frac{4Q}{\pi d^{2}}\approx140\text{ см/с}.$$
		Скорость $v_{1}$ газа у входа в пробирку относится к скорости $v_{2}$ у выхода из неё как давление $P_{1}$
		относится к давления $P_{2}$. В нашей установке $P_{1}=4$ атм, а $P_{2}=1$ атм, поэтому
		$$v_{2}=\frac{P_{2}}{P_{1}}v_{2}=35\text{ см/с}.$$
		Для углекислого газа $\mu=44$ г/моль, $C_{P}=40$ Дж/($\text{моль}\cdot\text{К}$); имеем
		$$\Delta T=\frac{\mu}{2C_{p}}\left(v^{2}_{2}-v^{2}_{1}\right)=7\cdot10^{-4}\text{ К}.$$
		Это изменение температуры ничтожно мало по сравнению с измеряемым эффектом (несколько градусов).
		\par В данной лабораторной работе исследуется коэффициент дифференциального эффекта Джоуля–Томсона для
		углекислого газа. По экспериментальным результатам оценивается коэффициент теплового расширения, постоянные в
		уравнении Ван-дер-Ваальса и температура инверсии углекислого газа. Начальная температура газа $T_{1}$ задается
		термостатом. Измерения проводятся при трех температурах: комнатной, 30 ◦C и 50 ◦C.
	
	\section{Экспериментальная установка}
		\par Схема установки для исследования эффекта Джоуля–Томсона в углекислом газе представлена на рисунке 1.
		Основным элементом установки является трубка 1 с пористой перегородкой 2, через которую пропускается
		исследуемый газ. Трубка имеет длину 80 мм и сделана из нержавеющей стали, обладающей, как известно, малой
		теплопроводностью. Диаметр трубки $d=3$ мм, толщина стенок 0.2 мм. Пористая перегородка расположена в конце
		трубки и представляет собой стеклянную пористую пробку со множеством узких и длинных каналов. Пористость и
		толщина пробки ($l=5$ мм) подобраны так, чтобы обеспечить оптимальный поток газа при перепаде давлений
		$\Delta P \leqslant 4$ атм (расход газа составляет около $10 \text{ см}^{3}\text{/с}$); при этом в результате эффекта
		Джоуля–Томсона создается достаточная разность температур.
		\begin{figure}[H]
			\begin{center}
				\includegraphics[width=\textwidth]{C:/Users/gromo/Desktop/Установка216.png}
				\caption{Схема установки для изучения эффекта Джоуля-Томсона}
			\end{center}
		\end{figure}
		\par Углекислый газ под повышенным давлением поступает в трубку через змеевик 5 из балластного баллона 6.
		Медный змеевик омывается водой и нагревает медленно протекающий через него газ до температуры воды в
		термостате. Температура воды измеряется термометром $T_{\text{в}}$, помещенным в термостате. Требуемая
		температура воды устанавливается и поддерживается во время эксперимента при помощи контактного термометра
		$T_{\text{к}}$.
		\par Давление газа в трубке измеряется манометром М и регулируется вентилем В (при открывании вентиля В, т. е.
		при повороте ручки против часовой стрелки, давление $P_{1}$ повышается). Манометр М измеряет разность между
		давлением внутри трубки и наружным (атмосферным) давлением. Так как углекислый газ после пористой 
		перегородки выходит в область с атмосферным давлением $P_{1}$, то этот манометр непосредственно измеряет
		перепад давления на входе и на выходе трубки $\Delta P=P_{1}-P_{2}.$
		\par Разность температур газа до перегородки и после нее измеряется дифференциальной термопарой медь ---
		константан. Константановая проволока диаметром 0.1 мм соединяет спаи 8 и 9, а медные проволоки (того же
		диаметра) подсоединены к цифровому вольтметру 7. Отвод тепла через проволоку столь малого сечения
		пренебрежимо мал. Для уменьшения теплоотвода трубка с пористой перегородкой помещена в трубу Дьюара 3, стенки
		которой посеребрены, для уменьшения теплоотдачи, связанной с излучением. Для уменьшения теплоотдачи за счет
		конвекции один конец трубы Дьюара уплотнен кольцом 4, а другой закрыт пробкой 10 из пенопласта. Такая пробка
		практически не создает перепада давлений между внутренней полостью трубы и атмосферой.
	
	\section{Результаты измерений и обработка данных}
		\subsection{Измерение коэффициента Джоуля-Томсона}
		\par Для определения коэффициента Джоуля Томпсона нужно определить коэффициент угла наклона графика
		$\Delta T\left(\Delta P\right)$. Для этого проведём измерения разности температур при разных перепадах давления.
		Разность температур определим через показания вольтметра по формуле:
		$\Delta T=(U(\Delta P)-U(0))/k(T_{\text{в}})$, где $k(T_{\text{в}})$ --- коэффициент перевода
		напряжения на термопаре в температуру; $U(0)$ --- напряжение при нулевой разности давлений. Погрешность
		измерения перепада давлений сотавляет 0.06 атм, а погрешность измерения напряжения составляет 1 мкВ
		(следовательно погрешность измерения разности температур --- 0.02 ℃).
		Занесём данные в таблицу 1.
		\begin{table}[H]
			\captionsetup{font={small}, labelformat=fullparents, labelsep=fill, labelfont=bf, justification=raggedleft,
			singlelinecheck=false, skip=-0.2cm}
			\caption{Результаты измерений}
			\begin{center}
				\begin{tabular}{|>{\centering}p{2.1cm}|>{\centering}p{1.8cm}
				 |>{\centering}p{1.8cm}|>{\centering}p{1.8cm}
				 |>{\centering}p{1.8cm}|>{\centering}p{1.8cm}
				 |>{\centering}p{1.8cm}|}
				 \hline
				 \multicolumn{7}{|c|}{$T=20.1$ ℃,  $U(0) = -12$ мкВ, $k=40.7$ мкВ/℃}\\
				 \hline 
				 \multicolumn{1}{|p{2.1cm}|}{$\Delta P$, дел} & \multicolumn{1}{p{1.9cm}|}{66} 
				 & \multicolumn{1}{p{1.9cm}|}{58} & \multicolumn{1}{p{1.9cm}|}{49.5} 
				 & \multicolumn{1}{p{1.9cm}|}{44} & \multicolumn{1}{p{1.9cm}|}{33} 
				 & \multicolumn{1}{p{1.9cm}|}{25} \\
				 \hline 
				 \multicolumn{1}{|p{2.1cm}|}{$\Delta P$, атм} & \multicolumn{1}{p{1.9cm}|}{3.83} 
				 & \multicolumn{1}{p{1.9cm}|}{3.37} & \multicolumn{1}{p{1.9cm}|}{2.87} 
				 & \multicolumn{1}{p{1.9cm}|}{2.56} & \multicolumn{1}{p{1.9cm}|}{1.92} 
				 & \multicolumn{1}{p{1.9cm}|}{1.45} \\
				 \hline 
				 \multicolumn{1}{|p{2.1cm}|}{$U(P)$, мкВ} & \multicolumn{1}{p{1.9cm}|}{-159} 
				 & \multicolumn{1}{p{1.9cm}|}{-134} & \multicolumn{1}{p{1.9cm}|}{-112} 
				 & \multicolumn{1}{p{1.9cm}|}{-91} & \multicolumn{1}{p{1.9cm}|}{-67} 
				 & \multicolumn{1}{p{1.9cm}|}{-47} \\
				 \hline
				 \multicolumn{1}{|p{2.1cm}|}{$\Delta U$, мкВ} & \multicolumn{1}{p{1.9cm}|}{-159} 
				 & \multicolumn{1}{p{1.9cm}|}{-134} & \multicolumn{1}{p{1.9cm}|}{-112} 
				 & \multicolumn{1}{p{1.9cm}|}{-91} & \multicolumn{1}{p{1.9cm}|}{-67} 
				 & \multicolumn{1}{p{1.9cm}|}{-47} \\
				 \hline
				 \multicolumn{1}{|p{2.1cm}|}{$\Delta T$, ℃} & \multicolumn{1}{p{1.9cm}|}{-3.91} 
				 & \multicolumn{1}{p{1.9cm}|}{-3.29} & \multicolumn{1}{p{1.9cm}|}{-2.75} 
				 & \multicolumn{1}{p{1.9cm}|}{-2.24} & \multicolumn{1}{p{1.9cm}|}{-1.65} 
				 & \multicolumn{1}{p{1.9cm}|}{-1.15} \\
				 \hline
				 \multicolumn{7}{|c|}{$T=30.1$ ℃,  $U(0) = -12$ мкВ, $k=41.6$ мкВ/℃}\\
				 \hline 
				 \multicolumn{1}{|p{2.1cm}|}{$\Delta P$, дел} & \multicolumn{1}{p{1.9cm}|}{66} 
				 & \multicolumn{1}{p{1.9cm}|}{57} & \multicolumn{1}{p{1.9cm}|}{49} 
				 & \multicolumn{1}{p{1.9cm}|}{44} & \multicolumn{1}{p{1.9cm}|}{33} 
				 & \multicolumn{1}{p{1.9cm}|}{25} \\
				 \hline 
				 \multicolumn{1}{|p{2.1cm}|}{$\Delta P$, атм} & \multicolumn{1}{p{1.9cm}|}{3.83} 
				 & \multicolumn{1}{p{1.9cm}|}{3.31} & \multicolumn{1}{p{1.9cm}|}{2.85} 
				 & \multicolumn{1}{p{1.9cm}|}{2.38} & \multicolumn{1}{p{1.9cm}|}{1.92} 
				 & \multicolumn{1}{p{1.9cm}|}{1.45} \\
				 \hline 
				 \multicolumn{1}{|p{2.1cm}|}{$U(P)$, мкВ} & \multicolumn{1}{p{1.9cm}|}{-147} 
				 & \multicolumn{1}{p{1.9cm}|}{-124} & \multicolumn{1}{p{1.9cm}|}{-102} 
				 & \multicolumn{1}{p{1.9cm}|}{-84} & \multicolumn{1}{p{1.9cm}|}{-68} 
				 & \multicolumn{1}{p{1.9cm}|}{-49} \\
				 \hline
				 \multicolumn{1}{|p{2.1cm}|}{$\Delta U$, мкВ} & \multicolumn{1}{p{1.9cm}|}{-143} 
				 & \multicolumn{1}{p{1.9cm}|}{-120} & \multicolumn{1}{p{1.9cm}|}{-98} 
				 & \multicolumn{1}{p{1.9cm}|}{-80} & \multicolumn{1}{p{1.9cm}|}{-64} 
				 & \multicolumn{1}{p{1.9cm}|}{-45} \\
				 \hline
				 \multicolumn{1}{|p{2.1cm}|}{$\Delta T$, ℃} & \multicolumn{1}{p{1.9cm}|}{-3.44} 
				 & \multicolumn{1}{p{1.9cm}|}{-2.88} & \multicolumn{1}{p{1.9cm}|}{-2.36} 
				 & \multicolumn{1}{p{1.9cm}|}{-1.92} & \multicolumn{1}{p{1.9cm}|}{-1.54} 
				 & \multicolumn{1}{p{1.9cm}|}{-1.08} \\
				 \hline
				 \multicolumn{7}{|c|}{$T=50.0$ ℃,  $U(0) = -12$ мкВ, $k=43.3$ мкВ/℃}\\
				 \hline 
				 \multicolumn{1}{|p{2.1cm}|}{$\Delta P$, дел} & \multicolumn{1}{p{1.9cm}|}{66} 
				 & \multicolumn{1}{p{1.9cm}|}{58} & \multicolumn{1}{p{1.9cm}|}{49} 
				 & \multicolumn{1}{p{1.9cm}|}{44} & \multicolumn{1}{p{1.9cm}|}{33} 
				 & \multicolumn{1}{p{1.9cm}|}{26} \\
				 \hline 
				 \multicolumn{1}{|p{2.1cm}|}{$\Delta P$, атм} & \multicolumn{1}{p{1.9cm}|}{3.83} 
				 & \multicolumn{1}{p{1.9cm}|}{3.37} & \multicolumn{1}{p{1.9cm}|}{2.85} 
				 & \multicolumn{1}{p{1.9cm}|}{2.56} & \multicolumn{1}{p{1.9cm}|}{1.92} 
				 & \multicolumn{1}{p{1.9cm}|}{1.51} \\
				 \hline 
				 \multicolumn{1}{|p{2.1cm}|}{$U(P)$, мкВ} & \multicolumn{1}{p{1.9cm}|}{-131} 
				 & \multicolumn{1}{p{1.9cm}|}{-113} & \multicolumn{1}{p{1.9cm}|}{-91} 
				 & \multicolumn{1}{p{1.9cm}|}{-74} & \multicolumn{1}{p{1.9cm}|}{-56} 
				 & \multicolumn{1}{p{1.9cm}|}{-42} \\
				 \hline
				 \multicolumn{1}{|p{2.1cm}|}{$\Delta U$, мкВ} & \multicolumn{1}{p{1.9cm}|}{-119} 
				 & \multicolumn{1}{p{1.9cm}|}{-101} & \multicolumn{1}{p{1.9cm}|}{-79} 
				 & \multicolumn{1}{p{1.9cm}|}{-62} & \multicolumn{1}{p{1.9cm}|}{-44} 
				 & \multicolumn{1}{p{1.9cm}|}{-30} \\
				 \hline
				 \multicolumn{1}{|p{2.1cm}|}{$\Delta T$, ℃} & \multicolumn{1}{p{1.9cm}|}{-2.75} 
				 & \multicolumn{1}{p{1.9cm}|}{-2.33} & \multicolumn{1}{p{1.9cm}|}{-1.82} 
				 & \multicolumn{1}{p{1.9cm}|}{-1.43} & \multicolumn{1}{p{1.9cm}|}{-1.02} 
				 & \multicolumn{1}{p{1.9cm}|}{-0.69} \\
				 \hline
			 \end{tabular}
		 	\end{center}
		\end{table}
		\par Как видно из графиков точки хорошо ложаться на прямые. Вычислим коэффициенты аппроксимирующих прямых и
		погрешности их определения. Угловой коэффицент прямой так же является коэффицентом Джоуля-Томпсона.
		Воспользуемся следующими формулами:
		\begin{equation}
			\mu_{\text{д-т}}= \frac{\langle\Delta T\Delta P\rangle -\langle \Delta P\rangle \langle \Delta T\rangle}
			{\langle \Delta P^2 \rangle - \langle \Delta P \rangle^2}
		\end{equation}
		\begin{equation}
			b = \langle \Delta T \rangle - k\langle \Delta P \rangle
		\end{equation}
		\begin{equation}
			\sigma_{\mu_{\text{д-т}}} = \frac{1}{\sqrt{n}} \sqrt{\frac{\langle \Delta T^2 \rangle - \langle \Delta T \rangle^2 }
			{\langle \Delta P^2 \rangle - \langle \Delta P \rangle^2} - k^2}
		\end{equation}
		\par Полученные результаты отображены на рисунке 2. 
		\subsection{Определение Ван-дер-Вальсовых коэффициентов}
		\par С помощью полученных в предыдущем пункте значений коэффициента Джоуля-Томпсона мы можем определить
		Коэффициенты Ван-дер-Вальса. Для этого воспользуемся формулой (3) и получим следующие выражения:
		\begin{equation}
   			b=\frac{C_{P}\left(\mu_{1}T_{1}-\mu_{2}T_{2}\right)}{T_{2}-T_{1}},
		\end{equation}
		\begin{equation}
   			a=\frac{1}{2}\left(\mu+bRT\right),
		\end{equation}
		\par Погрешности определения коэффициентов определим с помощью следующих выражений:
		\begin{equation}
   			\sigma_{b}=\frac{C_{P}}{\Delta T}\sqrt{T^{2}_{1}\sigma_{\mu_{1}}^{2}+T^{2}_{2}\sigma_{\mu_{2}}^{2}},
		\end{equation}
		\begin{equation}
   			\sigma_a=RT\sqrt{C^{2}_{P}\sigma_{\mu}^{2}+\sigma^{2}_{b}},
		\end{equation}
		\noindent где $\mu_{1}$ и $\mu_{2}$ коэффициенты Джоуля-Томпсона при температурах термостата $T_{1}$
		и $T_{2}$.
		\begin{figure}[H]
			\begin{minipage}[h]{0.49\linewidth}
				\begin{center}
					\input{C:/Users/gromo/PycharmProjects/plottest/plot9.tex} \\ a)
				\end{center}
			\end{minipage}
			\begin{minipage}[h]{0.49\linewidth}	
				\begin{center}
					\input{C:/Users/gromo/PycharmProjects/plottest/plot10.tex}\\ б)
				\end{center}
			\end{minipage}
			\hfill
			\begin{minipage}[h]{0.49\linewidth}
				\begin{center}
					\input{C:/Users/gromo/PycharmProjects/plottest/plot11.tex} \\ в)
				\end{center}
			\end{minipage}	
			\begin{minipage}[h]{0.49\linewidth}
				\begin{center}
					\par а) $T = 20.1$ ℃, $\mu = 1.15\pm0.03$ К/атм.
					\par б) $T = 30.1$ ℃, $\mu = 0.98\pm0.02$ К/атм.
					\par в) $T = 50.0$ ℃, $\mu = 0.89\pm0.03$ К/атм.
				\end{center}
			\end{minipage}
			\captionsetup{font={small}, justification=justified}
			\caption[figure]{Графики зависимости $\Delta T$ от $\Delta P$ при разных температурах}
		\end{figure}
		\par По полученным коэффициэнтам Ван-дер-Ваальса вычислим температуру инверсии
		$T_{\text{инв}}=\frac{2a}{Rb}$, её погрешность оценим формулой
		$\sigma_{T_{\text{инв}}}=T_{\text{инв}}\sqrt{\mathcal{E}^2_a+\mathcal{E}^2_b}$. Проведём расчёт для двух пар температур: 20 ℃ и 30 ℃, 30 ℃ и 50 ℃.
		Данные занесём в таблицу 2.
		\begin{table}[H]
			\captionsetup{font={small}, labelformat=fullparents, labelsep=fill, labelfont=bf, justification=raggedleft,
			singlelinecheck=false, skip=-0.2cm}
			\caption{Коэффициенты Вандер-дер-Ваальса}
			\begin{center}
				\begin{tabular}{|p{3.5cm}|p{3.5cm}|p{3.5cm}|p{3.5cm}|}
				\hline
				\multicolumn{2}{|>{\centering}p{7cm}|}{$T_{1}$ и $T_{2}$} &
				\multicolumn{2}{|>{\centering}p{7cm}|}{$T_{2}$ и $T_{3}$}  \\
				\hline
				$a=26\pm12 \ \frac{\text{атм}\cdot\text{м}^{6}}{\text{кмоль}}$ &
				$b=1.7\pm0.5 \ \frac{\text{ м}^{3}}{\text{кмоль}}$ &
				$a=7\pm6 \ \frac{\text{атм}\cdot\text{м}^{6}}{\text{кмоль}}$ &
				$b=0.2\pm0.3 \ \frac{\text{ м}^{3}}{\text{кмоль}}$ \\
				\hline
				\multicolumn{2}{|>{\centering}p{7cm}|}{$T_{\text{инв}}=400\pm200$ К} &
				\multicolumn{2}{|>{\centering}p{7cm}|}{$T_{\text{инв}}=963\pm1586$ К}  \\
				\hline
				\end{tabular}
			\end{center}
		\end{table}
	\section{Обсуждение результатов}
	\par В данной работе мы исследовали эффект Джоуля-Томпсона и определяли его коэффициент. Полученные результаты
	обладают хорошей точностью (порядка 3\%) и близки к табличным значениям. И вольтметр, и манометр обладают
	одинаковыми погрешностями (порядка 2\%), поэтому выделить главный источник неточностей сложно.
	\par Второй частью работы было определие коэффициентов Ван-дер-Ваальса. Не смотря на хорошую точность приборов,
	погрешность определения коэффициентов оказалась очень большой (больше 100\% в некоторых случаях). Многократные
	проверки применяемых формул не дали результатов (возможно, был неправильно выполнен перевод едениц измерения).
	Более правдоподобными выглядят результаты, полученные от данных при температурах 30.1 ℃ и 50.0 ℃, однако именно
	при в этом случае значения погрешностей выглядят слишком большими. Также причиной такого большого отклонения от
	табличных значений может являтся тот факт, что мы прводили измерения не при критических параметрах. Возможно,
	ошибка кроется и в теоретической модели, так как газ Ван-де-Ваальса не является хорошим приближением реального газа.
	\section{Вывод}
	\par Несмотря на неудачу во второй части работы, можно сказать, что она была выполнена качественно. Погрешности
	давления и температуры малы, и коэффициент Джоуля-Томсона совпал с табличными значениями. Тщательная
	проверка формул для расчёт коэффициентов Ван-дер-Ваальса, а также более точная теоретическая модель реального газа
	могут дать ответ на возникшие вопросы.
		
\end{document}
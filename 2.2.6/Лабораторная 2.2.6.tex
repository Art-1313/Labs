\documentclass[12pt,a4paper]{article}
\usepackage[utf8]{inputenc}
\usepackage[english,russian]{babel}
\usepackage{indentfirst}
\usepackage{misccorr}
\usepackage{graphicx}
\usepackage{amsmath}
\usepackage{amssymb}
\usepackage{circuitikz}
\usepackage[font={small}]{caption}
\usepackage[left=20mm, top=20mm, right=20mm, bottom=20mm, nohead]{geometry}
\usepackage{float}
\usepackage{tabularx}
\usepackage{array}
\usepackage{longtable}
\usepackage{pstool}
\usepackage{pgfplots}
\usepackage{hhline}
\usepackage{multirow}

\DeclareCaptionLabelSeparator{fill}{\\}
\DeclareCaptionLabelFormat{fullparents}{\bothIfFirst{#1}{~}#2}

\begin{document}

	\begin{titlepage}
		\begin{center}
			{\LARGE Отчёт по лабораторной работе 2.2.6.\\}
			\vspace*{10cm}
				\textbf{\LARGE Определение энергии активации по температурной зависимости вязкости жидкости.}	
			\vspace*{6cm}
		\end{center}
		\hfill\begin{minipage}{0.35\textwidth}
				Работу выполнил Грмов Артём\\
				ЛФИ Б02-006
		\end{minipage}
		\vspace{5cm}
		\begin{center}
 			Долгопрудный, 2021 г.
		\end{center}   
	\end{titlepage}
	
	\section{Аннотация}
		\noindent\textbf{Цель работы: }
		\begin{enumerate}
			\item измерение скорости падения шариков при разной температуре жидкости; 
			\item вычисление вязкости жидкости по закону Стокса и расчёт энергии активации.
		\end{enumerate}
		\noindent\textbf{В работе используются: } стеклянный цилиндр с исследуемой жидкостью (глицерин); термостат; секундомер;
		микроскоп; мелкие шарики (диаметром около 1 мм).
		\vspace{0.5 cm}
		\par По своим свойствам жидкости сходны как с газами, так и с твердыми телами. Подобно газам, жидкости принимают форму
		сосуда, в котором они находятся. Подобно твердым телам, они обладают сравнительно большой плотностью, с трудом поддаются
		сжатию. 
		\par Двойственный характер свойств жидкостей связан с особенностями движения их молекул. В жидкостях, как и в кристаллах,
		каждая молекула находится в потенциальной яме электрического поля, создаваемого окружающими молекулами. Глубина
		потенциальной ямы в жидкостях больше средней кинетической энергии колеблющейся молекулы, поэтому молекулы колеблются
		вокруг более или менее стабильных положений равновесия. Однако у жидкостей различие между этими двумя энергиями
		невелико, так что молекулы нередко выскакивают из «своей» потенциальной ямы и занимают место в другой.
		\par В отличие от твердых тел, жидкости обладают «рыхлой» структурой. В них имеются свободные места --- «дырки», благодаря
		чему молекулы могут перемещаться, покидая свое место и занимая одну из соседних дырок. Таким образом, молекулы медленно
		перемещаются внутри жидкости, пребывая часть времени около определенных мест равновесия и образуя картину меняющейся
		со временем пространственной решетки. На современном языке принято говорить, что \textit{в жидкости присутствует ближний,
		но не дальний порядок}, расположение молекул упорядочено в небольших объемах, но порядок перестает замечаться при
		увеличении расстояния.
		\par Как уже отмечалось, для того чтобы перейти в новое состояние, молекула должна преодолеть участки с большой
		потенциальной энергией, превышающей среднюю тепловую энергию молекул. Для этого тепловая энергия молекул должна ---
		вследствие флуктуации --- увеличиться на некоторую величину $W$, называемую \textit{энергией активации}.
		\par Отмеченный характер движения молекул объясняет как медленность диффузии в жидкостях, так и большую (по сравнению с
		газами) их вязкость. В газах вязкость объясняется происходящим при тепловом движении молекул переносом количества
		направленного движения. В жидкостях такие переходы существенно замедлены. Количество молекул, имеющих энергии больше
		$W$, в соответствии с формулой Больцмана экспоненциально зависит от $W$. Температурная зависимость вязкости жидкости
		выражается формулой:
		\begin{equation}
			\eta \sim Ae^{W/kT}.
		\end{equation}
		\noindent Из формулы (1) следует, что вязкость жидкости при повышении температуры должна резко уменьшаться. Если отложить
		на графике логарифм вязкости ln$\eta$ в зависимости от $1/T$, то согласно (1) должна получиться прямая линия, по угловому
		коэффициенту которой можно определить энергию активации молекулы $W$ исследуемой жидкости. Экспериментальные
		исследования показывают, что в небольших температурных интервалах эта формула неплохо описывает изменение вязкости с
		температурой.
		\par Для исследования температурной зависимости вязкости жидкости в данной работе используется метод Стокса, основанный
		на измерении скорости свободного падения шарика в жидкости.
		\par На всякое тело, двигающееся в вязкой жидкости, действует сила сопротивления. Стоксом было получено строгое решение
		задачи о ламинарном обтекании шарика безграничной жидкостью. В этом случае сила сопротивления $F$ определяется 
		формулой:
		\begin{equation}
			F=6\pi\eta rv,
		\end{equation}
		\noindent где $\eta$ --- вязкость жидкости, $v$ --- скорость шарика, $r$ --- его радиус.
		\par Рассмотрим свободное падение шарика в вязкой жидкости. На шарик действуют три силы: сила тяжести, архимедова сила и
		сила вязкости, зависящая от скорости.
		\par Найдем уравнение движения шарика в жидкости. По второму закону Ньютона:
		\begin{equation}
			Vg\left(\rho-\rho_{\text{ж}}\right)-6\pi\eta rv=V\rho\frac{dv}{dt},
		\end{equation}
		\noindent где $V$ --- объ	ём шарика, $\rho$ --- его плотность, $\rho_{\text{ж}}$ --- плотность жикости, $g$ --- ускорение свободного
		падения. Решая это уравнение, найдём
		\begin{equation}
			v\left(t\right)=v_{\text{уст}}-[v_{\text{уст}}-v\left(0\right)]e^{-t/\tau}.
		\end{equation}
		\noindent В формуле (4) приняты обозначения: $v\left(0\right)$ --- скорость шарика в момент начала его движения в жидкости,
		\begin{equation}
			v_{\text{уст}}=\frac{Vg\left(\rho-\rho_{\text{ж}}\right)}{6\pi\eta r}=\frac{2}{9}gr^2\frac{\left(\rho-\rho_{\text{ж}}\right)}{\eta},   			\text{             }\tau=\frac{V\rho}{6\pi\eta r}=\frac{2}{9}\frac{r^2\rho}{\eta}
		\end{equation}
		\par Как видно из (4), скорость шарика экспоненциально приближается к установившейся скорости $v_{\text{уст}}$.
		Установление скорости определяется величиной $\tau$, имеющей размерность времени и называющейся
		\textit{временем релаксации}. Если время падения в несколько раз больше времени релаксации, процесс установления скорости
		можно считать закончившимся.
		\par Измеряя на опыте установившуюся скорость падения шариков $v_{\text{уст}}$ и величины $r$, $\rho$, $\rho_{\text{ж}}$,
		можно определить вязкость жидкости по формуле, следующей из (5):
		\begin{equation}
			\eta=\frac{2}{9}gr^2\frac{\left(\rho-\rho_{\text{ж}}\right)}{v_{\text{уст}}}.
		\end{equation}
		
		\section{Экспериментальная установка}
		\par Для измерений используется стеклянный цилиндрический сосуд В, наполненный исследуемой жидкостью (глицерин).
		Диаметр сосуда $\approx$3 см, длина $\approx$40 см (точные размеры указаны на установке). На стенках сосуда нанесены две
		метки на некотором расстоянии друг от друга. Верхняя метка должна располагаться ниже уровня жидкости с таким расчетом,
		чтобы скорость шарика к моменту прохождения этой метки успевала установиться. Измеряя расстояние между метками с
		помощью линейки, а время падения с помощью секундомера, определяют скорость шарика $v_{\text{уст}}$. Сам сосуд B помещен
		в рубашку D, омываемую водой из термостата. При работающем термостате температура воды в рубашке D, а потому и
		температура жидкости 12 равна температуре воды в термостате.
		\par Радиусы шариков измеряются микрскопом. Для каждого шарика рекомендуется измерить несколько различных диаметров и
		вычислить среднее значение. Такое усреднение целесообразно, поскольку в работе используются шарики, форма которых может
		несколько отличаться от сферической.
		\par Схема прибора (в разрезе) и внешний вид термостата показаны на рис. 1 и рис. 2.
		\par Термостат и прибор для определения коэффициента вязкости жидкости показаны на рис. 1 (фотография термостата
		представлена на рис. 2, обозначения на рис. 1 и рис. 2 совпадают).
		\begin{figure}[H]
			\begin{minipage}[h]{0.49\linewidth}
				\includegraphics[width=0.99\textwidth]{C:/Users/gromo/Desktop/Установка.png}
				\caption{Установка для определения коэффициента вязкости жидкости}
			\end{minipage}
			\hfill
			\begin{minipage}[h]{0.49\linewidth}
				\includegraphics[width=0.99\textwidth]{C:/Users/gromo/Desktop/Термостат.png}
				\caption{Термостат}
			\end{minipage}
		\end{figure}
		\noindent\small{1 --- блок терморегулирования; 2 --- ванна; 3 --- индикаторное табло; 4 --- ручка установки температуры;
				5 --- кнопка переключения режимов установки/контроля температуры; 6 --- индикатор уровня жидкости;
				7 --- индикатор включения нагревателя; 8 --- сетевой выключатель прибора; 9 --- крышка; 
				10 --- входной и выходной патрубки насоса; 11 --- входной и выходной патрубки теплообменника.}
		
		\section{Результа измерений и обработка данных}
		\subsection{Измерение диаметров шаров}
		\par\normalsize{Определим с помощью микроскопа ($\sigma_{\text{мик}}=0.02$ мм) диаметры 20 шариков, участвующих в эксперименте.
		Будем проводить по три измерения для кажого шарика, затем вычислим средний диаметр. Занесём результаты измерений в
		таблицу 1.} 
		\begin{table}[H]
			\captionsetup{font={small}, labelformat=fullparents, labelsep=fill, labelfont=bf, justification=raggedleft,
							singlelinecheck=false, skip=-0.2cm}
			\caption{Результаты измерения диаметра шариков}
			\begin{center}
	  			\begin{tabular}{|>{\centering}p{2cm}|>{\centering}p{2cm}|>{\centering}p{2cm}|>{\centering}p{2cm}
	  							|>{\centering}p{2cm}|>{\centering}p{2cm}|>{\centering}p{2cm}|}
				      \hline
					\multicolumn{1}{|p{2cm}|}{\multirow{2}{2cm}{\centering № шара}} & 
					\multicolumn{3}{>{\centering}p{6cm}|}{Серия измерений $d$, мм} & 
					\multicolumn{1}{p{2cm}|}{\multirow{2}{2cm}{\centering$d_{\text{ср}}$, мм}} &
					\multicolumn{1}{p{2cm}|}{\multirow{2}{2cm}{\centering$\sigma_{\text{случ}}$, мм}} &
					\multicolumn{1}{p{2cm}|}{\multirow{2}{2cm}{\centering$\sigma_{\text{полн}}$, мм}} \\
					\hhline{~---~~~} & 1 & 2 & 3 & & & \\
					\hline
					\multicolumn{1}{|p{2cm}|}{1} & \multicolumn{1}{p{2cm}|}{2.14} & \multicolumn{1}{p{2cm}|}{2.06} 
					& \multicolumn{1}{p{2cm}|}{2.10} & \multicolumn{1}{p{2cm}|}{2.10} & \multicolumn{1}{p{2cm}|}{0.02} 
					& \multicolumn{1}{p{2cm}|}{0.03} \\
					\hline
					\multicolumn{1}{|p{2cm}|}{2} & \multicolumn{1}{p{2cm}|}{2.10} & \multicolumn{1}{p{2cm}|}{2.10} 
					& \multicolumn{1}{p{2cm}|}{2.10} & \multicolumn{1}{p{2cm}|}{2.10} & \multicolumn{1}{p{2cm}|}{0.00} 
					& \multicolumn{1}{p{2cm}|}{0.02} \\
					\hline
					\multicolumn{1}{|p{2cm}|}{3} & \multicolumn{1}{p{2cm}|}{0.78} & \multicolumn{1}{p{2cm}|}{0.78} 
					& \multicolumn{1}{p{2cm}|}{0.78} & \multicolumn{1}{p{2cm}|}{0.78} & \multicolumn{1}{p{2cm}|}{0.00} 
					& \multicolumn{1}{p{2cm}|}{0.02} \\		
					\hline
					\multicolumn{1}{|p{2cm}|}{4} & \multicolumn{1}{p{2cm}|}{2.08} & \multicolumn{1}{p{2cm}|}{2.10} 
					& \multicolumn{1}{p{2cm}|}{2.04} & \multicolumn{1}{p{2cm}|}{2.07} & \multicolumn{1}{p{2cm}|}{0.02} 
					& \multicolumn{1}{p{2cm}|}{0.02} \\
					\hline
					\multicolumn{1}{|p{2cm}|}{5} & \multicolumn{1}{p{2cm}|}{0.80} & \multicolumn{1}{p{2cm}|}{0.80} 
					& \multicolumn{1}{p{2cm}|}{0.78} & \multicolumn{1}{p{2cm}|}{0.79} & \multicolumn{1}{p{2cm}|}{0.01} 
					& \multicolumn{1}{p{2cm}|}{0.02} \\
					\hline
					\multicolumn{1}{|p{2cm}|}{6} & \multicolumn{1}{p{2cm}|}{0.64} & \multicolumn{1}{p{2cm}|}{0.68} 
					& \multicolumn{1}{p{2cm}|}{0.66} & \multicolumn{1}{p{2cm}|}{0.66} & \multicolumn{1}{p{2cm}|}{0.01} 
					& \multicolumn{1}{p{2cm}|}{0.02} \\
					\hline
					\multicolumn{1}{|p{2cm}|}{7} & \multicolumn{1}{p{2cm}|}{0.80} & \multicolumn{1}{p{2cm}|}{0.80} 
					& \multicolumn{1}{p{2cm}|}{0.80} & \multicolumn{1}{p{2cm}|}{0.80} & \multicolumn{1}{p{2cm}|}{0.00} 
					& \multicolumn{1}{p{2cm}|}{0.02} \\
					\hline
					\multicolumn{1}{|p{2cm}|}{8} & \multicolumn{1}{p{2cm}|}{2.02} & \multicolumn{1}{p{2cm}|}{2.04} 
					& \multicolumn{1}{p{2cm}|}{2.10} & \multicolumn{1}{p{2cm}|}{2.05} & \multicolumn{1}{p{2cm}|}{0.02} 
					& \multicolumn{1}{p{2cm}|}{0.03} \\
					\hline
					\multicolumn{1}{|p{2cm}|}{9} & \multicolumn{1}{p{2cm}|}{2.10} & \multicolumn{1}{p{2cm}|}{2.10} 
					& \multicolumn{1}{p{2cm}|}{2.10} & \multicolumn{1}{p{2cm}|}{2.10} & \multicolumn{1}{p{2cm}|}{0.00} 
					& \multicolumn{1}{p{2cm}|}{0.03} \\
					\hline
					\multicolumn{1}{|p{2cm}|}{10} & \multicolumn{1}{p{2cm}|}{2.08} & \multicolumn{1}{p{2cm}|}{2.10} 
					& \multicolumn{1}{p{2cm}|}{2.08} & \multicolumn{1}{p{2cm}|}{2.09} & \multicolumn{1}{p{2cm}|}{0.01} 
					& \multicolumn{1}{p{2cm}|}{0.02} \\
					\hline
					\multicolumn{1}{|p{2cm}|}{11} & \multicolumn{1}{p{2cm}|}{0.90} & \multicolumn{1}{p{2cm}|}{0.88} 
					& \multicolumn{1}{p{2cm}|}{0.88} & \multicolumn{1}{p{2cm}|}{0.89} & \multicolumn{1}{p{2cm}|}{0.01} 
					& \multicolumn{1}{p{2cm}|}{0.02} \\
					\hline
					\multicolumn{1}{|p{2cm}|}{12} & \multicolumn{1}{p{2cm}|}{0.58} & \multicolumn{1}{p{2cm}|}{0.58} 
					& \multicolumn{1}{p{2cm}|}{0.56} & \multicolumn{1}{p{2cm}|}{0.57} & \multicolumn{1}{p{2cm}|}{0.01} 
					& \multicolumn{1}{p{2cm}|}{0.02} \\
					\hline
					\multicolumn{1}{|p{2cm}|}{13} & \multicolumn{1}{p{2cm}|}{2.08} & \multicolumn{1}{p{2cm}|}{2.08} 
					& \multicolumn{1}{p{2cm}|}{2.08} & \multicolumn{1}{p{2cm}|}{2.08} & \multicolumn{1}{p{2cm}|}{0.00} 
					& \multicolumn{1}{p{2cm}|}{0.02} \\
					\hline				
				\end{tabular}
			\end{center}
		\end{table}
		\begin{table}[H]
			\begin{center}
				\begin{tabular}{|>{\centering}p{2cm}|>{\centering}p{2cm}|>{\centering}p{2cm}|>{\centering}p{2cm}
	  							|>{\centering}p{2cm}|>{\centering}p{2cm}|>{\centering}p{2cm}|}
	  				\hline
					\multicolumn{1}{|p{2cm}|}{14} & \multicolumn{1}{p{2cm}|}{2.10} & \multicolumn{1}{p{2cm}|}{2.08} 
					& \multicolumn{1}{p{2cm}|}{2.08} & \multicolumn{1}{p{2cm}|}{2.09} & \multicolumn{1}{p{2cm}|}{0.01} 
					& \multicolumn{1}{p{2cm}|}{0.02} \\
	  				\hline
	  				\multicolumn{1}{|p{2cm}|}{15} & \multicolumn{1}{p{2cm}|}{0.80} & \multicolumn{1}{p{2cm}|}{0.80} 
					& \multicolumn{1}{p{2cm}|}{0.84} & \multicolumn{1}{p{2cm}|}{0.81} & \multicolumn{1}{p{2cm}|}{0.01} 
					& \multicolumn{1}{p{2cm}|}{0.02} \\
					\hline	
	  				\multicolumn{1}{|p{2cm}|}{16} & \multicolumn{1}{p{2cm}|}{0.76} & \multicolumn{1}{p{2cm}|}{0.78} 
					& \multicolumn{1}{p{2cm}|}{0.80} & \multicolumn{1}{p{2cm}|}{0.78} & \multicolumn{1}{p{2cm}|}{0.01} 
					& \multicolumn{1}{p{2cm}|}{0.02} \\
					\hline
					\multicolumn{1}{|p{2cm}|}{17} & \multicolumn{1}{p{2cm}|}{2.08} & \multicolumn{1}{p{2cm}|}{2.10} 
					& \multicolumn{1}{p{2cm}|}{2.10} & \multicolumn{1}{p{2cm}|}{2.09} & \multicolumn{1}{p{2cm}|}{0.01} 
					& \multicolumn{1}{p{2cm}|}{0.02} \\
					\hline
					\multicolumn{1}{|p{2cm}|}{18} & \multicolumn{1}{p{2cm}|}{2.10} & \multicolumn{1}{p{2cm}|}{2.08} 
					& \multicolumn{1}{p{2cm}|}{2.08} & \multicolumn{1}{p{2cm}|}{2.09} & \multicolumn{1}{p{2cm}|}{0.01} 
					& \multicolumn{1}{p{2cm}|}{0.02} \\
					\hline
					\multicolumn{1}{|p{2cm}|}{19} & \multicolumn{1}{p{2cm}|}{0.80} & \multicolumn{1}{p{2cm}|}{0.78} 
					& \multicolumn{1}{p{2cm}|}{0.80} & \multicolumn{1}{p{2cm}|}{0.79} & \multicolumn{1}{p{2cm}|}{0.01} 
					& \multicolumn{1}{p{2cm}|}{0.02} \\
					\hline
					\multicolumn{1}{|p{2cm}|}{20} & \multicolumn{1}{p{2cm}|}{0.76} & \multicolumn{1}{p{2cm}|}{0.74} 
					& \multicolumn{1}{p{2cm}|}{0.76} & \multicolumn{1}{p{2cm}|}{0.75} & \multicolumn{1}{p{2cm}|}{0.01} 
					& \multicolumn{1}{p{2cm}|}{0.02} \\
					\hline
	  			\end{tabular}
			\end{center}
		\end{table}
		\par Случайные и полные погрешности вычислим по формулам:
		\begin{equation}
			\sigma_{\text{случ}}=\sqrt{\frac{1}{n(n-1)}\sum{\left(d_{i}-d_{\text{ср}}\right)^{2}}},
		\end{equation}
		\begin{equation}
			\sigma_{\text{полн}}=\sqrt{\sigma^{2}_{\text{cлуч}}+\sigma^{2}_{\text{мик}}}.
		\end{equation}
		\noindent полученные данные также занесём в таблицу 1.
		\subsection{Измерение установившейся скорости}
		\par Следущим шагом работы станет определение скоростей установившегося движения шаров при разных температурах
		глицерина. Для этого будем брасать их в сосуд и измерять время прохождения от одной метки до другой. Расстояние между
		метка составляет $l=20\pm0.05$ см. 
		\par Так же нам необходимо знать плотность шариков и плотность гицерина в зависимости от температуры. Согласно
		справочникам $\rho_{\text{стек}}=2.5\text{ кг/м}^{3}$ и $\rho_{\text{стал}}=7.8\text{ кг/м}^{3}$. Шарики с диаметром
		больше 2 мм сделаны из стекла, остальные из стали Данные о глицерине возьмём из учебника.
		\par Полученные данные занесём в таблицу 2.
		\begin{table}[H]
			\captionsetup{font={small}, labelformat=fullparents, labelsep=fill, labelfont=bf, justification=raggedleft,
							singlelinecheck=false, skip=-0.2cm}
			\caption{Результаты измерения диаметра шариков}
			\begin{center}
	  			\begin{tabular}{|>{\centering}p{2cm}|>{\centering}p{2cm}|>{\centering}p{2cm}|>{\centering}p{2cm}
	  							|>{\centering}p{2cm}|>{\centering}p{2cm}|>{\centering}p{2cm}|}
				      \hline
					\multicolumn{1}{|p{2cm}|}{№ шара}& \multicolumn{1}{p{2cm}|}{Материал} & \multicolumn{1}{p{2cm}|}{$T$, ℃} 
					& \multicolumn{1}{p{2cm}|}{$t$, с} & \multicolumn{1}{p{2cm}|}{$\rho_{\text{ж}}$, $\text{кг/м}^{3}$} 
					& \multicolumn{1}{p{2cm}|}{$\eta$, $\text{г/(с}\cdot\text{см)}$} 
					& \multicolumn{1}{p{2cm}|}{$\mathcal{E}_{\eta}$} \\
					\hline
					\multicolumn{1}{|p{2cm}|}{1} & \multicolumn{1}{p{2cm}|}{стекло} & \multicolumn{1}{p{2cm}|}{20} 
					& \multicolumn{1}{p{2cm}|}{73.2} & \multicolumn{1}{p{2cm}|}{1.260} & \multicolumn{1}{p{2cm}|}{10.91} 
					& \multicolumn{1}{p{2cm}|}{0.03} \\
					\hline
					\multicolumn{1}{|p{2cm}|}{2} & \multicolumn{1}{p{2cm}|}{стекло} & \multicolumn{1}{p{2cm}|}{20} 
					& \multicolumn{1}{p{2cm}|}{74.0} & \multicolumn{1}{p{2cm}|}{1.260} & \multicolumn{1}{p{2cm}|}{11.03} 
					& \multicolumn{1}{p{2cm}|}{0.03} \\
					\hline
					\multicolumn{1}{|p{2cm}|}{3} & \multicolumn{1}{p{2cm}|}{сталь} & \multicolumn{1}{p{2cm}|}{20} 
					& \multicolumn{1}{p{2cm}|}{95.1} & \multicolumn{1}{p{2cm}|}{1.260} & \multicolumn{1}{p{2cm}|}{10.31} 
					& \multicolumn{1}{p{2cm}|}{0.08} \\
					\hline
					\multicolumn{1}{|p{2cm}|}{4} & \multicolumn{1}{p{2cm}|}{стекло} & \multicolumn{1}{p{2cm}|}{30} 
					& \multicolumn{1}{p{2cm}|}{37.2} & \multicolumn{1}{p{2cm}|}{1.254} & \multicolumn{1}{p{2cm}|}{5.43} 
					& \multicolumn{1}{p{2cm}|}{0.03} \\
					\hline
					\multicolumn{1}{|p{2cm}|}{5} & \multicolumn{1}{p{2cm}|}{сталь} & \multicolumn{1}{p{2cm}|}{20} 
					& \multicolumn{1}{p{2cm}|}{91.7} & \multicolumn{1}{p{2cm}|}{1.260} & \multicolumn{1}{p{2cm}|}{10.29} 
					& \multicolumn{1}{p{2cm}|}{0.08} \\
					\hline
					\multicolumn{1}{|p{2cm}|}{6} & \multicolumn{1}{p{2cm}|}{сталь} & \multicolumn{1}{p{2cm}|}{30} 
					& \multicolumn{1}{p{2cm}|}{64.3} & \multicolumn{1}{p{2cm}|}{1.254} & \multicolumn{1}{p{2cm}|}{5.00} 
					& \multicolumn{1}{p{2cm}|}{0.09} \\
					\hline
					\multicolumn{1}{|p{2cm}|}{7} & \multicolumn{1}{p{2cm}|}{сталь} & \multicolumn{1}{p{2cm}|}{30} 
					& \multicolumn{1}{p{2cm}|}{42.4} & \multicolumn{1}{p{2cm}|}{1.254} & \multicolumn{1}{p{2cm}|}{4.84} 
					& \multicolumn{1}{p{2cm}|}{0.08} \\
					\hline
					\multicolumn{1}{|p{2cm}|}{8} & \multicolumn{1}{p{2cm}|}{стекло} & \multicolumn{1}{p{2cm}|}{30} 
					& \multicolumn{1}{p{2cm}|}{37.5} & \multicolumn{1}{p{2cm}|}{1.254} & \multicolumn{1}{p{2cm}|}{5.37} 
					& \multicolumn{1}{p{2cm}|}{0.03} \\
					\hline
					\multicolumn{1}{|p{2cm}|}{9} & \multicolumn{1}{p{2cm}|}{стекло} & \multicolumn{1}{p{2cm}|}{40} 
					& \multicolumn{1}{p{2cm}|}{17.8} & \multicolumn{1}{p{2cm}|}{1.251} & \multicolumn{1}{p{2cm}|}{2.67} 
					& \multicolumn{1}{p{2cm}|}{0.04} \\
					\hline
					\multicolumn{1}{|p{2cm}|}{10} & \multicolumn{1}{p{2cm}|}{стекло} & \multicolumn{1}{p{2cm}|}{40} 
					& \multicolumn{1}{p{2cm}|}{16.4} & \multicolumn{1}{p{2cm}|}{1.251} & \multicolumn{1}{p{2cm}|}{2.43} 
					& \multicolumn{1}{p{2cm}|}{0.04} \\
					\hline
					\multicolumn{1}{|p{2cm}|}{11} & \multicolumn{1}{p{2cm}|}{сталь} & \multicolumn{1}{p{2cm}|}{40} 
					& \multicolumn{1}{p{2cm}|}{17.6} & \multicolumn{1}{p{2cm}|}{1.251} & \multicolumn{1}{p{2cm}|}{2.47} 
					& \multicolumn{1}{p{2cm}|}{0.07} \\
					\hline
					\multicolumn{1}{|p{2cm}|}{12} & \multicolumn{1}{p{2cm}|}{сталь} & \multicolumn{1}{p{2cm}|}{40} 
					& \multicolumn{1}{p{2cm}|}{40.9} & \multicolumn{1}{p{2cm}|}{1.251} & \multicolumn{1}{p{2cm}|}{2.40} 
					& \multicolumn{1}{p{2cm}|}{0.11} \\
					\hline
					\multicolumn{1}{|p{2cm}|}{13} & \multicolumn{1}{p{2cm}|}{стекло} & \multicolumn{1}{p{2cm}|}{50} 
					& \multicolumn{1}{p{2cm}|}{9.3} & \multicolumn{1}{p{2cm}|}{1.247} & \multicolumn{1}{p{2cm}|}{1.37} 
					& \multicolumn{1}{p{2cm}|}{0.06} \\
					\hline
					\multicolumn{1}{|p{2cm}|}{14} & \multicolumn{1}{p{2cm}|}{стекло} & \multicolumn{1}{p{2cm}|}{50} 
					& \multicolumn{1}{p{2cm}|}{9.3} & \multicolumn{1}{p{2cm}|}{1.247} & \multicolumn{1}{p{2cm}|}{1.38} 
					& \multicolumn{1}{p{2cm}|}{0.06} \\
					\hline
					\multicolumn{1}{|p{2cm}|}{15} & \multicolumn{1}{p{2cm}|}{сталь} & \multicolumn{1}{p{2cm}|}{50} 
					& \multicolumn{1}{p{2cm}|}{11.3} & \multicolumn{1}{p{2cm}|}{1.247} & \multicolumn{1}{p{2cm}|}{1.33} 
					& \multicolumn{1}{p{2cm}|}{0.09} \\
					\hline	
	  				\multicolumn{1}{|p{2cm}|}{16} & \multicolumn{1}{p{2cm}|}{сталь} & \multicolumn{1}{p{2cm}|}{50} 
					& \multicolumn{1}{p{2cm}|}{11.4} & \multicolumn{1}{p{2cm}|}{1.247} & \multicolumn{1}{p{2cm}|}{1.24} 
					& \multicolumn{1}{p{2cm}|}{0.09} \\
					\hline
					\multicolumn{1}{|p{2cm}|}{17} & \multicolumn{1}{p{2cm}|}{стекло} & \multicolumn{1}{p{2cm}|}{60} 
					& \multicolumn{1}{p{2cm}|}{5.1} & \multicolumn{1}{p{2cm}|}{1.245} & \multicolumn{1}{p{2cm}|}{0.76} 
					& \multicolumn{1}{p{2cm}|}{0.10} \\
					\hline
				\end{tabular}
			\end{center}
		\end{table}
		\begin{table}[H]
			\begin{center}
	  			\begin{tabular}{|>{\centering}p{2cm}|>{\centering}p{2cm}|>{\centering}p{2cm}|>{\centering}p{2cm}
	  							|>{\centering}p{2cm}|>{\centering}p{2cm}|>{\centering}p{2cm}|}
				      \hline
				      \multicolumn{1}{|p{2cm}|}{18} & \multicolumn{1}{p{2cm}|}{стекло} & \multicolumn{1}{p{2cm}|}{60} 
					& \multicolumn{1}{p{2cm}|}{5.3} & \multicolumn{1}{p{2cm}|}{1.245} & \multicolumn{1}{p{2cm}|}{0.79} 
					& \multicolumn{1}{p{2cm}|}{0.10} \\
					\hline
					\multicolumn{1}{|p{2cm}|}{19} & \multicolumn{1}{p{2cm}|}{сталь} & \multicolumn{1}{p{2cm}|}{60} 
					& \multicolumn{1}{p{2cm}|}{6.7} & \multicolumn{1}{p{2cm}|}{1.245} & \multicolumn{1}{p{2cm}|}{0.75} 
					& \multicolumn{1}{p{2cm}|}{0.11} \\
					\hline
					\multicolumn{1}{|p{2cm}|}{20} & \multicolumn{1}{p{2cm}|}{сталь} & \multicolumn{1}{p{2cm}|}{60} 
					& \multicolumn{1}{p{2cm}|}{7.6} & \multicolumn{1}{p{2cm}|}{1.245} & \multicolumn{1}{p{2cm}|}{0.77} 
					& \multicolumn{1}{p{2cm}|}{0.10} \\
					\hline				
				\end{tabular}
			\end{center}
		\end{table}
		\par Погрешность измерения времени секундомером (с учётом реакции человека) $\sigma_{\text{сек}=0.5} с$. Погрешность
		измерения температуры пренебрежимо мала (меньше 0.3\%). Исподьзуя формулу (6) и формулу для нахождения погрешности
		степенной функции имеем:
		\begin{equation}
			\mathcal{E}_{\eta}=\sqrt{\left(\frac{\sigma_{t}}{t}\right)^{2}+4\left(\frac{\sigma_{d}}{d}\right)^{2}
			+\left(\frac{\sigma_{l}}{l}\right)^{2}}
		\end{equation}
		\subsection{Измерения числа Рейнольдса}
		\par Приведённая выше теория для расчёта вязкости справедлива, если обтекание шарика жидкостью имеет ламинарнй
		характер. Ламинарность определяется числом Рейнольдса Re$=vr\rho_{\text{ж}}$. Обтекание является ламинарным лишь при
		малых значених Re (меньше 0.5). Вычислим число Рейнольдса, а также найдём время релаксации $\tau$ из (5) и путь
		$S=v_{\text{уст}}\tau$. Погрешности снова вычислим по формулам для степенной функции. Результаты занесём в таблицу 3.
		\begin{table}[H]
			\captionsetup{font={small}, labelformat=fullparents, labelsep=fill, labelfont=bf, justification=raggedleft,
							singlelinecheck=false, skip=-0.2cm}
			\caption{Результаты измерения диаметра шариков}
			\begin{center}
	  			\begin{tabular}{|>{\centering}p{2cm}|>{\centering}p{2cm}|>{\centering}p{2cm}|>{\centering}p{2cm}
	  							|>{\centering}p{2cm}|>{\centering}p{2cm}|>{\centering}p{2cm}|}
				      \hline
					\multicolumn{1}{|p{2cm}|}{№ шара}& \multicolumn{1}{p{2cm}|}{Re} 
					& \multicolumn{1}{p{2cm}|}{$\mathcal{E}_{Re}$} 
					& \multicolumn{1}{p{2cm}|}{$\tau$, мс} & \multicolumn{1}{p{2cm}|}{$\mathcal{E}_{\tau}$} 
					& \multicolumn{1}{p{2cm}|}{$S$, мм} 
					& \multicolumn{1}{p{2cm}|}{$\mathcal{E}_{S}$} \\
					\hline
					\multicolumn{1}{|p{2cm}|}{1} & \multicolumn{1}{p{2cm}|}{0.00} & \multicolumn{1}{p{2cm}|}{0.03} 
					& \multicolumn{1}{p{2cm}|}{0.56} & \multicolumn{1}{p{2cm}|}{0.04} 
					& \multicolumn{1}{p{2cm}|}{0.00} & \multicolumn{1}{p{2cm}|}{0.04} \\
					\hline
					\multicolumn{1}{|p{2cm}|}{2} & \multicolumn{1}{p{2cm}|}{0.00} & \multicolumn{1}{p{2cm}|}{0.03} 
					& \multicolumn{1}{p{2cm}|}{0.56} & \multicolumn{1}{p{2cm}|}{0.04}  
					& \multicolumn{1}{p{2cm}|}{0.00} & \multicolumn{1}{p{2cm}|}{0.04} \\
					\hline
					\multicolumn{1}{|p{2cm}|}{3} & \multicolumn{1}{p{2cm}|}{0.00} & \multicolumn{1}{p{2cm}|}{0.08} 
					& \multicolumn{1}{p{2cm}|}{0.26} & \multicolumn{1}{p{2cm}|}{0.09}  
					& \multicolumn{1}{p{2cm}|}{0.00} & \multicolumn{1}{p{2cm}|}{0.09}\\		
					\hline
					\multicolumn{1}{|p{2cm}|}{4} & \multicolumn{1}{p{2cm}|}{0.01} & \multicolumn{1}{p{2cm}|}{0.04} 
					& \multicolumn{1}{p{2cm}|}{0.01} & \multicolumn{1}{p{2cm}|}{0.04} 
					& \multicolumn{1}{p{2cm}|}{0.01} & \multicolumn{1}{p{2cm}|}{0.04} \\
					\hline
					\multicolumn{1}{|p{2cm}|}{5} & \multicolumn{1}{p{2cm}|}{0.00} & \multicolumn{1}{p{2cm}|}{0.08} 
					& \multicolumn{1}{p{2cm}|}{0.27} & \multicolumn{1}{p{2cm}|}{0.09} 
					& \multicolumn{1}{p{2cm}|}{0.00} & \multicolumn{1}{p{2cm}|}{0.09} \\
					\hline
					\multicolumn{1}{|p{2cm}|}{6} & \multicolumn{1}{p{2cm}|}{0.00} & \multicolumn{1}{p{2cm}|}{0.10} 
					& \multicolumn{1}{p{2cm}|}{0.38} & \multicolumn{1}{p{2cm}|}{0.12} 
					& \multicolumn{1}{p{2cm}|}{0.00} & \multicolumn{1}{p{2cm}|}{0.12} \\
					\hline
					\multicolumn{1}{|p{2cm}|}{7} & \multicolumn{1}{p{2cm}|}{0.00} & \multicolumn{1}{p{2cm}|}{0.08} 
					& \multicolumn{1}{p{2cm}|}{0.57} & \multicolumn{1}{p{2cm}|}{0.09} 
					& \multicolumn{1}{p{2cm}|}{0.00} & \multicolumn{1}{p{2cm}|}{0.09} \\
					\hline
					\multicolumn{1}{|p{2cm}|}{8} & \multicolumn{1}{p{2cm}|}{0.01} & \multicolumn{1}{p{2cm}|}{0.04} 
					& \multicolumn{1}{p{2cm}|}{1.09} & \multicolumn{1}{p{2cm}|}{0.04} 
					& \multicolumn{1}{p{2cm}|}{0.01} & \multicolumn{1}{p{2cm}|}{0.05} \\
					\hline
					\multicolumn{1}{|p{2cm}|}{9} & \multicolumn{1}{p{2cm}|}{0.06} & \multicolumn{1}{p{2cm}|}{0.05} 
					& \multicolumn{1}{p{2cm}|}{2.29} & \multicolumn{1}{p{2cm}|}{0.04} 
					& \multicolumn{1}{p{2cm}|}{0.03} & \multicolumn{1}{p{2cm}|}{0.05} \\
					\hline
					\multicolumn{1}{|p{2cm}|}{10} & \multicolumn{1}{p{2cm}|}{0.07} & \multicolumn{1}{p{2cm}|}{0.05} 
					& \multicolumn{1}{p{2cm}|}{2.49} & \multicolumn{1}{p{2cm}|}{0.05} 
					& \multicolumn{1}{p{2cm}|}{0.03} & \multicolumn{1}{p{2cm}|}{0.06} \\
					\hline
					\multicolumn{1}{|p{2cm}|}{11} & \multicolumn{1}{p{2cm}|}{0.03} & \multicolumn{1}{p{2cm}|}{0.08} 
					& \multicolumn{1}{p{2cm}|}{1.38} & \multicolumn{1}{p{2cm}|}{0.09} 
					& \multicolumn{1}{p{2cm}|}{0.02} & \multicolumn{1}{p{2cm}|}{0.09} \\
					\hline
					\multicolumn{1}{|p{2cm}|}{12} & \multicolumn{1}{p{2cm}|}{0.01} & \multicolumn{1}{p{2cm}|}{0.11} 
					& \multicolumn{1}{p{2cm}|}{0.59} & \multicolumn{1}{p{2cm}|}{0.13}  
					& \multicolumn{1}{p{2cm}|}{0.00} & \multicolumn{1}{p{2cm}|}{0.13} \\
					\hline
					\multicolumn{1}{|p{2cm}|}{13} & \multicolumn{1}{p{2cm}|}{0.20} & \multicolumn{1}{p{2cm}|}{0.08} 
					& \multicolumn{1}{p{2cm}|}{4.37} & \multicolumn{1}{p{2cm}|}{0.06}  
					& \multicolumn{1}{p{2cm}|}{0.09} &\multicolumn{1}{p{2cm}|}{0.08} \\
					\hline
					\multicolumn{1}{|p{2cm}|}{14} & \multicolumn{1}{p{2cm}|}{0.20} & \multicolumn{1}{p{2cm}|}{0.08} 
					& \multicolumn{1}{p{2cm}|}{4.37} & \multicolumn{1}{p{2cm}|}{0.06}  
					& \multicolumn{1}{p{2cm}|}{0.09} & \multicolumn{1}{p{2cm}|}{0.08} \\
					\hline
					\multicolumn{1}{|p{2cm}|}{15} & \multicolumn{1}{p{2cm}|}{0.07} & \multicolumn{1}{p{2cm}|}{0.10} 
					& \multicolumn{1}{p{2cm}|}{2.15} & \multicolumn{1}{p{2cm}|}{0.11} 
					& \multicolumn{1}{p{2cm}|}{0.04} & \multicolumn{1}{p{2cm}|}{0.11} \\
					\hline	
	  				\multicolumn{1}{|p{2cm}|}{16} & \multicolumn{1}{p{2cm}|}{0.07} & \multicolumn{1}{p{2cm}|}{0.10} 
					& \multicolumn{1}{p{2cm}|}{2.13} & \multicolumn{1}{p{2cm}|}{0.11} 
					& \multicolumn{1}{p{2cm}|}{0.04} & \multicolumn{1}{p{2cm}|}{0.12} \\
					\hline
					\multicolumn{1}{|p{2cm}|}{17} & \multicolumn{1}{p{2cm}|}{0.67} & \multicolumn{1}{p{2cm}|}{0.14} 
					& \multicolumn{1}{p{2cm}|}{7.96} & \multicolumn{1}{p{2cm}|}{0.10} 
					& \multicolumn{1}{p{2cm}|}{0.31} & \multicolumn{1}{p{2cm}|}{0.14} \\
					\hline
					\multicolumn{1}{|p{2cm}|}{18} & \multicolumn{1}{p{2cm}|}{0.62} & \multicolumn{1}{p{2cm}|}{0.14} 
					& \multicolumn{1}{p{2cm}|}{7.66} & \multicolumn{1}{p{2cm}|}{0.10} 
					& \multicolumn{1}{p{2cm}|}{0.29} & \multicolumn{1}{p{2cm}|}{0.14} \\
					\hline
					\multicolumn{1}{|p{2cm}|}{19} & \multicolumn{1}{p{2cm}|}{0.20} & \multicolumn{1}{p{2cm}|}{0.13} 
					& \multicolumn{1}{p{2cm}|}{3.62} & \multicolumn{1}{p{2cm}|}{0.12}  
					& \multicolumn{1}{p{2cm}|}{0.11} & \multicolumn{1}{p{2cm}|}{0.14} \\
					\hline
					\multicolumn{1}{|p{2cm}|}{20} & \multicolumn{1}{p{2cm}|}{0.16} & \multicolumn{1}{p{2cm}|}{0.13} 
					& \multicolumn{1}{p{2cm}|}{3.19} & \multicolumn{1}{p{2cm}|}{0.12}  
					& \multicolumn{1}{p{2cm}|}{0.08} & \multicolumn{1}{p{2cm}|}{0.14} \\
					\hline				
				\end{tabular}
			\end{center}
		\end{table} 
		\par По таблице видно, что в двух опытах не соблюдены условия ламинарности обтекания. Будем учитывать этот факт при
		дальнейшем аналтзе данных.
		\subsection{Вычисление энергии активации}
		\par Для вычисления $W$ построим график  ln$\eta$ от $1/T$. Затем с
		по МНК вычислим угловой коэффициент. С помощью него найдём энергию активации по
		формуле:
		\begin{equation}
			W=ak_{\text{Б}},
		\end{equation}
		\noindent где $a$ --- угловой коэффицент, а $k_{\text{Б}}$ --- постоянная Больцмана.
		\par Для построения графика (Рис. 3) построим таблицу 4, в которой запишем значения ln$\eta$ от $1/T$.
		\begin{table}[H]
			\captionsetup{font={small}, labelformat=fullparents, labelsep=fill, labelfont=bf, justification=raggedleft,
							singlelinecheck=false, skip=-0.2cm}
			\caption{Результаты измерения диаметра шариков}
			\begin{center}
				\begin{tabular}{|p{2.2cm}|p{1.0cm}|p{1.0cm}|p{1.0cm}|p{1.0cm}|
								p{1.0cm}|p{1.0cm}|p{1.0cm}|p{1.0cm}|p{1.0cm}|p{1.0cm}|}
					\hline
					ln$\eta$ & 2.39 & 2.40 & 2.33 & 1.69 & 2.33 & 1.61 & 1.58 & 1.68 & 0.98 & 0.89 \\
					\hline
					$\sigma_{\text{ln}\eta}$ & 0.03 & 0.03 & 0.08 & 0.03 & 0.08 & 0.09 & 0.08 & 0.03 & 0.04 & 0.04 \\
					\hline
					$100/T$, 1/K & 0.34 & 0.34 & 0.34 & 0.33 & 0.34 & 0.33 & 0.33 & 0.33 & 0.32 & 0.32 \\
					\hline
					ln$\eta$  & 0.90 & 0.88 & 0.32 & 0.32 & 0.29 & 0.21 & -0.27 & -0.24 & -0.28 & -0.26 \\
					\hline
					$\sigma_{\text{ln}\eta}$ & 0.07 & 0.11 & 0.06 & 0.06 & 0.09 & 0.09 & 0.10 & 0.10 & 0.11 & 0.10 \\
					\hline
					$1/T$, 1/K & 0.32 & 0.32 & 0.31 & 0.31 & 0.31 & 0.31 & 0.30 & 0.30 & 0.30 & 0.30 \\
					\hline
				\end{tabular}
			\end{center}
		\end{table}
		\begin{figure}[H]
			\begin{center}
				% This file was created with tikzplotlib v0.9.16.
\begin{tikzpicture}

\begin{axis}[
    height=11cm,
    tick align=inside,
    major tick length=0.2cm,
    minor tick length=0.1cm,
    tick pos=left,
    xmin=-4.1, xmax=4.1,
    xtick={-5, -4, -3, -2, -1, 0, 1, 2, 3, 4, 5},
    minor x tick num=4,
    xmajorgrids,
    minor x grid style={dotted,black},
    xminorgrids,
    xtick style={color=black},
    xlabel={$m$},
    ymin=-1.2, ymax=1.2,
    ytick={-1.5, -1, -0.5, 0, 0.5, 1, 1.5},
    minor y tick num=4,
    ymajorgrids,
    minor y grid style={dotted,black},
    yminorgrids,
    ytick style={color=black},
    ylabel={$x_m$, мм}
]
\path [draw=red, semithick]
(axis cs:1,0.14)
--(axis cs:1,0.18);

\path [draw=red, semithick]
(axis cs:2,0.48)
--(axis cs:2,0.52);

\path [draw=red, semithick]
(axis cs:3,0.7)
--(axis cs:3,0.74);

\path [draw=red, semithick]
(axis cs:4,1.02)
--(axis cs:4,1.06);

\path [draw=red, semithick]
(axis cs:-1,-0.18)
--(axis cs:-1,-0.14);

\path [draw=red, semithick]
(axis cs:-2,-0.5)
--(axis cs:-2,-0.46);

\path [draw=red, semithick]
(axis cs:-3,-0.74)
--(axis cs:-3,-0.7);

\path [draw=red, semithick]
(axis cs:-4,-1.08)
--(axis cs:-4,-1.04);

\addplot [semithick, black]
table {%
1 0.249999999998364
2 0.499999999996728
3 0.749999999995093
4 0.999999999993457
-1 -0.249999999998364
-2 -0.499999999996728
-3 -0.749999999995093
-4 -0.999999999993457
};
\addplot [semithick, red, mark=square*, mark size=3, mark options={solid}, only marks]
table {%
1 0.16
2 0.5
3 0.72
4 1.04
-1 -0.16
-2 -0.48
-3 -0.72
-4 -1.06
};
\end{axis}

\end{tikzpicture}

			\end{center}
			\captionsetup{font={small}, justification=justified}
			\caption[figure]{График зависимости ln$\eta$ от $1/T$}
		\end{figure}
		\par Погрешность логарифма вычисляется с помощью основного уравнения для погрешностей, в итоге придём к следущему 
		выражению:
		\begin{equation}
			\sigma_{\text{ln}\eta}=\frac{\sigma_{\eta}}{\eta}=\mathcal{E}_{\eta}
		\end{equation}
		\par С помощью метода наименьших квадратов вычислим характеристики прямой ln$\eta=b+a\frac{1}{T}$:
		\begin{equation}
			a=\frac{\langle\text{ln}\eta/T\rangle-\langle\text{ln}\eta\rangle\langle1/T\rangle}
					{\langle1/T^{2}\rangle-\langle1/T\rangle^{2}},
			\text{       }b=\langle\text{ln}\eta\rangle-a\langle1/T\rangle
		\end{equation}
		\par Погрешность определения коэфициента $a$ вчислим по следующей формуле:
		\begin{equation}
			\sigma_a=\frac{1}{\sqrt{N}}
			\sqrt{\frac{\langle\text{ln}^{2}\eta\rangle-\langle\text{ln}\eta\rangle^{2}}
						{\langle1/T^{2}\rangle-\langle1/T\rangle^{2}}-a^2},
		\end{equation}
		\noindent где $N$ --- число шариков.
		\par Теперь нетрудно вычислить энергию активации $W=557\pm6$ мЭв.
		
		\section{Обсуждение результатов}
		\par Точность измерения энергии активации оказалась довольно высокой (порядка 1\%). Наибольший вклад в погрешность
		измерений внесло измерение времени, погрешность иногда достигала значения в 10\%. Судя по графику (Рис. 3) все
		измерения легли на теоретическую прямую в пределах погрешности, что говорит о правильности наших допущений. Не
		обнаружено отклонений от прямолинейной зависимости, следовательно интервал рабочих температур был выбран верно.
		Небольшие отклонения от ламинарного течения также не сказались на результате.
		
		\section{Вывод} 
		\par Данная работа показывет, что метод Стокса является довольно точным методом определения вязкости жидкости и
		её энергии активации. Однако он может приводить к неправильным результатам при неламинарном обтекании тела.
		Дальнейшим развитием данной работы могло бы стать исследование применимости метода Стокса при больших интервалах
		рабочих температур. Так же может быть исследована зависимоть парметра $A$ в формуле (1) от тумпературы. Все 
		поставленные в работе цели были достигнуты. Для повышения точности следует точнее измерять время.
		
\end{document}